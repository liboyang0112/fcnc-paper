%-------------------------------------------------------------------------------
%DIF LATEXDIFF DIFFERENCE FILE
%DIF DEL ../fcnc-paper_v0.5/ANA-TOPQ-2019-17-PAPER.tex   Wed Feb 23 22:46:52 2022
%DIF ADD ANA-TOPQ-2019-17-PAPER.tex                      Thu Feb 24 16:06:42 2022
% This file provides a skeleton ATLAS paper.
%-------------------------------------------------------------------------------
\pdfoutput=1
% The \pdfoutput command is needed by arXiv/JHEP/JINST to ensure use of pdflatex.
% It should be included in the first 5 lines of the file.
\pdfinclusioncopyfonts=1
% This command may be needed in order to get \ell in PDF plots to appear. Found in
% https://tex.stackexchange.com/questions/322010/pdflatex-glyph-undefined-symbols-disappear-from-included-pdf
%-------------------------------------------------------------------------------
% Specify where ATLAS LaTeX style files can be found.
\newcommand*{\ATLASLATEXPATH}{latex/}
% Use this variant if the files are in a central location, e.g. $HOME/texmf.
% \newcommand*{\ATLASLATEXPATH}{}
%-------------------------------------------------------------------------------
%\documentclass[cernpreprint, PAPER, atlasdraft=false, texlive=2016, UKenglish]{\ATLASLATEXPATH atlasdoc}
\documentclass[PAPER, coverpage, atlasdraft=true, texlive=2016, UKenglish]{\ATLASLATEXPATH atlasdoc}
%\documentclass[PAPER, atlasdraft=true, texlive=2016, UKenglish]{\ATLASLATEXPATH atlasdoc}
%\documentclass[CONF, atlasdraft=true, texlive=2016, UKenglish]{\ATLASLATEXPATH atlasdoc}
% The language of the document must be set: usually UKenglish or USenglish.
% british and american also work!
% Commonly used options:
%  atlasdraft=true|false This document is an ATLAS draft.
%  texlive=YYYY          Specify TeX Live version (2016 is default).
%  coverpage             Create ATLAS draft cover page for collaboration circulation.
%                        See atlas-draft-cover.tex for a list of variables that should be defined.
%  cernpreprint          Create front page for a CERN preprint.
%                        See atlas-preprint-cover.tex for a list of variables that should be defined.
%  NOTE                  The document is an ATLAS note (draft).
%  PAPER                 The document is an ATLAS paper (draft).
%  CONF                  The document is a CONF note (draft).
%  PUB                   The document is a PUB note (draft).
%  BOOK                  The document is of book form, like an LOI or TDR (draft)
%  txfonts=true|false    Use txfonts rather than the default newtx
%  paper=a4|letter       Set paper size to A4 (default) or letter.

%-------------------------------------------------------------------------------
% Extra packages:
\usepackage[block=none]{\ATLASLATEXPATH atlaspackage}
%\usepackage{\ATLASLATEXPATH atlaspackage}
% Commonly used options:
%  biblatex=true|false   Use biblatex (default) or bibtex for the bibliography.
%  backend=bibtex        Use the bibtex backend rather than biber.
%  subfigure|subfig|subcaption  to use one of these packages for figures in figures.
%  minimal               Minimal set of packages.
%  default               Standard set of packages.
%  full                  Full set of packages.
%-------------------------------------------------------------------------------
% Style file with biblatex options for ATLAS documents.
\usepackage{\ATLASLATEXPATH atlasbiblatex}

% Useful macros
\usepackage{\ATLASLATEXPATH atlasphysics}

\usepackage{\ATLASLATEXPATH atlascover}
\usepackage{multirow}
\usepackage{FCNCPaths}
\usepackage{FCNCsupport-defs}
\usepackage{float}
%add line numbers in draft
\usepackage{lineno}
\linenumbers
% See doc/atlas_physics.pdf for a list of the defined symbols.
% Default options are:
%   true:  journal, misc, particle, unit, xref
%   false: BSM, heppparticle, hepprocess, hion, jetetmiss, math, process, other, texmf
% See the package for details on the options.

% Files with references for use with biblatex.
% Note that biber gives an error if it finds empty bib files.
\addbibresource{ANA-TOPQ-2019-17-PAPER.bib}
\addbibresource{bib/ATLAS.bib}
\addbibresource{bib/CMS.bib}
\addbibresource{bib/ConfNotes.bib}
\addbibresource{bib/PubNotes.bib}
\addbibresource{bib/ATLAS-useful.bib}

% Paths for figures - do not forget the / at the end of the directory name.
\graphicspath{{logos/}{figures/}}

% Add you own definitions here (file ANA-TOPQ-2019-17-PAPER-defs.sty).
\usepackage{ANA-TOPQ-2019-17-PAPER-defs}

% for special fonts
\pdfinclusioncopyfonts=1 

%-------------------------------------------------------------------------------
% Generic document information
%-------------------------------------------------------------------------------

% Title, abstract and document
%-------------------------------------------------------------------------------
% This file contains the title, author and abstract.
% It also contains all relevant document numbers used by the different cover pages.
%-------------------------------------------------------------------------------
% Title
%\AtlasTitle{Search for flavor-changing neutral current $t\rightarrow Hq$ ($q=u,c$) decays in pp collisions at $\sqrt{s}$=13 TeV with the ATLAS detector }
%DIF 98c98-99
%DIF < \AtlasTitle{Search for flavour-changing neutral current $tqH$ interactions of the top quark and the Higgs boson which decays into a pair of $\tau$-leptons in pp collisions at $\sqrt{s}=13~\tev$ with the ATLAS detector\vspace{-0.1em}}
%DIF -------
\AtlasTitle{Search for flavour-changing neutral current interactions of the top quark and the Higgs %DIF > 
boson which decays into a pair of $\tau$-leptons in pp collisions at $\sqrt{s}=13~\tev$ with the ATLAS detector\vspace{-0.10em} } %DIF > 
%DIF -------

% Draft version:
% Should be 1.0 for the first circulation, and 2.0 for the second circulation.
% If given, adds draft version on front page, a 'DRAFT' box on top of each other page, 
% and line numbers.
% Comment or remove in final version.
\AtlasVersion{0.5}
% Abstract - % directly after { is important for correct indentation
\AtlasAbstract{%
%DIF 108a109
\vspace{0.10em} %DIF > 
%DIF -------
A search for flavour-changing neutral current (FCNC) $tqH$ interactions involving a top quark, an up-type quark ($q=u, c$), and a
Standard Model (SM) Higgs boson decaying into a $\tau$-lepton pair ($H\rightarrow \tau^+\tau^-$) is presented.
%DIF 110d112
%DIF < %The Higgs boson decays into a pair of $\tau$-leptons, $H\rightarrow \tau^+\tau^-$, is considered.
%DIF -------
The search is based on a dataset of $pp$ collisions at $\sqrt{s}=13~\tev$ recorded with the ATLAS detector at the 
Large Hadron Collider that corresponds to an integrated luminosity of 139 fb$^{-1}$.
Two processes are considered:  single top quark FCNC production in association with a Higgs boson ($pp\rightarrow tH$), and top quark pair production in
which one of top quarks decays into $Wb$ and the other decays into $qH$ through the FCNC interactions.
%DIF 115d116
%DIF < %top quark with FCNC decay of $t\rightarrow qH$. 
%DIF -------
The search selects events with two hadronically decaying $\tau$-lepton candidates ($\tauhad$) or at least one $\tauhad$ 
with an additional lepton (e,$\mu$),
as well as multiple jets. Event kinematics are used to separate signal from the background through a multivariate discriminant.  
%DIF 119-121c119
%DIF < %Multivariate techniques are used to separate the signal from the background on the basis of their different kinematics.
%DIF < %No significant excess of events above the background expectation is found,
%DIF < A slight excess of data is observed above the expected SM background with a significance of 2.6$\sigma$,
%DIF -------
A slight excess of data is observed above the expected SM background with a significance of 2.3$\sigma$, %DIF > 
%DIF -------
and 95\% CL upper limits on the $t\to qH$ branching ratios are derived.
Observed (expected) 95\% CL upper limits are set on the $t\to cH$ and $t\to uH$ branching ratios of $9.9 \times 10^{-4}$ ($5.0^{+2.2}_{-1.4}\times 10^{-4}$) and $7.2\times 10^{-4}$ ($3.6^{+1.7}_{-1.0}\times 10^{-4}$), respectively. 
The corresponding combined observed (expected) upper limits on the dimension-6 operator Wilson coefficients in
%DIF 125c123
%DIF < the effective $tqH$ couplings are $C_{c\phi} <1.38\, (0.97)$ and $C_{u\phi} <1.18\, (0.83)$, respectively.} 
%DIF -------
the effective $tqH$ couplings are $C_{c\phi} <1.38\, (0.97)$ and $C_{u\phi} <1.18\, (0.83)$ for the new physics scale $\Lambda$ at 1~TeV, respectively.}  %DIF > 
%DIF -------
%$|\lambda_{tcH}|$ and $|\lambda_{tuH}|$ couplings are 0.064 (0.055) and 0.066 (0.055), respectively. 
%These are the most restrictive direct bounds on $tqH$ interactions obtained so far.%}
% Author - this does not work with revtex (add it after \begin{document})
\author{The ATLAS Collaboration}

% ATLAS reference code, to help ATLAS members to locate the paper
\AtlasRefCode{TOPQ-2019-17}
%\AtlasRefCode{ATLAS-CONF-2018-049}

% ATLAS date - arXiv submission; usually filled in by the Physics Office
% \AtlasDate{\today}

% ATLAS heading - heading at top of title page. Set for TDR etc.
%\AtlasHeading{ATLAS ABC TDR}

% Submission journal and final reference

% \AtlasJournal{Phys.\ Lett.\ B.}
%\AtlasJournalRef{JHEP 05 (2019) 123}
%\AtlasDOI{10.1007/JHEP05(2019)123}

% CERN preprint number ---> can be found in CDS entry for Draft 2!
%\PreprintIdNumber{CERN-EP-2018-295}

 %-------------------------------------------------------------------------------
% The following information is needed for the cover page. The commands are only defined
% if you use the coverpage option in atlasdoc or use the atlascover package
%-------------------------------------------------------------------------------

% List of supporting notes  (leave as null \AtlasCoverSupportingNote{} if you want to skip this option)
%\AtlasCoverSupportingNote{Search for $\ttbar \to WbHq$, $H \to b\bar{b}$}{https://cds.cern.ch/record/2257631}
%\AtlasCoverSupportingNote{Search for $\ttbar \to WbHq$, $H \to \tau^+\tau^-$}{https://cds.cern.ch/record/2273683}
%\AtlasCoverSupportingNote{Combination of $\ttbar \to WbHq$ searches}{https://cds.cern.ch/record/2312520/}
\AtlasCoverSupportingNote{Search for flavor-changing neutral currents tHq interactions with, $H \to \tau^+\tau^-$}{https://cds.cern.ch/record/2687866}
%
% OR (the 2nd option is deprecated, especially for CONF and PUB notes)
%
% Supporting material TWiki page  (leave as null \AtlasCoverTwikiURL{} if you want to skip this option)
% \AtlasCoverTwikiURL{https://twiki.cern.ch/twiki/bin/view/Atlas/WebHome}

% Comment deadline
\AtlasCoverCommentsDeadline{1 March 2022}

% Analysis team members - contact editors should no longer be specified
% as there is a generic email list name for the editors
\AtlasCoverAnalysisTeam{
Boyang Li, Mingming Xia, Wei-Ming Yao, Xin Chen}
%Combination: Peter Onyisi, Harish Potti}

% Editorial Board Members - indicate the Chair by a (chair) after his/her name
% Give either all members at once (then they appear on one line), or separately
%\AtlasCoverEdBoardMember{Not assigned yet!}

 \AtlasCoverEdBoardMember{Luca Fiorini~(chair)}
 \AtlasCoverEdBoardMember{Edson Carquin Lopez}
 \AtlasCoverEdBoardMember{Michele Weber}

% Editors egroup
\AtlasCoverEgroupEditors{atlas-TOPQ-2019-17-editors@cern.ch}

% EdBoard egroup
\AtlasCoverEgroupEdBoard{atlas-TOPQ-2019-17-editorial-board@cern.ch}

\AtlasJournal{JHEP}








 

% Author and title for the PDF file
\hypersetup{pdftitle={ATLAS document},pdfauthor={The ATLAS Collaboration}}

%-------------------------------------------------------------------------------
% Content
%-------------------------------------------------------------------------------
%DIF PREAMBLE EXTENSION ADDED BY LATEXDIFF
%DIF UNDERLINE PREAMBLE %DIF PREAMBLE
\RequirePackage[normalem]{ulem} %DIF PREAMBLE
\RequirePackage{color}\definecolor{RED}{rgb}{1,0,0}\definecolor{BLUE}{rgb}{0,0,1} %DIF PREAMBLE
\providecommand{\DIFadd}[1]{{\protect\color{blue}\uwave{#1}}} %DIF PREAMBLE
\providecommand{\DIFdel}[1]{{\protect\color{red}\sout{#1}}}                      %DIF PREAMBLE
%DIF SAFE PREAMBLE %DIF PREAMBLE
\providecommand{\DIFaddbegin}{} %DIF PREAMBLE
\providecommand{\DIFaddend}{} %DIF PREAMBLE
\providecommand{\DIFdelbegin}{} %DIF PREAMBLE
\providecommand{\DIFdelend}{} %DIF PREAMBLE
\providecommand{\DIFmodbegin}{} %DIF PREAMBLE
\providecommand{\DIFmodend}{} %DIF PREAMBLE
%DIF FLOATSAFE PREAMBLE %DIF PREAMBLE
\providecommand{\DIFaddFL}[1]{\DIFadd{#1}} %DIF PREAMBLE
\providecommand{\DIFdelFL}[1]{\DIFdel{#1}} %DIF PREAMBLE
\providecommand{\DIFaddbeginFL}{} %DIF PREAMBLE
\providecommand{\DIFaddendFL}{} %DIF PREAMBLE
\providecommand{\DIFdelbeginFL}{} %DIF PREAMBLE
\providecommand{\DIFdelendFL}{} %DIF PREAMBLE
%DIF LISTINGS PREAMBLE %DIF PREAMBLE
\RequirePackage{listings} %DIF PREAMBLE
\RequirePackage{color} %DIF PREAMBLE
\lstdefinelanguage{DIFcode}{ %DIF PREAMBLE
%DIF DIFCODE_UNDERLINE %DIF PREAMBLE
  moredelim=[il][\color{red}\sout]{\%DIF\ <\ }, %DIF PREAMBLE
  moredelim=[il][\color{blue}\uwave]{\%DIF\ >\ } %DIF PREAMBLE
} %DIF PREAMBLE
\lstdefinestyle{DIFverbatimstyle}{ %DIF PREAMBLE
	language=DIFcode, %DIF PREAMBLE
	basicstyle=\ttfamily, %DIF PREAMBLE
	columns=fullflexible, %DIF PREAMBLE
	keepspaces=true %DIF PREAMBLE
} %DIF PREAMBLE
\lstnewenvironment{DIFverbatim}{\lstset{style=DIFverbatimstyle}}{} %DIF PREAMBLE
\lstnewenvironment{DIFverbatim*}{\lstset{style=DIFverbatimstyle,showspaces=true}}{} %DIF PREAMBLE
%DIF END PREAMBLE EXTENSION ADDED BY LATEXDIFF

\begin{document}

\maketitle

\tableofcontents

%Introduction
%-------------------------------------------------------------------------------
\section{Introduction}
\label{sec:intro}
%-------------------------------------------------------------------------------
After the discovery of the Higgs boson in 2012 at the Large Hadron Collider (LHC) by the ATLAS~\cite{Aad:2012tfa} and 
CMS~\cite{Chatrchyan:2012ufa} collaborations, 
%Since the observation of Higgs boson by the ATLAS and CMS experiments~\cite{Aad:2012tfa,Chatrchyan:2012ufa} at
a comprehensive programme of measurements % of its properties
has been conducted to explore this new particle. Measurements so far proved to be consistent with the Standard Model (SM) predictions. 
However, the program is ongoing and precision measurements as well as searches for rare new physics processes beyond the Standar Model (BSM)
are underway. One such possibility is flavour-changing neutral-current (FCNC) interactions between the Higgs boson, 
the top quark, and an up-quark, $tqH$ ($q=u, c$), which has been actively searched for by the ATLAS and CMS collaborations.  
%with increased precision and proved to be consistent with the Standard Model (SM) predictions. 
%for nearly a decade, by which no significant sign of deviation from Standard Model (SM) was observed until today.
%However the LHC still provides an excellent opportunity to search for rare new physics processes beyond the Standard Model (BSM).  
%there are still chances to find new physics by improving the sensitivity of previous searches.
%The flavour-changing neutral-current (FCNC) interactions 
%between the Higgs boson, the top quark, and a $u$- or $c$-quark, $tqH$ ($q=u,c$) has been actively searched for by the ATLAS and CMS collaborations. %is still an active topic.
Since the Higgs boson is lighter than the top quark~\cite{Aad:2015zhl},
%with a measured mass $m_H=125.09 \pm 0.24~\gev$~\cite{Aad:2015zhl}, 
such interactions could manifest themselves as FCNC top-quark decays~\DIFdelbegin \DIFdel{\mbox{%DIFAUXCMD
\cite{Agashe:2013hma}}\hspace{0pt}%DIFAUXCMD
, }\DIFdelend \DIFaddbegin \DIFadd{(}\DIFaddend $t\to qH$\DIFaddbegin \DIFadd{)~\mbox{%DIFAUXCMD
\cite{Agashe:2013hma}}\hspace{0pt}%DIFAUXCMD
}\DIFaddend .  
In the Standard Model (SM), the FCNC interaction is forbidden at tree level and suppressed at higher orders due to the Glashow-Iliopoulos-Maiani (GIM) mechanism~\cite{Glashow:1970gm}. The $t\to qH$ branching fraction in the SM is calculated to be exceedingly small, $\BR(t\to qH)\approx10^{-15}$~\cite{Eilam:1990zc,Mele:1998ag,AguilarSaavedra:2004wm,Zhang:2013xya}. 
%such decays are suppressed relative to the dominant $t\to Wb$ decay mode, since $tqH$ 
%interactions are forbidden at the tree level and suppressed even at higher orders in the perturbative expansion due to the 
%Glashow--Iliopoulos--Maiani (GIM) mechanism~\cite{Glashow:1970gm}.
%As a result, the SM predictions for the $t \to Hq$ branching 
%ratios ($\BR$) are exceedingly small, $\BR(t\to Hu) \sim 10^{-17} $ and $\BR(t\to Hc) \sim 10^{-15}$~\cite{Eilam:1990zc,Mele:1998ag,AguilarSaavedra:2004wm,Zhang:2013xya}, making them undetectable in the foreseeable future.
In contrast, large enhancements of these branching ratios are possible in some scenarios beyond the SM.
Examples include quark-singlet models~\cite{AguilarSaavedra:2002kr}, two-Higgs-doublet models (2HDM)~\cite{Branco:2hdm2012} with or without flavour violation,
%of type I, with explicit flavour conservation,
%and of type II, such as
the minimal supersymmetric SM (MSSM)~\cite{Bejar:2000ub, Guasch:1999jp,Cao:2007dk,Cao:2014udj},
supersymmetric models with R-parity violation~\cite{Eilam:2001dh}, composite Higgs models with partial compositeness~\cite{Azatov:2014lha}, 
or warped extra dimensions models with SM fermions in the bulk~\cite{Azatov:2009na}. 
In these scenarios, branching ratios can be as high as $\BR(t\to qH) \sim 10^{-5}$. 
An even larger branching ratio of  $\BR(t\to cH) \sim 10^{-3}$ can be reached in 2HDM without explicit flavour conservation (type III),
since a tree-level FCNC coupling is no longer forbidden by any symmetry~\cite{Cheng:1987rs,Baum:2008qm,Chen:2013qta,Chiang:2015cba,Crivellin:2015hha,Botella:2015hoa, Gori:2017tvg,Chiang:2017fjr}. 
%While other FCNC top couplings ($tq\gamma$, $tqZ$, $tqg$) are also enhanced in these scenarios beyond the SM, 
%the largest enhancements are typically found for the $tqH$ couplings, and in particular the $tcH$ coupling~\cite{Agashe:2013hma}.
Therefore, an observation of an enhanced rate of this decay would be a clear evidence for new physics.
%The $tqH$ interaction could also potentially open up more
%Higgs decay channels, such as $H\rightarrow t^*q\rightarrow Wbq$, but they are likely suppressed due to the $t-H$ mass difference,
%which could be interesting for future studies.
Furthermore, if the $tqH$ interaction exists, the associated single-top and  Higgs  production process through this interaction would enhance 
%becomes non-zero, enhancing 
the total
production cross section of $pp\rightarrow tH$.
The study of this process will therefore also contribute to the FCNC interaction searches~\cite{Greljo:2014dka}.
The $tH$ associated production in the SM is expected to have a cross section of $92^{+7}_{-12}$ fb at LHC 13 TeV centre of mass energy~\cite{deFlorian:2016spz}.

Searches for $t \to qH$ decays have been performed by the ATLAS and CMS collaborations, taking advantage of the large samples
of top-quark pair ($\ttbar$) events collected in proton-proton ($pp$) collisions at centre-of-mass energies of $\sqrt{s}=7~\tev$ and $8~\tev$~\cite{Aad:2014dya,Aad:2015pja,Khachatryan:2016atv} during Run~1 of the LHC, as well as at $\sqrt{s}=13~\tev$~\cite{fcnc36} using early Run~2 data.
In these searches, one of the top quarks is required to decay into $Wb$, while the other top quark decays into $qH$ with a small branching ratio 
 $\BR(t\to qH)$, denoted as $\ttbar \to WbHq$.\footnote{In the following, $WbHq$ is used to denote both $W^+b H\bar{q}$ and its charge conjugate, $HqW^- \bar{b}$. Similarly, 
$WbWb$ is used to denote $W^+b W^- \bar{b}$.}  The Higgs boson is assumed to have a mass of $m_H=125~\gev$ and to decay as predicted by
the SM.
%The assumption of using SM-like Higgs boson branching ratios is motivated by the fact that measurements of the flavour-diagonal Higgs boson couplings by the ATLAS and CMS collaborations are in agreement with the SM prediction within about 10\%~\cite{Khachatryan:2016vau,Sirunyan:2018koj}. 
%Furthermore, typical beyond-the-SM scenarios that predict significant enhancements to $\BR(t\to Hq)$, also predict modifications to the Higgs boson branching ratios at the few percent level or below, well beyond the current experimental precision.
%Some of the most sensitive single-channel searches have been performed in the $H\to\gamma\gamma$ decay mode, which
%has a small branching ratio of $\BR(H\to \gamma\gamma)\simeq 0.2\%$, but benefits from having a very small background contamination 
%and excellent diphoton mass re\-so\-lu\-tion. 
%Searches targeting signatures with two same-charge leptons or three leptons (electrons or muons), generically referred to as multileptons,
%are able to exploit a branching ratio that is significantly larger for the $H \rightarrow WW^*, \tau\tau$ decay modes than for the $H \rightarrow \gamma\gamma$ decay mode,
%are able to exploit a significantly larger branching ratio for the Higgs boson decay into $H \to WW^*, \tau\tau$ compared to the $H\to\gamma\gamma$ decay mode, 
%and are also characterised by relatively small backgrounds.
%However, in general they do not have good mass resolution,\footnote{An exception is the $H\to ZZ^*\to \ell^+\ell^- \ell^{\prime +}\ell^{\prime -}$ 
%($\ell, \ell^\prime = e, \mu$) decay mode, which has a very small branching ratio and thus is not promising for this search.} 
%so any excess would be hard to interpret as originating from $t \to Hq$ decays.
%Finally, searches have also been performed exploiting the dominant Higgs boson decay mode, $H\to b\bar{b}$, which has a branching ratio 
%of $\BR(H\to b\bar{b})\simeq 58\%$.
Compared to Run~1, the Run~2 searches, summarised in Table~\ref{tab:limits_summary_ref}, benefit from the increased $\ttbar$ cross section at $\sqrt{s}=13~\tev$, as well as the larger integrated luminosity.
Using 36.1~fb$^{-1}$ of data at $\sqrt{s}=13~\tev$, the ATLAS Collaboration has derived upper limits at 95\% confidence level (CL) of branching ratio
$\BR(t\to cH)<0.22\%$ using $H\to \gamma\gamma$ decays~\cite{Aaboud:2017mfd}, $\BR(t\to cH)<0.16\%$ based on
%signatures with two same-charge light leptons (electrons or muons) or three light leptons 
multilepton (electron or muon) signatures resulting from 
$H \to  WW^*, ZZ^*$, $H\to \tau^+\tau^-$ in which both $\tau$-leptons decay leptonically~\cite{Aaboud:2018pob},
$\BR(t\to cH)<0.42\%$ using $H\to b\bar{b}$ decay~\cite{fcnc36}, and of $\BR(t\to cH)<0.19\%$ using $H\to \tau^+\tau^-$ decay in which at least
one of $\tau$-leptons decays hadronically~\cite{fcnc36}.  
These upper limits are derived assuming that the branching ratio $\BR(t\to uH)$ is null. Similar upper limits are obtained for $\BR(t\to uH)$ assuming $\BR(t\to cH)=0$.
%The searches with  $H\to \tau^+\tau^-$ and $H\to b\bar{b}$ decays are later published~\cite{fcnc36}, unfortunately with no FCNC signal found.
Combining all of the ATLAS searches using 36.1~fb$^{-1}$ Run~2 data, upper limits at 95\% CL on the branching fractions are 
set at $\BR(t\to cH)<0.11\%$ assuming $\BR(t\to uH)=0$, and $\BR(t\to uH)<0.12\%$ assuming $\BR(t\to cH)=0$~\cite{fcnc36}.

The CMS Collaboration performed a similar search using  
$H\to b\bar{b}$ decays~\cite{Sirunyan:2017uae} with 35.9 fb$^{-1}$ of data at $\sqrt{s}=13~\tev$, resulting 
in upper limits of $\BR(t\to cH)<0.47\%$ and $\BR(t\to uH)<0.47\%$, in each case neglecting the other decay mode.
Compared with previous searches, the search in Ref.~\cite{Sirunyan:2017uae} considers in addition the contribution to the signal from 
$pp \to tH$ production~\cite{Greljo:2014dka}. 
%CMS recently updated their search for $tqH$ in the $H\to b\bar{b}$ channel using
%their full Run 2 data and  set observed(expected) upper limits on $\mathcal{B}(t\to Hu)$ of $7.9\times10^{-4}(1.1\times10^{-3})$ and $\mathcal{B}(t\to Hc)$ of $9.4\times10^{-4}(8.6\times10^{-4})$~\cite{CMS:2021gfa}.
%CMS also recently reported a search for $tqH$ in the $H\rightarrow \gamma\gamma$ mode using the full Run-2 data and set observed(expected) upper limits on $\mathcal{B}(t\to Hu)$ of $1.9\times10^{-4}(3.1\times10^{-4})$ and $\mathcal{B}(t\to Hc)$ of $7.3\times10^{-4}(5.1\times10^{-4})$~\cite{CMS-PAS-TOP-20-007}.


\begin{table}[t!]
\caption{\DIFdelbeginFL %DIFDELCMD < \small{Summary of 95\% CL upper limits on $\BR(t \to Hc)$ and $\BR(t \to Hu)$.}%%%
\DIFdelendFL \DIFaddbeginFL \small{Summary of 95\% CL upper limits on $\BR(t \to Hc)$ and $\BR(t \to Hu)$ obtained from ATLAS and CMS Collaborations with Run 2 data.}\DIFaddendFL }
\begin{center}
\DIFdelbeginFL %DIFDELCMD < \tiny
%DIFDELCMD < %%%
\DIFdelendFL \DIFaddbeginFL \small 
\DIFaddendFL \begin{tabular}{ccccc}
\toprule\toprule
& &\multirow{2}{*}{$\mathcal{L}$ [fb$^{-1}$]} & \DIFdelbeginFL %DIFDELCMD < \multicolumn{1}{c}{95\% CL observed upper limits} & \multicolumn{1}{c}{95\% CL observed upper limits}  %%%
\DIFdelendFL \DIFaddbeginFL \multicolumn{2}{c}{95\% CL observed upper limits}  \DIFaddendFL \\
& & 										    & \multicolumn{1}{c}{on $\BR(t \to Hc)$}            & \multicolumn{1}{c}{on $\BR(t \to Hu)$} \\
\midrule
\multirow{5}{*}{ATLAS}
& $H \to b\bar{b}$~\cite{fcnc36}                                          & 36.1         & $4.2 \times 10^{-3}$ & $5.2 \times 10^{-3}$ \\
& $H \to \gamma\gamma$~\cite{Aaboud:2017mfd}                              & 36.1         & $2.2 \times 10^{-3}$  & $2.4 \times 10^{-3}$  \\
& $H \to \tau\tau$ ($\lephad$, $\hadhad$)~\cite{fcnc36}                   & 36.1         & $1.9 \times 10^{-3}$  & $1.7 \times 10^{-3}$  \\ 
& $H \to WW^*, \tau\tau, ZZ^*$ ($2\ell$SS, $3\ell$)~\cite{Aaboud:2018pob} & 36.1         & $1.6 \times 10^{-3}$  & $1.9 \times 10^{-3}$\\ 
& Combination~\cite{fcnc36}                                               & 36.1         & $1.1 \times 10^{-3}$  & $1.2 \times 10^{-3}$  \\\midrule
\DIFdelbeginFL %DIFDELCMD < \multirow{3}{*}{CMS} 
%DIFDELCMD < %%%
\DIFdelendFL \DIFaddbeginFL \multirow{1}{*}{CMS} 
\DIFaddendFL & $H \to b\bar{b}$~\cite{Sirunyan:2017uae}                                & 35.9         & $4.7 \times 10^{-3}$  & $4.7 \times 10^{-3}$  \\
%& $H \to b\bar{b}$~\cite{CMS:2021gfa}                                     & 137          & $9.4 \times 10^{-4}$  & $7.9 \times 10^{-4}$  \\
%& $H \to \gamma\gamma$~\cite{CMS-PAS-TOP-20-007}                          & 137          & $7.3 \times 10^{-4}$  & $1.9 \times 10^{-4}$  \\
% 
\DIFdelbeginFL %DIFDELCMD < 

%DIFDELCMD < \midrule
%DIFDELCMD < %%%
\DIFdelendFL \bottomrule\bottomrule
\end{tabular}
\label{tab:limits_summary_ref}
\end{center}
\end{table}




% \begin{center}
% \begin{tabular}{lccc}
% \hline \hline
% Analysis                           & Dataset & $\mathcal{L}$ [fb$^{-1}$] & Ref. \\
% \hline
% \Hyy\ (including \ttH, \Hyy)       & \multirow{4}{*}{2015--2017} & $79.8$ & (\cite{HIGG-2016-21}), \cite{HIGG-2018-13} \\
% \hfourl\ (including \ttH, \hfourl) &                             & $79.8$ & (\cite{HIGG-2016-22}), \cite{HIGG-2018-13} \\
% \VH, \hbb                          &                             & $79.8$ & \cite{HIGG-2018-04,HIGG-2018-50} \\
% \hmm                               &                             & $79.8$ & (\cite{HIGG-2016-10}) \\
% \hline
% \hwwenmun                          & \multirow{6}{*}{2015--2016} & $36.1$ & \cite{HIGG-2016-07} \\
% \htt                               &                             & $36.1$ & \cite{HIGG-2017-07} \\
% \VBF, \hbb                         &                             & $24.5$ -- $30.6$ & \cite{HIGG-2016-30} \\
% \ttH, \hbb\ and \ttH\ multilepton  &                             & $36.1$ & \cite{HIGG-2017-03,HIGG-2017-02,HIGG-2018-13}\\
% \Hinv                              &                             & $36.1$ & \cite{EXOT-2016-37,HIGG-2016-28,EXOT-2016-23,HIGG-2018-54}\\
% Off-shell \Hllll\ and \Hllvv\      &                             & $36.1$ & \cite{HIGG-2017-06} \\
% \hline \hline
% \end{tabular}
% \end{center}
% \label{tab:lumi}
% \end{table}




%Upper limits on the branching ratios $\BR(t\to Hq)$ ($q=u,c$) can be translated to upper limits on the non-flavour-diagonal Yukawa couplings $\lamHq$ 
%appearing in the following Lagrangian~\cite{Harnik:2012pb}:
%\begin{equation}
%{\cal L}_{\rm FCNC} = -\lambda_{t_L q_R} \bar{t}_L q_R H - \lambda_{q_L t_R} \bar{q}_L t_R H  + h.c.
%\end{equation}
%The branching ratio $\BR(t\to Hq)$ is estimated as the ratio of its partial width~\cite{Zhang:2013xya} to the SM $t \to Wb$ partial width~\cite{Denner:1990ns}, 
%which is assumed to be dominant. Both predicted partial widths include next-to-leading-order (NLO) QCD corrections.
%Using the expression derived in Sect. 1.2 of Ref.~\cite{ATL-COM-PHYS-2016-1664}, the coupling $|\lamHq|$ can be extracted as $| \lamHq | = (1.92 \pm 0.02) \sqrt{\BR(t\to Hq)}$.
%The $\lamHq$ coupling corresponds to the sum in quadrature of the couplings relative to the two possible chirality combinations of the quark fields, 
%$\lamHq \equiv \sqrt{ |\lambda_{t_Lq_R}|^2 +   |\lambda_{q_L t_R}|^2 }$~\cite{Harnik:2012pb}.

%The searches presented in this paper are focused on the dominant fermionic decay modes of the Higgs boson.

%This paper is focused on the tau decay mode of the Higgs boson with the complete dataset collected during the Run~2 from 2016 to 2018.
The analysis reported here targets Higgs boson decays to 
%This paper is focused on the decay mode of the Higgs boson into
$\tau$-leptons with the complete Run~2 dataset collected in 2015-2018. The corresponding integrated luminosity is \DIFdelbegin \DIFdel{139.0 }\DIFdelend \DIFaddbegin \DIFadd{139 }\DIFaddend fb$^{-1}$. Both $\ttbar \to WbHq$ decay and $pp \to tH$ production are targeted. The dataset is divided into several final states depending on the production mode and
the  $W$ decay. Top pair production events where the \DIFdelbegin \DIFdel{W }\DIFdelend \DIFaddbegin \DIFadd{$W$ }\DIFaddend boson decays hadronically (leptonicall) are referred to as $t_h$ (\DIFdelbegin \DIFdel{$t_l$}\DIFdelend \DIFaddbegin \DIFadd{$t_{\ell}$}\DIFaddend ). The contribution of 
$W\rightarrow\tau\nu$ is included as \DIFdelbegin \DIFdel{$\tau_{\text{lep}}$ in $t_l$ }\DIFdelend \DIFaddbegin \DIFadd{$\lep$ in $t_{\ell}$ }\DIFaddend when the $\tau$ decays into a light lepton (electron or muon) or as $\thad$ 
in $t_h$ when the $\tau$ decays hadronically.
%either hadronically or leptonically from the $t\rightarrow Wb$ decay, which are referred as $t_h$ and $t_l$. The contribution of $W\rightarrow\tau\nu$ is included either in
%$t_l$ when the $\tau$ decays into a light lepton (electron or muon) as $\tau_{\text{lep}}$ or in $t_h$ when the $\tau$ decays hadronically as $\thad$.
The $H\rightarrow \tau\tau$ decay is detected in
either $\tlhad$ or $\thadhad$ final states when the top decays hadronically ($t_h$) and only in $\thadhad$ when the top decay leptonically (\DIFdelbegin \DIFdel{$t_l$}\DIFdelend \DIFaddbegin \DIFadd{$t_{\ell}$}\DIFaddend ). 
Events with two hadronically decaying $\tau$\DIFdelbegin \DIFdel{-lepton candidate (}%DIFDELCMD < \tauhad%%%
\DIFdel{) }\DIFdelend \DIFaddbegin \DIFadd{-leptons }\DIFaddend without electron or muon are \DIFdelbegin \DIFdel{selected as the }\DIFdelend \DIFaddbegin \DIFadd{named }\DIFaddend hadronic channels. Events with at least one \tauhad with an additional electron or muon are \DIFdelbegin \DIFdel{selected as the }\DIFdelend \DIFaddbegin \DIFadd{named }\DIFaddend leptonic channels. More signal regions are exploited here compared to the previous FCNC $\Htautau$ search conducted using partial Run~2 data\DIFaddbegin \DIFadd{~\mbox{%DIFAUXCMD
\cite{fcnc36}}\hspace{0pt}%DIFAUXCMD
}\DIFaddend .
In addition, an improved treatment of misidentified $\tau$-leptons in simulation and data-driven
fake estimations from \DIFdelbegin \DIFdel{QCD }\DIFdelend multi-jet background are implemented.
% and more
%$$t\bar{t}$ control regions (CRtt) are used for calibration of misidentified (fake) taus in Monte Carlo and data-driven fake estimations from QCD multi-jet background.
Finally, a multivariate technique based on boosted decision trees, is used to discriminate between the signal and the background on the basis of their different kinematics distributions. 

%The expected sensitivity is about a factor of two relative to the most sensitive results~\cite{fcnc36} up until now. 


% ATLAS detector
%-------------------------------------------------------------------------------
\section{ATLAS detector}
\label{sec:detector}
%-------------------------------------------------------------------------------

The ATLAS detector~\cite{PERF-2007-01} at the LHC covers almost the entire solid angle around the collision point,\footnote{ATLAS 
uses a right-handed coordinate system with its origin at the nominal interaction point (IP) in the 
centre of the detector.  
%and the $z$-axis coinciding with the axis of the beam pipe.
The $x$-axis points from
the IP to the centre of the LHC ring, %and
the $y$-axis points upward,
and the $z$-axis coincides with the axis of the beam pipe.
Cylindrical coordinates ($r$,$\phi$) are used 
in the transverse plane, $\phi$ being the azimuthal angle around the beam pipe. The pseudorapidity is defined in 
terms of the polar angle $\theta$ as $\eta = - \ln \tan(\theta/2)$.
Angular distance is measured in units of $\Delta R\equiv \sqrt{(\Delta\eta)^2+(\Delta\phi)^2}$.} and consists of an inner tracking detector surrounded by a thin superconducting solenoid producing a
2~T axial magnetic field, electromagnetic and hadronic calorimeters, and a muon spectrometer incorporating three large toroid magnet assemblies with eight coils each. The inner detector contains a high-granularity silicon pixel detector, including the %newly-installed
insertable B-layer~\cite{IBL1,IBL2,Abbott:2018ikt}, installed in 2014, and a silicon microstrip tracker, together providing a precise reconstruction of tracks of charged particles in the pseudorapidity range $|\eta|<2.5$.
The inner detector also includes a transition radiation tracker that provides tracking and electron identification for $|\eta|<2.0$.
%The electromagnetic (EM) sampling calorimeter uses lead as the absorber material 
%and liquid-argon (LAr) as the active medium, and is divided into barrel ($|\eta|<1.475$) and end-cap ($1.375<|\eta|<3.2$) regions.  
%Hadron calorimetry is also based on the sampling technique, with either scintillator tiles or LAr as the active medium, and with 
%steel, copper, or tungsten as the absorber material. The calorimeters cover $|\eta|<4.9$. 
%Hadronic calorimetry is provided by the steel/scintillating-tile calorimeter,
%segmented into three barrel structures within $|\eta| < 1.7$, and two copper/LAr hadronic endcap calorimeters.
%The solid angle coverage is completed with forward copper/LAr and tungsten/LAr calorimeter modules
%optimised for electromagnetic and hadronic measurements respectively.
The calorimeter system covers the pseudorapidity range $|\eta| < 4.9$. Within the region $|\eta|< 3.2$, electromagnetic (EM) calorimetry is provided by barrel and endcap high-granularity lead/liquid-argon (LAr) sampling calorimeters, with an additional thin LAr presampler covering $|\eta| < 1.8$, to correct for energy loss in material upstream of the calorimeters. Hadronic calorimetry is provided by %the
a steel/scintillator-tile calorimeter, segmented into three barrel structures within $|\eta| < 1.7$, and two copper/LAr hadronic endcap calorimeters.
The solid angle coverage is completed with forward copper/LAr and tungsten/LAr calorimeter modules optimised for electromagnetic and hadronic measurements, respectively.
The calorimeters are surrounded by a muon spectrometer within a magnetic field provided by air-core toroid magnets with a bending integral of about \SI{2.5}{\tesla\metre} in the barrel and up to \SI{6.0}{\tesla\metre} in the endcaps. 
The muon spectrometer measures the trajectories of muons with $|\eta|<2.7$ using multiple layers of high-precision tracking chambers, and is instrumented with separate trigger chambers covering $|\eta|<2.4$. A two-level trigger system~\cite{Aaboud:2016leb}, consisting of a hardware-based level-1 trigger followed by a software-based high-level trigger, is used to reduce the event rate to a maximum of around \SI{1}{\kHz} for offline storage.
An extensive software suite~\cite{ATL-SOFT-PUB-2021-001} is used in the reconstruction and analysis of real and simulated data, in detector operations, and in the trigger and data acquisition systems of the experiment.


% Object reconstruction
%-------------------------------------------------------------------------------
\section{Event reconstruction}
\label{sec:objects}
%-------------------------------------------------------------------------------

Events are selected from $pp$ collisions at $\sqrt{s}$=13 TeV recorded by the ATLAS detector during 2015-2018.
Only events for which all relevant subsystems were operational are considered.
Events are required to have at least one vertex with two or more tracks with transverse momentum
%The event reconstruction is affected by multiple $pp$ collisions in a single bunch crossing and by collisions
%in neighbouring bunch crossings, referred to as pile-up. 
%Interaction vertices from the $pp$ collisions are reconstructed from at least two tracks with transverse momentum
($\pt$) larger than $400~\mev$ that are consistent with originating from the 
beam collision region in the $x$--$y$ plane. If more than one primary vertex candidate is found, the
candidate whose associated tracks form the largest sum of squared $\pt$~\cite{ATL-PHYS-PUB-2015-026}
is selected as the hard-scatter primary vertex.

Electron candidates~\cite{ATLAS-CONF-2016-024,Aaboud:2018ugz} are reconstructed from energy 
clusters in the EM calorimeter that are matched to reconstructed tracks in the inner detector;
electron candidates in the transition region between the EM barrel and endcap calorimeters 
($1.37 < |\eta_{\textrm{cluster}}| < 1.52$) are excluded.
Electron candidates are required to have $|\eta_{\textrm{cluster}}| < 2.47$, and to satisfy tight likelihood-based identification 
criteria~\cite{ATLAS-CONF-2016-024} based on calorimeter, tracking and combined variables that provide 
separation between electrons and jets. 

Muon candidates~\cite{Aad:2016jkr} are reconstructed by matching track segments in %the various
different layers of the muon spectrometer to tracks found in the inner detector;
the resulting muon candidates are re-fitted using the complete track information from both detector systems.
Muon candidates are required to satisfy medium identification criteria~\cite{Aad:2016jkr}. 

Electron (muon) candidates are matched to the primary vertex by requiring that the significance of their transverse impact parameter, $d_0$, 
satisfies $|d_0/\sigma(d_0)|<5\,(3)$, where $\sigma(d_0)$ is the measured uncertainty in $d_0$,
and by requiring that their longitudinal impact parameter, $z_0$, satisfies $|z_0 \sin\theta|<0.5$~mm.
%To reduce the impact of the trigger efficiency uncertainty around the threshold, the leptons are required to have \pt 1~GeV above the threshold.
To further reduce the background from non-prompt leptons, photon conversions and hadrons, lepton candidates are also required to be isolated 
in the tracker and in the calorimeter.
A track-based lepton isolation criterion is defined by calculating the quantity $I_R = \sum \pt^{\textrm{trk}}$ and $E_R = \sum E^{\textrm{clst}}$, where
the scalar sum includes all tracks and energy deposits (excluding the lepton candidate itself) within the cone defined by $\Delta R<0.2 (0.3)$ around the %axis
direction of the electron (muon). The electron (muon) candidates \DIFaddbegin \DIFadd{($\ell=e$ or $\mu$) }\DIFaddend are required to satisfy both
%DIF > $I_R/\pt^l < 0.2~(0.3)$ and $E_R/\pt^l < 0.2~(0.3)$.
$I_R/\pt^\ell < 0.2~(0.3)$ and $E_R/\pt^\ell < 0.2~(0.3)$.

Tight isolation working points are also applied in some channels to reduce fake and non-prompt lepton contributions based on a trained isolation boosted decision tree \texttt{PromptLeptonVeto}(PLIV), which is to identify non-prompt light leptons using lifetime information associated with a track jet that matches the selected light lepton~\cite{ATLAS-CONF-2019-045}. These additional reconstructed charged particle tracks inside the jet can be
used to increase the precision for identifying the displaced decay vertex of heavy flavor (b, c) hadrons that produced non-prompt leptons.
The tight working points are used for high $\pt$ lepton.
The scale factors for the efficiencies of the tight \texttt{PLIV} working points are measured using the tag and probe method
with \DIFdelbegin \DIFdel{$Z\rightarrow l^+l^-$ ($l=e$ or $\mu$) }\DIFdelend \DIFaddbegin \DIFadd{$Z\rightarrow \ell^+\ell^-$ }\DIFaddend events. These scale factors have also been checked for the electron or muon from the tau decay
using $Z\rightarrow\tau\tau\rightarrow e\mu4\nu$ samples, which is consistent at 2\% level. To be conservative, an
additional uncertainty of $\pm 2\%$ is considered for the PLIV efficiency for the tau-lepton in the lepton+$\thad$ channels.
%Additionally, the $\Htautau$ search requires leptons to
%satisfy a calorimeter-based isolation criterion; a requirement is made on the sum of the transverse energy of
%topological clusters within the cone defined by $\Delta R<0.2$ around the lepton, after subtracting the contribution
%from the energy deposit of the lepton itself and correcting for pile-up effects, to be less than a $\pt$-dependent 
%percentage of the lepton energy. 
%The combined efficiency of track-based and calorimeter-based isolation requirements made by the $\Htautau$ 
%search is 90\% (99\%) at $\pt=25\;(60)~\gev$.

%a Gradient isolation working points are used for both electrons and muons. The working points are based on cut maps derived from the $Z\to l^+l^-$ Monte Carlo samples. The efficiency maps are simply derived from the sum of transverse energy of the clusters deposited in the calorimeter in the given cone. The efficiency for the leptons with a certain $\pt$ is $(0.1143*\pt[\GeV]+92.14)\%*(0.1143*\pt[\GeV]+92.14)\%$.
%draft 1 version
%Candidate jets are reconstructed with the anti-$k_t$ algorithm~\cite{Cacciari:2008gp,Cacciari:2005hq,Cacciari:2011ma} with a
%radius parameter $R=0.4$, using topological clusters~\cite{Aad:2016upy} 
%built from energy deposits in the calorimeters calibrated to the electromagnetic scale. 
Candidate jets are reconstructed using the anti-$k_t$ algorithm~\cite{Cacciari:2008gp,Cacciari:2005hq} with a
radius parameter $R=0.4$ applied to topological energy clusters~\cite{Aad:2016upy} and charged-particle tracks, processed using
a particle-flow algorithm~\cite{Aad:2017epj77}. %newref Eur. Phys. J. C 77 (2017) 466
%as implemented in the \fastjet\ package~\cite{Cacciari:2011ma}.  
%Jet reconstruction in the calorimeter starts from topological clustering~\cite{Aad:2016upy} of individual calorimeter cells calibrated to the electromagnetic energy scale. 
The reconstructed jets are then calibrated to the particle level by the application of a jet energy scale 
derived from simulation and in situ corrections based on $\sqrt{s}=13~\tev$ data~\cite{Aaboud:2017jcu}. %(TBD: check PFlow jets description)
The calibrated jets used are required to have $\pt > 25~\gev$ and $|\eta| < 2.5$.
Jet four-momenta are corrected for pile-up effects using the jet-area method~\cite{Cacciari:2008gn}.

Quality criteria are imposed to reject events that contain any jets arising from non-collision sources
or detector noise~\cite{ATLAS-CONF-2015-029}.  To reduce the contamination due to jets originating from pile-up interactions,
additional requirements are imposed on the jet vertex tagger (JVT)~\cite{Aad:2015ina} output for jets with $\pt<60~\gev$ and $|\eta| < 2.4$.

Jets containing $b$-hadrons are identified ($b$-tagged) via the DL1r tagger~\cite{Aad:2019epj79,ATL-PHYS-PUB-2017-013} %newref
%\cite{Aad:2015ydr,ATL-PHYS-PUB-2016-012} 
that uses multivariate techniques to combine information about the impact parameters of displaced tracks and the  topological properties 
of secondary and tertiary decay vertices reconstructed within the jet. For each jet, a value for the multivariate $b$-tagging discriminant is 
calculated. A jet is considered $b$-tagged if this value is above the threshold corresponding to
an average 70\% efficiency to tag a $b$-quark jet, with a light-jet\footnote{Light-jet refers to a jet originating from the hadronisation of a light quark ($u$, $d$, $s$) or a gluon.} rejection factor of about 385 and a charm-jet rejection factor of about 12, as determined for jets with
$\pt >20~\gev$ and $|\eta|<2.5$ in simulated $\ttbar$ events~\cite{Aad:2019epj79}.

Hadronically decaying $\tau$-lepton ($\had$) candidates are reconstructed from energy clusters in the calorimeters and
associated inner-detector tracks~\cite{ATL-PHYS-PUB-2019-033}. %newref
%~\cite{ATL-PHYS-PUB-2015-045}.
Candidates are required to have either one or three associated tracks,
with a total charge of $\pm 1$. Candidates are required to have \pt 5 GeV higher than the trigger threshold, and $|\eta|<2.5$,
excluding the EM calorimeter's transition region.
A Recurrent Neural Network (RNN)~\cite{Graves:2012SCI}
%~\cite{Breiman:1984jka,Friedman:2002we,Freund:1997xna}
%A. Graves, Supervised Sequence Labelling with Recurrent Neural Networks, Studies in Computational Intelligence 385, Springer, 2012.
using calorimeter- and tracking-based variables is used to identify $\had$ candidates and reject jet backgrounds.
Three working points labelled loose, medium and tight are defined, and correspond to different $\had$ identification efficiency values, with the efficiency designed to be independent of $\pt$. The $\Htautau$ search uses the medium
working point for the nominal selection.
%, while the loose working point is used for background estimation.
The medium working point has a combined reconstruction and identification efficiency of 75\% (60\%) for one-prong (three-prong) $\had$ 
decays, and an expected rejection factor against light-jets of 35 (240)~\cite{ATL-PHYS-PUB-2019-033}. 
Electrons that are reconstructed as one-prong $\had$ candidates are removed via a BDT trained to reject electrons.
Events with a $b$-tagged $\had$ candidate are rejected.
%(TBD: check RNN description and update then numbers)
%Overlaps between candidate objects are removed sequentially, following this order: muons, electrons,
%$\tauhad$ (only for the $\Htautau$ search), and jets. In the $\Hbb$ search, firstly, electron candidates that lie 
%within $\Delta R = 0.01$ of a muon candidate are removed to suppress contributions from muon bremsstrahlung. 
%Overlaps between electron and jet candidates are resolved next, and finally, overlaps between remaining jet candidates 
%and muon candidates are removed. Clusters from identified electrons are not excluded during jet reconstruction. 
%In order to avoid double-counting of electrons as jets, the closest jet whose axis is within ${\Delta}R = 0.2$ of an electron 
%is discarded. If the electron is within ${\Delta}R = 0.4$ of the axis of any jet after this initial removal, the jet is retained and  the electron is removed.
%The overlap removal procedure between the remaining jet candidates and muon candidates is designed to remove those muons 
%that are likely to have arisen in the decay chain of hadrons and to retain the overlapping jet instead. 
%Jets and muons may also appear in close proximity when the jet results from high-$\pt$ muon bremsstrahlung, 
%and in such cases the jet should be removed and the muon retained. Such jets are characterised by having very 
%few matching inner-detector tracks. Selected muons that satisfy $\Delta R(\mu,{\textrm{jet}}) < 0.04+10~\gev/\pt^\mu$ are rejected
%if the jet has at least three tracks originating from the primary vertex; otherwise the jet is removed and the muon is kept.
%In the $\Htautau$ search, a fixed cone size of $\Delta R=0.2$ is used to determine the overlap between
%candidate objects, and only the highest-$\pt$ (leading) or the two leading $\tauhad$ candidates (depending on the
%analysis channel, see Section~\ref{sec:data_presel}) are considered to resolve their overlap with jets.

To avoid double counting reconstructed objects, an overlap removal procedure is applied.
%Overlaps between reconstructed objects are removed sequentially.
%Firstly,
Electron candidates that lie 
within $\Delta R = 0.01$ of a muon candidate are removed to suppress contributions from muon bremsstrahlung. 
%Overlaps between electron and jet candidates are resolved next, and finally, overlaps between remaining jet candidates 
%and muon candidates are removed.
Energy clusters from identified electrons are not excluded during jet reconstruction. 
In order to avoid double-counting of electrons as jets, the closest jet whose axis is within ${\Delta}R = 0.2$ of an electron 
is discarded if the jet is not $b$-tagged, otherwise the electron is removed.
If the electron is within ${\Delta}R = 0.4$ of the axis of any jet after this initial removal, the jet is retained and  the electron is removed.
The overlap removal procedure between the remaining jet candidates and muon candidates is designed to remove those muons 
that are likely to have arisen in the decay of hadrons and to retain the overlapping jet instead. 
Jets and muons may also appear in close proximity when the jet results from high-$\pt$ muon bremsstrahlung, 
and in such cases the jet should be removed and the muon retained. Such jets are characterised by having very 
few matching inner-detector tracks. Selected muons that satisfy $\Delta R(\mu,{\textrm{jet}}) < 0.2$ are rejected
if the jet is either $b$-tagged or has at least three tracks originating from the primary vertex; otherwise the jet is removed and the muon is kept.
The $\tauhad$ within a $\Delta R=0.2$ cone of an electron or muon are removed.
In order to avoid double-counting of $\tauhad$ as jets, the closest jet whose axis is
within ${\Delta}R = 0.2$ of a $\tauhad$ is discarded if the jet is not $b$-tagged, otherwise the $\tauhad$ is removed. 
% TEXT FROM BOYANG. NEED TO RECHECK WHAT WE HAVE ABOVE IS ACCURATE
%In the $\Htautau$ search, the objects are removed with the following sequence: if two electrons have overlapping second-layer cluster, or shared tracks, the electron with lower $\pt$ is removed; $\tauhad$ within a $\Delta R=0.2$ cone of an electron or muon are removed; if a muon sharing an ID track with an electron and the muon is calo-tagged, the muon is removed, otherwise the electron is removed; jets within a $\Delta R=0.2$ cone of an electron are removed; electrons within a $\Delta R=0.4$ cone of a jet are removed; when a muon ID track is ghost associated to a jet or within a $\Delta R=0.2$ cone of a jet, the jet is removed if it has less than 3 tracks with $\pt>500$ MeV or has a relative small $\pt$ ($\pt^{\mu}>0.5\pt^{\text{jet}} \text{ and } \pt^{\mu}>0.7[\text{the scalar sum of the } \pt \text{'s of the jet tracks with } \pt>500$ MeV]); muons within a $\Delta R=0.4$ cone of a jet are removed; jets within a $\Delta R=0.2$ cone of the leading $\tauhad$ ($\lephad$), or with the two leading $\tauhad$'s ($\hadhad$), are removed.

%firstly, the electron within a ${\Delta}R = 0.2$ cone of muons are excluded, then the $\tau_{had}$'s or jets within a ${\Delta}R = 0.2$ cone of an electron or muon are excluded. Finally, the jets within a ${\Delta}R = 0.2$ cone of the leading $\had$ ($\lephad$), or with the two leading $\had$’s ($\hadhad$) are excluded.

The missing transverse momentum $\mpt$ (with magnitude $\met$) is defined as the negative vector sum of the 
$\pt$ of all selected and calibrated objects in the event, including a term to account for momentum from soft particles 
in the event which are not associated with any of the selected objects. 
This soft term is calculated from inner-detector tracks matched to the selected primary vertex to make it more resilient to
contamination from pile-up interactions~\cite{Aaboud:2018tkc}.




% Data sample and event preselection
%-------------------------------------------------------------------------------
\section{Data sample and event preselection}
\label{sec:data_presel}
%-------------------------------------------------------------------------------

The search is based on a dataset of $pp$ collisions at $\sqrt{s}=13~\tev$ with 25 ns bunch spacing collected from 2015 to 2018, corresponding to an integrated luminosity of $139~\ifb$.
Only events recorded with a single-electron trigger, a single-muon trigger, or a \DIFdelbegin \DIFdel{di-$\tau$ }\DIFdelend \DIFaddbegin \DIFadd{di-tau }\DIFaddend trigger under stable beam conditions 
and for which all detector subsystems were operational are considered for analysis. While the events recorded by di-lepton triggers fall into the control regions used for \DIFdelbegin \DIFdel{fake-tau }\DIFdelend \DIFaddbegin \DIFadd{fake tau }\DIFaddend background estimation.
The number of $pp$ interactions per bunch crossing in this dataset ranges from about 8 to 45, with an average of 24.

Single-electron and single-muon triggers with low $\pt$ thresholds and lepton isolation requirements are combined in a logical OR 
with higher-threshold triggers but with a looser identification criterion and without any isolation requirement.
The lowest $\pt$ threshold used for muons is 20 (26)~\gev\ in 2015 (2016-2018), while for electrons the threshold is 24 (26)~\gev.
%For ditau triggers, the $\pt$ threshold of the leading (trailing) $\tauhad$ candidate is 35 (25)~\gev\ with the medium identification required.
For \DIFdelbegin \DIFdel{di-$\tau$ }\DIFdelend \DIFaddbegin \DIFadd{di-tau }\DIFaddend triggers, the $\pt$ threshold of the leading (subleading) \DIFdelbegin \DIFdel{$\had$ }\DIFdelend \DIFaddbegin \DIFadd{tau }\DIFaddend candidate is 35 (25)~\gev.
Events satisfying the trigger selection are required to have at least one primary vertex candidate.
To reduce the impact of the trigger efficiency uncertainty around the threshold, the leptons are required to have \DIFaddbegin \DIFadd{a }\DIFaddend \pt \DIFaddbegin \DIFadd{of }\DIFaddend 1~GeV above the threshold. 

The signal final state including \DIFdelbegin \DIFdel{an }\DIFdelend \DIFaddbegin \DIFadd{a light lepton (}\DIFaddend electron or muon\DIFaddbegin \DIFadd{) }\DIFaddend or $\lep$ are referred to as leptonic channel, otherwise hadronic channel.
The events in the leptonic channel are recorded by single-electron or single-muon trigger, required to have exactly one electron or muon that matches, with $\Delta R < 0.15$, the lepton reconstructed by the trigger. Further requirements are defined targeting different signal decay modes:  
\begin{itemize}
\item $t_h\lephad$: Targeting $t_hH$ and $t_ht(qH)$ \DIFaddbegin \DIFadd{final states }\DIFaddend with $H\to\lephad$ decay. Exactly one $\had$ with opposite-side charge to \DIFdelbegin \DIFdel{$\tau_l$ }\DIFdelend \DIFaddbegin \DIFadd{$\lep$ }\DIFaddend is required, including at least three jets with exactly one $b$-jet.
\item \DIFdelbegin \DIFdel{$t_l\hadhad$: Targeting $t_lH$ and $t_lt(qH)$ }\DIFdelend \DIFaddbegin \DIFadd{$t_{\ell}\hadhad$: Targeting $t_{\ell}H$ and $t_{\ell}t(qH)$ final states }\DIFaddend with $H\to\hadhad$ decay. Exactly \DIFdelbegin \DIFdel{one }\DIFdelend \DIFaddbegin \DIFadd{a light }\DIFaddend lepton and two opposite-sign $\had$ are selected, including at least 1 jet with exactly one $b$-jet.
\item \DIFdelbegin \DIFdel{$t_l\had$}\DIFdelend \DIFaddbegin \DIFadd{$t_{\ell}\had$}\DIFaddend : Also targeting \DIFdelbegin \DIFdel{$t_lH$ and $t_lt(qH)$ }\DIFdelend \DIFaddbegin \DIFadd{$t_{\ell}H$ and $t_{\ell}t(qH)$ final states }\DIFaddend with $H\to\hadhad$ decay when one of the $\had$ fails the tau reconstruction or identification so that there is only
  one $\had$ candidate reconstructed. Exactly one $\had$ with same-sign charge to the \DIFaddbegin \DIFadd{light }\DIFaddend lepton is required, including at least 2 jets with exactly one $b$-jet.
\end{itemize}

The events in the hadronic channel are recorded by di-tau trigger, required to have both leading and subleading $\had$ passed the trigger. Further requirements are defined targeting signal decay modes:
\begin{itemize}
\item $t_h\hadhad$: Targeting $t_hH$ and $t_ht(qH)$ \DIFaddbegin \DIFadd{final states }\DIFaddend with $H\to\hadhad$ decay. Exactly two $\had$ with opposite-sign charge are selected, including at least 3 jet with exactly one $b$-jet.
\end{itemize}

The above requirements apply to the reconstructed objects defined in Section~\ref{sec:objects}.
%, which are in general different between both searches. 
These requirements are referred to as the preselection and are summarised in Table~\ref{tab:preselection}. 
\DIFaddbegin 

\DIFaddend %%%%%%%%%%%%%%%
\begin{table*}[t!]
\caption{\small{Summary of preselection requirements. 
The leading and subleading $\had$ candidates are denoted by $\tau_{\mathrm{had,1}}$ and $\tau_{\mathrm{had,2}}$ respectively.}}
\begin{center}
\begin{tabular}{c|ccc|c}
\toprule\toprule
\multirow{2}{*}{Requirement} &  \multicolumn{3}{c|}{leptonic channel}  & \DIFdelbeginFL \DIFdelFL{hadronic channel }\DIFdelendFL \DIFaddbeginFL \multicolumn{1}{c}{hadronic channel} \DIFaddendFL \\ 
%DIF > \multirow{2}{*}{Requirement} &  \multicolumn{3}{c|}{leptonic channel}  & hadronic channel \\ 
& $t_h\lephad$ & \DIFdelbeginFL \DIFdelFL{$t_l\hadhad$ }\DIFdelendFL \DIFaddbeginFL \DIFaddFL{$t_{\ell}\hadhad$ }\DIFaddendFL &  \DIFdelbeginFL \DIFdelFL{$t_l\had$ }\DIFdelendFL \DIFaddbeginFL \DIFaddFL{$t_{\ell}\had$ }\DIFaddendFL & $t_h\hadhad$\\
\midrule
Trigger & \multicolumn{3}{c|}{single-lepton trigger} & di-$\tau$ trigger  \\
Leptons  & \multicolumn{3}{c|}{=1 isolated $e$ or $\mu$}  & =0 isolated $e$ or $\mu$ \\
$\had$  & $=$1 $\had$ & $\geq$2 $\had$ & $=$1 $\had$ & $\geq$2 $\had$ \\
Electric charge ($Q$) & $Q_\ell \times Q_{\tau_{\mathrm{had,1}}} < 0$ & $Q_{\tau_{\mathrm{had,1}}} \times Q_{\tau_{\mathrm{had,2}}} < 0$ & $Q_\ell \times Q_{\tau_{\mathrm{had,1}}} > 0$ & $Q_{\tau_{\mathrm{had,1}}} \times Q_{\tau_{\mathrm{had,2}}} < 0$ \\
Jets  &  3, $\geq$ 4 jets & $\geq$1 jets & 2, $\geq$3 jets & 3, $\geq$4 jets \\
$b$-tagging & \DIFdelbeginFL %DIFDELCMD < \multicolumn{3}{c|}{=1 $b$-jet} %%%
\DIFdelendFL \DIFaddbeginFL \multicolumn{3}{c|}{=1 $b$-jets} \DIFaddendFL & =1 $b$\DIFdelbeginFL \DIFdelFL{-jet}\DIFdelendFL \DIFaddbeginFL \DIFaddFL{-jets}\DIFaddendFL \\
\bottomrule\bottomrule
\end{tabular}
\label{tab:preselection}
\end{center}
\end{table*}
%%%%%%%%%%%%%%%


%% Signal and background modelling
%%-------------------------------------------------------------------------------
\section{Signal and background modelling}
\label{sec:signal_background_model}
%-------------------------------------------------------------------------------

Signal and most background processes are modelled using Monte Carlo (MC) simulation.
After the event preselection, the main background is $\ttbar$ production, often in association with jets, denoted by $\ttbar$+jets in the following.
Small contributions arise from single-top-quark, $W/Z$+jets, multijet and diboson ($WW,WZ,ZZ$) production, as well as from the associated 
production of a vector boson $V$ ($V=W,Z$) or a Higgs boson and a $\ttbar$ pair ($\ttbar V$ and $\ttbar H$). All backgrounds 
with prompt leptons, i.e.\ those originating from the decay of a $W$ boson, a $Z$ boson, or a $\tau$-lepton,
are estimated using samples of simulated events and initially normalised to their theoretical cross sections.
In the simulation, the top-quark and SM Higgs boson masses are set to $172.5~\gev$ and $125~\gev$, respectively,
and the Higgs boson is allowed to decay into all SM particles with branching ratios calculated using \textsc{Hdecay}~\cite{Djouadi:1997yw}.  
Backgrounds with non-prompt electrons or muons, with photons or jets misidentified as electrons, or with jets misidentified as $\had$ candidates, 
generically referred to as fake leptons, are estimated using data-driven methods. 
The background prediction is further improved during the statistical analysis by performing a likelihood 
fit to data using several signal-depleted analysis regions, as discussed in Sections~\ref{sec:strategy_Htautau}.

%-------------------------------------------------------------------------------
\subsection{Simulated signal and background processes}
\label{sec:simulations}
%-------------------------------------------------------------------------------

Samples of simulated $\ttbar \to WbHq$ events were generated with the next-to-leading-order (NLO) generator\footnote{In the following, 
the order of a generator should be understood as referring to the order in the strong coupling constant at which the matrix-element calculation 
is performed.} {\amcatnlolong}~2.4.3~\cite{Alwall:2014hca}  (referred to in the following as {\amcatnlo}) with the NNPDF3.0 NLO~\cite{Ball:2014uwa} parton distribution function (PDF) set and interfaced to {\pythia} 8.212~\cite{Sjostrand:2007gs} with the NNPDF2.3 LO~\cite{Ball:2012cx} PDF set for the modelling of parton showering, hadronisation, and the underlying event. 
The A14~\cite{ATLASUETune4} set of tuned parameters in {\pythia} controlling the description of multiparton interactions and  
initial- and final-state radiation, referred to as the tune, was used.
The signal sample is normalised to the same total cross section as used for the inclusive $t\bar{t}\to WbWb$ sample (see discussion below) and
assuming an arbitrary branching ratio of $\BR_{\mathrm{ref}}(t\to Hq)=1\%$.
The case of both top quarks decaying into $Hq$ is neglected in the analysis given the existing upper limits on $\BR(t \to Hq)$ (Section~\ref{sec:intro}).

The nominal sample used to model the $\ttbar$ background was generated with the NLO generator {\powheg}~v2 \cite{Frixione:2007nw,Nason:2004rx,Frixione:2007vw,Alioli:2010xd} using the NNPDF3.0 NLO PDF set. The {\powheg} model parameter $h_{\textrm{damp}}$, which controls 
matrix element to parton shower matching and effectively regulates the high-$\pt$ radiation, was set to 1.5 times the top-quark mass. 
The parton showers, hadronisation, and underlying event were modelled by {\pythia}~8.210 with the NNPDF2.3 LO PDF set in combination with the A14 tune.
Alternative $\ttbar$ simulation samples used to derive systematic uncertainties are described in Section~\ref{sec:syst_bkgmodeling}. 
The generated $\ttbar$ samples are normalised to a theoretical cross section of $832^{+46}_{-51}$~pb, 
computed using \textsc{Top++}~v2.0~\cite{Czakon:2011xx} at next-to-next-to-leading order (NNLO), 
including resummation of next-to-next-to-leading logarithmic (NNLL) soft gluon 
terms~\cite{Cacciari:2011hy,Baernreuther:2012ws,Czakon:2012zr,Czakon:2012pz,Czakon:2013goa}.

Samples of single-top-quark events corresponding to the $t$-channel production mechanism were generated with the 
{\powheg}~v1~\cite{Frederix:2012dh} generator, using the 4F scheme  for the NLO matrix-element calculations
and the fixed 4F \textsc{CT10}f\textsc{4}~\cite{Lai:2010vv} PDF set.
Samples corresponding to the $tW$- and $s$-channel production mechanisms were generated 
with {\powheg}~v1 using the CT10 PDF set. Overlaps between the $\ttbar$ and $tW$ final states were avoided by using 
the diagram removal scheme~\cite{Frixione:2005vw}.
The parton showers, hadronisation and the underlying event were modelled using {\pythia}~6.428~\cite{Sjostrand:2006za} 
with the CTEQ6L1~\cite{Pumplin:2002vw,Nadolsky:2008zw} PDF set 
in combination with the Perugia 2012 tune~\cite{Skands:2010ak}.
The single-top-quark samples are normalised to the approximate NNLO theoretical cross 
sections~\cite{Kidonakis:2011wy,Kidonakis:2010ux,Kidonakis:2010tc}. 

Samples of $W/Z$+jets events were generated with the {\sherpa}~2.2.1~\cite{Gleisberg:2008ta} generator. 
The matrix element was calculated for up to two partons at NLO and up to four partons at LO using 
\textsc{Comix}~\cite{Gleisberg:2008fv} and \textsc{OpenLoops}~\cite{Cascioli:2011va}. The matrix-element calculation 
is merged with the {\sherpa} parton shower~\cite{Schumann:2007mg} using the ME+PS@NLO prescription~\cite{Hoeche:2012yf}. 
The PDF set used for the matrix-element calculation is NNPDF3.0 NNLO~\cite{Ball:2014uwa} with a dedicated parton shower tuning developed for {\sherpa}. 
Separate samples were generated for different $W/Z$+jets categories using filters for a $b$-jet 
($W/Z$+$\geq$1$b$+jets), a $c$-jet and no $b$-jet ($W/Z$+$\geq$1$c$+jets), and with a veto on $b$- and $c$-jets 
($W/Z$+light-jets), which are combined into the inclusive $W/Z$+jets samples.
Both the $W$+jets and $Z$+jets samples are normalised to their respective inclusive NNLO theoretical 
cross sections calculated with \textsc{FEWZ}~\cite{Anastasiou:2003ds}.

Samples of $WW/WZ/ZZ$+jets events were generated with {\sherpa}~2.2.1 using the CT10 PDF set
and include processes containing up to four electroweak vertices. 
In the case of $WW/WZ$+jets ($ZZ$+jets) the matrix element was calculated for zero (up to one) additional partons 
at NLO and up to three partons at LO using the same procedure as for the $W/Z$+jets samples. 
The final states simulated require one of the bosons to decay leptonically and the other hadronically.
All diboson samples are normalised to their NLO theoretical cross sections provided by {\sherpa}. 

Samples of $\ttbar V$ and $\ttbar H$ events were generated with {\amcatnlo}~2.2.1, using NLO matrix elements and the NNPDF3.0 NLO PDF set,
and interfaced to {\pythia}~8.210 with the NNPDF2.3 LO PDF set and the A14 tune. 
Instead, the $\ttbar V$ samples used in the $\Hbb$ search are based on LO matrix elements computed for up to two additional partons 
using the NNPDF3.0 NLO PDF set, and merged using the CKKW-L approach~\cite{Lonnblad:2001iq}.
The $\ttbar V$ samples are normalised to the NLO cross section computed with {\amcatnlo}, while the $\ttbar H$ sample is normalised using 
the NLO cross section recommended in Ref.~\cite{deFlorian:2016spz}.

All generated samples, except those produced with the {\sherpa}~\cite{Gleisberg:2008ta} event generator, 
utilise \textsc{EvtGen}~1.2.0~\cite{Lange:2001uf} to model the decays of heavy-flavour hadrons. 
To model the effects of pile-up, events from minimum-bias interactions were generated using {\pythia}~8.186~\cite{Sjostrand:2007gs}  
in combination with the A2 tune~\cite{ATL-PHYS-PUB-2011-014}, 
and overlaid onto the simulated hard-scatter events according to the luminosity profile of the recorded data. 
The generated events were processed through a simulation~\cite{Aad:2010ah} of the ATLAS detector geometry and response 
using \textsc{Geant4}~\cite{Agostinelli:2002hh}. A faster simulation, where the full \textsc{Geant4} simulation of
the calorimeter response is replaced by a detailed parameterisation of the shower shapes~\cite{FastCaloSim},
was adopted for some of the samples used to estimate systematic uncertainties in background modelling.
Simulated events were processed through the same reconstruction software as the data, and corrections were applied so that the object identification 
efficiencies, energy scales and energy resolutions match those determined from data control samples.

%-------------------------------------------------------------------------------
\subsection{Backgrounds with fake leptons}
\label{sec:fakeleptons}
%-------------------------------------------------------------------------------

\subsubsection{Fake electrons and muons}
Electrons and muons only appear in the leptonic channels where there is exactly one electron or muon and at least one \had. Since the fake rate of \had candidate is much larger than electron or muon. So in the majority of the fake lepton events, the \had candidate is misidentified and the electron or muon is real, which is also proved by monte carlo studies. However the multi-jet background has large event rate and can contribute with both electron or muon and \had faked to $t_l\had$ where the fake-lepton dominates and the jet multiplicity is low.  This kind of event is modelled by ABCD method widely used in ATLAS physics analyses.

\subsubsection{Fake $\tau$-lepton candidates}
\label{sec:faketaus}
The background with one or more fake $\had$ candidates mainly arises from $\ttbar$ or
multijet production, depending on the search channel. 
Studies based on the simulation show that, for all the above processes, fake $\had$ candidates primarily result from the 
misidentification of light-quark jets, with the contribution from $b$-quarks and gluon jets playing a subdominant role.
It is also found that the fake rate decreases for all jet flavours as the $\had$ candidate $\pt$ increases.

In the hadronic channels, this background is estimated partially from data by defining control regions (CR) enriched in fake $\had$ candidates via loosened $\had$ requirements and partially from monte carlo. These CRs do not overlap with the main search regions (SRs), discussed in Section~\ref{sec:strategy_Htautau}. The CR selection requirements are analogous to those used to define the different SRs, except that the trailing $\had$ candidate is required to fail the medium $\had$ identification.

The fake $\had$ background prediction in a given SR is modelled by the distribution (referred to as the fake $\had$ template) derived from data in the corresponding CR. The fake $\had$ template is defined as the data distribution from which the contributions from the simulated backgrounds with real $\had$ candidates, originating primarily from 
$W(\to \tau\nu)$+jets and $Z(\to \tau\tau)$+jets, are subtracted. In the $\lephad$ channel, simulation studies indicate that the fake $\had$ background composition is consistent between the SR and the CR, and dominated by $\ttbar$ production. In the $\hadhad$ channel, the fake $\had$ background is expected to be dominated by multijet production. However, simulation studies indicate that the contribution of $\ttbar$ events to the fake $\had$ background is higher in the SR than in the CR. Therefore, an appropriate number of simulated $\ttbar$ events with fake $\had$ candidates in the CR is added to the fake $\had$ template to match the fake $\had$ background composition in the SR. 
In both the $\lephad$ and $\hadhad$ channels, the fake $\had$ template in each SR is initially normalised to the estimated fake $\had$ background yield, 
defined as the data yield minus the contributions from the simulated backgrounds with real $\had$ candidates (assuming no signal contribution).
During the statistical analysis, the normalisation of the fake $\had$ background in each SR is allowed to vary freely in the fit to data, as discussed in Section~\ref{sec:results_Htautau}.
%-------------------------------------------------------------------------------
%-------------------------------------------------------------------------------
\section{Simulated samples}
\label{sec:simulations}
%-------------------------------------------------------------------------------

An overview of the \DIFdelbegin \DIFdel{MC }\DIFdelend \DIFaddbegin \DIFadd{Monte Carlo (MC) }\DIFaddend generators used for the main signal and background samples is summarized in Table~\ref{mob}.
\DIFdelbegin %DIFDELCMD < 

%DIFDELCMD < %%%
\DIFdelend Samples of simulated $\ttbar \to WbHq$ ($tt(qH)$) events were generated with the next-to-leading-order (NLO) generator\footnote{In the following, 
the order of a generator should be understood as referring to the order in the strong coupling constant at which the matrix-element (ME) calculation 
is performed.} {\powheg}~v2 \cite{Frixione:2007nw,Nason:2004rx,Frixione:2007vw,Alioli:2010xd}
%{\amcatnlolong}~2.4.3~\cite{Alwall:2014hca}  (referred to in the following as {\amcatnlo})
with the NNPDF3.0 NLO~\cite{Ball:2014uwa} parton distribution function (PDF) set and interfaced to {\pythia} 8.212~\cite{Sjostrand:2007gs} with the NNPDF2.3 LO~\cite{Ball:2012cx} PDF set for the modelling of parton showering (PS), hadronisation, and the underlying event. 
The A14~\cite{ATLASUETune4} set of tuned parameters in {\pythia} controlling the description of multiparton interactions and  
initial- and final-state radiation, referred to as the tune, was used.
The signal sample is normalised to the same total cross section as used for the inclusive $t\bar{t}\to WbWb$ sample (see discussion below) and
assuming a benchmark branching ratio of $\BR_{\mathrm{ref}}(t\to qH)=0.1\%$.
The case of both top quarks decaying into $qH$ is neglected in the analysis given the existing upper limits on $\BR(t \to qH)$ (Section~\ref{sec:intro}).

The $tH$ signal events were generated by {\amcatnlolong}~2.6.2~\cite{Alwall:2014hca}  (referred to in the following as {\amcatnlo})
with the NNPDF3.0 NLO parton distribution function (PDF) set and interfaced to {\pythia} 8.212 with the NNPDF2.3 LO PDF set for the modelling of parton showering,
hadronisation, and the underlying event with the A14 tune.
Depending on either up quark or charm quark involved in the FCNC production, the effective Lagrangian of $tqH$ interaction are parametrised using
dim-6 operators~\cite{fcnc_production_theory}. We obtain $\sigma(ug\to tH)$ = 0.711 pb and $\sigma(cg\to tH)$ = 0.103 pb using $\BR_{\mathrm{ref}}(t\to qH)=0.1\%$ as benchmark.   

The nominal sample used to model the $\ttbar$ background was generated with the NLO generator {\powheg}~v2
%\cite{Frixione:2007nw,Nason:2004rx,Frixione:2007vw,Alioli:2010xd}
using the NNPDF3.0 NLO PDF set. The {\powheg} model parameter $h_{\textrm{damp}}$, which controls 
matrix element to parton shower matching and effectively regulates the high-$\pt$ radiation, was set to 1.5 times the top-quark mass. 
The parton showers, hadronisation, and underlying event were modelled by {\pythia}~8.210 with the NNPDF2.3 LO PDF set in combination with the A14 tune.
Alternative $\ttbar$ simulation samples used to derive parton shower systematic uncertainties are described in Section~\ref{sec:syst_bkgmodeling}. 
The generated $\ttbar$ samples are normalised to a theoretical cross section of $832^{+46}_{-51}$~pb, 
computed using \textsc{Top++}~v2.0~\cite{Czakon:2011xx} at next-to-next-to-leading order (NNLO), 
including resummation of next-to-next-to-leading logarithmic (NNLL) soft gluon 
terms~\cite{Cacciari:2011hy,Baernreuther:2012ws,Czakon:2012zr,Czakon:2012pz,Czakon:2013goa}.

Samples of single-top-quark events corresponding to the $t$-channel production mechanism were generated with the 
{\powheg}~v2~\cite{Frederix:2012dh} generator, using the 4F scheme  for the NLO matrix-element calculations
and the corresponding NNPDF3.0 NLO set of PDFs.
Samples corresponding to the $tW$- and $s$-channel production mechanisms were generated 
with {\powheg}~v2 using the five-flavour scheme. Overlaps between the $\ttbar$ and $tW$ final states were avoided by using 
the diagram removal scheme~\cite{Frixione:2005vw}.
The parton showers, hadronisation and the underlying event were modelled using {\pythia}~8.230~\cite{Sjostrand:2006za} using the A14 tune and the NNPDF2.3 LO PDF set.
%with the CTEQ6L1~\cite{Pumplin:2002vw,Nadolsky:2008zw} PDF set %in combination with the Perugia 2012 tune~\cite{Skands:2010ak}.
The single-top-quark samples are normalised to the approximate NNLO theoretical cross 
sections~\cite{Kidonakis:2011wy,Kidonakis:2010ux,Kidonakis:2010tc}. 

Samples of $W/Z$+jets events were generated with the {\sherpa}~2.2.1~\cite{Gleisberg:2008ta} generator. 
The matrix element was calculated for up to two partons at NLO and up to four partons at LO using 
\textsc{Comix}~\cite{Gleisberg:2008fv} and \textsc{OpenLoops}~\cite{Cascioli:2011va}. The matrix-element calculation 
is merged with the {\sherpa} parton shower~\cite{Schumann:2007mg} using the ME+PS@NLO prescription~\cite{Hoeche:2012yf}. 
The PDF set used for the matrix-element calculation is NNPDF3.0 NNLO~\cite{Ball:2014uwa} with a dedicated parton shower tuning developed for {\sherpa}. 
Separate samples were generated for different $W/Z$+jets categories using filters for a $b$-jet 
($W/Z$+$\geq$1$b$+jets), a $c$-jet and no $b$-jet ($W/Z$+$\geq$1$c$+jets), and with a veto on $b$- and $c$-jets 
($W/Z$+light-jets), which are combined into the inclusive $W/Z$+jets samples.
Both the $W$+jets and $Z$+jets samples are normalised to their respective inclusive NNLO theoretical 
cross sections calculated with \textsc{FEWZ}~\cite{Anastasiou:2003ds}.

Samples of $WW/WZ/ZZ$+jets events were generated with {\sherpa}~2.2.1 using the CT10 PDF set
and include processes containing up to four electroweak vertices. 
In the case of $WW/WZ$+jets ($ZZ$+jets) the matrix element was calculated for zero (up to one) additional partons 
at NLO and up to three partons at LO using the same procedure as for the $W/Z$+jets samples. 
The final states simulated require one of the bosons to decay leptonically and the other hadronically.
All diboson samples are normalised to their NLO theoretical cross sections provided by {\sherpa}. 

Samples of $\ttbar V$ and $\ttbar H$ events were generated with {\amcatnlo}~2.2.1, using NLO matrix elements and the NNPDF3.0 NLO PDF set,
and interfaced to {\pythia}~8.210 with the NNPDF2.3 LO PDF set and the A14 tune. 
%Instead, the $\ttbar V$ samples used in the $\Hbb$ search are based on LO matrix elements computed for up to two additional partons 
%using the NNPDF3.0 NLO PDF set, and merged using the CKKW-L approach~\cite{Lonnblad:2001iq}.
The $\ttbar V$ samples are normalised to the NLO cross section computed with {\amcatnlo}, while the $\ttbar H$ sample is normalised using 
the NLO cross section recommended in Ref.~\cite{deFlorian:2016spz}.

Samples of $WH$ and $ZH$, collectively referred to as $VH$ are generated using {\powheg}~v2 \cite{Frixione:2007nw,Nason:2004rx,Frixione:2007vw,Alioli:2010xd}
and interfaced to {\pythia}~8.210 with the NNPDF2.3 LO PDF set and the A14 tune.
The contribution of $tH$ associated production is also considered as part of SM Higgs background.
The sample is generated using {\amcatnlolong}~2.6.2~\cite{Alwall:2014hca} and interfaced with {\pythia}~8.210 with CT10PDF
and A14 tune. The MC predictions are normalised using the NLO cross section recommended in Ref.~\cite{deFlorian:2016spz}. 

All generated samples, except those produced with the {\sherpa}~\cite{Gleisberg:2008ta} event generator, 
utilise \textsc{EvtGen}~1.2.0~\cite{Lange:2001uf} to model the decays of heavy-flavour hadrons. 
To model the effects of pile-up, events from minimum-bias interactions were generated using {\pythia}~8.186~\cite{Sjostrand:2007gs}  
in combination with the A2 tune~\cite{ATL-PHYS-PUB-2011-014}, 
and overlaid onto the simulated hard-scatter events according to the luminosity profile of the recorded data. 
The generated events were processed through a simulation~\cite{Aad:2010ah} of the ATLAS detector geometry and response 
using \textsc{Geant4}~\cite{Agostinelli:2002hh}. A faster simulation, where the full \textsc{Geant4} simulation of
the calorimeter response is replaced by a detailed parameterisation of the shower shapes~\cite{FastCaloSim},
was adopted for some of the samples used to estimate systematic uncertainties in background modelling.
Simulated events were processed through the same reconstruction software as the data, and corrections were applied so that the object identification 
efficiencies, energy scales and energy resolutions match those determined from data control samples.


%DIF >   \begin{table}
%DIF >   \footnotesize
%DIF >   \centering
%DIF >   \caption{Overview of the MC generators used for the main signal and background samples.}
%DIF >   \begin{tabular}[h]{l|c|c|c|c|c|c}
%DIF >   \hline \hline
%DIF >   \multirow{2}{*}{Process} & \multicolumn{2}{c|}{Generator} & \multicolumn{2}{c|}{PDF set} & \multirow{2}{*}{Tune} & \multirow{2}{*}{Order} \\ \cline{2-5}
%DIF >           &  ME   &  PS    &   ME  & PS &   &  \\\hline
%DIF >   $tt(qH)$ Signal & Powheg & Pythia8 & NNPDF30NLO & NNPDF23LO & A14 & NLO \\ \hline
%DIF >   $tH$ Signal & MadGraph5\_aMC@NLO & Pythia8 & NNPDF30NLO & NNPDF23LO & A14 & NLO \\ \hline
%DIF >   $W/Z$+jets & \multicolumn{2}{c|}{Sherpa 2.2.1} & \multicolumn{2}{c|}{NNPDF30NNLO} & Sherpa & NLO/LO \\ \hline
%DIF >   \ttbar & Powheg & Pythia8 & NNPDF30NLO & NNPDF23LO & A14 & NLO \\ \hline
%DIF >   Single top & Powheg & Pythia8 & NNPDF30NLO & NNPDF23LO & A14 & NLO \\ \hline
%DIF >   $t\bar{t}X$ & MadGraph5\_aMC@NLO & Pythia8 & NNPDF30NLO & NNPDF23LO & A14 & NLO \\ \hline
%DIF >   Diboson & \multicolumn{2}{c|}{Sherpa 2.2.1} & \multicolumn{2}{c|}{NNPDF30NNLO} & Sherpa & NLO/LO \\ \hline\hline
%DIF >   \end{tabular}
%DIF >   \label{mob}
%DIF >   \end{table}
\DIFaddbegin 

\DIFaddend \begin{table}
\footnotesize
\centering
\caption{Overview of the MC generators at which the matrix-element calculation (ME) at the leading or the next-leading order (Order), 
the parton shower model (PS), and the parton distribution function set (PDF) used for the main signal and background samples.}
\begin{tabular}[h]{l|c|c|c|c|c|c}
\hline \hline
\multirow{2}{*}{Process} & \multicolumn{2}{c|}{Generator} & \multicolumn{2}{c|}{PDF set} & \multirow{2}{*}{Tune} & \multirow{2}{*}{Order} \\ \cline{2-5}
        &  ME   &  PS    &  ME  & PS &   &  \\\hline
$tt(qH)$ Signal & {\powheg} & {\pythia}~8 & NNPDF30NLO & NNPDF23LO & A14 & NLO \\ \hline
$tH$ Signal & {\amcatnlolong} & {\pythia}~8 & NNPDF30NLO & NNPDF23LO & A14 & NLO \\ \hline
$W/Z$+jets & \multicolumn{2}{c|}{{\sherpa}~2.2.1} & \multicolumn{2}{c|}{NNPDF30NNLO} & {\sherpa} & NLO/LO \\ \hline
\ttbar & {\powheg} & {\pythia}~8 & NNPDF30NLO & NNPDF23LO & A14 & NLO \\ \hline
Single top & {\powheg} & {\pythia}~8 & NNPDF30NLO & NNPDF23LO & A14 & NLO \\ \hline
$t\bar{t}X$ & {\amcatnlolong} & {\pythia}~8 & NNPDF30NLO & NNPDF23LO & A14 & NLO \\ \hline
Diboson & \multicolumn{2}{c|}{{\sherpa}~2.2.1} & \multicolumn{2}{c|}{NNPDF30NNLO} & {\sherpa} & NLO/LO \\ \hline\hline
\end{tabular}
\label{mob}
\end{table}



%This kind of event is modelled by data-driven ABCD method
%by cutting on $\met$ and the tight lepton isolation using PLIV, which is widely used in ATLAS physics analyses.

%\subsubsection{Fake $\tau$-lepton candidates}
%\label{sec:faketaus}
%The background with one or more fake $\had$ candidates mainly arises from $\ttbar$ or
%multijet production, depending on the search channel. 
%Studies based on the simulation show that, for all the above processes, fake $\had$ candidates primarily result from the 
%misidentification of light-quark jets, with the contribution from $b$-quarks and gluon jets playing a subdominant role.
%It is also found that the fake rate decreases for all jet flavours as the $\had$ candidate $\pt$ increases.

%In the hadronic channels, this background is estimated partially from data by defining control regions (CR) enriched in fake $\had$ candidates via loosened $\had$ requirements with% proper fake factors and partially from monte carlo. These CRs do not overlap with the main search regions (SRs), discussed in Section~\ref{sec:strategy_Htautau}. The CR selection %requirements are analogous to those used to define the different SRs, except that the trailing $\had$ candidate is required to fail the medium $\had$ identification.

%In the leptonic channels, the events with real electron or muon and fake taus are modelled by calibrated monte carlo with scale factors derived from the dedicated $t\bar t$
%control regions ($CR_{tt}$) using the SM $t\bar t$ decay of dilepton events and semileptonically double-btagged lepton-jet events.
%The calibration is done depending on different source of the fake taus and \pt. The calibration factor are derived in the dedicated control regions discussed in Section~\ref{sec:st%rategy_Htautau}.


% Strategy Htautau
%-------------------------------------------------------------------------------
\section{Analysis strategy}
\label{sec:strategy_Htautau}
%-------------------------------------------------------------------------------

The analysis strategy adopted in the $\Htautau$ search is similar to the one used in Ref.~\cite{fcnc36,Chen:2015nta} but extended to more search channels.
\DIFdelbegin %DIFDELCMD < 

%DIFDELCMD < %%%
\subsection{\DIFdel{Event categorisation and kinematic reconstruction}}
%DIFAUXCMD
\addtocounter{subsection}{-1}%DIFAUXCMD
%DIFDELCMD < \label{sec:htautau_reco_cat}
%DIFDELCMD < 

%DIFDELCMD < %%%
\DIFdelend %DIF > \subsection{Event categorisation and kinematic reconstruction}
%DIF > \label{sec:htautau_reco_cat}
The $tt(qH)$ and $tH$ signal being probed are characterised by the presence of $\tau$-leptons from the decay of 
the Higgs boson, where the remaining top quark decays into $Wb$. There is an additional $q$-jet from the FCNC $t\to qH$ decay in the top pair production. 
%at least  four (three) jets in the decay (production) mode, only one of which originates from a $b$-quark.
If the $W$ boson or one of the $\tau$-leptons decays leptonically, an isolated electron or muon, together with significant $\met$ is also expected.
In a significant fraction of the events, the lowest $\pt$ jet from the hadronic $W$ boson decay fails the minimum $\pt$ requirement of $25~\gev$,
resulting in only three reconstructed jets where the production mode is dominant. 
%some signal events mixed with production mode having only three reconstructed jets.
In order to optimise the sensitivity of the search, the selected events are categorised into seven SRs based on the number of light leptons,
$\had$ candidates, and on the number of light flavored jets:
\DIFdelbegin \DIFdel{$t_l\hadhad$, $t_l\had$}\DIFdelend \DIFaddbegin \DIFadd{$t_{\ell}\hadhad$, $t_{\ell}\had$}\DIFaddend -1j, \DIFdelbegin \DIFdel{$t_l\had$}\DIFdelend \DIFaddbegin \DIFadd{$t_{\ell}\had$}\DIFaddend -2j, $t_h\lephad$-2j, $t_h\lephad$-3j, $t_h\hadhad$-2j, and $t_h\hadhad$-3j, as shown in Table~\ref{tab:srcr}. 

%DIF >    \begin{table}
%DIF >    \centering
%DIF >    \caption{Overview of the signal regions and the control regions used for fake tau scale factor derivation in leptonic channels.}
%DIF >    \label{tab:srcr}
%DIF >    \begin{tabular}[h]{c|c|c|c|c|c|c}
%DIF >    \hline \hline
%DIF >    \multicolumn{2}{c|}{Regions} & $b$-jet & light flavor jets        & lepton & hadronic taus & charge\\ \hline
%DIF >    \multirow{7}{*}{SR}&$t_{\ell}\thadhad$     & 1     & any                                & 1      & 2             & $\thadhad$ OS\\ \cline{2-7}
%DIF >    &$t_{\ell}\tauhad$-1j  & 1     & 1                                   & 1      & 1                     & $t_{\ell}\tauhad$ SS\\ \cline{2-7}
%DIF >    &$t_{\ell}\tauhad$-2j  & 1     & 2                                        & 1      & 1                     & $t_{\ell}\tauhad$ SS\\ \cline{2-7}
%DIF >    &$t_h\tlhad$-2j   & 1     & 2                           & 1      & 1             & $\tlhad$ OS\\ \cline{2-7}
%DIF >    &$t_h\tlhad$-3j   & 1     & $\ge3$                      & 1      & 1             & $\tlhad$ OS\\ \cline{2-7}
%DIF >    &$t_h\thadhad$-2j & 1     & 2                            & 0      & 2             & $\thadhad$ OS\\ \cline{2-7}
%DIF >    &$t_h\thadhad$-3j & 1     & $\ge3$                       & 0      & 2             & $\thadhad$ OS\\ \hline
%DIF >    \multirow{6}{*}{CRtt}&$t_{\ell}t_{\ell}1b\tauhad$ & 1     & any                           & 2      & 1                     & $t_{\ell}t_{\ell}$ OS\\ \cline{2-7}
%DIF >    &$t_{\ell}t_{\ell}2b\tauhad$      & 2     & any                           & 2      & 1                     & $t_{\ell}t_{\ell}$ OS\\ \cline{2-7}
%DIF >    &$t_{\ell}t_h2b\tauhad$-2jSS & 2     & 2                             & 1      & 1             & $t_{\ell}\tauhad$ SS\\ \cline{2-7}
%DIF >    &$t_{\ell}t_h2b\tauhad$-2jOS & 2     & 2                             & 1      & 1             & $t_{\ell}\tauhad$ OS\\ \cline{2-7}
%DIF >    &$t_{\ell}t_h2b\tauhad$-3jSS & 2     & $\ge3$                        & 1      & 1             & $t_{\ell}\tauhad$ SS\\ \cline{2-7}
%DIF >    &$t_{\ell}t_h2b\tauhad$-3jOS & 2     & $\ge3$                & 1      & 1             & $t_{\ell}\tauhad$ OS\\ \hline
%DIF >    \end{tabular}
%DIF >    \end{table}
\DIFaddbegin 


\DIFaddend \begin{table}
\centering
\caption{Overview of the signal regions (SR) and the $t\bar{t}$ control regions (CRtt) used for fake tau scale factor derivation in leptonic channels. Leptons are required to have either same-sign (SS) or opposite-sign (OS) charges in each region.}
\label{tab:srcr}
\begin{tabular}[h]{c|c|c|c|c|c|c}
\hline \hline
\multicolumn{2}{c|}{Regions} & $b$-jet & light flavor jets        & lepton & hadronic taus & charge\\ \hline
\multirow{7}{*}{SR}&\DIFdelbeginFL \DIFdelFL{$t_l\thadhad$     }\DIFdelendFL \DIFaddbeginFL \DIFaddFL{$t_{\ell}\thadhad$     }\DIFaddendFL & 1     & $\ge0$                                & 1      & 2             & $\thadhad$ OS\\ \cline{2-7}
&\DIFdelbeginFL \DIFdelFL{$t_l\tauhad$}\DIFdelendFL \DIFaddbeginFL \DIFaddFL{$t_{\ell}\tauhad$}\DIFaddendFL -1j  & 1     & 1                                   & 1      & 1                     & \DIFdelbeginFL \DIFdelFL{$t_l\tauhad$ }\DIFdelendFL \DIFaddbeginFL \DIFaddFL{$t_{\ell}\tauhad$ }\DIFaddendFL SS\\ \cline{2-7}
&\DIFdelbeginFL \DIFdelFL{$t_l\tauhad$}\DIFdelendFL \DIFaddbeginFL \DIFaddFL{$t_{\ell}\tauhad$}\DIFaddendFL -2j  & 1     & 2                                        & 1      & 1                     & \DIFdelbeginFL \DIFdelFL{$t_l\tauhad$ }\DIFdelendFL \DIFaddbeginFL \DIFaddFL{$t_{\ell}\tauhad$ }\DIFaddendFL SS\\ \cline{2-7}
&$t_h\tlhad$-2j   & 1     & 2                           & 1      & 1             & $\tlhad$ OS\\ \cline{2-7}
&$t_h\tlhad$-3j   & 1     & $\ge3$                      & 1      & 1             & $\tlhad$ OS\\ \cline{2-7}
&$t_h\thadhad$-2j & 1     & 2                            & 0      & 2             & $\thadhad$ OS\\ \cline{2-7}
&$t_h\thadhad$-3j & 1     & $\ge3$                       & 0      & 2             & $\thadhad$ OS\\ \hline
\multirow{6}{*}{CRtt}&\DIFdelbeginFL \DIFdelFL{$t_lt_l1b\tauhad$ }\DIFdelendFL \DIFaddbeginFL \DIFaddFL{$t_{\ell}t_{\ell}1b\tauhad$ }\DIFaddendFL & 1     & $\ge0$                            & 2      & 1                     & \DIFdelbeginFL \DIFdelFL{$t_lt_l$ }\DIFdelendFL \DIFaddbeginFL \DIFaddFL{$t_{\ell}t_{\ell}$ }\DIFaddendFL OS\\ \cline{2-7}
&\DIFdelbeginFL \DIFdelFL{$t_lt_l2b\tauhad$      }\DIFdelendFL \DIFaddbeginFL \DIFaddFL{$t_{\ell}t_{\ell}2b\tauhad$      }\DIFaddendFL & 2     & $\ge0$                            & 2      & 1                     & \DIFdelbeginFL \DIFdelFL{$t_lt_l$ }\DIFdelendFL \DIFaddbeginFL \DIFaddFL{$t_{\ell}t_{\ell}$ }\DIFaddendFL OS\\ \cline{2-7}
&\DIFdelbeginFL \DIFdelFL{$t_lt_h2b\tauhad$}\DIFdelendFL \DIFaddbeginFL \DIFaddFL{$t_{\ell}t_h2b\tauhad$}\DIFaddendFL -2jSS & 2     & 2                             & 1      & 1             & \DIFdelbeginFL \DIFdelFL{$t_l\tauhad$ }\DIFdelendFL \DIFaddbeginFL \DIFaddFL{$t_{\ell}\tauhad$ }\DIFaddendFL SS\\ \cline{2-7}
&\DIFdelbeginFL \DIFdelFL{$t_lt_h2b\tauhad$}\DIFdelendFL \DIFaddbeginFL \DIFaddFL{$t_{\ell}t_h2b\tauhad$}\DIFaddendFL -2jOS & 2     & 2                             & 1      & 1             & \DIFdelbeginFL \DIFdelFL{$t_l\tauhad$ }\DIFdelendFL \DIFaddbeginFL \DIFaddFL{$t_{\ell}\tauhad$ }\DIFaddendFL OS\\ \cline{2-7}
&\DIFdelbeginFL \DIFdelFL{$t_lt_h2b\tauhad$}\DIFdelendFL \DIFaddbeginFL \DIFaddFL{$t_{\ell}t_h2b\tauhad$}\DIFaddendFL -3jSS & 2     & $\ge3$                        & 1      & 1             & \DIFdelbeginFL \DIFdelFL{$t_l\tauhad$ }\DIFdelendFL \DIFaddbeginFL \DIFaddFL{$t_{\ell}\tauhad$ }\DIFaddendFL SS\\ \cline{2-7}
&\DIFdelbeginFL \DIFdelFL{$t_lt_h2b\tauhad$}\DIFdelendFL \DIFaddbeginFL \DIFaddFL{$t_{\ell}t_h2b\tauhad$}\DIFaddendFL -3jOS & 2     & $\ge3$                & 1      & 1             & \DIFdelbeginFL \DIFdelFL{$t_l\tauhad$ }\DIFdelendFL \DIFaddbeginFL \DIFaddFL{$t_{\ell}\tauhad$ }\DIFaddendFL OS\\ \hline
\end{tabular}
\end{table}







%This event categorisation is primarily motivated by the different quality of the event kinematic reconstruction, depending on the amount 
%of $\met$ in the event (larger in $\lephad$ events compared with $\hadhad$ events), and whether a jet from the hadronic top-quark decay %is missing or not (events with exactly three jets or at least four jets).
This event categorisation is primarily motivated by optimizing the sensitivity in each signal region that targets either leptonic or hadronic top-quark
decays as well as the Higgs decays into either $\lephad$ or $\hadhad$ final states.   
The event kinematic reconstruction is used in the $t_h\hadhad$ and $t_h\lephad$ channels to suppress the background by constraining the
di-tau mass and the missing transverse energy in the event~\cite{Chen:2015nta}.
\DIFaddbegin 

\DIFaddend %based on the strategy used in Ref.~\cite{Chen:2015nta}, and is summarised below.

For the $t_ht(qH)$ events, the jet from $t\to qH$, denoted as the FCNC jet ($q$-jet), should be a hard narrow jet from the
decay chain $t\to qH\to q\tau\tau$, with taus reconstructed as $\lephad$ or $\hadhad$.
There should be at least four jets among which the one with the smaller angular distance to the visible decay of the di-tau,
is considered as the $q$-jet since the FCNC top decay products are likely boosted closer together than other jets. 
If there are more than two jets besides the $q$- and $b$-jet, the jets from $W$ boson decay are chosen from the combination
with the invariant mass closest to the $W$ boson mass. There is the chance that one of the jets fails the $\pt$ requirement and \DIFaddbegin \DIFadd{is }\DIFaddend not reconstructed.
This kind of events will fall into $t_hH$ \DIFaddbegin \DIFadd{final state}\DIFaddend .
%while the FCNC top resonance is
%still reconstructable that the missing jet is likely from the decay of $W$ boson.
%given the big chance that the jet which is missing is from $W$ decay.
\DIFdelbegin \DIFdel{In the $t_hH$ events, there }\DIFdelend \DIFaddbegin \DIFadd{There }\DIFaddend are three jets coming from top hadronic decay including the $b$-jet and \DIFdelbegin \DIFdel{two }\DIFdelend \DIFaddbegin \DIFadd{a pair of }\DIFaddend opposite-sign \DIFdelbegin \DIFdel{$\had$ }\DIFdelend \DIFaddbegin \DIFadd{taus }\DIFaddend from the Higgs boson decay, where
both $t_h$ and \DIFdelbegin \DIFdel{H }\DIFdelend \DIFaddbegin \DIFadd{$H$ }\DIFaddend can be reconstructed.  
%. So a Higgs resonance formed by the taus and a top resonance formed by
%the jets are expected.

The four-momenta of the invisible decay products from the decay of the \DIFdelbegin \DIFdel{tau-leptons 
}\DIFdelend \DIFaddbegin \DIFadd{$\tau$-leptons 
}\DIFaddend are estimated using a kinematics fit. The fit is done by minimising a $\chi^2$ function based on the Gaussian constraints on the Higgs boson mass and the
measured $\met$ within their expected resolutions. The resolution on the Higgs boson mass is assumed to be $20~\gev$, \DIFdelbegin \DIFdel{respectively, }\DIFdelend while the resolution on the measured $\met$ is parameterised as a linear function of 
\DIFdelbegin \DIFdel{$\sqrt{\sum E_T}$, where $\sum E_T$ }\DIFdelend \DIFaddbegin \DIFadd{$\sqrt{\sum E_{\text{T}}}$, where $\sum E_{\text{T}}$ }\DIFaddend is the scalar sum of the $\pt$ of all physics objects contributing to the $\met$ reconstruction~\cite{Aaboud:2018tkc}.
After the $\chi^2$ minimisation, the Higgs boson \DIFdelbegin \DIFdel{four-momentum, and hence its invariant mass, }\DIFdelend \DIFaddbegin \DIFadd{$\pt$ }\DIFaddend as well as the 
\DIFdelbegin \DIFdel{four-momentum }\DIFdelend \DIFaddbegin \DIFadd{$\pt$ }\DIFaddend of the parent top quarks \DIFdelbegin \DIFdel{, }\DIFdelend are determined with better resolution in the signal events. 

For the  \DIFdelbegin \DIFdel{$t_lt(qH)$ and $t_lH$ events where the }\DIFdelend \DIFaddbegin \DIFadd{$t_{\ell}t(qH)$ and $t_{\ell}H$ events where }\DIFaddend $W$ boson from $t\to W b$ decay decays leptonically,
%has a large momentum and the flying direction is unknown.
the \DIFdelbegin \DIFdel{kinematics }\DIFdelend \DIFaddbegin \DIFadd{kinematic }\DIFaddend fit is no longer feasible due to extra neutrino from $W\rightarrow \ell\nu$ decay. The kinematic variables are calculated using the visible
objects only. After the event reconstruction, a number of variables are used to discriminate the signal from the background using a multivariate analysis described
in Section~\ref{sec:tmva}.

%With the event topology reconstructed, a number of variables are defined for signal and background separation used in the multivariate analysis discussed below.

%DIF < In the  $t_lt(qH)$ and $t_lH$ signal, the additional $q$ jet in the decay mode can still be found, but since the $W$ boson decays leptonically, there is a neutrino with large momentum and the flying direction is unknown. The kinematics fit is no longer feasible in the $t_l\hadhad$ channel. The variables calculated from the visible objects are directly used in the multivariate analysis. However, the kinematics fit is still performed for the $t_l\had$ channels where the lepton and $\had$ are treated as di-tau candidate.
%DIF > In the  $t_{\ell}t(qH)$ and $t_{\ell}H$ signal, the additional $q$ jet in the decay mode can still be found, but since the $W$ boson decays leptonically, there is a neutrino with large momentum and the flying direction is unknown. The kinematics fit is no longer feasible in the $t_{\ell}\hadhad$ channel. The variables calculated from the visible objects are directly used in the multivariate analysis. However, the kinematics fit is still performed for the $t_{\ell}\had$ channels where the lepton and $\had$ are treated as di-tau candidate.

\DIFaddbegin \FloatBarrier

%DIF >  Background estimation
%DIF > -------------------------------------------------------------------------------
\section{\DIFadd{Background estimation}}
\label{sec:background_model}
%DIF > -------------------------------------------------------------------------------

\DIFadd{Most background processes are modelled using MC simulation.
After the event preselection, the main background is $\ttbar$ production, often in association with jets, denoted by $\ttbar$+jets in the following.
Small contributions arise from single-top-quark, $W/Z$+jets, multi-jet and diboson ($WW,WZ,ZZ$) production, as well as from the associated 
production of a vector boson $V$ ($V=W,Z$) or a Higgs boson and a $\ttbar$ pair ($\ttbar V$ and $\ttbar H$). All backgrounds 
with prompt leptons, i.e.\ those originating from the decay of a $W$ boson, a $Z$ boson, or a $\tau$-lepton,
are estimated using samples of simulated events and initially normalised to their theoretical cross sections.
In the simulation, the top-quark and SM Higgs boson masses are set to $172.5~\gev$ and $125~\gev$, respectively,
and the Higgs boson is forced to decay into $H\to \tau\tau$ with branching ratio calculated using }\textsc{\DIFadd{Hdecay}}\DIFadd{~\mbox{%DIFAUXCMD
\cite{Djouadi:1997yw}}\hspace{0pt}%DIFAUXCMD
.  
Backgrounds with non-prompt light leptons (electron or muon), with photons or jets misidentified as electrons, or with jets misidentified as tau candidates, 
generically referred to as }\texttt{\DIFadd{fake}} \DIFadd{leptons, are estimated using data-driven methods. 
The background prediction is further improved during the statistical analysis by performing a likelihood 
fit to data using several signal-depleted control regions as shown in Table~\ref{tab:srcr}.
}

%DIF > -------------------------------------------------------------------------------
%DIF > \subsection{Backgrounds with fake leptons}
%DIF > \label{sec:fakeleptons}
%DIF > -------------------------------------------------------------------------------
\DIFaddend \subsection{\DIFdelbegin \DIFdel{Multivariate discriminant}\DIFdelend \DIFaddbegin \DIFadd{Backgrounds with fake tau leptons}\DIFaddend }
\DIFaddbegin \label{sec:faketaus}
\DIFadd{The background with one or more fake tau candidates mainly arises from $\ttbar$ or
multi-jet production, depending on the search channels.
Studies based on simulation show that, for all the above processes, fake tau candidates primarily result from the
misidentification of light jets and $b$-quark jets.
It is also found that the fake rate decreases for all jet flavours as the tau candidate $\pt$ increases.
}

\DIFadd{In the leptonic channels, the events with prompt electron or muon and fake taus are modelled by calibrating MC with scale factors (SF)
derived from the dedicated $t\bar t$
control regions (CRtt) using dileptonic decays of $t\bar t$ events and semileptonic decays of $t\bar t$ events with two
$b$-jets, summarized in Table~\ref{tab:srcr}. 	
%DIF > the SM $t\bar t$ decay of dilepton events and semileptonically double-btagged lepton-jet events, summarized in Table~\ref{tab:srcr},
%DIF > aimed at fake taus from different origins.
There are four kinds of fake taus that need to be calibrated: Type-1 fake taus from $W$-jets ($\tau_{W}$)
with opposite-sign (OS) of charge to the light lepton;
Type-2 $\tau_{W}$'s with same-sign (SS) charge to the light lepton; Type-3 fake taus originating from $b$-hadron decays; Type-4 fake taus from light-flavour hadron decays.
The control regions are defined similar to the signal regions but with an additional $b$-jet or light lepton.
%DIF > $t_{\ell}t_{\ell}1b\thad$, $t_{\ell}t_{\ell}2b\thad$, $t_{\ell}t_h2b\thad$-2jSS, $t_{\ell}t_h2b\thad$-2jOS, $t_{\ell}t_h2b\thad$-3jSS, and $t_{\ell}t_h2b\thad$-3jOS.
The dilepton regions ($t_{\ell}t_{\ell}1b\thad$ and $t_{\ell}t_{\ell}2b\thad$) are used to calibrate Type-3 and Type-4 fake taus. The semileptonic
regions ($t_{\ell}t_h2b\thad$-2jOS and $t_{\ell}t_h2b\thad$-3jOS) where $\thad$ and light lepton have opposite charge are used to calibrate Type-1 fake taus.
Similarly for Type-2, the semileptonic regions ($t_{\ell}t_h2b\thad$-2jSS and $t_{\ell}t_h2b\thad$-3jSS) where $\thad$ and light lepton have same charge are used.
A simultaneous fit to data is made to derive the scale factors for fake taus in MC, which consist of total of 24 parameters
depending on four types, three $\pt$ bins, two bins for 1- and 3-prong taus separately.
These scale factors are then used to correct the MC estimated fakes in the corresponding  signal regions with a single $b$-jet.
In the $t_{\ell}\thadhad$ channel, both taus can be misidentified, so the calibration is applied to each tau separately, following the same procedure used in the $\tlhad$ channel.
%DIF > using the lepton and fake tau charges, then the scale factors are multiplied together.
The nominal value of scale factors will vary according to their uncertainties in the final fit.
%DIF > In the control regions with single lepton, $\met > 20$GeV and at least 2 light jets and 2 b-tagged jets are required to ensure that QCD contribution is negligible.
%DIF > The calibration is done depending on different source of the fake taus and \pt. The calibration factor are derived in the dedicated
%DIF > control regions discussed in Section~\ref{sec:strategy_Htautau}.
}

\DIFadd{In the hadronic channels, the contribution of fakes is estimated partially from data by defining control regions (CR)
enriched in misidentified
tau candidates via loosened tau requirements with proper fake factors and partially from MC~\mbox{%DIFAUXCMD
\cite{ATLAS-CONF-2021-044}}\hspace{0pt}%DIFAUXCMD
.
These CRs do not overlap with the main signal regions, discussed in Section~\ref{sec:strategy_Htautau}.
The CR selection requirements are analogous to those used to define signal regions, except
that the subleading tau candidate is required to fail the medium tau identification, but it still passes a loose requirement.
The contribution of fakes with subleading tau candidate can be calculated by rescaling the templates of loose taus in the CR
with fake factors (FF).
The templates are produced by subtracting all MC background contributions with real subleading taus from data.
The FFs are computed as
the ratio of the Data-MC ($\mathrm{real~tau}$) yields in which the tau candidate is passed or failed the medium tau ID selection
in the $W$+jets control region.
%DIF > passing to failing the medium tau ID in the $W$+jets control region.
%DIF > We have compared
FFs obtained from the $W$+jets and from the same-sign $\thadhad$ control regions are compared and the differences are treated as a systematic uncertainty.
}

\subsection{\DIFadd{Background with fake light leptons}}
\DIFadd{The background originating from non-prompt or misidentified light leptons is primarily from the multi-jet production.
%DIF > Electrons and muons only appear in the leptonic channels where exactly one electron or muon and at least one $\had$ are required.
%DIF > Since the rate of $\had$ candidate is much larger than for electron or muon, in the majority of the fake lepton events, the $\had$ candidate is misidentified and
%DIF > the electron or muon is prompt based on MC studies.
%DIF > However the multi-jet background has large event rate and can contribute with both electron or muon and $\had$ faked, especially to $t_{\ell}\had$ where the fake-lepton dominates. and the jet multiplicity is low.
The contribution from these events is estimated with a data-driven method called ABCD.
The estimate is based on the number of tight and loose light-leptons that passed and failed the PLIV cut in the low and high $\met$ regions. The number of
fake light leptons in the signal region is evaluated by scaling the number of loose leptons in the high $\met$ region by the ratio of tight and loose leptons in the
low $\met$ region. The contributions of prompt leptons and calibrated fake taus in the number of tight and loose leptons are subtracted using MC simulation.
A closure test is made for the background estimations in the signal depleted low BDT regions. The data are in good agreement with the background prediction in the
$t_{\ell}\had$ channels where the fake light leptons are important, while they are negligible in other leptonic channels. 
}


%DIF >  BDT
\section{\DIFadd{Multivariate discriminant}}
\DIFaddend \label{sec:tmva}

%%%%%%%%%%%%%%%
\begin{table*}[t!]
  \caption{\small{The number of discriminating variables (n) used in the training of BDT in each SR. 
The rank of the discriminating variables relative to one another according to their importance in the training is reported from highest (1) to 
lowest (n). Variables whose ranking is missing are not included in the training of the corresponding SR. The description of each variable is provided in the text.}}
%Variables are ranked from highest (1) to lowest (n) relative to one another by their
%      importance in the training. Variables which do not have a ranking are not included in the training. 
%The values represent their ra importance rankings of each variable and cross($\times$) denotes the variable is not used in the training.
%      The descriptions of each variable are provided in the following text.}}
\label{tab:importance}
 \centering
 \begin{tabular}{cccccccc} \toprule\toprule
   & $t_{l}\tauhad$-1j                                  &  $t_{h}\tlhad$-2j   &  $t_{l}\tauhad$-2j & $t_{h}\tlhad$-3j & \DIFdelbeginFL \DIFdelFL{$t_l\thadhad$     }\DIFdelendFL \DIFaddbeginFL \DIFaddFL{$t_{\ell}2\tauhad$     }\DIFaddendFL & \DIFdelbeginFL \DIFdelFL{$t_h\thadhad$}\DIFdelendFL \DIFaddbeginFL \DIFaddFL{$t_h2\tauhad$}\DIFaddendFL -2j & \DIFdelbeginFL \DIFdelFL{$t_h\thadhad$}\DIFdelendFL \DIFaddbeginFL \DIFaddFL{$t_h2\tauhad$}\DIFaddendFL -3j       \\\midrule
   Total variables~(n)                           & $12$ & $15$ & $12$ & $17$ & $15$ & $12$ & $12$ \\\midrule 
 \DIFdelbeginFL \DIFdelFL{$m_{\text{W}}$                                      }\DIFdelendFL \DIFaddbeginFL \DIFaddFL{$m_{\text{jj}}$                                      }\DIFaddendFL &   &             &           & $9$      &       & $6$      & $7$\\
 $\chi^{2}$                                          &   &             &           & $14$     &       &  &       \\
 \text{max}($\eta_{\tau}$)                           & $4$       &             &  $4$              &  & $10$          &  &        \\
 \DIFdelbeginFL \DIFdelFL{$m^{\text{T}}_{\text{W}}$                           }\DIFdelendFL \DIFaddbeginFL \DIFaddFL{$m^{W}_{\text{T}}$                           }\DIFaddendFL & $11$      &             &  $8$              &  & $13$          &  &         \\
 \DIFdelbeginFL \DIFdelFL{$m_{\tau,\tau}$                                     }\DIFdelendFL \DIFaddbeginFL \DIFaddFL{$m_{\tau\tau,\text{fit}}$                                     }\DIFaddendFL &   &  $2$                &           & $3$      &       & $1$      & $1$          \\
 \DIFdelbeginFL \DIFdelFL{$m_{\text{t},\text{SM}}$                            }\DIFdelendFL \DIFaddbeginFL \DIFaddFL{$m_{\text{bjj},\text{fit}}$                            }\DIFaddendFL &   &  $1$                &           & $2$      &       & $3$      & $4$          \\
 \DIFdelbeginFL \DIFdelFL{${\pt}_{\ell}$                                 }\DIFdelendFL \DIFaddbeginFL \DIFaddFL{${\pt}^{\ell}$                                 }\DIFaddendFL & $12$      &  $15$               &  $12$             & $17$     &       &  &         \\
 \DIFdelbeginFL \DIFdelFL{$m_{\text{t},\text{FCNC}}$                          }\DIFdelendFL \DIFaddbeginFL \DIFaddFL{$m_{\tau\tau\text{q},\text{fit}}$                          }\DIFaddendFL &   &             &           &  &       & $10$     & $6$\\
 \DIFdelbeginFL \DIFdelFL{$m_{\text{t},\text{SM},\text{vis}}$                 }\DIFdelendFL \DIFaddbeginFL \DIFaddFL{$m_{\text{bjj}}$                 }\DIFaddendFL & $3$       &             &  $5$              &  & $4$           &  &         \\
 \DIFdelbeginFL \DIFdelFL{${\pt}_{\tauhadvis}$                                 }\DIFdelendFL \DIFaddbeginFL \DIFaddFL{${\pt}_{\tau1}$                                 }\DIFaddendFL & $1$       &  $4$                &  $1$              & $1$      & $5$           & $11$   & $10$           \\
 $\met$                                              & $5$       &  $11$               &  $10$             & $13$     & $6$           & $7$    & $13$          \\
 \DIFdelbeginFL \DIFdelFL{$m_{\tau\tau,\text{vis}}$                           }\DIFdelendFL \DIFaddbeginFL \DIFaddFL{$m_{\tau\tau}$                           }\DIFaddendFL & $10$      &  $14$               &  $11$             & $6$      & $1$           & $2$    & $2$          \\
 \DIFdelbeginFL \DIFdelFL{$E_{\text{vis}~\tau1}/E_{\tau1}$                  }\DIFdelendFL \DIFaddbeginFL \DIFaddFL{$E_{\tau1}/E_{\tau1,\text{fit}}$                  }\DIFaddendFL &   &  $10$               &           & $12$     &       & $8$    & $8$          \\
 \DIFdelbeginFL \DIFdelFL{$E_{\text{vis}~\tau2}/E_{\tau2}$                  }\DIFdelendFL \DIFaddbeginFL \DIFaddFL{$E_{\tau2}/E_{\tau2,\text{fit}}$                  }\DIFaddendFL &   &  $7$                &           & $4$      &       & $9$    & $11$         \\
 \DIFdelbeginFL \DIFdelFL{$P_{\text{T,\tauhadvis}} $                          }\DIFdelendFL \DIFaddbeginFL \DIFaddFL{${\pt}_{\tautau} $                          }\DIFaddendFL &   &             &           &  & $9$           &  &         \\
 \DIFdelbeginFL \DIFdelFL{$m_{\text{t},\text{FCNC},\text{vis}}$               }\DIFdelendFL \DIFaddbeginFL \DIFaddFL{$m_{\tau\tau\text{q}}$               }\DIFaddendFL &   &             &           &  & $3$           &  &        \\
 $\Delta\phi(\tau\tau,\met)$                         &   &  $6$                            &           & $16$     &       & $13$   & $12$         \\
 $\met\text{centrality}$                             &   &  $13$               &           & $15$     &       & $12$   & $9$         \\
 \text{\DIFdelbeginFL \DIFdelFL{Min}\DIFdelendFL \DIFaddbeginFL \DIFaddFL{min}\DIFaddendFL }(\DIFdelbeginFL \DIFdelFL{$m_{\tau\tau \text{q-jet}}$}\DIFdelendFL \DIFaddbeginFL \DIFaddFL{$m_{\tau\tau \text{j}}$}\DIFaddendFL )             & $9$       &             &  $3$              &  & $14$          &  &         \\
 \text{\DIFdelbeginFL \DIFdelFL{Min}\DIFdelendFL \DIFaddbeginFL \DIFaddFL{min}\DIFaddendFL }($\Delta R(\ell,\tau)$)                               & $8$       &  $9$                &  $9$              & $10$     & $15$          &  &         \\
 $\Delta R(\tau,\tau)$                               &   &             &           &  & $2$           & $4$    & $3$             \\
 \DIFdelbeginFL \DIFdelFL{$\Delta R(\ell,\text{b-jet})$                       }\DIFdelendFL \DIFaddbeginFL \DIFaddFL{$\Delta R(\ell,\text{$b$-jet})$                       }\DIFaddendFL & $2$       &  $3$                &  $2$              & $8$      & $12$          &  &         \\
 \DIFdelbeginFL \DIFdelFL{$\Delta R(\tau1,\text{b-jet})$                       }\DIFdelendFL \DIFaddbeginFL \DIFaddFL{$\Delta R(\tau1,\text{$b$-jet})$                       }\DIFaddendFL & $6$       &  $5$                &  $6$              & $7$      & $11$          &  &        \\
 \DIFdelbeginFL \DIFdelFL{$\Delta R(\ell+\text{b-jet},\tau\tau )$             }\DIFdelendFL \DIFaddbeginFL \DIFaddFL{$\Delta R(\ell+\text{$b$-jet},\tau\tau )$             }\DIFaddendFL &   &             &           &  & $7$           &  &         \\
 $\Delta R(\tau1,\text{light-jet})$                   & $7$       &  $8$                &  $7$              & $5$      & $8$           & $5$    & $5$    \\
 \text{\DIFdelbeginFL \DIFdelFL{Min}\DIFdelendFL \DIFaddbeginFL \DIFaddFL{min}\DIFaddendFL }(\DIFdelbeginFL \DIFdelFL{$m_{\text{light-jet},\text{light-jet}}$}\DIFdelendFL \DIFaddbeginFL \DIFaddFL{$m_{jj}$}\DIFaddendFL ) &   &  $12$               &           & $11$     &       &  &         \\
% $m_{\text{W}}$                                      & $\times$  &  $\times$           &  $\times$         & $9$      & $\times$      & $6$      & $7$\\
% $\chi^{2}$                                          & $\times$  &  $\times$           &  $\times$         & $14$     & $\times$      & $\times$ & $\times$      \\
% \text{max}($\eta_{\tau}$)                           & $4$       &  $\times$           &  $4$              & $\times$ & $10$          & $\times$ & $\times$       \\
% $m^{\text{T}}_{\text{W}}$                           & $10$      &  $\times$           &  $8$              & $\times$ & $13$          & $\times$ & $\times$        \\
% $m_{\tau,\tau}$                                     & $\times$  &  $2$                &  $\times$         & $3$      & $\times$      & $1$      & $1$          \\
% $m_{\text{t},\text{SM}}$                            & $\times$  &  $1$                &  $\times$         & $2$      & $\times$      & $3$      & $4$          \\
% $p_{\text{T},\ell}$                                 & $12$      &  $15$               &  $12$             & $17$     & $\times$      & $\times$ & $\times$        \\
% $m_{\text{t},\text{FCNC}}$                          &  $\times$ &  $\times$           &  $\times$         & $\times$ & $\times$      & $10$     & $6$\\
% $m_{\text{t},\text{SM},\text{vis}}$                 & $3$       &  $\times$           &  $5$              & $\times$ & $4$           & $\times$ & $\times$        \\
% $p_{\text{T},\tau}$                                 & $1$       &  $4$                &  $1$              & $1$      & $5$           & $11$   & $10$           \\
% $\met$                                              & $5$       &  $11$               &  $10$             & $13$     & $6$           & $7$    & $13$          \\
% $m_{\tau\tau,\text{vis}}$                           & $10$      &  $14$               &  $11$             & $6$      & $1$           & $2$    & $2$          \\
% $E_{\text{vis}~\tau,1}/E_{\tau,1}$                  & $\times$  &  $10$               &  $\times$         & $12$     & $\times$      & $8$    & $8$          \\
% $E_{\text{vis}~\tau,2}/E_{\tau,2}$                  & $\times$  &  $7$                &  $\times$         & $4$      & $\times$      & $9$    & $11$         \\
% $P_{\text{T,\tauhadvis}} $                          & $\times$  &  $\times$           &  $\times$         & $\times$ & $9$           & $\times$ & $\times$        \\
% $m_{\text{t},\text{FCNC},\text{vis}}$               & $\times$  &  $\times$           &  $\times$         & $\times$ & $3$           & $\times$ & $\times$       \\
% $\Delta\phi(\tau\tau,\met)$                         & $\times$  &  $6$    			   &  $\times$         & $16$     & $\times$      & $13$   & $12$         \\
% $\met\text{centrality}$                             & $\times$  &  $13$               &  $\times$         & $15$     & $\times$      & $12$   & $9$         \\
% \text{Min}($m_{\tau\tau \text{q-jet}}$)             & $9$       &  $\times$           &  $3$              & $\times$ & $14$          & $\times$ & $\times$        \\
% $\Delta R(\ell,\tau)$                               & $8$       &  $9$                &  $9$              & $10$     & $15$          & $\times$ & $\times$        \\
% $\Delta R(\tau,\tau)$                               & $\times$  &  $\times$           &  $\times$         & $\times$ & $2$           & $4$    & $3$             \\
% $\Delta R(\ell,\text{b-jet})$                       & $2$       &  $3$                &  $2$              & $8$      & $12$          & $\times$ & $\times$        \\
% $\Delta R(\tau,\text{b-jet})$                       & $6$       &  $5$                &  $6$              & $7$      & $11$          & $\times$ & $\times$       \\
% $\Delta R(\ell+\text{b-jet},\tau\tau )$             & $\times$  &  $\times$           &  $\times$         & $\times$ & $7$           & $\times$ & $\times$        \\
% $\Delta R(\tau,\text{light-jet})$                   & $7$       &  $8$                &  $7$              & $5$      & $8$           & $5$    & $5$    \\
% \text{Min}($m_{\text{light-jet},\text{light-jet}}$) & $\times$  &  $12$               &  $\times$         & $11$     & $\times$      & $\times$ & $\times$        \\
 \bottomrule\bottomrule\\
 \end{tabular}
\end{table*}




%%%%%%%%%%%%%%%
%%%  \begin{table*}[t!]
%%%    \caption{\small{Discriminating variables used in the training of the BDT for hadronic channel.
%%%        The values in percent (\%) represent the separation and importance of each variable.
%%%      The descriptions of each variable are provided in the following text.}}
%%%  \label{tab:importance_xTFW}
%%%  \centering
\begin{tabular}{ccc} \toprule\toprule
  & $t_h\thadhad$-2j & $t_h\thadhad$-3j\\\midrule
$m_{\text{W}}$                               & $7.62$ / $6.54$ & $7.00$ / $8.00$\\
$m_{\text{t},\text{SM}}$                            & $10.61$ / $10.55$ & $9.66$ / $9.64$\\
$p_{\text{T},\tau }$                         & $5.70$ / $6.96$ & $5.86$ / $5.13$\\
$E^{\text{T}}_{\text{miss}}$                        & $7.07$ / $5.76$ & $4.34$ / $5.77$\\
$m_{\tau\tau,\text{vis}}$                  & $10.72$ / $10.95$ & $11.26$ / $10.92$\\
$m_{\tau ,\tau }$                     & $14.36$ / $14.63$ & $14.77$ / $13.77$\\
$m_{\text{t},\text{FCNC}}$                          & $5.81$ / $7.01$ & $7.49$ / $8.24$\\
$\Delta R(\tau,\tau)$               & $10.27$ / $9.58$ & $10.00$ / $9.12$\\
$\Delta\phi(\tau\tau,P^{\text{T}}_{\text{miss}})$ & $2.69$ / $5.71$ & $4.51$ / $5.03$\\
$E^{\text{T}}_{\text{miss}} \text{centrality}$             & $3.89$ / $4.68$ & $5.87$ / $5.51$\\
$E_{\text{vis}~\tau ,1}/E_{\tau ,1}$         & $6.56$ / $5.50$ & $6.32$ / $4.74$\\
$E_{\text{vis}~\tau ,2}/E_{\tau ,2}$         & $6.45$ / $5.21$ & $4.72$ / $7.37$\\
$\Delta R(\tau,\text{light~jet},\text{min})$       & $8.26$ / $6.93$ & $8.20$ / $6.75$\\
\bottomrule\bottomrule\\
\end{tabular}

%%%  \end{table*}
%%%  
%%%  \begin{table*}[t!]
%%%    \caption{\small{Discriminating variables used in the training of the BDT for leptonic channel.
%%%        The values in percent (\%) represent the separation and importance of each variable.
%%%      The descriptions of each variable are provided in the following text.}}
%%%  \label{tab:importance_tthML}
%%%  \centering
\begin{tabular}{cccccc} \toprule\toprule
 & $t_{l}\tauhad$-1j & $t_{h}\tlhad$-2j & $t_{l}\tauhad$-2j & $t_{h}\tlhad$-3j & $t_l\thadhad$\\\midrule
 $m_{\text{W}}$ &  / &  / &  / & $5.96$ / $6.84$ &  /\\
$\chi^{2}$ &  / &  / &  / & $5.35$ / $5.08$ &  /\\
\text{max}($\eta_{\tau}$) & $8.98$ / $8.97$ &  / & $9.35$ / $10.04$ &  / & $6.27$ / $6.14$\\
$m^{\text{T}}_{\text{W}}$ & $6.97$ / $6.34$ &  / & $8.39$ / $8.15$ &  / & $4.78$ / $5.84$\\
$m_{\tau,\tau}$ &  / & $7.99$ / $7.85$ &  / & $7.28$ / $7.86$ &  /\\
$m_{\text{t},\text{SM}}$ &  / & $8.20$ / $8.13$ &  / & $7.54$ / $7.24$ &  /\\
$p_{\text{T},\ell}$ & $4.51$ / $5.17$ & $3.16$ / $3.93$ & $4.60$ / $5.62$ & $2.82$ / $3.52$ &  /\\
$m_{\text{t},\text{SM},\text{vis}}$ & $10.36$ / $10.05$ &  / & $9.15$ / $9.10$ &  / & $7.50$ / $7.06$\\
$p_{\text{T},\tau}$ & $12.28$ / $10.93$ & $7.52$ / $7.60$ & $11.63$ / $12.32$ & $7.68$ / $7.96$ & $7.28$ / $8.18$\\
$\met$ & $8.16$ / $6.83$ & $6.36$ / $6.28$ & $6.37$ / $5.72$ & $5.38$ / $4.47$ & $7.27$ / $6.11$\\
$m_{\tau\tau,\text{vis}}$ & $6.40$ / $6.95$ & $5.79$ / $6.65$ & $5.31$ / $4.89$ & $6.18$ / $6.00$ & $10.35$ / $10.09$\\
$E_{\text{vis}~\tau,1}/E_{\tau,1}$ &  / & $6.39$ / $5.90$ &  / & $5.35$ / $5.35$ &  /\\
$E_{\text{vis}~\tau,2}/E_{\tau,2}$ &  / & $6.94$ / $7.40$ &  / & $6.69$ / $6.54$ &  /\\
$P_{\text{T,\tauhadvis}} $ &  / &  / &  / &  / & $6.49$ / $6.36$\\
$m_{\text{t},\text{FCNC},\text{vis}}$ &  / &  / &  / &  / & $8.01$ / $7.43$\\
$\Delta\phi(\tau\tau,\met)$ &  / & $7.02$ / $6.96$ &  / & $4.97$ / $5.58$ &  /\\
$\met\text{centrality}$ &  / & $6.04$ / $5.22$ &  / & $5.13$ / $5.06$ &  /\\
\text{Min}($m_{\tau\tau \text{q-jet}}$) & $7.70$ / $8.20$ &  / & $9.55$ / $9.30$ &  / & $4.65$ / $4.11$\\
$\Delta R(\ell,\tau)$ & $7.75$ / $9.07$ & $6.56$ / $7.50$ & $8.33$ / $8.51$ & $5.73$ / $5.08$ & $4.07$ / $4.59$\\
$\Delta R(\tau,\tau)$ &  / &  / &  / &  / & $8.87$ / $9.27$\\
$\Delta R(\ell,\text{b-jet})$ & $11.00$ / $10.88$ & $7.69$ / $7.18$ & $10.10$ / $9.52$ & $6.10$ / $6.30$ & $5.37$ / $5.85$\\
$\Delta R(\tau,\text{b-jet})$ & $8.06$ / $8.40$ & $7.30$ / $6.84$ & $8.69$ / $8.48$ & $6.12$ / $6.07$ & $5.41$ / $5.65$\\
$\Delta R(\ell+\text{b-jet},\tau\tau )$ &  / &  / &  / &  / & $6.90$ / $6.82$\\
$\Delta R(\tau,\text{light-jet})$ & $7.83$ / $8.20$ & $6.88$ / $6.93$ & $8.51$ / $8.35$ & $6.26$ / $5.76$ & $6.78$ / $6.50$\\
\text{Min}($m_{\text{light-jet},\text{light-jet}}$) &  / & $6.15$ / $5.64$ &  / & $5.47$ / $5.30$ &  /\\
\bottomrule\bottomrule\\
\end{tabular}



%%%  \end{table*}

Boosted decision trees (BDT) \DIFaddbegin \DIFadd{implemented in the TMVA framework}\DIFaddend ~\cite{Hocker:2007ht} are used in each SR to improve the separation between signal and background. 
%The separate training exploits differences in event kinematics across SRs.  
%draft 1 version
In the training, all signal events from $tt(qH)$ and $tH$ are merged together for $tuH$ and $tcH$. All background sources from SM processes
(including both real and fake \DIFdelbegin \DIFdel{$\had$ }\DIFdelend \DIFaddbegin \DIFadd{tau }\DIFaddend contributions) are also used in the training.

A large set of potential variables are investigated in each SR separately. The discrimination of a given variable is quantified by the "separation"(measures the degree of overlap between background and signal shape) and "importance"(depicts the power of the variable to the classification of \DIFdelbegin \DIFdel{two distributions}\DIFdelend \DIFaddbegin \DIFadd{the events}\DIFaddend ) provided by the TMVA package~\cite{Hocker:2007ht}, and only those variables whose importance is \DIFdelbegin \DIFdel{large }\DIFdelend \DIFaddbegin \DIFadd{larger }\DIFaddend than 2\% were kept.
The BDT input variables in each SR and their importance are listed in Table~\ref{tab:importance}. The discriminating variables used are:
\begin{itemize}
\item $\met$ is the missing transverse momentum.
\item \DIFdelbegin \DIFdel{${\pt}_{\tauhadvis} $ }\DIFdelend \DIFaddbegin \DIFadd{${\pt}_{\tau1} $ }\DIFaddend is the transverse momentum of the leading tau candidate.
\item \DIFdelbegin \DIFdel{${\pt}_{\text{sub}-\tau}$ }\DIFdelend \DIFaddbegin \DIFadd{${\pt}_{\tau2}$ }\DIFaddend is the transverse momentum of the \DIFdelbegin \DIFdel{sub-leading }\DIFdelend \DIFaddbegin \DIFadd{subleading }\DIFaddend tau candidate.
\item \DIFdelbegin \DIFdel{${\pt}_{\ell}$ }\DIFdelend \DIFaddbegin \DIFadd{${\pt}^{\ell}$ }\DIFaddend is the transverse momentum of the leading light lepton.
\item $\chi^2$ derived from the kinematic fit of the neutrino momentum.
\item \DIFdelbegin \DIFdel{$m_{t,\text{SM}}$ is the }\DIFdelend \DIFaddbegin \DIFadd{$m_{\text{bjj},\text{fit}}$ is the fitted }\DIFaddend invariant mass of the $b$-jet and the two jets from the $W$ decay, and reflects the top mass in the decay $t\to Wb \to j_1j_2b$. This variable is only defined for the 4-jet $t_hH$ and $t_ht(qH)$ events.
\item \DIFdelbegin \DIFdel{$m^{T}_{\text{W}}$ }\DIFdelend \DIFaddbegin \DIFadd{$m^{W}_{\text{T}}$ }\DIFaddend is the transverse mass calculated from the lepton and $\met$ in the leptonic channels, defined as
\begin{equation}
m\DIFdelbegin \DIFdel{^{T}_{\text{W}} }\DIFdelend \DIFaddbegin \DIFadd{^{W}_{\text{T}} }\DIFaddend = \sqrt{2 {\pt}_{\ell} E_{\text{T}}^{\text{miss}} \left(1-\cos\Delta\phi_{\ell,\text{miss}} \right)},  
\end{equation}
where $\Delta\phi_{\ell,\text{miss}}$ is the azimuth angle between the light-lepton and $\met$.  
\item \DIFdelbegin \DIFdel{$m_{\tau,\tau}$ }\DIFdelend \DIFaddbegin \DIFadd{$m_{\tau\tau,\text{fit}}$ }\DIFaddend is the fitted invariant mass of the tau candidates and reconstructed neutrinos \DIFdelbegin \DIFdel{in }\DIFdelend \DIFaddbegin \DIFadd{for }\DIFaddend the $t_hH$ \DIFdelbegin \DIFdel{, }\DIFdelend \DIFaddbegin \DIFadd{and }\DIFaddend $t_ht(qH)$ \DIFdelbegin \DIFdel{channels}\DIFdelend \DIFaddbegin \DIFadd{events}\DIFaddend . 
\item \DIFdelbegin \DIFdel{$m_{\text{W}}$ }\DIFdelend \DIFaddbegin \DIFadd{$m_{\text{jj}}$ }\DIFaddend is the reconstructed invariant mass of two light-jets from the $W$ decay with the mass closet to the $W$ mass.
\item \DIFdelbegin \DIFdel{$m_{\text{t},\text{FCNC}}$ }\DIFdelend \DIFaddbegin \DIFadd{$m_{\tau\tau\text{q},\text{fit}}$ }\DIFaddend is the fitted invariant mass of the FCNC-decaying top quark reconstructed from di-tau candidates, $q$-jet and reconstructed neutrinos.
\item \DIFdelbegin \DIFdel{$m_{\tau\tau,\text{vis}}$ }\DIFdelend \DIFaddbegin \DIFadd{$m_{\tau\tau}$ }\DIFaddend is the visible invariant mass of the di-tau \DIFdelbegin \DIFdel{candidates}\DIFdelend \DIFaddbegin \DIFadd{system}\DIFaddend . %or the lepton and $\had$ candidate when there is only one $\had$.
\item \DIFdelbegin \DIFdel{${\pt}_{\tau\tau,\text{vis}}$ }\DIFdelend \DIFaddbegin \DIFadd{${\pt}_{\tau\tau}$ }\DIFaddend is the visible $\pT$ of the di-tau \DIFdelbegin \DIFdel{candidates}\DIFdelend \DIFaddbegin \DIFadd{system}\DIFaddend .
\item \DIFdelbegin \DIFdel{$m_{\text{t},\text{FCNC},\text{vis}}$ }\DIFdelend \DIFaddbegin \DIFadd{$m_{\tau\tau,\text{q}}$ }\DIFaddend is the reconstructed visible mass of the FCNC-decaying top quark.
\item \DIFdelbegin \DIFdel{$m_{\text{t},\text{SM},\text{vis}}$ }\DIFdelend \DIFaddbegin \DIFadd{$m_{\text{bjj}}$ }\DIFaddend is the invariant mass of the lepton and the $b$-jet, which reflects the visible top quark mass.
\item \text{\DIFdelbegin \DIFdel{Min}\DIFdelend \DIFaddbegin \DIFadd{min}\DIFaddend }(\DIFdelbegin \DIFdel{$M_{\tau\tau \text{q-jet}}$}\DIFdelend \DIFaddbegin \DIFadd{$m_{\tau\tau \text{j}}$}\DIFaddend ) is the visible mass of the di-tau candidates (include leptonic tau) and the light-flavor jet, minimized by choosing different jet, reflects the invariant masss of the visible FCNC top decaying product, an alternative to variable \DIFdelbegin \DIFdel{$m_{\text{t},\text{FCNC},\text{vis}}$}\DIFdelend \DIFaddbegin \DIFadd{$m_{\tau\tau\text{q}}$}\DIFaddend .
\item \text{\DIFdelbegin \DIFdel{Min}\DIFdelend \DIFaddbegin \DIFadd{min}\DIFaddend }(\DIFdelbegin \DIFdel{$M_{\text{light-jet},\text{light-jet}}$}\DIFdelend \DIFaddbegin \DIFadd{$m_{\text{jj}}$}\DIFaddend ) is the invariant mass of two light-flavor jets, minimized by choosing different jets, reflects the invariant mass of the \DIFdelbegin \DIFdel{W }\DIFdelend \DIFaddbegin \DIFadd{$W$ }\DIFaddend candidate.
  %an alternative of $m_{\text{W}}$.
\item $\met$ centrality is a measure of how central the $\met$ lies between the two tau candidates in the transverse plane, and is defined as
\begin{eqnarray}
\begin{array}{l}
\met~\text{centrality} = {(x+y)}/{\sqrt{x^2+y^2}}, \\
\text{with}~x=\frac{\sin(\phi_{\text{miss}}-\phi_{\tau_1})}{\sin(\phi_{\tau_2}-\phi_{\tau_1})}, \quad  y=\frac{\sin(\phi_{\tau_2}-\phi_{\text{miss}})}{\sin(\phi_{\tau_2}-\phi_{\tau_1})} ,
\end{array}
\label{eq:eq3}
\end{eqnarray}
\item \DIFdelbegin \DIFdel{$E_{\text{vis}~\tau\text{i}}/E_{\tau\text{i}},\text{i}=1,2$ }\DIFdelend \DIFaddbegin \DIFadd{$E_{\tau\text{i}}/E_{\tau\text{i},\text{fit}}$ (\text{i}=1,2) }\DIFaddend is the momentum fraction carried by the visible decay products from the \DIFdelbegin \DIFdel{tau mother}\DIFdelend \DIFaddbegin \DIFadd{leading and subleading tau decays}\DIFaddend . It is based on the best-fit 4-momentum of the neutrino(s) according to the event reconstruction algorithm in this section. For the $\tauhad$ decay mode, the visible decay products carry most of the tau energy since there is only a single neutrino in the final state.% which is evident in the excess around 1 in Figure \ref{fig:x12_fit}. 
%\item $\Delta R(\ell+\text{b-jet},\tau\tau)$ is the angular distance between the lepton+$b$-jet and di-tau candidates.
%\item $\Delta R(\ell,\text{b-jet})$ is the angular distance between the lepton and $b$-jet.
%\item $\Delta R(\tau,\text{b-jet})$ is the angular distance between the tau candidate and $b$-jet. If there are two taus in the event, the leading one is selected for the calculation.
%DIF < \item \text{Max}($\eta_{\tau}$) is the maximum $\eta$ value among the tau candidates.
%DIF > \item \text{max}($\eta_{\tau}$) is the maximum $\eta$ value among the tau candidates.
%\item $\Delta R(\ell,\tau)$ is the angular distance between the lepton and the closest tau candidate in the leptonic channels.
%\item $\Delta R(\tau,\text{q-jet})$ is the angular distance between the tau candidate and the reconstructed $q$-jet. If there are two taus in the event, the leading one is used.
%\item $\Delta R(\tau,\tau)$ is the angular distance between two tau candidates, in case of $t_h\tlhad$ channels, the definition is the same as $\Delta R(\ell,\tau)$.
\item $\Delta\phi(\tau\tau,\met)$ is the azimuthal angle between the $\met$ and di-tau \DIFaddbegin \DIFadd{system }\DIFaddend $\pT$.
\item \DIFdelbegin \DIFdel{$\Delta R(i,j)$ }\DIFdelend \DIFaddbegin \DIFadd{$\Delta R(a,b)$ }\DIFaddend is the angular distance between \text{\DIFdelbegin \DIFdel{i}\DIFdelend \DIFaddbegin \DIFadd{a}\DIFaddend } and \text{\DIFdelbegin \DIFdel{j}\DIFdelend \DIFaddbegin \DIFadd{b}\DIFaddend } objects \DIFaddbegin \DIFadd{in the event}\DIFaddend . 
%\item $\Delta R(\tau1,\text{light-jet})$ is the minimum angle distance between the tau candidate and the light-jet. If there are two taus in the event, the leading one is used.
\end{itemize}
\DIFaddbegin 

\DIFaddend A comparison between data and the predicted background for the leading \DIFdelbegin \DIFdel{$\had$ }\DIFdelend \DIFaddbegin \DIFadd{tau }\DIFaddend $\pt$ distribution in the SRs 
%DIF < some of these variables in three main decay final states($t_l\hadhad$, $t_h\hadhad$, and $t_h\lephad$)
%DIF > some of these variables in three main decay final states($t_{\ell}\hadhad$, $t_h\hadhad$, and $t_h\lephad$)
is shown in Figure~\ref{fig:taupt_prefit}.
%DIF < each of the SRs considered is shown in Figures~\ref{fig:mva_input_hadhad} -~\ref{fig:mva_input_lhadhad}.
%DIF > each of the SRs considered is shown in Figures~\ref{fig:mva_input_hadhad} -~\ref{fig:mva_input_{\ell}hadhad}.
%The comparison between the data and predicted background after preselection for the distributions of two of the most 
%discriminating BDT input variables in the $\hadhad$ channel before the fit to data (``Pre-Fit'').
%The expected signals are also shown after scaled up for the shape comparison of the distributions.
%by a large factor normalized to the total number of events in the predicted
%background for the shape comparison.   
%corresponding to $\BR(t\to Hq)=0.2\%$ is also shown.
The \DIFdelbegin \DIFdel{first and the last bins in the figures contain the underflow and overflow respectively}\DIFdelend \DIFaddbegin \DIFadd{last bin of each plot contains the overflow entries}\DIFaddend .
The bottom panel displays the ratio of data to the background (``Bkg'') prediction as discussed in section~\ref{sec:background_model}\DIFdelbegin \DIFdel{below}\DIFdelend .
The hashed area represents the total uncertainty of the background.
A good description of the data by the background model is observed in all cases.
The final observable used to extract \DIFdelbegin \DIFdel{signal events is defined as }\DIFdelend \DIFaddbegin \DIFadd{the signal contribution is }\DIFaddend the BDT distribution in each SR corresponding to either $tuH$ or $tcH$ signal.
\DIFaddbegin 

\DIFaddend %Figure~\ref{fig:asimov_prefitbdt} show the postfits to the data of BDT observable in each SRs with all background processes constrained to the SM expectation.
%The level of discrimination between signal and background achieved by the BDTs is illustrated in Figure~\ref{fig:overtrain_hadhad}-~\ref{fig:overtrain_lhadhad}.
%%%%%%%%%%%%%%%%%%%%%%%%%%%%%%%%%%%%%%%
%\input{\FCNCFigures/tex/BDTinput}
\begin{figure}[H]
\centering
\begin{tabular}{@{}ccc@{}}
%\includegraphics[page=7,width=0.33\textwidth]{\FCNCFigures/tthML/showFake/faketau/postfit/NOMINAL_fancpaper/reg1l2tau1bnj_os/tau_pt_0.pdf} &
%\includegraphics[page=7,width=0.33\textwidth]{\FCNCFigures/tthML/showFake/faketau/postfit/NOMINAL_fancpaper/reg1l1tau1b1j_ss_vetobtagwp70_highmet/tau_pt_0.pdf}&
%\includegraphics[page=7,width=0.33\textwidth]{\FCNCFigures/tthML/showFake/faketau/postfit/NOMINAL_fancpaper/reg1l1tau1b2j_ss_vetobtagwp70_highmet/tau_pt_0.pdf}\\
\DIFdelbeginFL %DIFDELCMD < \includegraphics[page=1,width=0.33\textwidth]{figures/reg1l2tau1bnj_os.pdf} %%%
\DIFdelendFL \DIFaddbeginFL \includegraphics[page=1,width=0.33\textwidth]{figures/new_pt/reg1l2tau1bnj_os.pdf} \DIFaddendFL &
\DIFdelbeginFL %DIFDELCMD < \includegraphics[page=1,width=0.33\textwidth]{figures/reg1l1tau1b1j_ss.pdf}%%%
\DIFdelendFL \DIFaddbeginFL \includegraphics[page=1,width=0.33\textwidth]{figures/new_pt/reg1l1tau1b1j_ss.pdf}\DIFaddendFL &
\DIFdelbeginFL %DIFDELCMD < \includegraphics[page=1,width=0.33\textwidth]{figures/reg1l1tau1b2j_ss.pdf}%%%
\DIFdelendFL \DIFaddbeginFL \includegraphics[page=1,width=0.33\textwidth]{figures/new_pt/reg1l1tau1b2j_ss.pdf}\DIFaddendFL \\
(a1) \DIFdelbeginFL %DIFDELCMD < \pT%%%
\DIFdelFL{(}%DIFDELCMD < \tauhad%%%
\DIFdelFL{) in $t_l\thadhad$ }\DIFdelendFL & (a2) \DIFdelbeginFL %DIFDELCMD < \pT%%%
\DIFdelFL{(}%DIFDELCMD < \tauhad%%%
\DIFdelFL{) in  $t_l\tauhad$-1j}\DIFdelendFL & (a3) \DIFdelbeginFL %DIFDELCMD < \pT%%%
\DIFdelFL{(}%DIFDELCMD < \tauhad%%%
\DIFdelFL{) in $t_l\tauhad$-2j}\DIFdelendFL \\
\DIFdelbeginFL %DIFDELCMD < \includegraphics[page=1,width=0.33\textwidth]{figures/reg1l1tau1b2j_os.pdf}%%%
\DIFdelendFL %DIF > (a1) \pT(\tauhad) in $t_{\ell}\thadhad$ & (a2) \pT(\tauhad) in  $t_{\ell}\tauhad$-1j& (a3) \pT(\tauhad) in $t_{\ell}\tauhad$-2j\\
\DIFaddbeginFL \includegraphics[page=1,width=0.33\textwidth]{figures/new_pt/reg1l1tau1b2j_os.pdf}\DIFaddendFL &
\DIFdelbeginFL %DIFDELCMD < \includegraphics[page=1,width=0.33\textwidth]{figures/reg1l1tau1b3j_os.pdf}%%%
\DIFdelendFL \DIFaddbeginFL \includegraphics[page=1,width=0.33\textwidth]{figures/new_pt/reg1l1tau1b3j_os.pdf}\DIFaddendFL &
\DIFdelbeginFL %DIFDELCMD < \includegraphics[page=1,width=0.33\textwidth]{figures/reg2mtau1b2jos_vetobtagwp70_highmet.pdf}%%%
\DIFdelendFL \DIFaddbeginFL \includegraphics[page=1,width=0.33\textwidth]{figures/new_pt/reg2mtau1b2jos_vetobtagwp70_highmet.pdf}\DIFaddendFL \\
(b1) \DIFdelbeginFL %DIFDELCMD < \pT%%%
\DIFdelFL{(}%DIFDELCMD < \tauhad%%%
\DIFdelFL{) in $t_h\tlhad$-2j }\DIFdelendFL & (b2) \DIFdelbeginFL %DIFDELCMD < \pT%%%
\DIFdelFL{(}%DIFDELCMD < \tauhad%%%
\DIFdelFL{) in  $t_h\tlhad$-3j }\DIFdelendFL & (b3) \DIFdelbeginFL %DIFDELCMD < \pT%%%
\DIFdelFL{(}%DIFDELCMD < \tauhad%%%
\DIFdelFL{) in $t_h\thadhad$-2j }\DIFdelendFL \\
\DIFdelbeginFL %DIFDELCMD < \includegraphics[page=1,width=0.33\textwidth]{figures/reg2mtau1b3jos_vetobtagwp70_highmet.pdf}%%%
\DIFdelendFL %DIF > (b1) \pT(\tauhad) in $t_h\tlhad$-2j & (b2) \pT(\tauhad) in  $t_h\tlhad$-3j & (b3) \pT(\tauhad) in $t_h\thadhad$-2j \\
\DIFaddbeginFL \includegraphics[page=1,width=0.33\textwidth]{figures/new_pt/reg2mtau1b3jos_vetobtagwp70_highmet.pdf}\DIFaddendFL &\\
(c1) \DIFdelbeginFL %DIFDELCMD < \pT%%%
\DIFdelFL{(}%DIFDELCMD < \tauhad%%%
\DIFdelFL{) in $t_h\thadhad$-3j}\DIFdelendFL \\
%DIF > (c1) \pT(\tauhad) in $t_h\thadhad$-3j\\
\end{tabular}
\caption{The leading $\thad$ $p_T$  distributions are compared between the expected background and $tuH$ signals in: \DIFdelbeginFL \DIFdelFL{$t_l\thadhad$ }\DIFdelendFL \DIFaddbeginFL \DIFaddFL{$t_{\ell}\thadhad$ }\DIFaddendFL (a1),  \DIFdelbeginFL \DIFdelFL{$t_l\tauhad$}\DIFdelendFL \DIFaddbeginFL \DIFaddFL{$t_{\ell}\tauhad$}\DIFaddendFL -1j (a2),  \DIFdelbeginFL \DIFdelFL{$t_l\tauhad$}\DIFdelendFL \DIFaddbeginFL \DIFaddFL{$t_{\ell}\tauhad$}\DIFaddendFL -2j (a3), $t_h\tlhad$-2j (b1), $t_h\tlhad$-3j (b2), $t_h\thadhad$-2j (b3), and $t_h\thadhad$-3j (c1) before the fit to data ('Pre-Fit'). The uncertainty band includes both the statistical and systematic uncertainties in the background prediction. Overflow bins are included respectively in the last bin. Others (Rare) includes single top, and $V$+jets and other small backgrounds in the leptonic (hadronic) 
channels. 
%( only subtau real refers to $t\bar{t}$ MC with leading tau faked by other object, subleading tau from real contribution. 
The lower panels show the ratios of the data to the background prediction.}
\label{fig:taupt_prefit}
\end{figure}




% %\input{\FCNCFigures/tex/BDT}
% \begin{figure}[H]
% \centering
% \begin{tabular}{@{}ccc@{}}
% \includegraphics[page=7,width=0.33\textwidth]{\FCNCFigures/tthML/showFake/faketau/postfit/NOMINAL_fancpaper/reg1l2tau1bnj_os/BDTG_test.pdf} &
% \includegraphics[page=7,width=0.33\textwidth]{\FCNCFigures/tthML/showFake/faketau/postfit/NOMINAL_fancpaper/reg1l1tau1b1j_ss_vetobtagwp70_highmet/BDTG_test.pdf}&
% \includegraphics[page=7,width=0.33\textwidth]{\FCNCFigures/tthML/showFake/faketau/postfit/NOMINAL_fancpaper/reg1l1tau1b2j_ss_vetobtagwp70_highmet/BDTG_test.pdf}\\
%DIF <  (a1) BDT discriminant in $t_l\thadhad$ & (a2) BDT discriminant in  $t_l\tauhad$-1j& (a3) BDT discriminant in $t_l\tauhad$-2j\\
%DIF >  (a1) BDT discriminant in $t_{\ell}\thadhad$ & (a2) BDT discriminant in  $t_{\ell}\tauhad$-1j& (a3) BDT discriminant in $t_{\ell}\tauhad$-2j\\
% \includegraphics[page=7,width=0.33\textwidth]{\FCNCFigures/tthML/showFake/faketau/postfit/NOMINAL_fancpaper/reg1l1tau1b2j_os_vetobtagwp70_highmet/BDTG_test.pdf} &
% \includegraphics[page=7,width=0.33\textwidth]{\FCNCFigures/tthML/showFake/faketau/postfit/NOMINAL_fancpaper/reg1l1tau1b3j_os_vetobtagwp70_highmet/BDTG_test.pdf} &
% \includegraphics[page=7,width=0.33\textwidth]{\FCNCFigures/xTFW/showFake/NOMINAL/reg2mtau1b2jos_vetobtagwp70_highmet/BDTG_test.pdf} \\
% (b1) BDT discriminant in $t_h\tlhad$-2j & (b2) BDT discriminant in  $t_h\tlhad$-3j & (b3) BDT discriminant in $t_h\thadhad$-2j \\
% \includegraphics[page=7,width=0.33\textwidth]{\FCNCFigures/xTFW/showFake/NOMINAL/reg2mtau1b3jos_vetobtagwp70_highmet/BDTG_test.pdf} & \\
% (c1) BDT discriminant in$t_h\thadhad$-3j\\
% \end{tabular}
%DIF <  \caption{ The BDT output distributions are compared between the expected background and $tqH$ signals in: $t_l\thadhad$ (a1),  $t_l\tauhad$-1j (a2),  $t_l\tauhad$-2j (a3),$t_h\tlhad$-2j (b1), $t_h\tlhad$-3j (b2), $t_h\thadhad$-2j (b3), and $t_h\thadhad$-3j (c1). Only statistical uncertainties are being shown (?). Underflow and overflow bins are included
%DIF >  \caption{ The BDT output distributions are compared between the expected background and $tqH$ signals in: $t_{\ell}\thadhad$ (a1),  $t_{\ell}\tauhad$-1j (a2),  $t_{\ell}\tauhad$-2j (a3),$t_h\tlhad$-2j (b1), $t_h\tlhad$-3j (b2), $t_h\thadhad$-2j (b3), and $t_h\thadhad$-3j (c1). Only statistical uncertainties are being shown (?). Underflow and overflow bins are included
% respectively in the first and last bins. Empty data bins here are always blinded based on our strategy. The real tau contributions shown from $t\bar{t}$ and other MC including % diboson, single top, and V+jets.}
% \label{fig:asimov_prefitbdt}
% \end{figure}



\DIFdelbegin %DIFDELCMD < \FloatBarrier
%DIFDELCMD < 

%DIFDELCMD < %%%
%DIF <  Background estimation
%DIF < -------------------------------------------------------------------------------
\section{\DIFdel{Background estimation}}
%DIFAUXCMD
\addtocounter{section}{-1}%DIFAUXCMD
%DIFDELCMD < \label{sec:background_model}
%DIFDELCMD < %%%
%DIF < -------------------------------------------------------------------------------
%DIFDELCMD < 

%DIFDELCMD < %%%
\DIFdel{Most background processes are modelled using Monte Carlo (MC) simulation.
After the event preselection, the main background is $\ttbar$ production, often in association with jets, denoted by $\ttbar$+jets in the following.
Small contributions arise from single-top-quark, $W/Z$+jets, multi-jet and diboson ($WW,WZ,ZZ$) production, as well as from the associated 
production of a vector boson $V$ ($V=W,Z$) or a Higgs boson and a $\ttbar$ pair ($\ttbar V$ and $\ttbar H$). All backgrounds 
with prompt leptons, i.e.\ those originating from the decay of a $W$ boson, a $Z$ boson, or a $\tau$-lepton,
are estimated using samples of simulated events and initially normalised to their theoretical cross sections.
In the simulation, the top-quark and SM Higgs boson masses are set to $172.5~\gev$ and $125~\gev$, respectively,
and the Higgs boson is forced to decay into $H\to \tau\tau$ with branching ratio calculated using }\textsc{\DIFdel{Hdecay}}%DIFAUXCMD
\DIFdel{~\mbox{%DIFAUXCMD
\cite{Djouadi:1997yw}}\hspace{0pt}%DIFAUXCMD
.  
Backgrounds with non-prompt light leptons (electron or muon), with photons or jets misidentified as electrons, or with jets misidentified as $\had$ candidates, 
generically referred to as }\texttt{\DIFdel{fake}} %DIFAUXCMD
\DIFdel{leptons, are estimated using data-driven methods. 
The background prediction is further improved during the statistical analysis by performing a likelihood 
fit to data using several signal-depleted analysis regions, as discussed in Sections~\ref{sec:strategy_Htautau}.
}%DIFDELCMD < 

%DIFDELCMD < %%%
%DIF < -------------------------------------------------------------------------------
%DIF < \subsection{Backgrounds with fake leptons}
%DIF < \label{sec:fakeleptons}
%DIF < -------------------------------------------------------------------------------
\subsection{\DIFdel{Backgrounds with fake $\had$ leptons}}
%DIFAUXCMD
\addtocounter{subsection}{-1}%DIFAUXCMD
%DIFDELCMD < \label{sec:faketaus}
%DIFDELCMD < %%%
\DIFdel{The background with one or more fake $\had$ candidates mainly arises from $\ttbar$ or
multi-jet production, depending on the search channels.
Studies based on simulation show that, for all the above processes, fake $\had$ candidates primarily result from the
misidentification of light jets and $b$-quark jets.
It is also found that the fake rate decreases for all jet flavours as the $\had$ candidate $\pt$ increases.
}%DIFDELCMD < 

%DIFDELCMD < %%%
\DIFdel{In the leptonic channels, the events with prompt electron or muon and fake taus are modelled by calibrating MC with scale factors (SF)
derived from the dedicated $t\bar t$
control regions ($CR_{tt}$) using dileptonic decays of $t\bar t$ events and semileptonic decays of $t\bar t$ events with two
$b$-jets, summarized in Table~\ref{tab:srcr}. 	
%DIF < the SM $t\bar t$ decay of dilepton events and semileptonically double-btagged lepton-jet events, summarized in Table~\ref{tab:srcr},
%DIF < aimed at fake taus from different origins.
There are four kinds of fake taus that need to be calibrated: Type-1 fake taus from $W$-jets ($\tau_{W}$)
with opposite-sign (OS) of charge to the light lepton;
Type-2 $\tau_{W}$'s with same-sign (SS) charge to the light lepton; Type-3 fake taus originating from $b$-hadron decays; Type-4 fake taus from light-flavour hadron decays.
The control regions are defined similar to the signal regions but with an additional $b$-jet or light lepton.
%DIF < $t_lt_l1b\thad$, $t_lt_l2b\thad$, $t_lt_h2b\thad$-2jSS, $t_lt_h2b\thad$-2jOS, $t_lt_h2b\thad$-3jSS, and $t_lt_h2b\thad$-3jOS.
The dilepton regions ($t_lt_l1b\thad$ and $t_lt_l2b\thad$) are used to calibrate Type-3 and Type-4 fake taus. The semileptonic
regions ($t_lt_h2b\thad$-2jOS and $t_lt_h2b\thad$-3jOS) where $\thad$ and light lepton have opposite charge are used to calibrate Type-1 fake taus.
Similarly for Type-2, the semileptonic regions ($t_lt_h2b\thad$-2jSS and $t_lt_h2b\thad$-3jSS) where $\thad$ and light lepton have same charge are used.
A simultaneous fit to data is made to derive the scale factors for fake taus in MC, which consist of total of 24 parameters
depending on four types, three $\pt$ bins, two bins for 1- and 3-prong taus separately.
These scale factors are then used to correct the MC estimated fakes in the corresponding  signal regions with a single $b$-jet.
In the $t_l\thadhad$ channel, both taus can be misidentified, so the calibration is applied to each tau separately, following the same procedure used in the $\tlhad$ channel.
%DIF < using the lepton and fake tau charges, then the scale factors are multiplied together.
The nominal value of scale factors will vary according to their uncertainties in the final fit.
%DIF < In the control regions with single lepton, $\met > 20$GeV and at least 2 light jets and 2 b-tagged jets are required to ensure that QCD contribution is negligible.
%DIF < The calibration is done depending on different source of the fake taus and \pt. The calibration factor are derived in the dedicated
%DIF < control regions discussed in Section~\ref{sec:strategy_Htautau}.
}%DIFDELCMD < 

%DIFDELCMD < %%%
\DIFdel{In the hadronic channels, the contribution of fakes is estimated partially from data by defining control regions (CR)
enriched in misidentified
$\had$ candidates via loosened $\had$ requirements with proper fake factors and partially from MC~\mbox{%DIFAUXCMD
\cite{ATLAS-CONF-2021-044}}\hspace{0pt}%DIFAUXCMD
.
These CRs do not overlap with the main signal regions, discussed in Section~\ref{sec:strategy_Htautau}.
The CR selection requirements are analogous to those used to define signal regions, except
that the subleading $\had$ candidate is required to fail the medium $\had$ identification (Loose).
The contribution of fakes with subleading $\had$ candidate can be calculated by rescaling the templates of Loose taus in the CR
with proper fake factors (FF).
The templates are produced by subtracting all MC background contributions with real sub-leading taus from data.
The FFs are computed as
the ratio of the Data-MC ($\mathrm{real~tau}$) yields passing to failing the medium tau ID in the $W$+jets control region. We have compared
FF obtained from the $W$+jets and the same-sign $\thadhad$ control regions and the differences are treated as a systematic uncertainty.
}%DIFDELCMD < 

%DIFDELCMD < %%%
\subsection{\DIFdel{Background with fake light leptons}}
%DIFAUXCMD
\addtocounter{subsection}{-1}%DIFAUXCMD
\DIFdel{The background originating from non-prompt or misidentified light leptons is primarily from the QCD multi-jet production.
%DIF < Electrons and muons only appear in the leptonic channels where exactly one electron or muon and at least one $\had$ are required.
%DIF < Since the rate of $\had$ candidate is much larger than for electron or muon, in the majority of the fake lepton events, the $\had$ candidate is misidentified and
%DIF < the electron or muon is prompt based on MC studies.
%DIF < However the multi-jet background has large event rate and can contribute with both electron or muon and $\had$ faked, especially to $t_l\had$ where the fake-lepton dominates. and the jet multiplicity is low.
The contribution from these events is estimated with a data-driven method called ABCD.
The estimate is based on the number of tight and loose light-leptons that passed and failed the PLIV cut in the low and high $\met$ regions. The number of
fake light leptons in the signal region is evaluated by scaling the number of loose leptons in the high $\met$ region by the ratio of tight and loose leptons in the
low $\met$ region. The contributions of prompt leptons and calibrated $\had$ fakes in the number of tight and loose leptons are subtracted using MC simulation.
A closure test is made for the background estimations in the signal depleted low BDT regions. The data are in good agreement with the background prediction in the
$t_l\had$ channels where the fake light leptons are important while negligible in other leptonic channels. 
}%DIFDELCMD < 

%DIFDELCMD < %%%
\DIFdelend % Systematic uncertainties
%-------------------------------------------------------------------------------
\section{Systematic uncertainties}
\label{sec:systematics}
%-------------------------------------------------------------------------------

Several sources of systematic uncertainty that can affect the normalisation of signal 
and background and/or the shape of their corresponding discriminant distributions are considered.
Each source is considered to be uncorrelated with the other sources.  
Correlations of a given systematic uncertainty are maintained across processes and channels 
as appropriate. The following sections describe the systematic uncertainties considered.
\DIFaddbegin \DIFadd{Table~\ref{tab:had_sys_impact} shows a summary of the dominant systematics uncertainties on the measured $\BR(H\to \tautau)$ branching ratio
resulting from the fits to the data in the signal regions as described in Section~\ref{sec:result}.
}\DIFaddend 

\subsection{Luminosity}
\label{sec:syst_lumi}

The uncertainty in the integrated luminosity is 1.7\%, affecting the overall normalisation of all processes estimated from the simulation. 
It is derived, following a methodology similar to that detailed in Ref.~\cite{Aaboud:2016hhf}, and using the LUCID-2 detector 
for the baseline luminosity measurements \cite{Avoni:2018iuv}, from a calibration of the luminosity scale using $x$--$y$ beam-separation scans.

\subsection{Reconstructed objects}
\label{sec:syst_objects}

Uncertainties associated with electrons, muons, and $\had$ candidates arise from the trigger, reconstruction,  
identification and isolation (in the case of electrons and muons) efficiencies, as well as the momentum scale and resolution. 
These are measured using $Z\to \ell^+\ell^-$ and $J/\psi\to \ell^+\ell^-$ events ($\ell =e, \mu$)~\cite{ATLAS-CONF-2016-024,Aad:2016jkr} 
in the case of electrons and muons, and using $Z\to \tau^+\tau^-$ events in the case of $\had$ candidates~\cite{ATLAS-CONF-2017-029}.

Uncertainties associated with jets arise from the jet energy scale
and resolution, and the efficiency to pass the JVT requirements. 
The largest contribution results from the jet energy scale, whose uncertainty dependence on jet $\pt$ and $\eta$, jet flavour, and pile-up treatment, 
is split into 43 uncorrelated components that are treated independently~\cite{Aaboud:2017jcu}. The total JES uncertainty is
below 5\% for most jets and below 1\% for central jets with $\pt$ between 300 GeV and 2 TeV. The difference between the JER
in data and MC is represented by one NP. It is applied on the MC by smearing the jet $\pt$ within the prescribed uncertainty.

Uncertainties associated with energy scales and resolutions of leptons and jets 
are propagated to $\met$. Additional uncertainties originating from the modelling 
of the underlying event, in particular its impact on the $\pt$ scale and resolution 
of unclustered energy, are negligible.

Efficiencies to tag $b$-jets and $c$-jets in the simulation are corrected to match the efficiencies in data by $\pt$-dependent factors,
whereas the light-jet efficiency is scaled by $\pt$- and $\eta$-dependent factors.
The $b$-jet efficiency is measured in a data sample enriched in $\ttbar$ events~\cite{Aad:2019epj79}, %EPJ C79(2019)970 {Aaboud:2018xwy},
  while the $c$-jet efficiency is measured
using $\ttbar$ events~\cite{ATLAS-CONF-2018-001} or $W$+$c$-jet events~\cite{Aad:2015ydr}. 
The light-jet efficiency is measured in a multi-jet data sample enriched in light-flavour jets~\cite{ATLAS-CONF-2018-006}.
The uncertainties in these scale factors include a total of 44 independent sources affecting $b$-jets, 19 source affecting $c$-jets, and 19 sources affecting light-jets. 
These systematic uncertainties are taken as uncorrelated between $b$-jets, $c$-jets, and light-jets.

The uncertainty on the pileup reweighing is evaluated by varying the pileup scale factors
by 1$\sigma$ based on the reweighing of the average interactions per bunch crossing. However, this
uncertainty is highly correlated with the luminosity uncertainty and may be an overestimate.

\subsection{Background modelling}
\label{sec:syst_bkgmodeling}

A number of sources of systematic uncertainty affecting the modelling of $t\bar{t}$+jets are considered: the choice of the renormalisation and factorisation scale in the matrix-element calculation, the choice of the matching scale when matching the matrix elements to the parton show generator, the uncertainty in the value of $\alpha_s$ when modeling initial-state radiation (ISR), the choice of the renormalisation scale when modeling final-state radiation (FSR).
%%An uncertainty of  6\% is assigned to the inclusive $\ttbar$ production
%%cross section~\cite{Czakon:2011xx}, including contributions from varying the factorisation and renormalisation 
%%scales, as well as from the top-quark mass, the PDF and $\alpha_{\textrm{S}}$. The latter two represent the largest contribution 
%%to the overall theoretical uncertainty in the cross section and were calculated using the PDF4LHC prescription~\cite{Botje:2011sn} 
%%with the NNPDF3.0(NLO), CT10 NNLO~\cite{Lai:2010vv,Gao:2013xoa} and NNPDF2.3(LO) 5F FFN~\cite{Ball:2012cx} PDF sets.

The hdamp parameter (which controls the amount of radiation produced by the parton shower in
POWHEG-BOX v2) is set to 1.5$m_t$ in the nominal case. Alternative samples are generated with hdamp=3$m_t$. The
difference between two samples is treated as one of the systematics as ``$t\bar{t}$ hdamp''. 
%The uncertainty associated with the choice of NLO generator is derived by comparing the nominal prediction from
%{\powheg}+{\pythiaeight} with a prediction from \textsc{Sherpa}~2.2.1. For the latter, the matrix-element calculation is performed 
%for up to two partons at NLO and up to four partons at LO using \textsc{Comix} and \textsc{OpenLoops}, and
%merged with the {\sherpa} parton shower using the ME+PS@NLO prescription.

The uncertainty due to the choice of parton shower and hadronisation (PS \& Had) model is derived 
by comparing the predictions from {\powheg} interfaced either to {\pythiaeight} or {\herwig7}.
The latter uses the MMHT2014 LO~\cite{Harland-Lang:2014zoa} PDF set in combination with the H7UE tune~\cite{Bellm:2015jjp}.
The uncertainty in the modelling of additional radiation from the PS are assessed by
varying the corresponding parameter of the A14 set~\cite{ATL-PHYS-PUB-2016-004} and by varying the radiation renormalisation and factorisation scales
by a factor of 2.0 and 0.5, respectively. 
%The uncertainty in the modelling of additional radiation is assessed with two alternative {\powheg}+{\pythiaeight} samples:
%a sample with increased radiation (referred to as radHi) is obtained by decreasing the renormalisation and factorisation scales  
%by a factor of two, %doubling the $h_{\textrm{damp}}$ parameter,
%and using the Var3c upward variation of the A14 parameter set;
%a sample with decreased radiation (referred to as radLow) is obtained by increasing the scales by a factor of two  %and using the Var3c downward variation of the A14 set~\cite{ATL-PHYS-PUB-2016-004}.


Another significant background in hadronic channel stems from the $Z\rightarrow \tau\tau$ samples, several sources of uncertainty are
considered for these samples: the PDF variation considering the standard deviation of the 100 NNPDF replicas event weights
of NNPDF3.0nnlo~\cite{Ball:2015NNPDF} PDF
set used in Sherpa, renormalisation ($\mu_{R}$) and factorisation scales ($\mu_{F}$), jet-to-parton matching uncertainty, resummation scale uncertainty,
variation in the choice of $\alpha_{S}$, alternative PDF variation evaluated comparing predictions from NNPDF3.0nnlo PDF set (nominal)
with MMHT2014nnlo68cl and CT14nnlo~\cite{Lai:2010vv,Gao:2013xoa} PDF sets.

Uncertainties affecting the normalisation of the $V$+jets background are estimated separately for $V$+light-jets, $V$+$\geq$1$c$+jets,
and $V$+$\geq$1$b$+jets subprocesses. The total normalisation uncertainty of $V$+jets processes is estimated \DIFaddbegin \DIFadd{to be }\DIFaddend approximately 30\% by comparing the
data and total background prediction in the different analysis regions considered, but requiring exactly zero $b$-jet.
%Agreement between data and predicted background 
%in these modified regions, which are dominated by $V$+light-jets, is found to be within approximately 30\%. This bound is taken to 
%be the normalisation uncertainty, correlated across all $V$+jets subprocesses. 
%Since {\sherpa}~2.2 has been found to underestimate $V$+heavy-flavour production by about a factor
%of 1.3~\cite{Aaboud:2017xsd}, additional 30\% normalisation uncertainties are assumed for $V$+$\geq$1$c$+jets and $V$+$\geq$1$b$+jets
%subprocesses, considered uncorrelated between them.

%To estimate the uncertainty originating from ISR modelling in the single top samples, the same procedure as $t\bar{t}$ is used. One NP reflects the symmetrised effect of the two shower variations Var3cDown and Var3cUp, and the other NP with name ``scale'' describe the symmetrised effect from the independent variations of the renormalisation and factorisation scale. The impact of the uncertainty from FSR modelling is estimated by reweighing the nominal single top sample. The two variations $\mu^{FSR}_{R} \times 0.5$ and $\mu^{FSR}_{R} \times 2$ are considered.  The standard deviation of 100 NNPDF3.0nnlo variations is calculated Uncertainties on the PDF are evaluated following the recommended prescription in Ref.~\cite{ttRun2}.

%The values of the scale-parameters $\mu_{f}$ and $\mu_{f}$ are varied by factors of 2.0 and 0.5 with respect to their default values to obtain the uncertainties related to the renormalization and factorization scales in ttV samples. Uncertainties on the PDF are evaluated following the same procedure in Ref.~\cite{ttZRun2}.

%Following the recommendations of the PMG group, the uncertainties to be considered for diboson background include~\cite{dibosonRes} scale variations of the renormalization and factorization scale,
%PDF variations and $\alpha_{S}$ variations.

%A $+5.8\%-9.2\%$ normalization uncertainty is considered for the ttH background, corresponding to the scale and $\alpha_{S}$ uncertainties in the NLO cross-section computation. A constant PDF uncertainty of $\pm3.6$\% is assumed, as it was done for the previous measurement. Therefore, a overall systematics with variation 9\% is entered in the final fit model~\cite{ttZRun2}.

Uncertainties affecting the modelling of the single-top-quark background include a 
$+5\%$/$-4\%$ uncertainty of the total cross section estimated as a weighted average 
of the theoretical uncertainties in $t$-, $tW$- and $s$-channel production~\cite{Kidonakis:2011wy,Kidonakis:2010ux,Kidonakis:2010tc}.
Additional uncertainties associated with the parton shower, hadronisation and ISR/FSR radiations are also considered using the same procedure
as in $t\bar t$. Uncertainties of the diboson background normalisation include variations of the renormalization and factorization scale, NNPDF3.0nnlo variations and $\alpha_{S}$ variations. Uncertainties of the $\ttbar V$ and $\ttbar H$ cross sections are 12\% and $+5.8\%-9.2\%$, respectively,
from the uncertainties of their respective NLO theoretical cross sections~\cite{ttZRun2}. 

%by comparing the nominal
%%%samples with alternative samples where generator parameters are varied.
%%%For the $t$- and $tW$-channel processes, an uncertainty due to the choice of parton shower and hadronisation model is derived 
%%%by comparing events produced by {\powheg} interfaced to {\pythia}~8 or {\herwigpp}.
%%%These uncertainties are treated as fully correlated among single-top-quark production processes, but uncorrelated with the
%%%corresponding uncertainty of the $\ttbar$+jets background.
%%%An additional systematic uncertainty in $tW$-channel production concerning the separation 
%%%between $t\bar{t}$ and $tW$ at NLO is assessed by comparing
%%%the nominal sample, which uses the diagram removal scheme~\cite{Frixione:2008yi}, with an alternative sample
%%%using the diagram subtraction scheme~\cite{Frixione:2008yi}.

%%%%(it is assumed that two jets originate from the $W/Z$ decay, as in $WW/WZ \to \ell \nu jj$). 
%%%%Therefore, the total normalisation uncertainty is $5\% \oplus \sqrt{N-2}\times 24\%$, where $N$ is the selected jet multiplicity,  
%%%%resulting in 34\%, 42\%, and 48\%, for events with exactly 4 jets, exactly 5 jets, and $\geq$6 jets, respectively. 
%%%%Recent comparisons between data and {\sherpa}~2.1.1 for $WZ(\to \ell\nu\ell\ell) + \geq$4 jets show
%%%%agreement within the experimental uncertainty of approximately 40\%~\cite{Aaboud:2016yus}, which further justifies the above uncertainty.
%%%%Given the very small contribution of this background to the total prediction, the final result is not affected by the assumed modelling
%%%%uncertainties.
%%%%



The statistical uncertainties on the fake \DIFdelbegin \DIFdel{$\had$ }\DIFdelend \DIFaddbegin \DIFadd{tau }\DIFaddend background calibration in the leptonic channel are applied with uncorrelated uncertainties for different fake \DIFdelbegin \DIFdel{$\had$ }\DIFdelend \DIFaddbegin \DIFadd{tau }\DIFaddend sources and
different $\pt$ slices. The uncertainty of the ABCD method is applied with a normalisation factor for muon and electron respectively considering both,
its statistical fluctuation and the differences between signal regions.
The uncertainties in the fake factor method applied in hadronic channel includes statistical uncertainties of each fake factor and different sets of fake factors derived from different signal depleted CRs.

\subsection{Signal modelling}
\label{sec:syst_sigmodeling}

Several normalisation and shape uncertainties are taken into account for the $\Hq$ and $pp\to tH$ signals.
Uncertainties of the Higgs boson branching ratios are taken \DIFdelbegin \DIFdel{into account
following the recommendation }\DIFdelend \DIFaddbegin \DIFadd{from the measured values }\DIFaddend in Ref.~\DIFdelbegin \DIFdel{\mbox{%DIFAUXCMD
\cite{deFlorian:2016spz}}\hspace{0pt}%DIFAUXCMD
.
}\DIFdelend \DIFaddbegin \DIFadd{\mbox{%DIFAUXCMD
\cite{Zyla:2020zbs}}\hspace{0pt}%DIFAUXCMD
.
%DIF > into account following the recommendation in Ref.~\cite{deFlorian:2016spz}.
}\DIFaddend The uncertainty of ISR, FSR, scale and PDF are considered in all SRs. The parton shower uncertainties are estimated by comparing
the nominal to an alternative sample interfaced with {\herwig7}.
%the predictions from {\powheg} interfaced either to {\pythiaeight} or {\herwig7}. 
%treated correlatedly with $\ttbar$ samples. 


\DIFdelbegin %DIFDELCMD < \begin{table}[h!]
%DIFDELCMD < %%%
%DIFDELCMD < \caption{%
{%DIFAUXCMD
\DIFdelFL{List of relative uncertainties on the upper limits of $\BR(t\to qH)$ ($q=u,c$) obtained from the combined fit. The uncertainties are symmetrised for presentation and grouped into the categories described in the text.}}
%DIFAUXCMD
%DIFDELCMD < \label{tab:had_sys_impact}
%DIFDELCMD < \begin{center}
%DIFDELCMD < \begin{tabular}{%
%DIFDELCMD < @{}l%
%DIFDELCMD < S%
%DIFDELCMD < S%
%DIFDELCMD < @{}%
%DIFDELCMD < }
%DIFDELCMD < \toprule
%DIFDELCMD < \multirow{2}{*}{Source of uncertainty}      & \multicolumn{2}{c}{$\Delta\BR/\BR [\%]$} \\
%DIFDELCMD <                                             & \multicolumn{1}{c}{$t\rightarrow uH$} & \multicolumn{1}{c}{$t\rightarrow cH$} \\\midrule
%DIFDELCMD < %%%
\DIFdelFL{Lepton                                  }%DIFDELCMD < & %%%
\DIFdelFL{0.6           }%DIFDELCMD < &%%%
\DIFdelFL{1.0         }%DIFDELCMD < \\
%DIFDELCMD < %%%
\DIFdelFL{Met                                     }%DIFDELCMD < & %%%
\DIFdelFL{0.7           }%DIFDELCMD < &%%%
\DIFdelFL{0.8         }%DIFDELCMD < \\
%DIFDELCMD < %%%
\DIFdelFL{Luminosity and Pileup                   }%DIFDELCMD < & %%%
\DIFdelFL{1.0           }%DIFDELCMD < &%%%
\DIFdelFL{1.3         }%DIFDELCMD < \\
%DIFDELCMD < %%%
\DIFdelFL{Theoretical uncertainty in other MC     }%DIFDELCMD < & %%%
\DIFdelFL{2.1           }%DIFDELCMD < &%%%
\DIFdelFL{2.9         }%DIFDELCMD < \\
%DIFDELCMD < %%%
\DIFdelFL{Jets                                    }%DIFDELCMD < & %%%
\DIFdelFL{2.4           }%DIFDELCMD < &%%%
\DIFdelFL{3.2         }%DIFDELCMD < \\
%DIFDELCMD < %%%
\DIFdelFL{Flavour tagging                         }%DIFDELCMD < & %%%
\DIFdelFL{2.7           }%DIFDELCMD < &%%%
\DIFdelFL{3.7         }%DIFDELCMD < \\
%DIFDELCMD < %%%
\DIFdelFL{Misidentified $\tau$                    }%DIFDELCMD < & %%%
\DIFdelFL{3.4           }%DIFDELCMD < &%%%
\DIFdelFL{4.8         }%DIFDELCMD < \\
%DIFDELCMD < %%%
\DIFdelFL{Theoretical uncertainty in $t\bar{t}$   }%DIFDELCMD < & %%%
\DIFdelFL{2.9           }%DIFDELCMD < &%%%
\DIFdelFL{4.3         }%DIFDELCMD < \\
%DIFDELCMD < %%%
\DIFdelFL{Hadronic $\tau$ decays                  }%DIFDELCMD < & %%%
\DIFdelFL{3.3           }%DIFDELCMD < &%%%
\DIFdelFL{4.4         }%DIFDELCMD < \\
%DIFDELCMD < %%%
\DIFdelFL{Theoretical uncertainty in signal       }%DIFDELCMD < & %%%
\DIFdelFL{5.3           }%DIFDELCMD < &%%%
\DIFdelFL{7.0         }%DIFDELCMD < \\\midrule
%DIFDELCMD < %%%
\DIFdelFL{Total statistics uncertainty            }%DIFDELCMD < & %%%
\DIFdelFL{5.1           }%DIFDELCMD < &%%%
\DIFdelFL{7.0         }%DIFDELCMD < \\
%DIFDELCMD < %%%
\DIFdelFL{Total systematic uncertainty            }%DIFDELCMD < & %%%
\DIFdelFL{11.2          }%DIFDELCMD < &%%%
\DIFdelFL{15.4        }%DIFDELCMD < \\ \midrule
%DIFDELCMD < %%%
\DIFdelFL{Total                                   }%DIFDELCMD < & %%%
\DIFdelFL{12.3          }%DIFDELCMD < &%%%
\DIFdelFL{16.9        }%DIFDELCMD < \\
%DIFDELCMD < \bottomrule
%DIFDELCMD < \end{tabular} 
%DIFDELCMD < \end{center}
%DIFDELCMD < \end{table}
%DIFDELCMD < %%%
\DIFdelend %DIF > \begin{table}[h!]
%DIF > \caption{List of relative uncertainties on the upper limits of $\BR(t\to qH)$ ($q=u,c$) obtained from the combined fit. The uncertainties are symmetrised for presentation and grouped into the categories described in the text.}
%DIF > \label{tab:had_sys_impact}
%DIF > \begin{center}
%DIF > \begin{tabular}{%
%DIF > @{}l%
%DIF > S%
%DIF > S%
%DIF > @{}%
%DIF > }
%DIF > \toprule\toprule
%DIF > \multirow{2}{*}{Source of uncertainty}      & \multicolumn{2}{c}{$\Delta\BR/\BR [\%]$} \\
%DIF >                                             & \multicolumn{1}{c}{$t\rightarrow uH$} & \multicolumn{1}{c}{$t\rightarrow cH$} \\\midrule
%DIF > Lepton                                  & 0.6           &1.0         \\
%DIF > Met                                     & 0.7           &0.8         \\
%DIF > Luminosity and Pileup                   & 1.0           &1.3         \\
%DIF > Theoretical uncertainty in other MC     & 2.1           &2.9         \\
%DIF > Jets                                    & 2.4           &3.2         \\
%DIF > Flavour tagging                         & 2.7           &3.7         \\
%DIF > Misidentified $\tau$                    & 3.4           &4.8         \\
%DIF > Theoretical uncertainty in $t\bar{t}$   & 2.9           &4.3         \\
%DIF > Hadronic $\tau$ decays                  & 3.3           &4.4         \\
%DIF > Theoretical uncertainty in signal       & 5.3           &7.0         \\\midrule
%DIF > Total statistics uncertainty            & 5.1           &7.0         \\
%DIF > Total systematic uncertainty            & 11.2          &15.4        \\ \midrule
%DIF > Total                                   & 12.3          &16.9        \\


%DIF > Lepton ID                               & 0.6           &1.0         \\
%DIF > $\met$                                  & 0.7           &0.8         \\
%DIF > Fake lepton  modeling                   & 0.9           &1.1         \\
%DIF > Luminosity and Pileup                   & 0.9           &1.3         \\
%DIF > Other MC modeling                       & 2.1           &2.9         \\
%DIF > JES and JER                             & 2.4           &3.2         \\
%DIF > Flavour tagging                         & 2.7           &3.7         \\
%DIF > $t\bar{t}$ modeling                     & 2.9           &4.3         \\
%DIF > Fake $\tau$ modeling                    & 3.2           &4.6         \\
%DIF > $\tau$ ID                               & 3.3           &4.4         \\
%DIF > Signal modeling including Br($H\to\tau\tau$)       & 5.3           &7.0         \\\midrule
%DIF > Total systematic uncertainty            & 11.2          &15.5        \\ 
%DIF > Total statistics uncertainty            & 5.1           &7.0         \\\midrule
%DIF > Total                                   & 12.3          &17.0        \\
\DIFaddbegin 



%DIF > tuH:
%DIF > Fake LEP method    0.00919939  ( +0.00919939, -0.00919939 )
%DIF > Fake method    0.0323159  ( +0.0323159, -0.0323159 )
%DIF > FullSyst    0.112111  ( +0.112111, -0.112111 )
%DIF > Gammas    0.0509968  ( +0.0509968, -0.0509968 )
%DIF > Instrument    0.0093934  ( +0.0093934, -0.0093934 )
%DIF > JET    0.0235705  ( +0.0235705, -0.0235705 )
%DIF > Lepton    0.00568277  ( +0.00568277, -0.00568277 )
%DIF > MET    0.00704844  ( +0.00704844, -0.00704844 )
%DIF > SignalTheory    0.0526104  ( +0.0526104, -0.0526104 )
%DIF > TAU    0.032714  ( +0.032714, -0.032714 )
%DIF > Theory    0.0210686  ( +0.0210686, -0.0210686 )
%DIF > btag    0.027376  ( +0.027376, -0.027376 )
%DIF > ttbarTheory    0.0289194  ( +0.0289194, -0.0289194 )

%DIF > tcH:
%DIF > Fake LEP method    0.010914  ( +0.010914, -0.010914 )
%DIF > Fake method    0.0455017  ( +0.0455017, -0.0455017 )
%DIF > FullSyst    0.154686  ( +0.154686, -0.154686 )
%DIF > Gammas    0.0700708  ( +0.0700708, -0.0700708 )
%DIF > Instrument    0.0131647  ( +0.0131647, -0.0131647 )
%DIF > JET    0.0320897  ( +0.0320897, -0.0320897 )
%DIF > Lepton    0.0104552  ( +0.0104552, -0.0104552 )
%DIF > MET    0.00841447  ( +0.00841447, -0.00841447 )
%DIF > SignalTheory    0.0699833  ( +0.0699833, -0.0699833 )
%DIF > TAU    0.0437292  ( +0.0437292, -0.0437292 )
%DIF > Theory    0.029182  ( +0.029182, -0.029182 )
%DIF > btag    0.0367613  ( +0.0367613, -0.0367613 )
%DIF > ttbarTheory    0.0427391  ( +0.0427391, -0.0427391 )

%DIF > \bottomrule\bottomrule
%DIF > \end{tabular} 
%DIF > \end{center}
%DIF > \end{table}



\DIFaddend %% Fake method    0.0476721  ( +0.0476721, -0.0476721 )
%% FullSyst    0.154686  ( +0.154686, -0.154686 )
%% Gammas    0.0700708  ( +0.0700708, -0.0700708 )
%% Instrument    0.0131647  ( +0.0131647, -0.0131647 )
%% JET    0.0320897  ( +0.0320897, -0.0320897 )
%% Lepton    0.0104552  ( +0.0104552, -0.0104552 )
%% MET    0.00841447  ( +0.00841447, -0.00841447 )
%% SignalTheory    0.0699833  ( +0.0699833, -0.0699833 )
%% TAU    0.0437292  ( +0.0437292, -0.0437292 )
%% Theory    0.029182  ( +0.029182, -0.029182 )
%% btag    0.0367613  ( +0.0367613, -0.0367613 )
%% ttbarTheory    0.0427391  ( +0.0427391, -0.0427391 )







%% Fake method   0.0341326  ( +0.0341326, -0.0341326 )
%% FullSyst      0.112111      ( +0.112111, -0.112111 )
%% Gammas        0.0509968       ( +0.0509968, -0.0509968 )
%% Instrument    0.0093934  ( +0.0093934, -0.0093934 )
%% JET           0.0235705  ( +0.0235705, -0.0235705 )
%% Lepton        0.00568277  ( +0.00568277, -0.00568277 )
%% MET           0.00704844  ( +0.00704844, -0.00704844 )
%% SignalTheory  0.0526104  ( +0.0526104, -0.0526104 )
%% TAU           0.032714  ( +0.032714, -0.032714 )
%% Theory        0.0210686  ( +0.0210686, -0.0210686 )
%% btag          0.027376  ( +0.027376, -0.027376 )
%% ttbarTheory   0.0289194  ( +0.0289194, -0.0289194 )



%\begin{table}[htbp]
%\caption{List of relative uncertainties of the signal strength from the combined fit in hadronic channel. The uncertainties are symmetrised for presentation and %grouped into the categories described in the text. %The quadrature sum of the individual uncertainties is not equal to the total uncertainty due to correlations %introduced by the fit
%}
%\small
%\centering
%\begin{tabular}{cc} \toprule\toprule
%$\text{Uncertainty}    $   & $\Delta\mu/\mu[\%]$ \\\midrule
%$\text{Fake modelling} $   & $13.5$ \\
%$\text{Instrument}     $   & $2.6$  \\
%$\text{Jets}           $   & $1.7$   \\
%$\text{Met}            $   & $0.4$   \\
%$\tau                  $   & $7.2$    \\
%$\text{b-tagging}       $  & $0.3$    \\
%$\text{Signal Theory}  $   & $1.5$   \\
%$\text{Other MC theory} $  & $2.7$    \\
%$t\bar{t} \text{theory}$  & $7.6$     \\\midrule
%$\text{Statistics}      $  & $11.2$   \\
%$\text{Total systematic}$  & $24.2$  \\ \midrule
%$\text{Total}  $           & $26.7$ \\
%\bottomrule\bottomrule
%\end{tabular}
%\label{tab:had_sys_impact}
%\end{table} 


%\begin{table}[htbp]
%\caption{List of relative uncertainties of the signal strength from the combined fit in leptonic channel. The uncertainties are symmetrised for presentation and %grouped into  the categories described in the text. %The quadrature sum of the individual uncertainties is not equal to the total uncertainty due to correlations %introduced by the fit
%}
%\small
%\centering
%\begin{tabular}{cc} \toprule\toprule
%$\text{Uncertainty}        $ & $  \Delta\mu/\mu[\%]  $\\\midrule
%$\text{Fake modelling}     $ & $       2.1             $\\
%$\text{Instrument}         $ & $       0.7             $\\
%$\text{Jets}               $ & $       2.8             $\\
%$\text{Met}                $ & $       1.2             $\\
%$\text{Lepton}             $ & $       0.7             $\\
%$\tau                      $ & $       2.3             $\\
%$\text{b-tagging}          $ & $       4.8             $\\
%$\text{Signal Theory}      $ & $       6.9             $\\
%$\text{Other MC theory}    $ & $       2.1             $\\
%$t\bar{t} \text{theory}    $ & $       3.7             $\\\midrule
%$\text{Statistics}         $ & $       5.9             $\\
%$\text{Total systematic}   $ & $       12.9            $\\\midrule
%$\text{Total}              $ & $       14.2            $\\
%\bottomrule\bottomrule
%\end{tabular}
%\label{tab:lep_sys_impact}
%\end{table} 





 

% Statistical analysis
\section{Statistical analysis}
\label{sec:stat_analysis}


The final \DIFdelbegin \DIFdel{BDT discriminant distributions across all analysis regions considered }\DIFdelend \DIFaddbegin \DIFadd{discriminant distributions are the BDT outputs of all the considered analysis regions. They }\DIFaddend are jointly analysed to test for the 
presence of a signal. The statistical analysis uses a binned likelihood function ${\cal L}(\mu,\theta)$ constructed as
a product of Poisson probability terms over all bins considered in the search. This function depends
on the signal-strength parameter $\mu$, defined as a factor multiplying the expected yield of $tH$ and $tt(qH)$ signal events
normalised to a reference branching ratio $\BR_{\mathrm{ref}}(t\to qH)=0.1\%$,
and $\theta$, a set of nuisance parameters that encode the effect of systematic uncertainties on the signal and background expectations. 
Therefore, the expected total number of events in a given bin depends on $\mu$ and $\theta$. 
All nuisance parameters are subject to Gaussian constraints in the likelihood.
For a given value of $\mu$, the nuisance parameters $\theta$ allow variations of the expectations for signal and background
according to the corresponding systematic uncertainties, and their fitted values result in the deviations from
the nominal expectations that globally provide the best fit to the data.
This procedure allows a reduction of the impact of systematic uncertainties on 
the search sensitivity by taking advantage of the highly populated background-dominated bins included in the likelihood fit.
%To verify the improved background prediction, fits under the background-only hypothesis are performed, 
%and differences between the data and the post-fit background prediction are checked 
%using kinematic variables other than the ones used in the fit. 
Statistical uncertainties in each bin of the predicted final discriminant distributions are taken into account by dedicated parameters in the fit.     
The best-fit $\BR(t\to qH)$ is obtained by performing a binned likelihood fit to the data under the signal-plus-background
hypothesis, maximising the likelihood function ${\cal L}(\mu,\theta)$ over $\mu$ and $\theta$.

The fitting procedure was initially validated through extensive studies using pseudo data, defined as the sum of all predicted backgrounds 
plus an injected signal of variable strength, as well as by performing fits to real data where bins of the final discriminant variable with 
a signal contamination above 10\% are excluded (referred to as blinding requirements).
In both cases, the robustness of the model for systematic uncertainties is established by verifying the stability of the fitted background 
when varying assumptions about some of the leading sources of uncertainty. 
After this, the blinding requirements
are removed in the data and a fit under the signal-plus-background hypothesis is performed. Further checks involve the comparison of the fitted 
nuisance parameters before and after removal of the blinding requirements, and their values are found to be consistent. In addition, it is verified that the 
fit is able to correctly determine the strength of a simulated signal injected into the real data.

The test statistic $q_\mu$ is defined as the profile likelihood ratio, 
$q_\mu = -2\ln({\cal L}(\mu,{\hat{\theta}}_\mu)/{\cal L}(\hat{\mu},\hat{\theta}))$,
where $\hat{\mu}$ and $\hat{\theta}$ are the values of the parameters that
maximise the likelihood function (subject to the constraint $0\leq \hat{\mu} \leq \mu$), and ${\hat{\theta}}_\mu$ are the values of the
nuisance parameters that maximise the likelihood function for a given value of $\mu$. 
The test statistic $q_\mu$ is evaluated with the {\textsc RooFit} package~\cite{Verkerke:2003ir,RooFitManual}.
%A related statistic is used to determine whether the observed data is compatible with the background-only hypothesis (the so-called discovery test)  
%by setting $\mu=0$ in the profile likelihood ratio and leaving $\hat{\mu}$ unconstrained: $q_0 = -2\ln({\cal L}(0,{\hat{\theta}}_0)/{\cal L}(\hat{\mu},\hat{\theta}))$.
%The $p$-value (referred to as $p_0$), representing the level of agreement between the data and the background-only hypothesis, is estimated by integrating
%%representing the probability of the data being compatible with the background-only hypothesis is estimated by integrating
%%the distribution of $q_0$ obtained from background-only pseudo-experiments, approximated using the asymptotic formulae given in Refs.~\cite{Cowan:2010js}, 
%the distribution of $q_0$ based on the asymptotic formulae in Ref.~\cite{Cowan:2010js}, 
%above the observed value of $q_0$ in the data. 
%%The observed $p_0$-value is checked for each explored signal scenario.
%%In the case of the data being compatible with the background-only hypothesis, 
%Upper limits on $\mu$, and thus on 
%$\BR(t\to Hq)$, are derived by using $q_\mu$ in the CL$_{\textrm{s}}$ method~\cite{Junk:1999kv,Read:2002hq}.
%For a given signal scenario, values of the $\BR(t\to Hq)$ yielding CL$_{\textrm{s}} < 0.05$, 
%where CL$_{\textrm{s}}$ is computed using the asymptotic approximation~\cite{Cowan:2010js}, are excluded at $\geq 95\%$ CL.

%In the absence of signal,
Exclusion limits are set on $\mu$ , and thus on
$\BR(t\to qH)$, are derived by using $q_\mu$ in the CL$_{\textrm{s}}$ method~\cite{Junk:1999kv,Read:2002hq}.
For a given signal scenario, values of the $\BR(t\to qH)$ yielding CL$_{\textrm{s}} < 0.05$,
where CL$_{\textrm{s}}$ is computed using the asymptotic approximation~\cite{Cowan:2010js}, are excluded at $\geq 95\%$ CL.


% Results
%-------------------------------------------------------------------------------
\section{Results}
\label{sec:result}
%-------------------------------------------------------------------------------

This section presents the results obtained from the individual channels, as well as their combination,
following the statistical analysis discussed in Section~\ref{sec:stat_analysis}.

A binned likelihood fit under the signal-plus-background hypothesis is performed on the BDT discriminant distributions in the seven 
analysis regions considered. The unconstrained parameter of the fit is the signal strength.
%and four independent parameters associated with the normalisation of the fake $\had$ background in each of the analysis regions. 
No significant pulls or constraints are obtained for the fitted nuisance parameters, resulting in a post-fit background prediction in each analysis region that is
very close to the pre-fit prediction, albeit with reduced uncertainties due to the anti-correlations among sources of systematic uncertainty resulting from the fit.
\DIFdelbegin \DIFdel{Figure~\ref{fig:Bonlyfit_data} shows a background only fit to the BDT discriminant distribution in the data.
Figure~\ref{fig:asimov_postfitbdtHc} and~\ref{fig:asimov_postfitbdtHu} }\DIFdelend %DIF > Figure~\ref{fig:Bonlyfit_data} shows a background only fit to the BDT discriminant distribution in the data.
\DIFaddbegin \DIFadd{Figure~\ref{fig:asimov_postfitbdtHu} and~\ref{fig:asimov_postfitbdtHc} }\DIFaddend show a post-fit with signal plus background (S+B) to the data for
the $tcH$ and $tuH$ search separately.
%both pre- and post-fit to data, in the case of the $\Hc$ search.  
%A similar comparison for the leptonic channel is shown in Figure~\ref{fig:tthML_trexPrefit} and \ref{fig:tthML_trexPrefit_1}.
The observed and predicted yields after fit to the data with background only are summarized in Table~\ref{tab:HtautauPostfitYieldsUnblind}.
%pre-fit and post-fit yields can be found in Appendix~\ref{sec:prepostfit_yields_Htautau_appendix}.
A slight excess of data is observed above background with a significance of \DIFdelbegin \DIFdel{2.6 }\DIFdelend \DIFaddbegin \DIFadd{2.3 }\DIFaddend $\sigma$, which is mainly in the most sensitive \DIFdelbegin \DIFdel{$t_l\thadhad$ }\DIFdelend \DIFaddbegin \DIFadd{$t_{\ell}\thadhad$ }\DIFaddend channel in the high BDT score region \DIFdelbegin \DIFdel{.
We have checked the }\DIFdelend \DIFaddbegin \DIFadd{as shown in
Table~\ref{tab:limits_summary}.
The }\DIFaddend kinematic distributions for the observed excess in
the high BDT region \DIFaddbegin \DIFadd{are checked}\DIFaddend . Within the large statistical uncertainty, the observed distributions are compatible with the background shapes, but also with a small signal contribution. \DIFaddbegin \DIFadd{There is no indication that the excess is from a specific data period.
%DIF > The fitted signal strength and its significance from individual channels and their combination are shown in Table~\ref{tab:limits_summary}.
}\DIFaddend %and found there are nothing unusual between data and expectation.
\DIFdelbegin %DIFDELCMD < \begin{figure}[H]
%DIFDELCMD < \centering
%DIFDELCMD < \begin{tabular}{@{}ccc@{}}
%DIFDELCMD < %%%
%DIF < \includegraphics[width=0.33\textwidth]{\FCNCFigures/unblinded/tthML/tcH_reg1l2tau1bnj_os_postFit_BOnly.pdf}&
%DIF < \includegraphics[width=0.33\textwidth]{\FCNCFigures/unblinded/tthML/tcH_reg1l1tau1b1j_ss_postFit_BOnly.pdf}&
%DIF < \includegraphics[width=0.33\textwidth]{\FCNCFigures/unblinded/tthML/tcH_reg1l1tau1b2j_ss_postFit_BOnly.pdf}\\
%DIF < (a1) BDT discriminant in $t_l\thadhad$ & (a2) BDT discriminant in  $t_l\tauhad$-1j& (a3) BDT discriminant in $t_l\tauhad$-2j\\
%DIF < \includegraphics[width=0.33\textwidth]{\FCNCFigures/unblinded/tthML/tcH_reg1l1tau1b2j_os_postFit_BOnly.pdf}&
%DIF < \includegraphics[width=0.33\textwidth]{\FCNCFigures/unblinded/tthML/tcH_reg1l1tau1b3j_os_postFit_BOnly.pdf}&
%DIF < \includegraphics[width=0.33\textwidth]{\FCNCFigures/unblinded/xTFW/tcH_reg2mtau1b2jos_vetobtagwp70_highmet_postFit_BOnly.pdf}\\
%DIF < (b1) BDT discriminant in $t_h\tlhad$-2j & (b2) BDT discriminant in  $t_h\tlhad$-3j & (b3) BDT discriminant in $t_h\thadhad$-2j \\
%DIF < \includegraphics[width=0.33\textwidth]{\FCNCFigures/unblinded/xTFW/tcH_reg2mtau1b3jos_vetobtagwp70_highmet_postFit_BOnly.pdf}& \\
%DIF < (c1) BDT discriminant in$t_h\thadhad$-3j\\
  %DIFDELCMD < \includegraphics[page=9,width=0.33\textwidth]{\FCNCFigures/tthML/showFake/faketau/postfit/NOMINAL/reg1l2tau1bnj_os/BDTG_test.pdf}&
%DIFDELCMD <   \includegraphics[page=9,width=0.33\textwidth]{\FCNCFigures/tthML/showFake/faketau/postfit/NOMINAL/reg1l1tau1b1j_ss_vetobtagwp70_highmet/BDTG_test.pdf}&
%DIFDELCMD <   \includegraphics[page=9,width=0.33\textwidth]{\FCNCFigures/tthML/showFake/faketau/postfit/NOMINAL/reg1l1tau1b2j_ss_vetobtagwp70_highmet/BDTG_test.pdf}\\
%DIFDELCMD < %%%
\DIFdelFL{(a1) BDT in $t_l\thadhad$ }%DIFDELCMD < & %%%
\DIFdelFL{(a2) BDT in  $t_l\tauhad$-1j}%DIFDELCMD < & %%%
\DIFdelFL{(a3) BDT in $t_l\tauhad$-2j}%DIFDELCMD < \\
%DIFDELCMD <   \includegraphics[page=9,width=0.33\textwidth]{\FCNCFigures/tthML/showFake/faketau/postfit/NOMINAL/reg1l1tau1b2j_os_vetobtagwp70_highmet/BDTG_test.pdf}&
%DIFDELCMD <   \includegraphics[page=9,width=0.33\textwidth]{\FCNCFigures/tthML/showFake/faketau/postfit/NOMINAL/reg1l1tau1b3j_os_vetobtagwp70_highmet/BDTG_test.pdf}&
%DIFDELCMD <   \includegraphics[width=0.33\textwidth]{figures/reg2mtau1b2jos_vetobtagwp70_highmet_pre_bonly.pdf}\\
%DIFDELCMD < %%%
\DIFdelFL{(b1) BDT in $t_h\tlhad$-2j }%DIFDELCMD < & %%%
\DIFdelFL{(b2) BDT in $t_h\tlhad$-3j }%DIFDELCMD < & %%%
\DIFdelFL{(b3) BDT in $t_h\thadhad$-2j }%DIFDELCMD < \\
%DIFDELCMD <   \includegraphics[width=0.33\textwidth]{figures/reg2mtau1b3jos_vetobtagwp70_highmet_pre_bonly.pdf}& \\
%DIFDELCMD < %%%
\DIFdelFL{(c1) BDT in$t_h\thadhad$-3j}%DIFDELCMD < \\
%DIFDELCMD < \end{tabular}
%DIFDELCMD < %%%
%DIFDELCMD < \caption{%
{%DIFAUXCMD
\DIFdelFL{The BDT output distributions are fitted with background only to the data: $t_l\thadhad$ (a1),  $t_l\tauhad$-1j (a2),  $t_l\tauhad$-2j (a3),
  $t_h\tlhad$-2j (b1), $t_h\tlhad$-3j (b2), $t_h\thadhad$-2j (b3), and $t_h\thadhad$-3j (c1).
  Statistical and systematic uncertainties are being shown. For different type of signals are also shown for comparing their shapes. (}%DIFDELCMD < {\textbf %%%
\DIFdelFL{to be updated?}%DIFDELCMD < }%%%
\DIFdelFL{)}}
%DIFAUXCMD
%DIFDELCMD < \label{fig:Bonlyfit_data}
%DIFDELCMD < \end{figure}
%DIFDELCMD < %%%
\DIFdelend %DIF >  \begin{figure}[H]
%DIF >  \centering
%DIF >  \begin{tabular}{@{}ccc@{}}
%DIF >  %\includegraphics[width=0.33\textwidth]{\FCNCFigures/unblinded/tthML/tcH_reg1l2tau1bnj_os_postFit_BOnly.pdf}&
%DIF >  %\includegraphics[width=0.33\textwidth]{\FCNCFigures/unblinded/tthML/tcH_reg1l1tau1b1j_ss_postFit_BOnly.pdf}&
%DIF >  %\includegraphics[width=0.33\textwidth]{\FCNCFigures/unblinded/tthML/tcH_reg1l1tau1b2j_ss_postFit_BOnly.pdf}\\
%DIF >  %(a1) BDT discriminant in $t_{\ell}\thadhad$ & (a2) BDT discriminant in  $t_{\ell}\tauhad$-1j& (a3) BDT discriminant in $t_{\ell}\tauhad$-2j\\
%DIF >  %\includegraphics[width=0.33\textwidth]{\FCNCFigures/unblinded/tthML/tcH_reg1l1tau1b2j_os_postFit_BOnly.pdf}&
%DIF >  %\includegraphics[width=0.33\textwidth]{\FCNCFigures/unblinded/tthML/tcH_reg1l1tau1b3j_os_postFit_BOnly.pdf}&
%DIF >  %\includegraphics[width=0.33\textwidth]{\FCNCFigures/unblinded/xTFW/tcH_reg2mtau1b2jos_vetobtagwp70_highmet_postFit_BOnly.pdf}\\
%DIF >  %(b1) BDT discriminant in $t_h\tlhad$-2j & (b2) BDT discriminant in  $t_h\tlhad$-3j & (b3) BDT discriminant in $t_h\thadhad$-2j \\
%DIF >  %\includegraphics[width=0.33\textwidth]{\FCNCFigures/unblinded/xTFW/tcH_reg2mtau1b3jos_vetobtagwp70_highmet_postFit_BOnly.pdf}& \\
%DIF >  %(c1) BDT discriminant in$t_h\thadhad$-3j\\
%DIF >    \includegraphics[page=9,width=0.33\textwidth]{\FCNCFigures/tthML/showFake/faketau/postfit/NOMINAL/reg1l2tau1bnj_os/BDTG_test.pdf}&
%DIF >    \includegraphics[page=9,width=0.33\textwidth]{\FCNCFigures/tthML/showFake/faketau/postfit/NOMINAL/reg1l1tau1b1j_ss_vetobtagwp70_highmet/BDTG_test.pdf}&
%DIF >    \includegraphics[page=9,width=0.33\textwidth]{\FCNCFigures/tthML/showFake/faketau/postfit/NOMINAL/reg1l1tau1b2j_ss_vetobtagwp70_highmet/BDTG_test.pdf}\\
%DIF >  (a1) BDT in $t_{\ell}\thadhad$ & (a2) BDT in  $t_{\ell}\tauhad$-1j& (a3) BDT in $t_{\ell}\tauhad$-2j\\
%DIF >    \includegraphics[page=9,width=0.33\textwidth]{\FCNCFigures/tthML/showFake/faketau/postfit/NOMINAL/reg1l1tau1b2j_os_vetobtagwp70_highmet/BDTG_test.pdf}&
%DIF >    \includegraphics[page=9,width=0.33\textwidth]{\FCNCFigures/tthML/showFake/faketau/postfit/NOMINAL/reg1l1tau1b3j_os_vetobtagwp70_highmet/BDTG_test.pdf}&
%DIF >    \includegraphics[width=0.33\textwidth]{figures/reg2mtau1b2jos_vetobtagwp70_highmet_pre_bonly.pdf}\\
%DIF >  (b1) BDT in $t_h\tlhad$-2j & (b2) BDT in $t_h\tlhad$-3j & (b3) BDT in $t_h\thadhad$-2j \\
%DIF >    \includegraphics[width=0.33\textwidth]{figures/reg2mtau1b3jos_vetobtagwp70_highmet_pre_bonly.pdf}& \\
%DIF >  (c1) BDT in$t_h\thadhad$-3j\\
%DIF >  \end{tabular}
%DIF >  \caption{ The BDT output distributions are fitted with background only to the data: $t_{\ell}\thadhad$ (a1),  $t_{\ell}\tauhad$-1j (a2),  $t_{\ell}\tauhad$-2j (a3),
%DIF >    $t_h\tlhad$-2j (b1), $t_h\tlhad$-3j (b2), $t_h\thadhad$-2j (b3), and $t_h\thadhad$-3j (c1).
%DIF >    Statistical and systematic uncertainties are being shown. For different type of signals are also shown for comparing their shapes. ({\textbf to be updated?})}
%DIF >  \label{fig:Bonlyfit_data}
%DIF >  \end{figure}


%\input{\FCNCFigures/tex/tthML_trexPrefit}
%\input{\FCNCFigures/tex/xTFW_trexPrefit}
\begin{figure}[H]
\centering
%\begin{tabular}{@{}ccc@{}}
%\includegraphics[width=0.33\textwidth]{\FCNCFigures/tthML/Limit/tcH_reg1l2tau1bnj_os_postFit.pdf}&
%\includegraphics[width=0.33\textwidth]{\FCNCFigures/tthML/Limit/tcH_reg1l1tau1b1j_ss_postFit.pdf}&
%\includegraphics[width=0.33\textwidth]{\FCNCFigures/tthML/Limit/tcH_reg1l1tau1b2j_ss_postFit.pdf}\\
%DIF < (a1) BDT discriminant in $t_l\thadhad$ & (a2) BDT discriminant in  $t_l\tauhad$-1j& (a3) BDT discriminant in $t_l\tauhad$-2j\\
%DIF > (a1) BDT discriminant in $t_{\ell}\thadhad$ & (a2) BDT discriminant in  $t_{\ell}\tauhad$-1j& (a3) BDT discriminant in $t_{\ell}\tauhad$-2j\\
%\includegraphics[width=0.33\textwidth]{\FCNCFigures/tthML/Limit/tcH_reg1l1tau1b2j_os_postFit.pdf}&
%\includegraphics[width=0.33\textwidth]{\FCNCFigures/tthML/Limit/tcH_reg1l1tau1b3j_os_postFit.pdf}&
%\includegraphics[width=0.33\textwidth]{\FCNCFigures/xTFW/Limit/tcH_reg2mtau1b2jos_vetobtagwp70_highmet_postFit.pdf}\\
%(b1) BDT discriminant in $t_h\tlhad$-2j & (b2) BDT discriminant in  $t_h\tlhad$-3j & (b3) BDT discriminant in $t_h\thadhad$-2j \\
%\includegraphics[width=0.33\textwidth]{\FCNCFigures/xTFW/Limit/tcH_reg2mtau1b3jos_vetobtagwp70_highmet_postFit.pdf}& \\
%(c1) BDT discriminant in$t_h\thadhad$-3j\\
%\end{tabular}
%DIF < \caption{ The BDT output distributions are fitted to the asimov S+B data in $tHc$ search: $t_l\thadhad$ (a1),  $t_l\tauhad$-1j (a2),  $t_l\tauhad$-2j (a3),
%DIF > \caption{ The BDT output distributions are fitted to the asimov S+B data in $tHc$ search: $t_{\ell}\thadhad$ (a1),  $t_{\ell}\tauhad$-1j (a2),  $t_{\ell}\tauhad$-2j (a3),
%  $t_h\tlhad$-2j (b1), $t_h\tlhad$-3j (b2), $t_h\thadhad$-2j (b3), and $t_h\thadhad$-3j (c1). Statistical and systematic uncertainties are being shown.}
%\label{fig:asimov_postfitbdtHc}

\begin{tabular}{@{}ccc@{}}
\DIFdelbeginFL %DIFDELCMD < \includegraphics[width=0.33\textwidth]{\FCNCFigures/unblinded/tthML/tcH_reg1l2tau1bnj_os_postFit.pdf}%%%
\DIFdelendFL \DIFaddbeginFL \includegraphics[width=0.3\textwidth]{figures/tuH_reg1l2tau1bnj_os.pdf}\DIFaddendFL &
\DIFdelbeginFL %DIFDELCMD < \includegraphics[width=0.33\textwidth]{\FCNCFigures/unblinded/tthML/tcH_reg1l1tau1b1j_ss_postFit.pdf}%%%
\DIFdelendFL \DIFaddbeginFL \includegraphics[width=0.3\textwidth]{figures/tuH_reg1l1tau1b1j_ss.pdf}\DIFaddendFL &
\DIFdelbeginFL %DIFDELCMD < \includegraphics[width=0.33\textwidth]{\FCNCFigures/unblinded/tthML/tcH_reg1l1tau1b2j_ss_postFit.pdf}%%%
\DIFdelendFL \DIFaddbeginFL \includegraphics[width=0.3\textwidth]{figures/tuH_reg1l1tau1b2j_ss.pdf}\DIFaddendFL \\
%DIF > (a1) BDT in $t_{\ell}\thadhad$ & (a2) BDT in  $t_{\ell}\tauhad$-1j& (a3) BDT in $t_{\ell}\tauhad$-2j\\
(a1)  \DIFdelbeginFL \DIFdelFL{BDT in $t_l\thadhad$ }\DIFdelendFL & (a2) \DIFdelbeginFL \DIFdelFL{BDT in  $t_l\tauhad$-1j}\DIFdelendFL & (a3) \DIFdelbeginFL \DIFdelFL{BDT in $t_l\tauhad$-2j}\DIFdelendFL \\
\DIFdelbeginFL %DIFDELCMD < \includegraphics[width=0.33\textwidth]{\FCNCFigures/unblinded/tthML/tcH_reg1l1tau1b2j_os_postFit.pdf}%%%
\DIFdelendFL \DIFaddbeginFL \includegraphics[width=0.3\textwidth]{figures/tuH_reg1l1tau1b2j_os.pdf}\DIFaddendFL &
\DIFdelbeginFL %DIFDELCMD < \includegraphics[width=0.33\textwidth]{\FCNCFigures/unblinded/tthML/tcH_reg1l1tau1b3j_os_postFit.pdf}%%%
\DIFdelendFL \DIFaddbeginFL \includegraphics[width=0.3\textwidth]{figures/tuH_reg1l1tau1b3j_os.pdf}\DIFaddendFL &
\DIFdelbeginFL %DIFDELCMD < \includegraphics[width=0.33\textwidth]{\FCNCFigures/unblinded/xTFW/tcH_reg2mtau1b2jos_vetobtagwp70_highmet_postFit.pdf}%%%
\DIFdelendFL \DIFaddbeginFL \includegraphics[width=0.3\textwidth]{figures/tuH_reg2mtau1b2jos.pdf}\DIFaddendFL \\
%DIF > (b1) BDT in $t_h\tlhad$-2j & (b2) BDT in  $t_h\tlhad$-3j & (b3) BDT in $t_h\thadhad$-2j \\
(b1) \DIFdelbeginFL \DIFdelFL{BDT in $t_h\tlhad$-2j }\DIFdelendFL & (b2)  \DIFdelbeginFL \DIFdelFL{BDT in  $t_h\tlhad$-3j }\DIFdelendFL & (b3)  \DIFdelbeginFL \DIFdelFL{BDT in $t_h\thadhad$-2j }\DIFdelendFL \\
\DIFdelbeginFL %DIFDELCMD < \includegraphics[width=0.33\textwidth]{\FCNCFigures/unblinded/xTFW/tcH_reg2mtau1b3jos_vetobtagwp70_highmet_postFit.pdf}%%%
\DIFdelendFL \DIFaddbeginFL \includegraphics[width=0.3\textwidth]{figures/tuH_reg2mtau1b3jos.pdf}\DIFaddendFL & \\
%DIF > (c1) BDT in$t_h\thadhad$-3j\\
(c1) \DIFdelbeginFL \DIFdelFL{BDT in$t_h\thadhad$-3j}\DIFdelendFL \\
\end{tabular}
\caption{ The BDT output distributions are fitted with S+B to the data in \DIFdelbeginFL \DIFdelFL{$tcH$ }\DIFdelendFL \DIFaddbeginFL \DIFaddFL{$tuH$ }\DIFaddendFL search: \DIFdelbeginFL \DIFdelFL{$t_l\thadhad$ }\DIFdelendFL \DIFaddbeginFL \DIFaddFL{$t_{\ell}\thadhad$ }\DIFaddendFL (a1),  \DIFdelbeginFL \DIFdelFL{$t_l\tauhad$}\DIFdelendFL \DIFaddbeginFL \DIFaddFL{$t_{\ell}\tauhad$}\DIFaddendFL -1j (a2),  \DIFdelbeginFL \DIFdelFL{$t_l\tauhad$}\DIFdelendFL \DIFaddbeginFL \DIFaddFL{$t_{\ell}\tauhad$}\DIFaddendFL -2j (a3),
  $t_h\tlhad$-2j (b1), $t_h\tlhad$-3j (b2), $t_h\thadhad$-2j (b3), and $t_h\thadhad$-3j (c1). Statistical and systematic uncertainties are shown. \DIFaddbeginFL \DIFaddFL{The signal shapes
  from $tt(uH)$, $tH$, and their sum are also compared using a normalisation of $\BR(t\to uH)=0.2\%$. 
}\DIFaddendFL }
\DIFdelbeginFL %DIFDELCMD < \label{fig:asimov_postfitbdtHc}
%DIFDELCMD < %%%
\DIFdelendFL \DIFaddbeginFL \label{fig:asimov_postfitbdtHu}
\DIFaddendFL \end{figure}

\begin{figure}[H]
\centering
\begin{tabular}{@{}ccc@{}}
%\includegraphics[width=0.33\textwidth]{\FCNCFigures/tthML/Limit/tuH_reg1l2tau1bnj_os_postFit.pdf}&
%\includegraphics[width=0.33\textwidth]{\FCNCFigures/tthML/Limit/tuH_reg1l1tau1b1j_ss_postFit.pdf}&
%\includegraphics[width=0.33\textwidth]{\FCNCFigures/tthML/Limit/tuH_reg1l1tau1b2j_ss_postFit.pdf}\\
%DIF < (a1) BDT discriminant in $t_l\thadhad$ & (a2) BDT discriminant in  $t_l\tauhad$-1j& (a3) BDT discriminant in $t_l\tauhad$-2j\\
%DIF > (a1) BDT discriminant in $t_{\ell}\thadhad$ & (a2) BDT discriminant in  $t_{\ell}\tauhad$-1j& (a3) BDT discriminant in $t_{\ell}\tauhad$-2j\\
%\includegraphics[width=0.33\textwidth]{\FCNCFigures/tthML/Limit/tuH_reg1l1tau1b2j_os_postFit.pdf}&
%\includegraphics[width=0.33\textwidth]{\FCNCFigures/tthML/Limit/tuH_reg1l1tau1b3j_os_postFit.pdf}&
%\includegraphics[width=0.33\textwidth]{\FCNCFigures/xTFW/Limit/tuH_reg2mtau1b2jos_vetobtagwp70_highmet_postFit.pdf}\\
%(b1) BDT discriminant in $t_h\tlhad$-2j & (b2) BDT discriminant in  $t_h\tlhad$-3j & (b3) BDT discriminant in $t_h\thadhad$-2j \\
%\includegraphics[width=0.33\textwidth]{\FCNCFigures/xTFW/Limit/tuH_reg2mtau1b3jos_vetobtagwp70_highmet_postFit.pdf}&\\
%(c1) BDT discriminant in$t_h\thadhad$-3j\\
%\end{tabular}
%DIF < \caption{ The BDT output distributions are fitted to the asimov data in $tHu$ search: $t_l\thadhad$ (a1),  $t_l\tauhad$-1j (a2),  $t_l\tauhad$-2j (a3),
%DIF > \caption{ The BDT output distributions are fitted to the asimov data in $tHu$ search: $t_{\ell}\thadhad$ (a1),  $t_{\ell}\tauhad$-1j (a2),  $t_{\ell}\tauhad$-2j (a3),
%  $t_h\tlhad$-2j (b1), $t_h\tlhad$-3j (b2), $t_h\thadhad$-2j (b3), and $t_h\thadhad$-3j (c1). Statistical and systematic uncertainties are being shown.}
%\label{fig:asimov_postfitbdtHu}

\DIFdelbeginFL %DIFDELCMD < \includegraphics[width=0.33\textwidth]{\FCNCFigures/unblinded/tthML/tuH_reg1l2tau1bnj_os_postFit.pdf}%%%
\DIFdelendFL \DIFaddbeginFL \includegraphics[width=0.3\textwidth]{figures/tcH_reg1l2tau1bnj_os.pdf}\DIFaddendFL &
\DIFdelbeginFL %DIFDELCMD < \includegraphics[width=0.33\textwidth]{\FCNCFigures/unblinded/tthML/tuH_reg1l1tau1b1j_ss_postFit.pdf}%%%
\DIFdelendFL \DIFaddbeginFL \includegraphics[width=0.3\textwidth]{figures/tcH_reg1l1tau1b1j_ss.pdf}\DIFaddendFL &
\DIFdelbeginFL %DIFDELCMD < \includegraphics[width=0.33\textwidth]{\FCNCFigures/unblinded/tthML/tuH_reg1l1tau1b2j_ss_postFit.pdf}%%%
\DIFdelendFL \DIFaddbeginFL \includegraphics[width=0.3\textwidth]{figures/tcH_reg1l1tau1b2j_ss.pdf}\DIFaddendFL \\
%DIF > (a1) BDT in $t_{\ell}\thadhad$ & (a2) BDT in  $t_{\ell}\tauhad$-1j& (a3) BDT in $t_{\ell}\tauhad$-2j\\
(a1)  \DIFdelbeginFL \DIFdelFL{BDT in $t_l\thadhad$ }\DIFdelendFL & (a2) \DIFdelbeginFL \DIFdelFL{BDT in  $t_l\tauhad$-1j}\DIFdelendFL & (a3) \DIFdelbeginFL \DIFdelFL{BDT in $t_l\tauhad$-2j}\DIFdelendFL \\
\DIFdelbeginFL %DIFDELCMD < \includegraphics[width=0.33\textwidth]{\FCNCFigures/unblinded/tthML/tuH_reg1l1tau1b2j_os_postFit.pdf}%%%
\DIFdelendFL \DIFaddbeginFL \includegraphics[width=0.3\textwidth]{figures/tcH_reg1l1tau1b2j_os.pdf}\DIFaddendFL &
\DIFdelbeginFL %DIFDELCMD < \includegraphics[width=0.33\textwidth]{\FCNCFigures/unblinded/tthML/tuH_reg1l1tau1b3j_os_postFit.pdf}%%%
\DIFdelendFL \DIFaddbeginFL \includegraphics[width=0.3\textwidth]{figures/tcH_reg1l1tau1b3j_os.pdf}\DIFaddendFL &
\DIFdelbeginFL %DIFDELCMD < \includegraphics[width=0.33\textwidth]{\FCNCFigures/unblinded/xTFW/tuH_reg2mtau1b2jos_vetobtagwp70_highmet_postFit.pdf}%%%
\DIFdelendFL \DIFaddbeginFL \includegraphics[width=0.3\textwidth]{figures/tcH_reg2mtau1b2jos.pdf}\DIFaddendFL \\
%DIF > (b1) BDT in $t_h\tlhad$-2j & (b2) BDT in  $t_h\tlhad$-3j & (b3) BDT in $t_h\thadhad$-2j \\
(b1) \DIFdelbeginFL \DIFdelFL{BDT in $t_h\tlhad$-2j }\DIFdelendFL & (b2)  \DIFdelbeginFL \DIFdelFL{BDT in  $t_h\tlhad$-3j }\DIFdelendFL & (b3)  \DIFdelbeginFL \DIFdelFL{BDT in $t_h\thadhad$-2j }\DIFdelendFL \\
\DIFdelbeginFL %DIFDELCMD < \includegraphics[width=0.33\textwidth]{\FCNCFigures/unblinded/xTFW/tuH_reg2mtau1b3jos_vetobtagwp70_highmet_postFit.pdf}%%%
\DIFdelendFL \DIFaddbeginFL \includegraphics[width=0.3\textwidth]{figures/tcH_reg2mtau1b3jos.pdf}\DIFaddendFL & \\
%DIF > (c1) BDT in$t_h\thadhad$-3j\\
(c1) \DIFdelbeginFL \DIFdelFL{BDT in$t_h\thadhad$-3j}\DIFdelendFL \\
\end{tabular}
\caption{ The BDT output distributions are fitted with S+B to the data in \DIFdelbeginFL \DIFdelFL{$tuH$ }\DIFdelendFL \DIFaddbeginFL \DIFaddFL{$tcH$ }\DIFaddendFL search: \DIFdelbeginFL \DIFdelFL{$t_l\thadhad$ }\DIFdelendFL \DIFaddbeginFL \DIFaddFL{$t_{\ell}\thadhad$ }\DIFaddendFL (a1),  \DIFdelbeginFL \DIFdelFL{$t_l\tauhad$}\DIFdelendFL \DIFaddbeginFL \DIFaddFL{$t_{\ell}\tauhad$}\DIFaddendFL -1j (a2),  \DIFdelbeginFL \DIFdelFL{$t_l\tauhad$}\DIFdelendFL \DIFaddbeginFL \DIFaddFL{$t_{\ell}\tauhad$}\DIFaddendFL -2j (a3),
  $t_h\tlhad$-2j (b1), $t_h\tlhad$-3j (b2), $t_h\thadhad$-2j (b3), and $t_h\thadhad$-3j (c1). Statistical and systematic uncertainties are shown.
  \DIFaddbeginFL \DIFaddFL{The signal shapes from $tt(cH)$, $tH$, and their sum are also compared using a normalisation of $\BR(t\to cH)=0.2\%$.
}\DIFaddendFL }
\DIFdelbeginFL %DIFDELCMD < \label{fig:asimov_postfitbdtHu}
%DIFDELCMD < %%%
\DIFdelendFL \DIFaddbeginFL \label{fig:asimov_postfitbdtHc}
\DIFaddendFL \end{figure}

\begin{table}[htbp]
\caption{
  Predicted and observed yields in each of the analysis regions considered. The background prediction is shown after a background only fit to data.
  Also shown are the signal expectations for $\Hc$ and
  $\Hu$ assuming $\BR(t\to cH)=0.1\%$ and $\BR(t\to uH)=0.1\%$ respectively. The contributions with real $\had$ candidates from $\ttbar$ and  $Z\to \ell^+\ell^-$ ($\ell = e, \mu$),
  diboson, $\ttbar V$, $\ttbar H$, single-top-quark, and other small backgrounds are combined into a single background source referred to as ``Other MC'' in the leptonic channels ,
  whereas single-top-quark and the small contributions are combined into ``Rare'' in the hadronic channels.
  The quoted uncertainties are the sum in quadrature of statistical and systematic uncertainties of the yields.\DIFdelbeginFL \DIFdelFL{(}%DIFDELCMD < {\textbf %%%
\DIFdelFL{to be updated?}%DIFDELCMD < }%%%
\DIFdelFL{)}\DIFdelendFL }
\small
\centering

\begin{tabular}{cccccc} \toprule\toprule
 & $t_l\tauhad$-1j & $t_l\tauhad$-2j & $t_h\tlhad$-3j &$t_h\tlhad$-2j  & $t_l\thadhad$ \\\midrule
  Double Fake            & \DIFdelbeginFL \DIFdelFL{$0 (0)       $  }\DIFdelendFL \DIFaddbeginFL \DIFaddFL{$--       $  }\DIFaddendFL & \DIFdelbeginFL \DIFdelFL{$0 (0)       $  }\DIFdelendFL \DIFaddbeginFL \DIFaddFL{$--       $  }\DIFaddendFL & \DIFdelbeginFL \DIFdelFL{$0 (0)       $  }\DIFdelendFL \DIFaddbeginFL \DIFaddFL{$--       $  }\DIFaddendFL &  \DIFdelbeginFL \DIFdelFL{$0 (0)       $  }\DIFdelendFL \DIFaddbeginFL \DIFaddFL{$--       $  }\DIFaddendFL & $73 \pm 24    $ \\ 
  $\bar{t}tV$            & $9.3 \pm 1.2 $  & $22.6 \pm 2.8$  & $23.5 \pm 3.0$  &  $13.7 \pm 1.7$  & $2.57 \pm 0.35$ \\ 
  SM Higgs               & $5.8 \pm 0.8 $  & $13.7 \pm 1.7$  & $32.8 \pm 3.5$  &  $13.5 \pm 2.5$  & $16.7 \pm 1.9 $ \\ 
  Diboson                & $32.6 \pm 3.4$  & $19.9 \pm 2.1$  & $36 \pm 4    $  &  $46 \pm 5    $  & $13.2 \pm 1.4 $ \\ 
  Other MC               & $35.6 \pm 3.1$  & $15.9 \pm 1.7$  & $226 \pm 21  $  &  $620 \pm 40  $  & $6.7 \pm 0.6  $  \\ 
  $Z\rightarrow\tau\tau$ & $0 \pm 6     $  & $9.1 \pm 2.2 $  & $500 \pm 60  $  &  $880 \pm 90  $  & $2.1 \pm 0.7  $ \\ 
  Lep Fake               & $212 \pm 30  $  & $80 \pm 10   $  & $292 \pm 26  $  &  $490 \pm 70  $  & $0.9 \pm 0.4  $ \\ 
  QCD Fake               & $670 \pm 200 $  & $310 \pm 90  $  & $180 \pm 70  $  &  $330 \pm 110 $  & \DIFdelbeginFL \DIFdelFL{$0 (0)        $  }\DIFdelendFL \DIFaddbeginFL \DIFaddFL{$--        $  }\DIFaddendFL \\ 
  b Fake                 & $960 \pm 140 $  & $1250 \pm 230$  & $710 \pm 140 $  &  $710 \pm 130 $  & $82 \pm 13    $ \\ 
  W-jet Fake             & $970 \pm 200 $  & $1090 \pm 240$  & $3300 \pm 500$  &  $3800 \pm 600$  & $5.5 \pm 1.8  $ \\ 
  Other Fake             & $3020 \pm 260$  & $2470 \pm 160$  & $1420 \pm 220$  &  $1320 \pm 320$  & $129 \pm 14   $ \\ 
  $\bar{t}t$             & $281 \pm 14  $  & $195 \pm 24  $  & $7100 \pm 400$  &  \DIFdelbeginFL \DIFdelFL{$11800 \pm 50$0 }\DIFdelendFL \DIFaddbeginFL \DIFaddFL{$11800 \pm 500$ }\DIFaddendFL & $7.7 \pm 2.7  $ \\ 
  Total background       & $6200 \pm 170$  & $5480 \pm 100$  & \DIFdelbeginFL \DIFdelFL{$13820 \pm 14$0 }\DIFdelendFL \DIFaddbeginFL \DIFaddFL{$13820 \pm 140$ }\DIFaddendFL &  \DIFdelbeginFL \DIFdelFL{$20000 \pm 17$0 }\DIFdelendFL \DIFaddbeginFL \DIFaddFL{$20000 \pm 170$ }\DIFaddendFL & $339 \pm 27   $ \\  \midrule
  tcH                    & $30 \pm 5    $  & $27 \pm 4    $  & $51 \pm 8    $  &  $34 \pm 6    $  & $36 \pm 5     $ \\
  tuH                    & $36 \pm 8    $  & $32 \pm 5    $  & $63 \pm 10   $  &  $45 \pm 7    $  & $48 \pm 7     $ \\ \midrule
  Data                   & $6353        $  & $5410        $  & $13804       $  &  $20000       $  & $351          $ \\ 
\bottomrule\bottomrule
\end{tabular}\\




\begin{tabular}{ccc} \toprule\toprule
& $t_{h}\thadhad$-2j & $t_{h}\thadhad$-3j\\\midrule
  $t\bar{t}V$              & $0.7 \pm 0.4 $ & $5.5 \pm 1.0 $  \\
  Diboson                  & $8.4 \pm 1.6 $ & $10.8 \pm 1.5$  \\
  Rare                     & $17.9 \pm 3.1$ & $10.2 \pm 2.6$  \\ 
  SM Higgs                 & $17.4 \pm 2.5$ & $25.9 \pm 3.1$  \\ 
  only $\tau_{sub}$ real   & $56 \pm 30   $ & $80 \pm 50   $  \\  
  $t\bar{t}$               & $221 \pm 28  $ & $220 \pm 40  $  \\
  Fake $\tau$              & $220 \pm 70  $ & $270 \pm 70  $  \\  
  $Z\rightarrow\tau\tau$   & $490 \pm 50  $ & $420 \pm 50  $  \\ 
  Total background         & $1040 \pm 35 $ & $1040 \pm 40 $  \\ \midrule
  tcH                      & \DIFdelbeginFL \DIFdelFL{$15.6 \pm 2.4$ }\DIFdelendFL \DIFaddbeginFL \DIFaddFL{$15.6 \pm 2.5$ }\DIFaddendFL & $42 \pm 8    $  \\ 
  tuH                      & $23 \pm 4    $ & $52 \pm 10   $  \\ \midrule
  Data                     & $1033       $& $1052 $       \\
\bottomrule\bottomrule
\end{tabular}
\label{tab:HtautauPostfitYieldsUnblind}
\end{table}




%Comparison between the data and prediction for the BDT discriminant distribution in the
%$\lephad$ channel, before and after the fit to data  (``Pre-Fit'' and ``Post-Fit'', respectively) under the signal-plus-background hypothesis.
%Shown are the ($\lephad$, 3j) region (a) pre-fit and (c) post-fit, and the ($\lephad$, $\geq$4j) region (b) pre-fit and (d) post-fit.
%The contributions with real $\had$ candidates from $\ttbar$,  $\ttbar V$, $\ttbar H$, and single-top-quark backgrounds are combined into
%a single background source referred to as ``Top (real $\had$)'', whereas the small contributions from 
%$Z\to \ell^+\ell^-$ ($\ell = e, \mu$) and diboson backgrounds are combined into ``Other''. 
%In the pre-fit figures the expected $\Hc$ signal (solid red) corresponding to $\BR(t\to Hc)=1\%$ is also shown,
%added to the background prediction. In the post-fit figures, the $\Hc$ signal is normalised using the best-fit branching ratio, 
%$\BR(t\to Hc)=(-4.4^{+9.9}_{-8.5})\times 10^{-4}$.
%The bottom panels display the ratios of data to either the SM background prediction before the fit (``Bkg'')  or the total signal-plus-background
%prediction after the fit (``Pred''). 
%The hashed area represents the total uncertainty of the background. 
%In the case of the pre-fit background uncertainty, the normalisation uncertainty of the fake $\had$ background is not included.
%The results are given in terms of exclusion limits despite a small excess of data events above the background expectation is found.
Upper limits are derived using the CL$_{\textrm{s}}$ method~\cite{Junk:1999kv,Read:2002hq}, and  
observed (expected) 95\% CL limits are set on $\BR(t\to cH)$ and $\BR(t\to uH)$:
$\BR(t\to cH)<9.9 \times 10^{-4}\,(5.0 \times 10^{-4})$, assuming $\BR(t\to uH)=0$,and $\BR(t\to uH)<7.2 \times 10^{-4}\,(3.6 \times 10^{-4})$, assuming $\BR(t\to cH)=0$.
These results are dominated by the leptonic channels, which has a sensitivity a factor of two better than that of the hadronic channels.
The expected sensitivity has been improved upon the previous ATLAS result based on 36 fb$^{-1}$ of data using $H\to \tau\tau$ decay~\cite{fcnc36} by a factor of~5. A factor of~2 improvement in sensitivity comes from the larger dataset, and a further factor of~2.5 comes from including
additional leptonic channels, $tH$ production, and improved the analysis techniques.

%The best-fit branching ratio obtained is $\BR(t\to Hc)=[xxx^{+yy}_{-yy}\,(\mathrm{stat})^{+zz}_{-zz}\,(\mathrm{syst})] \times 10^{-4}$, assuming $\BR(t\to Hu)=0$. 
%The best-fit normalisation factors for the fake $\had$ background are: $0.82 \pm 0.23$ in the ($\lephad$, 3j) region, $0.84^{+0.25}_{-0.28}$ in the ($\lephad$, $\geq$4j) region,
%$0.94^{+0.18}_{-0.17}$ in the ($\hadhad$, 3j) region, and $0.90 \pm 0.26$ in the ($\hadhad$, $\geq$4j) region.
%A similar fit is performed for the $tuH$ search, yielding $\BR(t\to Hu)=[xxx^{+yy}_{-yy}\,(\mathrm{stat})^{+zz}_{-zz}\,(\mathrm{syst})] \times 10^{-4}$,
%assuming $\BR(t\to Hc)=0$.
%The obtained normalisation factors for the fake $\had$ background agree within 1\% with those obtained by the $\Hc$ search.
In both cases, the results are dominated by the statistical uncertainty.
The main contributions to the total systematic uncertainty arise from the uncertainties on the measured $\BR(H\to \tautau)$ branching ratio,
affecting the normalization and factorization scales, $b$-tagging, the choice of parton shower and hadronization for $t\bar t$ modelling, and the fake $\had$ background estimation in the hadronic channels. Their relative impacts on the signal strength are summarized in \DIFdelbegin \DIFdel{Tables}\DIFdelend \DIFaddbegin \DIFadd{Table}\DIFaddend ~\ref{tab:had_sys_impact}.
%the fake $\had$ background estimation in the hadronic channels and the uncertainty associated
%with the different responses to quark-initiated and gluon-initiated jets. 
%o significant excess of data events above the background expectation is found, 
%nd observed (expected) 95\% CL limits are set on $\BR(t\to Hc)$ and $\BR(t\to Hu)$:
%\BR(t\to Hc)<xxx \times 10^{-3}\,(yyy \times 10^{-3})$ and $\BR(t\to Hu)<xxx \times 10^{-3}\,(yyy \times 10^{-3})$.
%hese results are dominated by the leptonic channels, which has a sensitivity a factor of two better than that of the hadronic channels.
%\begin{figure*}[t!]
%\begin{center}
%\includegraphics[width=0.7\textwidth]{\FCNCFigures/tcH_combined_Limit.pdf}
%\caption{\small {Summary of the best-fit $\BR(t\to Hc)$ for the individual channels as well as their combination,
%assuming $\BR(t\to Hu)=0$. (TBD: updated with best fit plots.)}}
%\label{fig:summary_printnum_hc} 
%\end{center}
%\end{figure*}
%%%%%%%%%%%%%%
%%%%%%%%%%%%%%
%\begin{figure*}[h!]
%\begin{center}
%\includegraphics[width=0.7\textwidth]{\FCNCFigures/tuH_combined_Limit.pdf}
%\caption{\small {Summary of the best-fit $\BR(t\to Hu)$ for the individual channels as well as their combination,
%assuming $\BR(t\to Hc)=0$. (TBD: updated with best fit plots.)}}
%\label{fig:summary_printnum_hu} 
%\end{center}
%\end{figure*}
%%%%%%%%%%%%%%
%The first set of combined results is obtained for each branching ratio separately, setting the other branching ratio to zero.
%The best-fit combined branching ratios are $\BR(t\to Hc)=[3.0^{+3.0}_{-2.7}\,(\mathrm{stat})^{+2.6}_{-2.1}\,(\mathrm{syst})] \times 10^{-4}$ and 
%$\BR(t\to Hu)=[4.2^{+3.2}_{-2.9}\,(\mathrm{stat})^{+2.6}_{-2.1}\,(\mathrm{syst})] \times 10^{-4}$.  
%%The difference between the central values of $\BR(t\to Hc)$ and $\BR(t\to Hu)$ originates from the ability of the $H \to b\bar{b}$ search to 
%%probe both decay modes separately.
%A comparison of the best-fit branching ratios for the individual searches and their combination is shown in Figure~\ref{fig:summary_printnum_hc} 
%for $\BR(t\to Hc)$ and Figure~\ref{fig:summary_printnum_hu} for $\BR(t\to Hu)$.
%The observed (expected) 95\% CL combined upper limits on the branching ratios are 
%$\BR(t\to Hc)<1.1 \times 10^{-3}\,(8.3 \times 10^{-4})$ and $\BR(t\to Hu)<1.2 \times 10^{-3}\,(8.3 \times 10^{-4})$.
A summary of the upper limits on the branching ratios obtained by the individual searches, as well as their combination, is given  
%%in Table~\ref{tab:limits_summary}, as is displayed in Figures~\ref{fig:limits_combo_1D_hc} and~\ref{fig:limits_combo_1D_hu}.
in Table~\ref{tab:limits_summary} and in Figures~\ref{fig:limits_combo_1D_hc}(a) and~\ref{fig:limits_combo_1D_hc}(b).

\DIFaddbegin \begin{table}[h!]
\caption{\DIFaddFL{List of relative uncertainties on $\BR(t\to qH)$ ($q=u,c$) obtained from the combined fit. The uncertainties are symmetrised for\
 presentation and grouped into the categories described in the text.}}
\label{tab:had_sys_impact}
\begin{center}
  \begin{tabular}{%
      @{}l%
      S
      S
      @{}
    }
    \toprule\toprule
    \multirow{2}{*}{Source of uncertainty}      & \multicolumn{2}{c}{$\Delta\BR/\BR [\%]$} \\
    & \multicolumn{1}{c}{$t\rightarrow uH$} & \multicolumn{1}{c}{$t\rightarrow cH$} \\\midrule
    \DIFaddFL{Lepton ID                               }& \DIFaddFL{0.6           }&\DIFaddFL{1.0         }\\
    \DIFaddFL{$\met$                                  }& \DIFaddFL{0.7           }&\DIFaddFL{0.8         }\\
    \DIFaddFL{Fake lepton  modeling                   }& \DIFaddFL{0.9           }&\DIFaddFL{1.1         }\\
    \DIFaddFL{JES and JER                             }& \DIFaddFL{2.4           }&\DIFaddFL{3.2         }\\
    \DIFaddFL{Flavour tagging                         }& \DIFaddFL{2.7           }&\DIFaddFL{3.7         }\\
    \DIFaddFL{$t\bar{t}$ modeling                     }& \DIFaddFL{2.9           }&\DIFaddFL{4.3         }\\
    \DIFaddFL{Other MC modeling                       }& \DIFaddFL{2.1           }&\DIFaddFL{2.9         }\\
    \DIFaddFL{Fake $\tau$ modeling                    }& \DIFaddFL{3.2           }&\DIFaddFL{4.6         }\\
    \DIFaddFL{Signal modeling including Br($H\to\tau\tau$)       }& \DIFaddFL{5.3           }&\DIFaddFL{7.0         }\\
    \DIFaddFL{$\tau$ ID                               }& \DIFaddFL{3.3           }&\DIFaddFL{4.4         }\\
    \DIFaddFL{Luminosity and Pileup                   }& \DIFaddFL{0.9           }&\DIFaddFL{1.3         }\\\midrule
    %DIF > Other MC modeling                       & 2.1           &2.9         \\
    %DIF > JES and JER                             & 2.4           &3.2         \\
    %DIF > Flavour tagging                         & 2.7           &3.7         \\
    %DIF > $t\bar{t}$ modeling                     & 2.9           &4.3         \\
    %DIF > Fake $\tau$ modeling                    & 3.2           &4.6         \\
    %DIF > $\tau$ ID                               & 3.3           &4.4         \\
    %DIF > Signal modeling including Br($H\to\tau\tau$)       & 5.3           &7.0         \\\midrule
    \DIFaddFL{Total systematic uncertainty            }& \DIFaddFL{11.2          }&\DIFaddFL{15.5        }\\
    \DIFaddFL{Total statistics uncertainty            }& \DIFaddFL{5.1           }&\DIFaddFL{7.0         }\\\midrule
    \DIFaddFL{Total                                   }& \DIFaddFL{12.3          }&\DIFaddFL{17.0        }\\
    \bottomrule\bottomrule
  \end{tabular}
\end{center}
\end{table}



\DIFaddend Upper limits on the branching ratios $\BR(t\to qH)$ ($q=u,c$) can be translated into upper limits on the dimension-6 (D6) operator Wilson coefficients appearing in the effective field theory Lagrangian for $tqH$ interaction~\cite{fcnc_production_theory}:
%
\begin{equation}
  \mathcal{L}_{EFT} = \frac{C^{i3}_{u\phi}}{\Lambda^{2}}(\phi^{\dagger}\phi)(\bar{q_{i}}t)\tilde{\phi} + \frac{C^{3i}_{u\phi}}{\Lambda^{2}}(\phi^{\dagger}\phi)(\bar{t}q_{i})\tilde{\phi}
  \label{eq:eq01}
\end{equation}
%
where the subscript i= 1, 2 represents the generation of the light quark fields ($q=u, c$).
The branching ratio $\BR(t\to qH)$ is estimated as the ratio of its partial width to the SM $t \to Wb$ partial width including next-to-leading-order QCD corrections and the coefficients can be extracted as $C_{q\phi} = \sqrt{1946.6~\BR(t\to qH)}$~\cite{fcnc_production_theory}. The $C_{q\phi}$ coefficient corresponds to the sum in quadrature of the coefficients relative to the two possible chirality combinations of the quark fields,
$C_{q\phi} =\sqrt{(C^{i3}_{q\phi})^2 + (C^{3i}_{q\phi})^2}$~\cite{fcnc_production_theory}. The observed (expected) upper limits on the D6 Wilson coefficients from the combination of the searches are $C_{c\phi}<1.38\,(0.97)$ and $C_{u\phi}<1.18\,(0.83)$ \DIFaddbegin \DIFadd{for the new physics scale $\Lambda$ at 1~TeV}\DIFaddend . 

%Upper limits on the branching ratios $\BR(t\to Hq)$ ($q=u,c$) can be translated into upper limits on the non-flavour-diagonal Yukawa couplings $\lamHq$ 
%appearing in the Lagrangian~\cite{Harnik:2012pb}:
%\begin{equation*}
%{\cal L}_\mathrm{FCNC} = -\lambda_{t_\mathrm{L} q_\mathrm{R}} \bar{t}_\mathrm{L} q_\mathrm{R} H - \lambda_{q_\mathrm{L} t_\mathrm{R}} \bar{q}_\mathrm{L} t_\mathrm{R} H  + \mathrm{h.c.}
%\end{equation*}
%The branching ratio $\BR(t\to Hq)$ is estimated as the ratio of its partial width~\cite{Zhang:2013xya} to the SM $t \to Wb$ partial width~\cite{Denner:1990ns}, 
%which is assumed to be dominant. Both predicted partial widths include next-to-leading-order QCD corrections.
%Using the expression derived in Ref.~\cite{Aad:2014dya}, the coupling $|\lamHq|$ can be extracted as $| \lamHq | = (1.92 \pm 0.02) \sqrt{\BR(t\to Hq)}$.
%The $\lamHq$ coupling corresponds to the sum in quadrature of the couplings relative to the two possible chirality combinations of the quark fields, 
%$\lamHq \equiv \sqrt{ |\lambda_{t_\mathrm{L} q_\mathrm{R}}|^2 +   |\lambda_{q_\mathrm{L} t_\mathrm{R}}|^2 }$~\cite{Harnik:2012pb}.
%The observed (expected) upper limits on the couplings from the combination of the searches are $|\lamHc|<0.064\,(0.055)$ and $|\lamHu|<0.066\,(0.055)$.

%%%%%%%%%%%%%%%
\DIFaddbegin 

%DIF > %%%% mu benchmark is 0.1%

%DIF >  \begin{table}[t!]
%DIF >  \caption{\small{Summary of 95\% CL upper limits on $\BR(t \to cH)$ and $\BR(t \to uH)$, in each case neglecting the other decay mode. }}
%DIF >  \begin{center}
%DIF >  \begin{tabular}{lcc}
%DIF >  \toprule\toprule
%DIF >   & \multicolumn{1}{c}{95\% CL upper limits} & \multicolumn{1}{c}{95\% CL upper limits}  \\
%DIF >   & \multicolumn{1}{c}{on $\BR(t \to cH)$} & \multicolumn{1}{c}{on $\BR(t \to uH)$} \\
%DIF >   &  Observed (Expected) & Observed (Expected)  \\
%DIF >  \midrule\midrule
%DIF >  hadronic  & $1.0 \times 10^{-3}$ ($9.8 \times 10^{-4}$) & $7.8 \times 10^{-4}$ ($7.8 \times 10^{-4}$) \\ 
%DIF >  leptonic  & $1.3 \times 10^{-3}$ ($5.9 \times 10^{-4}$) & $9.2 \times 10^{-4}$ ($4.2 \times 10^{-4}$) \\
%DIF >  \midrule
%DIF >  Combination  & $9.9 \times 10^{-4}$ ($5.0 \times 10^{-4}$) & $7.2 \times 10^{-4}$ ($3.6 \times 10^{-4}$) \\
%DIF >  \bottomrule\bottomrule
%DIF >  \end{tabular}
%DIF >  \label{tab:limits_summary}
%DIF >  \end{center}
%DIF >  \end{table}
%DIF >  %%%%%%%%%%%%%%%

\DIFaddend \begin{table}[t!]
  \caption{\DIFdelbeginFL %DIFDELCMD < \small{Summary of 95\% CL upper limits on $\BR(t \to cH)$ and $\BR(t \to uH)$, in each case neglecting the other decay mode. }%%%
\DIFdelendFL \DIFaddbeginFL \small{Summary of 95\% CL upper limits on $\BR(t \to cH)$ and $\BR(t \to uH)$, significance and fitted signal strength($\mu$) in signal regions with a
  benchmark branching ratio of $\BR(t \to qH)=0.1\%$}\DIFaddFL{.}\DIFaddendFL }
\begin{center}
\DIFdelbeginFL %DIFDELCMD < \begin{tabular}{lcc}
%DIFDELCMD < \toprule\toprule
%DIFDELCMD <  & \multicolumn{1}{c}{95\% CL upper limits} & \multicolumn{1}{c}{95\% CL upper limits}  \\
%DIFDELCMD <  & \multicolumn{1}{c}{on $\BR(t \to cH)$} & \multicolumn{1}{c}{on $\BR(t \to uH)$} \\
%DIFDELCMD <  &  %%%
\DIFdelFL{Observed (Expected) }%DIFDELCMD < & %%%
\DIFdelFL{Observed (Expected)  }%DIFDELCMD < \\
%DIFDELCMD < \midrule\midrule
%DIFDELCMD < %%%
\DIFdelFL{hadronic  }%DIFDELCMD < & %%%
\DIFdelFL{$1.0 \times 10^{-3}$ ($9.8 \times 10^{-4}$) }%DIFDELCMD < & %%%
\DIFdelFL{$7.8 \times 10^{-4}$ ($7.8 \times 10^{-4}$) }%DIFDELCMD < \\ 
%DIFDELCMD < %%%
\DIFdelFL{leptonic  }%DIFDELCMD < & %%%
\DIFdelFL{$1.3 \times 10^{-3}$ ($5.9 \times 10^{-4}$) }%DIFDELCMD < & %%%
\DIFdelFL{$9.2 \times 10^{-4}$ ($4.2 \times 10^{-4}$) }%DIFDELCMD < \\
%DIFDELCMD < \midrule
%DIFDELCMD < %%%
\DIFdelFL{Combination  }%DIFDELCMD < & %%%
\DIFdelFL{$9.9 \times 10^{-4}$ ($5.0 \times 10^{-4}$) }%DIFDELCMD < & %%%
\DIFdelFL{$7.2 \times 10^{-4}$ ($3.6 \times 10^{-4}$) }%DIFDELCMD < \\
%DIFDELCMD < \bottomrule\bottomrule
%DIFDELCMD < \end{tabular}
%DIFDELCMD < %%%
\DIFdelendFL \DIFaddbeginFL \resizebox{\textwidth}{!}{
\begin{tabular}{lcccccc}
\toprule\toprule

\multirow{3}{*}{Signal Regions} & \multicolumn{3}{c}{$t\to cH$}                                & \multicolumn{3}{c}{$t\to uH$}  \\
                                &  95\% CL upper limits[$10^{-3}$]  & Significance             &  $\mu$ &     95\% CL upper limits[$10^{-3}$]  & Significance   &  $\mu$  \\
                                &  \multicolumn{2}{c}{Observed (Expected)}                     &        &     \multicolumn{2}{c}{Observed (Expected)}& \\
\midrule
$t_h\thadhad$-2j                & $1.85$($2.80^{+1.30}_{-0.78}$)&$-0.96$($0.78$) & $-1.03^{+1.04}_{-1.04}$&$1.10$($1.65^{+0.79}_{-0.46}$)&$-0.90$($1.25$) &$-0.55^{+0.59}_{-0.59}$ \\
$t_h\thadhad$-3j                & $1.18$($1.06^{+0.50}_{-0.30}$)&$0.34$($1.87$)  & $0.16^{+0.47}_{-0.47}$ &$1.00$($0.89^{+0.42}_{-0.25}$)&$0.36$($2.13$)  &$0.14^{+0.40}_{-0.40}$  \\
Hadronic Combination            & $1.04$($0.98^{+0.46}_{-0.28}$)&$0.26$ ($1.99$) & $0.11^{+0.43}_{-0.43}$ &$0.78$($0.78^{+0.37}_{-0.22}$)&$0.11$($2.33$)  &$0.04^{+0.34}_{-0.34}$  \\
\midrule
$t_l\tauhad$-2j                 &$4.86$($4.32^{+1.89}_{-1.21}$) &$0.40$($0.48$)   &$0.81^{+2.04}_{-2.04}$  & $3.93$($3.55^{+1.56}_{-0.99}$) & $0.34$($0.58$)  &$0.57^{+1.66}_{-1.66}$\\
$t_l\tauhad$-1j                 &$3.94$($3.67^{+1.66}_{-1.03}$) &$0.24$($0.57$)   &$0.40^{+1.70}_{-1.70}$  & $3.10$($2.87^{+1.29}_{-0.80}$) & $0.24$($0.73$)  &$0.31^{+1.33}_{-1.33}$\\
$t_h\tlhad$-2j                  &$4.81$($5.85^{+2.90}_{-1.63}$) &$-0.52$($0.39$)  &$-1.36^{+2.56}_{-2.56}$ & $2.56$($3.05^{+1.38}_{-0.85}$) & $-0.48$($0.69$) &$-0.66^{+1.38}_{-1.38}$\\
$t_h\tlhad$-3j                  &$2.78$($2.79^{+1.36}_{-0.78}$) &$-0.04$($0.76$)  &$-0.04^{+1.26}_{-1.26}$ & $2.07$($2.09^{+0.94}_{-0.58}$) & $-0.05$($0.98$) &$-0.04^{+0.98}_{-0.98}$\\
$t_l\thadhad$                   &$1.41$($0.63^{+0.29}_{-0.18}$) &$2.64$($3.24$)   &$0.74^{+0.34}_{-0.34}$  & $1.01$($0.45^{+0.21}_{-0.13}$) & $2.64$($4.08$)  &$0.53^{+0.25}_{-0.25}$\\
Leptonic Combination            &$1.29$($0.59^{+0.27}_{-0.17}$) &$2.59$($3.34$)   &$0.68^{+0.32}_{-0.32}$  & $0.92$($0.42^{+0.19}_{-0.12}$) & $2.59$($4.23$)  &$0.48^{+0.23}_{-0.23}$\\
\midrule
Combination                     &$0.99$ ($0.50^{+0.22}_{-0.14}$)&$2.34$($3.69$)   &$0.51^{+0.25}_{-0.25}$ & $0.72$ ($0.36^{+0.17}_{-0.10}$)& $2.31$($4.49$)&$0.37^{+0.18}_{-0.18}$  \\
\bottomrule\bottomrule
\end{tabular}
}
\DIFaddendFL \label{tab:limits_summary}
\end{center}
\end{table}
%%%%%%%%%%%%%%%






%%%%%%%%%%%%%%
\begin{figure*}[h!]
\begin{center}
\begin{tabular}{@{}cc@{}}
\includegraphics[width=0.49\textwidth]{figures/tcH_Limits.pdf}&
\includegraphics[width=0.49\textwidth]{figures/tuH_Limits.pdf}\\
(a) tcH & (b) tuH \\
\end{tabular}
\caption{\small {95\% CL upper limits on $\BR(t\to cH)$(a) for the individual searches as well as their
combination, assuming $\BR(t\to uH)=0$. 95\% CL upper limits on $\BR(t\to uH)$(b) for the individual searches as well as their
combination, assuming $\BR(t\to cH)=0$. The observed limits (solid lines) are compared with the 
expected (median) limits under the background-only hypothesis (dotted lines). The surrounding shaded bands correspond to the 68\% and 95\% CL intervals around the expected limits, 
denoted by $\pm 1\sigma$ and $\pm 2\sigma$, respectively.
}}
\label{fig:limits_combo_1D_hc} 
\end{center}
\end{figure*}
%%%%%%%%%%%%%%

%%%%%%%%%%%%%%%%%%
%%%%\begin{figure*}[h!]
%%%%\begin{center}
%%%%\includegraphics[width=0.7\textwidth]{figures/tuH_combined_Limit.pdf}
%%%%\caption{\small {95\% CL upper limits on $\BR(t\to Hu)$ for the individual searches as well as their
%%%%combination, assuming $\BR(t\to Hc)=0$. The observed limits (solid lines) are compared with the 
%%%%expected (median) limits under the background-only
%%%%hypothesis (dotted lines). The surrounding shaded bands correspond to the 68\% and 95\% CL intervals around the expected limits, 
%%%%denoted by $\pm 1\sigma$ and $\pm 2\sigma$, respectively.
%%%%}}
%%%%\label{fig:limits_combo_1D_hu} 
%%%%\end{center}
%%%%\end{figure*}

A similar set of results can be obtained by simultaneously varying both branching ratios in the likelihood function.
Figure~\ref{fig:limits_combo_2D}(a) shows the 95\% CL upper limits on the branching ratios in the $\BR(t\to uH)$ versus $\BR(t\to cH)$ plane. 
%The small differences between the limiting values (on the $x$- and $y$-axes) of the branching ratio limits obtained in the two-dimensional scan and 
%those reported in Table~\ref{tab:limits_summary}, result from slightly different choices in the $\HML$ search  
%regarding the final discriminant, which in the two-dimensional case should be common to both signals, and its binning.
%\textbf{Add comment of what discriminant is used in this case and the caveat regarding the corresponding 1D limit.}
The corresponding upper limits on the D6 Wilson coefficients couplings in the $C_{u\phi}$ versus $C_{c\phi}$ plane are shown in Figure~\ref{fig:limits_combo_2D}(b).

\begin{figure*}[t!]
\begin{center}
\subfloat[]{\includegraphics[width=0.49\textwidth]{figures/2DLimits.pdf}}
\subfloat[]{\includegraphics[width=0.49\textwidth]{figures/Wilson_coefficient_smooth.pdf}}
\caption{\small {95\% CL upper limits (a) on the plane of $\BR(t\to uH)$ versus $\BR(t\to cH)$ and (b) on the plane 
of $C_{c\phi}$ versus $C_{u\phi}$ for the combination of the searches. The observed limits (solid lines) are compared with the expected (median) limits under the background-only hypothesis (dotted lines). The surrounding shaded bands correspond to the 68\% and 95\% CL intervals around the expected limits, 
denoted by $\pm 1\sigma$ and $\pm 2\sigma$, respectively.}}
%%=======
%%of $C_{u\phi}$ versus $C_{c\phi}$ for the combination of the searches. The observed limits (solid lines) are compared with the expected (median) limits under the background-only hypothesis (dotted %%lines). The surrounding shaded bands correspond to the 68\% and 95\% CL intervals around the expected limits, 
%%denoted by $\pm 1\sigma$ and $\pm 2\sigma$, respectively. ({\color{red} plot (b) needs to be updated}) }}
%%>>>>>>> e84999e0023e50e93b4b7507152380195248366c
\label{fig:limits_combo_2D} 
\end{center}
\end{figure*}
%%%%%%%%%%%%%%





\FloatBarrier

% Conclusion
%-------------------------------------------------------------------------------
\section{Conclusion}
\label{sec:conclusion}
%-------------------------------------------------------------------------------
A search for flavour-changing neutral current processes involving a top quark, an up-type quark ($q=u, c$), and a SM Higgs boson is presented. FCNC $tqH$ interaction is searched for both in $t\bar{t}$ decay mode, where one top quark decays \DIFdelbegin \DIFdel{following }\DIFdelend \DIFaddbegin \DIFadd{according to the }\DIFaddend SM processes and the other one decays through $t\rightarrow Hq$, and in production mode ($pp\rightarrow tH$) with a single top quark produced via FCNC interaction where $t\to Wb$. The search is based on a dataset of $pp$ collisions at $\sqrt{s}=13~\tev$ recorded with the ATLAS detector at the CERN Large Hadron Collider and corresponding to an integrated luminosity of 139 fb$^{-1}$. The search selects events with two hadronically decaying $\tau$-lepton candidates ($\tauhad$) or at least one $\tauhad$ with an additional lepton(e,$\mu$), a $b$-jet as well as multiple jets. Multivariate techniques based on BDT are used to distinguish the signal from the background on the basis of their different kinematics.
%The search sensitivities have been improved significantly compared to the previous analysis using 36$^{-1}$ of run-2 data~\cite{Aaboud2019SearchFT}. No significant excess of events above the background expectation is found.
A slight excess of data is observed above background with a significance of \DIFdelbegin \DIFdel{2.6}\DIFdelend \DIFaddbegin \DIFadd{2.3}\DIFaddend $\sigma$.  
\DIFdelbegin \DIFdel{We set }\DIFdelend 95\% CL upper limits \DIFaddbegin \DIFadd{are set }\DIFaddend on the $t\to Hq$ branching ratios and the corresponding
dimension-6 operator Wilson coefficients in the effective $tqH$ couplings. 
%about a factor of 2 improvement to 95\% CL upper limits on the $t\to Hq$ branching ratios are reported in Table~\ref{tab:limits_summary}, along with the Wilson coefficients with respect to the previous partial run-2 results~\cite{Aaboud2019SearchFT}.
Observed (expected) 95\% CL upper limits are set on the $t\to cH$ and $t\to uH$ branching ratios of $9.9\times10^{-4}$ ($5.0^{+2.2}_{-1.4}\times10^{-4}$) and $7.2\times10^{-4}$ ($3.6^{+1.7}_{-1.0}\times10^{-4}$).
%The observed (expected) upper limits on the D6 Wilson coefficients from the combination of the searches are set to $C_{c\phi}<1.38\,(0.97)$ and $C_{u\phi}<1.18\,(0.83)$, respectively.
The corresponding combined observed (expected) upper limits on the dimension-6 operator Wilson coefficients in
the effective $tqH$ couplings are $C_{c\phi} <1.38\, (0.97)$ and $C_{u\phi} <1.18\, (0.83)$ \DIFaddbegin \DIFadd{for the new physics scale $\Lambda$ at 1~TeV}\DIFaddend , respectively.
These results improve significantly upon previous ATLAS and CMS limits on the $t\rightarrow Hq$ branching ratios.



%-------------------------------------------------------------------------------
\section*{Acknowledgements}
%-------------------------------------------------------------------------------
% Acknowledgements for papers with collision data
% Version 24-Oct-2018

% Standard acknowledgements start here
%----------------------------------------------
We thank CERN for the very successful operation of the LHC, as well as the
support staff from our institutions without whom ATLAS could not be
operated efficiently.

We acknowledge the support of ANPCyT, Argentina; YerPhI, Armenia; ARC, Australia; BMWFW and FWF, Austria; ANAS, Azerbaijan; SSTC, Belarus; CNPq and FAPESP, Brazil; NSERC, NRC and CFI, Canada; CERN; CONICYT, Chile; CAS, MOST and NSFC, China; COLCIENCIAS, Colombia; MSMT CR, MPO CR and VSC CR, Czech Republic; DNRF and DNSRC, Denmark; IN2P3-CNRS, CEA-DRF/IRFU, France; SRNSFG, Georgia; BMBF, HGF, and MPG, Germany; GSRT, Greece; RGC, Hong Kong SAR, China; ISF and Benoziyo Center, Israel; INFN, Italy; MEXT and JSPS, Japan; CNRST, Morocco; NWO, Netherlands; RCN, Norway; MNiSW and NCN, Poland; FCT, Portugal; MNE/IFA, Romania; MES of Russia and NRC KI, Russian Federation; JINR; MESTD, Serbia; MSSR, Slovakia; ARRS and MIZ\v{S}, Slovenia; DST/NRF, South Africa; MINECO, Spain; SRC and Wallenberg Foundation, Sweden; SERI, SNSF and Cantons of Bern and Geneva, Switzerland; MOST, Taiwan; TAEK, Turkey; STFC, United Kingdom; DOE and NSF, United States of America. In addition, individual groups and members have received support from BCKDF, CANARIE, CRC and Compute Canada, Canada; COST, ERC, ERDF, Horizon 2020, and Marie Sk{\l}odowska-Curie Actions, European Union; Investissements d' Avenir Labex and Idex, ANR, France; DFG and AvH Foundation, Germany; Herakleitos, Thales and Aristeia programmes co-financed by EU-ESF and the Greek NSRF, Greece; BSF-NSF and GIF, Israel; CERCA Programme Generalitat de Catalunya, Spain; The Royal Society and Leverhulme Trust, United Kingdom. 

The crucial computing support from all WLCG partners is acknowledged gratefully, in particular from CERN, the ATLAS Tier-1 facilities at TRIUMF (Canada), NDGF (Denmark, Norway, Sweden), CC-IN2P3 (France), KIT/GridKA (Germany), INFN-CNAF (Italy), NL-T1 (Netherlands), PIC (Spain), ASGC (Taiwan), RAL (UK) and BNL (USA), the Tier-2 facilities worldwide and large non-WLCG resource providers. Major contributors of computing resources are listed in Ref.~\cite{ATL-GEN-PUB-2016-002}.
%----------------------------------------------


%The \texttt{atlaslatex} package contains the acknowledgements that were valid
%at the time of the release you are using.
%These can be found in the \texttt{acknowledgements} subdirectory.
%When your ATLAS paper or PUB/CONF note is ready to be published,
%download the latest set of acknowledgements from:\\
%\url{https://twiki.cern.ch/twiki/bin/view/AtlasProtected/PubComAcknowledgements}



%-------------------------------------------------------------------------------
%\clearpage
%\appendix
%\part*{Auxiliary Material}
%\addcontentsline{toc}{part}{Appendix}
%\addcontentsline{toc}{part}{Auxiliary Material}
%-------------------------------------------------------------------------------
%In a paper, an appendix is used for technical details that would otherwise disturb the flow of the paper.
%Such an appendix should be printed before the Bibliography.
%%-------------------------------------------------------------------------------
\section{Pre-fit and post-fit event yields in the $\Htautau$ search}
\label{sec:prepostfit_yields_Htautau_appendix}
%-------------------------------------------------------------------------------

Table~\ref{tab:Htautau_Prefit_Yields_Unblind} presents the observed and predicted yields in each of the analysis regions 
for the search before the fit to data. 
Tables~\ref{tab:Htautau_Postfit_Yields_Unblind_Hc} and~\ref{tab:Htautau_Postfit_Yields_Unblind_Hu} present the observed and predicted yields 
in each of the analysis regions after the fit to the data under the signal-plus-background hypothesis, assuming 
$tHc$ and $tHu$ as signal, respectively.

%%%%%%%%%%%%%%%%%%%%%%%%
\begin{table}[htbp]
\caption{
Predicted and observed yields in each of the analysis regions considered.
The prediction is shown before the fit to data. Also shown are the signal expectations for 
$\Hc$ and $\Hu$ assuming $\BR(t\to Hc)=0.1\%$ and $\BR(t\to Hu)=0.1\%$ respectively.
The contributions with real $\had$ candidates from $\ttbar$ and  $Z\to \ell^+\ell^-$ ($\ell = e, \mu$), diboson, $\ttbar V$, $\ttbar H$, single-top-quark, and other small backgrounds are combined into
a single background source referred to as ``Other MC'' in the leptonic channels , whereas single-top-quark and the small contributions are combined into ``Rare'' in the hadronic channels.  
The quoted uncertainties are the sum in quadrature of statistical and systematic uncertainties of the yields.}
\small
\centering
\begin{tabular}{|c|c|c|c|c|c|}
\hline 
 & $t_{l}\tau_{had}-1j$ & $t_{l}\tau_{had}-2j$ & $t_{h}\tau_{lep}\tau_{had}-3j$ & $t_{h}\tau_{lep}\tau_{had}-2j$ & $t_{l}\tau_{had}\tau_{had}$\\\hline 
  QCD Fake lep  & $916.84 \pm 469.22$ & $434.65 \pm 218.77$ & $233.82 \pm 155.13$ & $416.75 \pm 232.61$ & $0 \pm 0$ \\ 
  W-jet fake $\tau$   & $839.60 \pm 257.79$ & $963.47 \pm 297.97$ & $2934.6 \pm 2722.5$ & $3434.22 \pm 3525.35$ & $4.96 \pm 2.80$ \\ 
  Other Fake $\tau$   & $2918.71 \pm 315.44$ & $2625.87 \pm 196.24$ & $1767.93 \pm 721.15$ & $1635.44 \pm 910.48$ & $138.86 \pm 23.10$ \\ 
  b Fake $\tau$   & $811.72 \pm 149.76$ & $1015.84 \pm 244.27$ & $522.39 \pm 236.07$ & $573.09 \pm 171.86$ & $68.10 \pm 14.20$ \\ 
  %tuH   & 71.0825 \pm 8.98689 & 62.9807 \pm 4.44175 & 123.517 \pm 14.72 & 87.6489 \pm 6.581 & 93.0677 \pm 3.83729 \\ 
  Lep fake $\tau$    & $221.68 \pm 31.75$ & $87.81 \pm 10.74$ & $307.45 \pm 102.87$ & $563.25 \pm 84.76$ & $0.88 \pm 0.36$ \\ 
  Double Fake $\tau$    & $0 \pm 0$ & $0 \pm 0$ & $0 \pm 0$ & $0 \pm 0$ & $89.74 \pm 37.20$ \\ 
  $\bar{t}t$   & $280.14 \pm 20.08$ & $178.8 \pm 42.01$ & $7081.53 \pm 2237.99$ & $11610.1 \pm 1263.95$ & $5.10 \pm 4.28$ \\ 
  Other MC   & $84.62 \pm 8.00$ & $83.30 \pm 3.31$ & $880.00 \pm 233.62$ & $1591.41 \pm 254.97$ & $40.79 \pm 2.02$ \\\hline
  Total B & $6073 \pm 488$ & $5390 \pm 291$ & $13728 \pm 1192$ & $19824 \pm 2091$ & $348 \pm 42$ \\ \hline
  Data & $6353\pm80$ & $5390 \pm 291$ & $35942\pm190$ & $50412\pm225$ & $351\pm19$\\\hline
  $\bar{t}t\to bWcH$ & $57.59\pm0.61$ &$52.80\pm0.58$ & $133.14\pm1.19$ &$102.58\pm0.97$ & $66.43\pm0.65$\\
$cg\to tH$ & $2.23\pm0.04$ & $1.38\pm0.03$  & $4.47\pm0.08$ & $5.66\pm0.08$ & $5.10\pm0.06$\\
tcH~merged & $59.82\pm0.61$ & $54.19\pm0.58$ & $137.61\pm1.19$ & $108.24\pm0.98$ & $71.53\pm0.65$\\\hline
$\bar{t}t\to bWuH$ & $59.72\pm7.52$ & $55.47\pm3.91$ & $139.67\pm16.65$ & $105.62\pm7.93$ & $69.10\pm2.85$\\
$ug\to tH$ & $11.36\pm1.43$ &$7.51\pm0.53$ & $24.78\pm3.00$ &$28.60\pm2.15$ & $24.00\pm1.00$\\
tuH~merged & $71.08\pm9.00$ & $62.98\pm4.44$ & $164.45\pm19.60$ & $134.22\pm10.10$ & $93.10\pm3.84$\\\hline
 % Total  & 6144.39 \pm 487.946 & 5452.72 \pm 291.302 & 13851.2 \pm 1191.74 & 19911.9 \pm 2091.1 & 441.49 \pm 41.7833 \\ 
\hline 
\end{tabular} 
\begin{tabular}{|c|c|c|}
\hline 
 & $t_{h}\tau_{had}\tau_{had}-2j$ & $t_{h}\tau_{had}\tau_{had}-3j$\\
\hline 
  only $\tau_{sub}$ real   & $64.23 \pm 70.07$ & $84.46 \pm 100.98$ \\ 
  %tcH   & 31.4803 \pm 6.36964 & 84.4383 \pm 10.5275 \\ 
  %tuH   & 45.6609 \pm 6.76374 & 105.38 \pm 10.798 \\ 
  Diboson   & $8.40 \pm 1.65$ & $10.75 \pm 1.25$ \\ 
  Fake $\tau$ MC   & $277.11 \pm 148.41$ & $332.76 \pm 153.96$ \\ 
  $\bar{t}t$   & $225.85 \pm 34.02$ & $209.75 \pm 50.43$ \\ 
  $Z\rightarrow\tau\tau$   & $485.70 \pm 53.51$ & $406.02 \pm 64.48$ \\ 
  $\bar{t}tV$   & $0.75 \pm 0.44$ & $5.46 \pm 1.13$ \\ 
  Rare   & $17.84 \pm 2.96$ & $10.05 \pm 2.72$ \\ 
  SM Higgs   & $17.29 \pm 2.18$ & $25.88 \pm 2.55$ \\ 
\hline 
Total B  & $1097 \pm 197$ & $1085 \pm 230$ \\ \hline
data                  & $1033\pm32$                   & $1052\pm32$  \\\hline
$\bar{t}t\to bWcH$    & $28.13\pm5.70$        & $80.14\pm9.94$     \\
$cg\to tH$            & $3.35\pm0.67$                 & $4.30\pm0.53$      \\
tcH~merged            & $31.48\pm6.37$                & $84.44\pm10.53$     \\\hline
$\bar{t}t\to bWuH$    & $28.37\pm4.26$        & $83.75\pm8.40$     \\
$ug\to tH$            & $17.29\pm2.59$                & $21.63\pm2.40$     \\
tuH~merged            & $45.66\pm6.76$                & $105.38\pm10.80$    \\\hline
\end{tabular} 
%\input{\FCNCTables/tthML/showFake/faketau/postfit/NOMINAL_SROnly/yield_chart}
%\input{\FCNCTables/xTFW/showFake/NOMINAL/yield_chart}
%\begin{center}
%\begin{tabular}{l*{4}{c}}
%\hline\hline
% & $\lephad$, 3j & $\lephad$, $\geq$4j & $\hadhad$, 3j &  $\hadhad$, $\geq$4j \\
%\hline
%$\Hc$  &   $ 89 \pm 14 $ &   $ 226 \pm 43 $ &   $ 46 \pm 14 $ &   $ 122 \pm 32 $ \\ 
%$\Hu$  &   $ 100 \pm 17 $ &   $ 237 \pm 47 $ &   $ 32 \pm 10 $ &   $ 114 \pm 28 $ \\ 
%\hline
%Fake $\had$  &   $ 2828 \pm 78 $ &   $ 3200 \pm 100 $ &   $ 710 \pm 110 $ &   $ 500 \pm 62 $ \\
%Top (real $\had$)  &   $ 3840 \pm 720 $ &   $ 3160 \pm 890 $ &   $ 113 \pm 72 $ &   $ 117 \pm 35 $ \\  
%$Z \to \tau\tau$  &   $ 420 \pm 140 $ &   $ 320 \pm 120 $ &   $ 283 \pm 99 $ &   $ 267 \pm 96 $ \\ 
%Other  &   $ 168 \pm 56 $ &   $ 103 \pm 33 $ &   $ 8.9 \pm 2.5 $ &   $ 11.2 \pm 2.5 $ \\ 
%\hline
%Total background  &   $ 7260 \pm 730 $ &   $ 6770 \pm 880 $ &   $ 1120 \pm 120 $ &   $ 900 \pm 120 $ \\ 
%\hline
%Data  & 7259  & 6768  & 1119  & 894  \\
%\hline\hline    
%\end{tabular}
%%\\
%\vspace{0.2cm}
%
%\end{center}
%\vspace{-0.5cm}
\label{tab:Htautau_Prefit_Yields_Unblind}
\end{table} 
%%%%%%%%%%%%%%%%%%%%%%%%

%%%%%%%%%%%%%%%%%%%%%%%%
\begin{table}[htbp]
\caption{
Predicted and observed yields in each of the analysis regions considered.
The background prediction is shown after the fit to asimov data under the signal-plus-background hypothesis 
(assuming $\Hc$ as signal).
The contributions with real $\had$ candidates from $\ttbar$ and  $Z\to \ell^+\ell^-$ ($\ell = e, \mu$), diboson, $\ttbar V$, $\ttbar H$, single-top-quark, and other small backgrounds are combined into
a single background source referred to as ``Other MC'' in the leptonic channels , whereas single-top-quark and the small contributions are combined into ``Rare'' in the hadronic channels.
The quoted uncertainties are the sum in quadrature of statistical and systematic uncertainties of the yields, 
computed taking into account correlations among nuisance parameters and among processes.}
\small
\centering
\begin{tabular}{|c|c|c|c|c|c|}\hline 
  & $t_{l}\tau_{had}-1j$ & $t_{l}\tau_{had}2j$ & $t_{h}\tau_{lep}\tau_{had}-3j$ & $t_{h}\tau_{lep}\tau_{had}-2j$ & $t_{l}\tau_{had}\tau_{had}$\\\hline 
  QCD Fake lep   & $916.84 \pm 213.23$ & $434.65 \pm 98.21$ & $233.82 \pm 77.52$ & $416.75 \pm 107.82$ & $0 \pm 0$ \\ 
  W-jet fake $\tau$  & $839.60 \pm 176.84$ & $963.47 \pm 211.11$ & $2934.6 \pm 602.46$ & $3434.22 \pm 820.85$ & $4.96 \pm 1.58$ \\ 
  Other Fake $\tau$  & $2918.71 \pm 217.11$ & $2625.87 \pm 161.15$ & $1767.93 \pm 455.10$ & $1635.44 \pm 784.74$ & $138.86 \pm 14.05$ \\ 
  b Fake $\tau$  & $811.72 \pm 109.86$ & $1015.84 \pm 158.82$ & $522.39 \pm 69.31$ & $573.09 \pm 75.11$ & $68.10 \pm 8.32$ \\ 
    tcH   & $59.82 \pm xxx$ & $54.19 \pm xxx$ & $137.61 \pm xxx$ & $108.24 \pm xxx$ & $71.53 \pm xxx$ \\ 
  Lep fake $\tau$  & $221.68 \pm 29.45$ & $87.81 \pm 9.62$ & $307.45 \pm 33.81$ & $563.25 \pm 60.64$ & $0.88 \pm 0.34$ \\ 
  Double Fake $\tau$  & $0 \pm 0$ & $0 \pm 0$ & $0 \pm 0$ & $0 \pm 0$ & $89.74 \pm 23.31$ \\ 
  $\bar{t}t$   & $280.14 \pm 15.02$ & $178.8 \pm 26.84$ & $7081.53 \pm 363.91$ & $11610.1 \pm 475.80$ & $5.10 \pm 2.69$ \\ 
  Other MC   & $84.62 \pm 7.34$ & $83.30 \pm 3.10$ & $880.00 \pm 67.57$ & $1591.41 \pm 84.97$ & $40.79 \pm 1.75$ \\ 
\hline 
  Total  & $6144 \pm 170$ & $5453 \pm 94$ & $13851 \pm 138$ & $19912 \pm 164$ & $441 \pm 29$ \\ \hline 
\end{tabular} 
\begin{tabular}{|c|c|c|}
\hline 
 & $t_{h}\tau_{had}\tau_{had}-2j$ & $t_{h}\tau_{had}\tau_{had}-3j$\\
\hline 
  only $\tau_{sub}$ real   & $64.23 \pm 49.64$ & $84.46 \pm 66.13$ \\ 
  tcH   & $31.48 \pm 9.43$ & $84.44 \pm 22.34$ \\ 
%  tuH   & $45.66 \pm 13.02$ & $105.38 \pm 27.65$ \\ 
  Diboson   & $8.40 \pm 1.59$ & $10.75 \pm 1.10$ \\ 
  Fake $\tau$ MC   & $277.11 \pm 87.45$ & $332.77 \pm 93.32$ \\ 
  $\bar{t}t$   & $225.85 \pm 26.18$ & $209.75 \pm 31.01$ \\ 
  $Z\rightarrow\tau\tau$   & $485.70 \pm 43.14$ & $406.02 \pm 51.63$ \\ 
  $\bar{t}tV$   & $0.76 \pm 0.43$ & $5.46 \pm 1.03$ \\ 
  Rare   & $17.84 \pm 2.76$ & $10.05 \pm 2.53$ \\ 
  SM Higgs   & $17.30 \pm 1.90$ & $25.88 \pm 2.05$ \\ \hline 
  Total background  & $1097 \pm 38$ & $1085 \pm 45$ \\ 
\hline 
\end{tabular} 
%\begin{center}
%\begin{tabular}{l*{4}{c}}
%\hline\hline
% & $\lephad$, 3j & $\lephad$, $\geq$4j & $\hadhad$, 3j &  $\hadhad$, $\geq$4j \\
%\hline
%$\Hc$  &   $ -4.2 \pm 8.2 $ &   $ -11 \pm 21 $ &   $ -2.4 \pm 4.3 $ &   $ -10 \pm 11 $ \\ 
%\hline
%Fake $\had$  &   $ 2290 \pm 680 $ &   $ 2640 \pm 880 $ &   $ 640 \pm 110 $ &   $ 440 \pm 100 $ \\ 
%Top (real $\had$)  &   $ 4300 \pm 670 $ &   $ 3660 \pm 860 $ &   $ 147 \pm 84 $ &   $ 139 \pm 35 $ \\ 
%$Z \to \tau\tau$  &   $ 500 \pm 100 $ &   $ 359 \pm 90 $ &   $ 320 \pm 79 $ &   $ 306 \pm 76 $ \\ 
%Other  &   $ 178 \pm 45 $ &   $ 112 \pm 28 $ &   $ 9.6 \pm 2.6 $ &   $ 12.5 \pm 2.6 $ \\ 
%\hline
%Total  &   $ 7230 \pm 160 $ &   $ 6760 \pm 170 $ &   $ 1117 \pm 65 $ &   $ 893 \pm 45 $ \\
%\hline
%Data & 7259  & 6768  & 1119  & 894 \\ 
%\hline\hline    
%\end{tabular}
%%\\
%\vspace{0.2cm}
%
%\end{center}
%\vspace{-0.5cm}
\label{tab:Htautau_Postfit_Yields_Unblind_Hc}
\end{table} 
%%%%%%%%%%%%%%%%%%%%%%%%

%%%%%%%%%%%%%%%%%%%%%%%%
\begin{table}[htbp]
\caption{
Predicted and observed yields in each of the analysis regions considered.
The background prediction is shown after the fit to the asimov data under the signal-plus-background hypothesis 
(assuming $\Hu$ as signal).
The contributions with real $\had$ candidates from $\ttbar$ and  $Z\to \ell^+\ell^-$ ($\ell = e, \mu$), diboson, $\ttbar V$, $\ttbar H$, single-top-quark, and other small backgrounds are combined into
a single background source referred to as ``Other MC'' in the leptonic channels , whereas single-top-quark and the small contributions are combined into ``Rare'' in the hadronic channels.
The quoted uncertainties are the sum in quadrature of statistical and systematic uncertainties of the yields, 
computed taking into account correlations among nuisance parameters and among processes.
}
\small
\centering
\begin{tabular}{|c|c|c|c|c|c|}
  \hline
 & $t_{l}\tau_{had}-1j$ & $t_{l}\tau_{had}2j$ & $t_{h}\tau_{lep}\tau_{had}-3j$ & $t_{h}\tau_{lep}\tau_{had}-2j$ & $t_{l}\tau_{had}\tau_{had}$\\
\hline
  QCD Fake lep   & $916.84 \pm 213.23$ & $434.65 \pm 98.21$ & $233.82 \pm 77.52$ & $416.75 \pm 107.82$ & $0 \pm 0$ \\
  W-jet fake $\tau$  & $839.60 \pm 176.84$ & $963.47 \pm 211.11$ & $2934.6 \pm 602.46$ & $3434.22 \pm 820.85$ & $4.96 \pm 1.58$ \\
  Other Fake $\tau$  & $2918.71 \pm 217.11$ & $2625.87 \pm 161.15$ & $1767.93 \pm 455.10$ & $1635.44 \pm 784.74$ & $138.86 \pm 14.05$ \\
  b Fake $\tau$  & $811.72 \pm 109.86$ & $1015.84 \pm 158.82$ & $522.39 \pm 69.31$ & $573.09 \pm 75.11$ & $68.10 \pm 8.32$ \\
  tuH   & $71.08 \pm 14.95$ & $62.98 \pm 11.26$ & $123.52 \pm 20.39$ & $87.65 \pm 13.75$ & $93.07 \pm 14.64$ \\                                                                                           
  Lep fake $\tau$  & $221.68 \pm 29.45$ & $87.81 \pm 9.62$ & $307.45 \pm 33.81$ & $563.25 \pm 60.64$ & $0.88 \pm 0.34$ \\
  Double Fake $\tau$  & $0 \pm 0$ & $0 \pm 0$ & $0 \pm 0$ & $0 \pm 0$ & $89.74 \pm 23.31$ \\
  $\bar{t}t$   & $280.14 \pm 15.02$ & $178.8 \pm 26.84$ & $7081.53 \pm 363.91$ & $11610.1 \pm 475.80$ & $5.10 \pm 2.69$ \\
  Other MC   & $84.62 \pm 7.34$ & $83.30 \pm 3.10$ & $880.00 \pm 67.57$ & $1591.41 \pm 84.97$ & $40.79 \pm 1.75$ \\
\hline	
  Total  & $6144 \pm 170$ & $5453 \pm 94$ & $13851 \pm 138$ & $19912 \pm 164$ & $441 \pm 29$ \\ \hline
\end{tabular}
\begin{tabular}{|c|c|c|}
\hline
 & $t_{h}\tau_{had}\tau_{had}-2j$ & $t_{h}\tau_{had}\tau_{had}-3j$\\
\hline
  only $\tau_{sub}$ real   & $64.23 \pm 49.64$ & $84.46 \pm 66.13$ \\
  %  tcH   & $31.48 \pm 9.43$ & $84.44 \pm 22.34$ \\
  tuH   & $45.66 \pm 13.02$ & $105.38 \pm 27.65$ \\                                                                                                                                                       
  Diboson   & $8.40 \pm 1.59$ & $10.75 \pm 1.10$ \\
  Fake $\tau$ MC   & $277.11 \pm 87.45$ & $332.77 \pm 93.32$ \\	
  $\bar{t}t$   & $225.85 \pm 26.18$ & $209.75 \pm 31.01$ \\
  $Z\rightarrow\tau\tau$   & $485.70 \pm 43.14$ & $406.02 \pm 51.63$ \\	
  $\bar{t}tV$   & $0.76 \pm 0.43$ & $5.46 \pm 1.03$ \\
  Rare   & $17.84 \pm 2.76$ & $10.05 \pm 2.53$ \\
  SM Higgs   & $17.30 \pm 1.90$ & $25.88 \pm 2.05$ \\ \hline
  Total background  & $1097 \pm 38$ & $1085 \pm 45$ \\
\hline 
\end{tabular} 
%\begin{center}
%\begin{tabular}{l*{4}{c}}
%\hline\hline
% & $\lephad$, 3j & $\lephad$, $\geq$4j & $\hadhad$, 3j &  $\hadhad$, $\geq$4j \\
%\hline
%$\Hu$  &   $ -5.7 \pm 8.6 $ &   $ -14 \pm 21 $ &   $ -2,0 \pm 2.8 $ &   $ -7.1 \pm 9.8 $ \\ 
%\hline
%Fake $\had$  &   $ 2270 \pm 680 $ &   $ 2620 \pm 880 $ &   $ 640 \pm 110 $ &   $ 440 \pm 100 $ \\
%Top (real $\had$)  &   $ 4320 \pm 660 $ &   $ 3680 \pm 860 $ &   $ 148 \pm 84 $ &   $ 140 \pm 35 $ \\ 
%$Z \to \tau\tau$  &   $ 470 \pm 100 $ &   $ 359 \pm 89 $ &   $ 321 \pm 79 $ &   $ 308 \pm 77 $ \\ 
%Other  &   $ 177 \pm 44 $ &   $ 111 \pm 27 $ &   $ 9.7 \pm 2.6 $ &   $ 12.5 \pm 2.6 $ \\ 
%\hline
%Total  &   $ 7230 \pm 160 $ &   $ 6760 \pm 160 $ &   $ 1118 \pm 66 $ &   $ 892 \pm 45 $ \\ 
%\hline
%Data  & 7259  & 6768  & 1119  & 894  \\ 
%\hline\hline    
%\end{tabular}
%%\\
%\vspace{0.2cm}
%
%\end{center}
%\vspace{-0.5cm}
\label{tab:Htautau_Postfit_Yields_Unblind_Hu}
\end{table} 
%%%%%%%%%%%%%%%%%%%%%%%%

%\FloatBarrier
\clearpage

%-------------------------------------------------------------------------------
% If you use biblatex and either biber or bibtex to process the bibliography
% just say \printbibliography here
\printbibliography
% If you want to use the traditional BibTeX you need to use the syntax below.
%\bibliographystyle{bib/bst/atlasBibStyleWoTitle}
%\bibliography{,bib/ATLAS,bib/CMS,bib/ConfNotes,bib/PubNotes}
%-------------------------------------------------------------------------------

%-------------------------------------------------------------------------------
% Auxiliary material - comment out the following line if you do not have any
%\part*{Auxiliary material}
\addcontentsline{toc}{part}{Auxiliary material}
%-------------------------------------------------------------------------------



%%%%%%%%%%%%%%%%%%%%%%%%%%%%%%%%%%%%%%%
\begin{figure*}[t]
\begin{center}
\subfloat[]{\includegraphics[width=0.40\textwidth]{figures/Htautau/control_plots/x1_fit_lephad_3j_FR.pdf}}
\subfloat[]{\includegraphics[width=0.40\textwidth]{figures/Htautau/control_plots/x1_fit_lephad_4j_FR.pdf}} \\
\subfloat[]{\includegraphics[width=0.40\textwidth]{figures/Htautau/control_plots/x2_fit_lephad_3j_FR.pdf}}
\subfloat[]{\includegraphics[width=0.40\textwidth]{figures/Htautau/control_plots/x2_fit_lephad_4j_FR.pdf}} \\
\caption{$\Htautau$ search: Comparison between the data and predicted background for the distribution of some of the most
discriminating BDT input variables in the $\lephad$ channel before the fit to data (``Pre-Fit''). The distributions are shown for
$x_{1}^{\text{fit}}$ in (a) the ($\lephad$, 3j) region and (b) the ($\lephad$, $\geq$4j) region, and for
$x_{2}^{\text{fit}}$ in (c) the ($\lephad$, 3j)  region and (d) the ($\lephad$, $\geq$4j) region.
The contributions with real $\had$ candidates from $\ttbar$,  $\ttbar V$, $\ttbar H$, and single-top-quark backgrounds are combined into
a single background source referred to as ``Top (real $\had$)", whereas the small contributions from 
$Z\to \ell^+\ell^-$ ($\ell = e, \mu$) and diboson backgrounds are combined into ``Other''. 
The expected $\Hc$ signal (solid red) corresponding to $\BR(t\to Hc)=1\%$ is also shown,
added to the background prediction.
%The first and the last bins in all figures contain the underflow and overflow respectively.
The bottom panel displays the ratio of data to the SM background (``Bkg'') prediction.
The hashed area represents the total uncertainty of the background, excluding the normalisation uncertainty of the fake $\had$ background, 
which is determined via a likelihood fit to data.}
\label{fig:BDT_inputs_lephad_2}
\end{center}
\end{figure*}
%%%%%%%%%%%%%%%%%%%%%%%%%%%%%%%%%%%%%%%


%%%%%%%%%%%%%%%%%%%%%%%%%%%%%%%%%%%%%%%
\begin{figure*}[t]
\begin{center}
\subfloat[]{\includegraphics[width=0.40\textwidth]{figures/Htautau/control_plots/met_centrality_lephad_3j_FR.pdf}}
\subfloat[]{\includegraphics[width=0.40\textwidth]{figures/Htautau/control_plots/met_centrality_lephad_4j_FR.pdf}} \\
\subfloat[]{\includegraphics[width=0.40\textwidth]{figures/Htautau/control_plots/mtop_thc_lephad_3j_FR.pdf}}
\subfloat[]{\includegraphics[width=0.40\textwidth]{figures/Htautau/control_plots/mtop_thc_lephad_4j_FR.pdf}} \\
\caption{$\Htautau$ search: Comparison between the data and predicted background for the distribution of some of the most
discriminating BDT input variables in the $\lephad$ channel before the fit to data (``Pre-Fit''). The distributions are shown for
$\met$ $\phi$ centrality in (a) the ($\lephad$, 3j) region and (b) the ($\lephad$, $\geq$4j) region, and for
$m_{Hq}$ in (c) the ($\lephad$, 3j)  region and (d) the ($\lephad$, $\geq$4j) region.
The contributions with real $\had$ candidates from $\ttbar$,  $\ttbar V$, $\ttbar H$, and single-top-quark backgrounds are combined into
a single background source referred to as ``Top (real $\had$)", whereas the small contributions from 
$Z\to \ell^+\ell^-$ ($\ell = e, \mu$) and diboson backgrounds are combined into ``Other''. 
The expected $\Hc$ signal (solid red) corresponding to $\BR(t\to Hc)=1\%$ is also shown,
added to the background prediction.
%The first and the last bins in all figures contain the underflow and overflow respectively.
The bottom panel displays the ratio of data to the SM background (``Bkg'') prediction.
The hashed area represents the total uncertainty of the background, excluding the normalisation uncertainty of the fake $\had$ background, 
which is determined via a likelihood fit to data.}
\label{fig:BDT_inputs_lephad_3}
\end{center}
\end{figure*}
%%%%%%%%%%%%%%%%%%%%%%%%%%%%%%%%%%%%%%%


%%%%%%%%%%%%%%%%%%%%%%%%%%%%%%%%%%%%%%%
\begin{figure*}[t]
\begin{center}
\subfloat[]{\includegraphics[width=0.40\textwidth]{figures/Htautau/control_plots/MET_perp_lephad_3j_FR.pdf}}
\subfloat[]{\includegraphics[width=0.40\textwidth]{figures/Htautau/control_plots/MET_perp_lephad_4j_FR.pdf}} \\
\subfloat[]{\includegraphics[width=0.40\textwidth]{figures/Htautau/control_plots/MET_proj_lephad_3j_FR.pdf}}
\subfloat[]{\includegraphics[width=0.40\textwidth]{figures/Htautau/control_plots/MET_proj_lephad_4j_FR.pdf}} \\
\caption{$\Htautau$ search: Comparison between the data and predicted background for the distribution of some of the most
discriminating BDT input variables in the $\lephad$ channel before the fit to data (``Pre-Fit''). The distributions are shown for
$E_{\text{T},\perp}^{\text{miss}}$ in (a) the ($\lephad$, 3j) region and (b) the ($\lephad$, $\geq$4j) region, and for
$E_{\text{T},\parallel}^{\text{miss}}$ in (c) the ($\lephad$, 3j)  region and (d) the ($\lephad$, $\geq$4j) region.
The contributions with real $\had$ candidates from $\ttbar$,  $\ttbar V$, $\ttbar H$, and single-top-quark backgrounds are combined into
a single background source referred to as ``Top (real $\had$)", whereas the small contributions from 
$Z\to \ell^+\ell^-$ ($\ell = e, \mu$) and diboson backgrounds are combined into ``Other''. 
The expected $\Hc$ signal (solid red) corresponding to $\BR(t\to Hc)=1\%$ is also shown,
added to the background prediction.
%The first and the last bins in all figures contain the underflow and overflow respectively.
The bottom panel displays the ratio of data to the SM background (``Bkg'') prediction.
The hashed area represents the total uncertainty of the background, excluding the normalisation uncertainty of the fake $\had$ background, 
which is determined via a likelihood fit to data.}
\label{fig:BDT_inputs_lephad_4}
\end{center}
\end{figure*}
%%%%%%%%%%%%%%%%%%%%%%%%%%%%%%%%%%%%%%%



%%%%%%%%%%%%%%%%%%%%%%%%%%%%%%%%%%%%%%%
\begin{figure*}[t]
\begin{center}
\subfloat[]{\includegraphics[width=0.40\textwidth]{figures/Htautau/control_plots/x2_fit_hadhad_3j_FR.pdf}}
\subfloat[]{\includegraphics[width=0.40\textwidth]{figures/Htautau/control_plots/x2_fit_hadhad_4j_FR.pdf}} \\
\subfloat[]{\includegraphics[width=0.40\textwidth]{figures/Htautau/control_plots/ptL2_hadhad_3j_FR.pdf}}
\subfloat[]{\includegraphics[width=0.40\textwidth]{figures/Htautau/control_plots/ptL2_hadhad_4j_FR.pdf}} \\
\caption{$\Htautau$ search: Comparison between the data and predicted background for the distribution of two of the most 
discriminating BDT input variables in the $\hadhad$ channel before the fit to data (``Pre-Fit''). The distributions are shown for
$x_{2}^{\text{fit}}$ in (a) the ($\hadhad$, 3j) region and (b) the ($\hadhad$, $\geq$4j) region, and for
$p_{\text{T},2}$ in (c) the ($\hadhad$, 3j)  region and (d) the ($\hadhad$, $\geq$4j) region.
The contributions with real $\had$ candidates from $\ttbar$,  $\ttbar V$, $\ttbar H$, and single-top-quark backgrounds are combined into
a single background source referred to as ``Top (real $\had$)", whereas the small contributions from 
$Z\to \ell^+\ell^-$ ($\ell = e, \mu$) and diboson backgrounds are combined into ``Other''. 
The expected $\Hc$ signal (solid red) corresponding to $\BR(t\to Hc)=1\%$ is also shown,
added to the background prediction.
The first and the last bins in the figures in (c) and (d) contain the underflow and overflow respectively.
The bottom panel displays the ratio of data to the SM background (``Bkg'') prediction.
The hashed area represents the total uncertainty of the background, excluding the normalisation uncertainty of the fake $\had$ background, 
which is determined via a likelihood fit to data.} 
\label{fig:BDT_inputs_hadhad_2}
\end{center}
\end{figure*}
%%%%%%%%%%%%%%%%%%%%%%%%%%%%%%%%%%%%%%%

%%%%%%%%%%%%%%%%%%%%%%%%%%%%%%%%%%%%%%%
\begin{figure*}[t]
\begin{center}
\subfloat[]{\includegraphics[width=0.40\textwidth]{figures/Htautau/control_plots/met_centrality_hadhad_3j_FR.pdf}}
\subfloat[]{\includegraphics[width=0.40\textwidth]{figures/Htautau/control_plots/met_centrality_hadhad_4j_FR.pdf}} \\
\subfloat[]{\includegraphics[width=0.40\textwidth]{figures/Htautau/control_plots/mtop_thc_hadhad_3j_FR.pdf}}
\subfloat[]{\includegraphics[width=0.40\textwidth]{figures/Htautau/control_plots/mtop_thc_hadhad_4j_FR.pdf}} \\
\caption{$\Htautau$ search: Comparison between the data and predicted background for the distribution of two of the most 
discriminating BDT input variables in the $\hadhad$ channel before the fit to data (``Pre-Fit''). The distributions are shown for
$\met$ $\phi$ centrality  in (a) the ($\hadhad$, 3j) region and (b) the ($\hadhad$, $\geq$4j) region, and for
$m_{Hq}$ in (c) the ($\hadhad$, 3j)  region and (d) the ($\hadhad$, $\geq$4j) region.
The contributions with real $\had$ candidates from $\ttbar$,  $\ttbar V$, $\ttbar H$, and single-top-quark backgrounds are combined into
a single background source referred to as ``Top (real $\had$)", whereas the small contributions from 
$Z\to \ell^+\ell^-$ ($\ell = e, \mu$) and diboson backgrounds are combined into ``Other''. 
The expected $\Hc$ signal (solid red) corresponding to $\BR(t\to Hc)=1\%$ is also shown,
added to the background prediction.
The first and the last bins in the figures in (c) and (d) contain the underflow and overflow respectively.
The bottom panel displays the ratio of data to the SM background (``Bkg'') prediction.
The hashed area represents the total uncertainty of the background, excluding the normalisation uncertainty of the fake $\had$ background, 
which is determined via a likelihood fit to data.} 
\label{fig:BDT_inputs_hadhad_3}
\end{center}
\end{figure*}
%%%%%%%%%%%%%%%%%%%%%%%%%%%%%%%%%%%%%%%


%%%%%%%%%%%%%%%%%%%%%%%%%%%%%%%%%%%%%%%
\begin{figure*}[t]
\begin{center}
\subfloat[]{\includegraphics[width=0.40\textwidth]{figures/Htautau/control_plots/MET_perp_hadhad_3j_FR.pdf}}
\subfloat[]{\includegraphics[width=0.40\textwidth]{figures/Htautau/control_plots/MET_perp_hadhad_4j_FR.pdf}} \\
\subfloat[]{\includegraphics[width=0.40\textwidth]{figures/Htautau/control_plots/MET_proj_hadhad_3j_FR.pdf}}
\subfloat[]{\includegraphics[width=0.40\textwidth]{figures/Htautau/control_plots/MET_proj_hadhad_4j_FR.pdf}} \\
\caption{$\Htautau$ search: Comparison between the data and predicted background for the distribution of two of the most 
discriminating BDT input variables in the $\hadhad$ channel before the fit to data (``Pre-Fit''). The distributions are shown for
$E_{\text{T},\perp}^{\text{miss}}$ in (a) the ($\hadhad$, 3j) region and (b) the ($\hadhad$, $\geq$4j) region, and for
$E_{\text{T},\parallel}^{\text{miss}}$ in (c) the ($\hadhad$, 3j)  region and (d) the ($\hadhad$, $\geq$4j) region.
The contributions with real $\had$ candidates from $\ttbar$,  $\ttbar V$, $\ttbar H$, and single-top-quark backgrounds are combined into
a single background source referred to as ``Top (real $\had$)", whereas the small contributions from 
$Z\to \ell^+\ell^-$ ($\ell = e, \mu$) and diboson backgrounds are combined into ``Other''. 
The expected $\Hc$ signal (solid red) corresponding to $\BR(t\to Hc)=1\%$ is also shown,
added to the background prediction.
The first and the last bins in the figures in (c) and (d) contain the underflow and overflow respectively.
The bottom panel displays the ratio of data to the SM background (``Bkg'') prediction.
The hashed area represents the total uncertainty of the background, excluding the normalisation uncertainty of the fake $\had$ background, 
which is determined via a likelihood fit to data.} 
\label{fig:BDT_inputs_hadhad_4}
\end{center}
\end{figure*}
%%%%%%%%%%%%%%%%%%%%%%%%%%%%%%%%%%%%%%%


%%%%%%%%%%%%%%%%%%%%%%%%%%%%%%%%%%%%%%%
\begin{figure*}[t]
\begin{center}
\includegraphics[width=0.90\textwidth]{figures/Combo/ranking/Ranking_tt.pdf} \\
\caption{$\Htautau$ search: Ranking of the nuisance parameters included in the fit according to their impact on the measured signal strength $\mu$, in
the case of the $\Hc$ search.
Only the 10 most highly ranked parameters are shown. Nuisance parameters corresponding to MC statistical uncertainties are not included here. 
The empty blue rectangles correspond to the pre-fit impact on $\mu$ and the filled blue ones to the post-fit impact on $\mu$, 
both referring to the upper scale. The impact of each nuisance parameter, $\Delta\mu$, is computed by comparing the nominal best-fit value of $\mu$ 
with the result of the fit when fixing the considered nuisance parameter to its best-fit value, $\hat{\theta}$, shifted by its pre-fit (post-fit) uncertainties 
$\pm\Delta\theta$ ($\pm\Delta\hat{\theta}$). The black points show the pulls of the nuisance parameters relative to their nominal values, $\theta_{0}$. 
These pulls and their relative post-fit errors, $\Delta\hat{\theta}/\Delta\theta$, refer to the lower scale.
The parameter $\mathrm{k}($Fake~$\had)$ refers to the floating normalisation of the fake $\had$ background, for which the pre-fit impact on $\mu$ is not defined, 
and for which both $\theta_{0}$ and $\Delta\theta$ are set to 1.
For experimental uncertainties that are decomposed into several independent sources, NP I and NP II correspond to the first and second nuisance parameters, 
ordered by its impact on $\mu$.} 
\label{fig:ranking_tautau}
\end{center}
\end{figure*}
%%%%%%%%%%%%%%%%%%%%%%%%%%%%%%%%%%%%%%%

%-------------------------------------------------------------------------------

%-------------------------------------------------------------------------------
% Extra tables etc. for HepData - comment in the following line if you have any
% \include{-hepdata}
%-------------------------------------------------------------------------------

%%\clearpage % ATLAS Collaboration author list
% Reference date of TOPQ-2017-07 is 2018-08-20
% Data extracted on 20-Aug-2018 for paper reference TOPQ-2017-07
% at 2:36pm
 
\begin{flushleft}
{\Large The ATLAS Collaboration}

\bigskip

M.~Aaboud$^\textrm{\scriptsize 34d}$,    
G.~Aad$^\textrm{\scriptsize 99}$,    
B.~Abbott$^\textrm{\scriptsize 124}$,    
D.C.~Abbott$^\textrm{\scriptsize 100}$,    
O.~Abdinov$^\textrm{\scriptsize 13,*}$,    
B.~Abeloos$^\textrm{\scriptsize 128}$,    
D.K.~Abhayasinghe$^\textrm{\scriptsize 91}$,    
S.H.~Abidi$^\textrm{\scriptsize 164}$,    
O.S.~AbouZeid$^\textrm{\scriptsize 39}$,    
N.L.~Abraham$^\textrm{\scriptsize 153}$,    
H.~Abramowicz$^\textrm{\scriptsize 158}$,    
H.~Abreu$^\textrm{\scriptsize 157}$,    
Y.~Abulaiti$^\textrm{\scriptsize 6}$,    
B.S.~Acharya$^\textrm{\scriptsize 64a,64b,o}$,    
S.~Adachi$^\textrm{\scriptsize 160}$,    
L.~Adam$^\textrm{\scriptsize 97}$,    
L.~Adamczyk$^\textrm{\scriptsize 81a}$,    
J.~Adelman$^\textrm{\scriptsize 119}$,    
M.~Adersberger$^\textrm{\scriptsize 112}$,    
A.~Adiguzel$^\textrm{\scriptsize 12c,ag}$,    
T.~Adye$^\textrm{\scriptsize 141}$,    
A.A.~Affolder$^\textrm{\scriptsize 143}$,    
Y.~Afik$^\textrm{\scriptsize 157}$,    
C.~Agheorghiesei$^\textrm{\scriptsize 27c}$,    
J.A.~Aguilar-Saavedra$^\textrm{\scriptsize 136f,136a}$,    
F.~Ahmadov$^\textrm{\scriptsize 77,ae}$,    
G.~Aielli$^\textrm{\scriptsize 71a,71b}$,    
S.~Akatsuka$^\textrm{\scriptsize 83}$,    
T.P.A.~{\AA}kesson$^\textrm{\scriptsize 94}$,    
E.~Akilli$^\textrm{\scriptsize 52}$,    
A.V.~Akimov$^\textrm{\scriptsize 108}$,    
G.L.~Alberghi$^\textrm{\scriptsize 23b,23a}$,    
J.~Albert$^\textrm{\scriptsize 173}$,    
P.~Albicocco$^\textrm{\scriptsize 49}$,    
M.J.~Alconada~Verzini$^\textrm{\scriptsize 86}$,    
S.~Alderweireldt$^\textrm{\scriptsize 117}$,    
M.~Aleksa$^\textrm{\scriptsize 35}$,    
I.N.~Aleksandrov$^\textrm{\scriptsize 77}$,    
C.~Alexa$^\textrm{\scriptsize 27b}$,    
D.~Alexandre$^\textrm{\scriptsize 19}$,    
T.~Alexopoulos$^\textrm{\scriptsize 10}$,    
M.~Alhroob$^\textrm{\scriptsize 124}$,    
B.~Ali$^\textrm{\scriptsize 138}$,    
G.~Alimonti$^\textrm{\scriptsize 66a}$,    
J.~Alison$^\textrm{\scriptsize 36}$,    
S.P.~Alkire$^\textrm{\scriptsize 145}$,    
C.~Allaire$^\textrm{\scriptsize 128}$,    
B.M.M.~Allbrooke$^\textrm{\scriptsize 153}$,    
B.W.~Allen$^\textrm{\scriptsize 127}$,    
P.P.~Allport$^\textrm{\scriptsize 21}$,    
A.~Aloisio$^\textrm{\scriptsize 67a,67b}$,    
A.~Alonso$^\textrm{\scriptsize 39}$,    
F.~Alonso$^\textrm{\scriptsize 86}$,    
C.~Alpigiani$^\textrm{\scriptsize 145}$,    
A.A.~Alshehri$^\textrm{\scriptsize 55}$,    
M.I.~Alstaty$^\textrm{\scriptsize 99}$,    
B.~Alvarez~Gonzalez$^\textrm{\scriptsize 35}$,    
D.~\'{A}lvarez~Piqueras$^\textrm{\scriptsize 171}$,    
M.G.~Alviggi$^\textrm{\scriptsize 67a,67b}$,    
B.T.~Amadio$^\textrm{\scriptsize 18}$,    
Y.~Amaral~Coutinho$^\textrm{\scriptsize 78b}$,    
A.~Ambler$^\textrm{\scriptsize 101}$,    
L.~Ambroz$^\textrm{\scriptsize 131}$,    
C.~Amelung$^\textrm{\scriptsize 26}$,    
D.~Amidei$^\textrm{\scriptsize 103}$,    
S.P.~Amor~Dos~Santos$^\textrm{\scriptsize 136a,136c}$,    
S.~Amoroso$^\textrm{\scriptsize 44}$,    
C.S.~Amrouche$^\textrm{\scriptsize 52}$,    
F.~An$^\textrm{\scriptsize 76}$,    
C.~Anastopoulos$^\textrm{\scriptsize 146}$,    
L.S.~Ancu$^\textrm{\scriptsize 52}$,    
N.~Andari$^\textrm{\scriptsize 142}$,    
T.~Andeen$^\textrm{\scriptsize 11}$,    
C.F.~Anders$^\textrm{\scriptsize 59b}$,    
J.K.~Anders$^\textrm{\scriptsize 20}$,    
K.J.~Anderson$^\textrm{\scriptsize 36}$,    
A.~Andreazza$^\textrm{\scriptsize 66a,66b}$,    
V.~Andrei$^\textrm{\scriptsize 59a}$,    
C.R.~Anelli$^\textrm{\scriptsize 173}$,    
S.~Angelidakis$^\textrm{\scriptsize 37}$,    
I.~Angelozzi$^\textrm{\scriptsize 118}$,    
A.~Angerami$^\textrm{\scriptsize 38}$,    
A.V.~Anisenkov$^\textrm{\scriptsize 120b,120a}$,    
A.~Annovi$^\textrm{\scriptsize 69a}$,    
C.~Antel$^\textrm{\scriptsize 59a}$,    
M.T.~Anthony$^\textrm{\scriptsize 146}$,    
M.~Antonelli$^\textrm{\scriptsize 49}$,    
D.J.A.~Antrim$^\textrm{\scriptsize 168}$,    
F.~Anulli$^\textrm{\scriptsize 70a}$,    
M.~Aoki$^\textrm{\scriptsize 79}$,    
J.A.~Aparisi~Pozo$^\textrm{\scriptsize 171}$,    
L.~Aperio~Bella$^\textrm{\scriptsize 35}$,    
G.~Arabidze$^\textrm{\scriptsize 104}$,    
J.P.~Araque$^\textrm{\scriptsize 136a}$,    
V.~Araujo~Ferraz$^\textrm{\scriptsize 78b}$,    
R.~Araujo~Pereira$^\textrm{\scriptsize 78b}$,    
A.T.H.~Arce$^\textrm{\scriptsize 47}$,    
R.E.~Ardell$^\textrm{\scriptsize 91}$,    
F.A.~Arduh$^\textrm{\scriptsize 86}$,    
J-F.~Arguin$^\textrm{\scriptsize 107}$,    
S.~Argyropoulos$^\textrm{\scriptsize 75}$,    
J.-H.~Arling$^\textrm{\scriptsize 44}$,    
A.J.~Armbruster$^\textrm{\scriptsize 35}$,    
L.J.~Armitage$^\textrm{\scriptsize 90}$,    
A~Armstrong$^\textrm{\scriptsize 168}$,    
O.~Arnaez$^\textrm{\scriptsize 164}$,    
H.~Arnold$^\textrm{\scriptsize 118}$,    
M.~Arratia$^\textrm{\scriptsize 31}$,    
O.~Arslan$^\textrm{\scriptsize 24}$,    
A.~Artamonov$^\textrm{\scriptsize 109,*}$,    
G.~Artoni$^\textrm{\scriptsize 131}$,    
S.~Artz$^\textrm{\scriptsize 97}$,    
S.~Asai$^\textrm{\scriptsize 160}$,    
N.~Asbah$^\textrm{\scriptsize 57}$,    
E.M.~Asimakopoulou$^\textrm{\scriptsize 169}$,    
L.~Asquith$^\textrm{\scriptsize 153}$,    
K.~Assamagan$^\textrm{\scriptsize 29}$,    
R.~Astalos$^\textrm{\scriptsize 28a}$,    
R.J.~Atkin$^\textrm{\scriptsize 32a}$,    
M.~Atkinson$^\textrm{\scriptsize 170}$,    
N.B.~Atlay$^\textrm{\scriptsize 148}$,    
K.~Augsten$^\textrm{\scriptsize 138}$,    
G.~Avolio$^\textrm{\scriptsize 35}$,    
R.~Avramidou$^\textrm{\scriptsize 58a}$,    
M.K.~Ayoub$^\textrm{\scriptsize 15a}$,    
A.M.~Azoulay$^\textrm{\scriptsize 165b}$,    
G.~Azuelos$^\textrm{\scriptsize 107,at}$,    
A.E.~Baas$^\textrm{\scriptsize 59a}$,    
M.J.~Baca$^\textrm{\scriptsize 21}$,    
H.~Bachacou$^\textrm{\scriptsize 142}$,    
K.~Bachas$^\textrm{\scriptsize 65a,65b}$,    
M.~Backes$^\textrm{\scriptsize 131}$,    
P.~Bagnaia$^\textrm{\scriptsize 70a,70b}$,    
M.~Bahmani$^\textrm{\scriptsize 82}$,    
H.~Bahrasemani$^\textrm{\scriptsize 149}$,    
A.J.~Bailey$^\textrm{\scriptsize 171}$,    
J.T.~Baines$^\textrm{\scriptsize 141}$,    
M.~Bajic$^\textrm{\scriptsize 39}$,    
C.~Bakalis$^\textrm{\scriptsize 10}$,    
O.K.~Baker$^\textrm{\scriptsize 180}$,    
P.J.~Bakker$^\textrm{\scriptsize 118}$,    
D.~Bakshi~Gupta$^\textrm{\scriptsize 8}$,    
S.~Balaji$^\textrm{\scriptsize 154}$,    
E.M.~Baldin$^\textrm{\scriptsize 120b,120a}$,    
P.~Balek$^\textrm{\scriptsize 177}$,    
F.~Balli$^\textrm{\scriptsize 142}$,    
W.K.~Balunas$^\textrm{\scriptsize 133}$,    
J.~Balz$^\textrm{\scriptsize 97}$,    
E.~Banas$^\textrm{\scriptsize 82}$,    
A.~Bandyopadhyay$^\textrm{\scriptsize 24}$,    
S.~Banerjee$^\textrm{\scriptsize 178,k}$,    
A.A.E.~Bannoura$^\textrm{\scriptsize 179}$,    
L.~Barak$^\textrm{\scriptsize 158}$,    
W.M.~Barbe$^\textrm{\scriptsize 37}$,    
E.L.~Barberio$^\textrm{\scriptsize 102}$,    
D.~Barberis$^\textrm{\scriptsize 53b,53a}$,    
M.~Barbero$^\textrm{\scriptsize 99}$,    
T.~Barillari$^\textrm{\scriptsize 113}$,    
M-S.~Barisits$^\textrm{\scriptsize 35}$,    
J.~Barkeloo$^\textrm{\scriptsize 127}$,    
T.~Barklow$^\textrm{\scriptsize 150}$,    
R.~Barnea$^\textrm{\scriptsize 157}$,    
S.L.~Barnes$^\textrm{\scriptsize 58c}$,    
B.M.~Barnett$^\textrm{\scriptsize 141}$,    
R.M.~Barnett$^\textrm{\scriptsize 18}$,    
Z.~Barnovska-Blenessy$^\textrm{\scriptsize 58a}$,    
A.~Baroncelli$^\textrm{\scriptsize 72a}$,    
G.~Barone$^\textrm{\scriptsize 29}$,    
A.J.~Barr$^\textrm{\scriptsize 131}$,    
L.~Barranco~Navarro$^\textrm{\scriptsize 171}$,    
F.~Barreiro$^\textrm{\scriptsize 96}$,    
J.~Barreiro~Guimar\~{a}es~da~Costa$^\textrm{\scriptsize 15a}$,    
R.~Bartoldus$^\textrm{\scriptsize 150}$,    
A.E.~Barton$^\textrm{\scriptsize 87}$,    
P.~Bartos$^\textrm{\scriptsize 28a}$,    
A.~Basalaev$^\textrm{\scriptsize 134}$,    
A.~Bassalat$^\textrm{\scriptsize 128}$,    
R.L.~Bates$^\textrm{\scriptsize 55}$,    
S.J.~Batista$^\textrm{\scriptsize 164}$,    
S.~Batlamous$^\textrm{\scriptsize 34e}$,    
J.R.~Batley$^\textrm{\scriptsize 31}$,    
M.~Battaglia$^\textrm{\scriptsize 143}$,    
M.~Bauce$^\textrm{\scriptsize 70a,70b}$,    
F.~Bauer$^\textrm{\scriptsize 142}$,    
K.T.~Bauer$^\textrm{\scriptsize 168}$,    
H.S.~Bawa$^\textrm{\scriptsize 150}$,    
J.B.~Beacham$^\textrm{\scriptsize 122}$,    
T.~Beau$^\textrm{\scriptsize 132}$,    
P.H.~Beauchemin$^\textrm{\scriptsize 167}$,    
P.~Bechtle$^\textrm{\scriptsize 24}$,    
H.C.~Beck$^\textrm{\scriptsize 51}$,    
H.P.~Beck$^\textrm{\scriptsize 20,r}$,    
K.~Becker$^\textrm{\scriptsize 50}$,    
M.~Becker$^\textrm{\scriptsize 97}$,    
C.~Becot$^\textrm{\scriptsize 44}$,    
A.~Beddall$^\textrm{\scriptsize 12d}$,    
A.J.~Beddall$^\textrm{\scriptsize 12a}$,    
V.A.~Bednyakov$^\textrm{\scriptsize 77}$,    
M.~Bedognetti$^\textrm{\scriptsize 118}$,    
C.P.~Bee$^\textrm{\scriptsize 152}$,    
T.A.~Beermann$^\textrm{\scriptsize 74}$,    
M.~Begalli$^\textrm{\scriptsize 78b}$,    
M.~Begel$^\textrm{\scriptsize 29}$,    
A.~Behera$^\textrm{\scriptsize 152}$,    
J.K.~Behr$^\textrm{\scriptsize 44}$,    
A.S.~Bell$^\textrm{\scriptsize 92}$,    
G.~Bella$^\textrm{\scriptsize 158}$,    
L.~Bellagamba$^\textrm{\scriptsize 23b}$,    
A.~Bellerive$^\textrm{\scriptsize 33}$,    
M.~Bellomo$^\textrm{\scriptsize 157}$,    
P.~Bellos$^\textrm{\scriptsize 9}$,    
K.~Belotskiy$^\textrm{\scriptsize 110}$,    
N.L.~Belyaev$^\textrm{\scriptsize 110}$,    
O.~Benary$^\textrm{\scriptsize 158,*}$,    
D.~Benchekroun$^\textrm{\scriptsize 34a}$,    
M.~Bender$^\textrm{\scriptsize 112}$,    
N.~Benekos$^\textrm{\scriptsize 10}$,    
Y.~Benhammou$^\textrm{\scriptsize 158}$,    
E.~Benhar~Noccioli$^\textrm{\scriptsize 180}$,    
J.~Benitez$^\textrm{\scriptsize 75}$,    
D.P.~Benjamin$^\textrm{\scriptsize 6}$,    
M.~Benoit$^\textrm{\scriptsize 52}$,    
J.R.~Bensinger$^\textrm{\scriptsize 26}$,    
S.~Bentvelsen$^\textrm{\scriptsize 118}$,    
L.~Beresford$^\textrm{\scriptsize 131}$,    
M.~Beretta$^\textrm{\scriptsize 49}$,    
D.~Berge$^\textrm{\scriptsize 44}$,    
E.~Bergeaas~Kuutmann$^\textrm{\scriptsize 169}$,    
N.~Berger$^\textrm{\scriptsize 5}$,    
L.J.~Bergsten$^\textrm{\scriptsize 26}$,    
J.~Beringer$^\textrm{\scriptsize 18}$,    
S.~Berlendis$^\textrm{\scriptsize 7}$,    
N.R.~Bernard$^\textrm{\scriptsize 100}$,    
G.~Bernardi$^\textrm{\scriptsize 132}$,    
C.~Bernius$^\textrm{\scriptsize 150}$,    
F.U.~Bernlochner$^\textrm{\scriptsize 24}$,    
T.~Berry$^\textrm{\scriptsize 91}$,    
P.~Berta$^\textrm{\scriptsize 97}$,    
C.~Bertella$^\textrm{\scriptsize 15a}$,    
G.~Bertoli$^\textrm{\scriptsize 43a,43b}$,    
I.A.~Bertram$^\textrm{\scriptsize 87}$,    
G.J.~Besjes$^\textrm{\scriptsize 39}$,    
O.~Bessidskaia~Bylund$^\textrm{\scriptsize 179}$,    
M.~Bessner$^\textrm{\scriptsize 44}$,    
N.~Besson$^\textrm{\scriptsize 142}$,    
A.~Bethani$^\textrm{\scriptsize 98}$,    
S.~Bethke$^\textrm{\scriptsize 113}$,    
A.~Betti$^\textrm{\scriptsize 24}$,    
A.J.~Bevan$^\textrm{\scriptsize 90}$,    
J.~Beyer$^\textrm{\scriptsize 113}$,    
R.~Bi$^\textrm{\scriptsize 135}$,    
R.M.~Bianchi$^\textrm{\scriptsize 135}$,    
O.~Biebel$^\textrm{\scriptsize 112}$,    
D.~Biedermann$^\textrm{\scriptsize 19}$,    
R.~Bielski$^\textrm{\scriptsize 35}$,    
K.~Bierwagen$^\textrm{\scriptsize 97}$,    
N.V.~Biesuz$^\textrm{\scriptsize 69a,69b}$,    
M.~Biglietti$^\textrm{\scriptsize 72a}$,    
T.R.V.~Billoud$^\textrm{\scriptsize 107}$,    
M.~Bindi$^\textrm{\scriptsize 51}$,    
A.~Bingul$^\textrm{\scriptsize 12d}$,    
C.~Bini$^\textrm{\scriptsize 70a,70b}$,    
S.~Biondi$^\textrm{\scriptsize 23b,23a}$,    
M.~Birman$^\textrm{\scriptsize 177}$,    
T.~Bisanz$^\textrm{\scriptsize 51}$,    
J.P.~Biswal$^\textrm{\scriptsize 158}$,    
C.~Bittrich$^\textrm{\scriptsize 46}$,    
D.M.~Bjergaard$^\textrm{\scriptsize 47}$,    
J.E.~Black$^\textrm{\scriptsize 150}$,    
K.M.~Black$^\textrm{\scriptsize 25}$,    
T.~Blazek$^\textrm{\scriptsize 28a}$,    
I.~Bloch$^\textrm{\scriptsize 44}$,    
C.~Blocker$^\textrm{\scriptsize 26}$,    
A.~Blue$^\textrm{\scriptsize 55}$,    
U.~Blumenschein$^\textrm{\scriptsize 90}$,    
Dr.~Blunier$^\textrm{\scriptsize 144a}$,    
G.J.~Bobbink$^\textrm{\scriptsize 118}$,    
V.S.~Bobrovnikov$^\textrm{\scriptsize 120b,120a}$,    
S.S.~Bocchetta$^\textrm{\scriptsize 94}$,    
A.~Bocci$^\textrm{\scriptsize 47}$,    
D.~Boerner$^\textrm{\scriptsize 179}$,    
D.~Bogavac$^\textrm{\scriptsize 112}$,    
A.G.~Bogdanchikov$^\textrm{\scriptsize 120b,120a}$,    
C.~Bohm$^\textrm{\scriptsize 43a}$,    
V.~Boisvert$^\textrm{\scriptsize 91}$,    
P.~Bokan$^\textrm{\scriptsize 51,169}$,    
T.~Bold$^\textrm{\scriptsize 81a}$,    
A.S.~Boldyrev$^\textrm{\scriptsize 111}$,    
A.E.~Bolz$^\textrm{\scriptsize 59b}$,    
M.~Bomben$^\textrm{\scriptsize 132}$,    
M.~Bona$^\textrm{\scriptsize 90}$,    
J.S.~Bonilla$^\textrm{\scriptsize 127}$,    
M.~Boonekamp$^\textrm{\scriptsize 142}$,    
H.M.~Borecka-Bielska$^\textrm{\scriptsize 88}$,    
A.~Borisov$^\textrm{\scriptsize 140}$,    
G.~Borissov$^\textrm{\scriptsize 87}$,    
J.~Bortfeldt$^\textrm{\scriptsize 35}$,    
D.~Bortoletto$^\textrm{\scriptsize 131}$,    
V.~Bortolotto$^\textrm{\scriptsize 71a,71b}$,    
D.~Boscherini$^\textrm{\scriptsize 23b}$,    
M.~Bosman$^\textrm{\scriptsize 14}$,    
J.D.~Bossio~Sola$^\textrm{\scriptsize 30}$,    
K.~Bouaouda$^\textrm{\scriptsize 34a}$,    
J.~Boudreau$^\textrm{\scriptsize 135}$,    
E.V.~Bouhova-Thacker$^\textrm{\scriptsize 87}$,    
D.~Boumediene$^\textrm{\scriptsize 37}$,    
C.~Bourdarios$^\textrm{\scriptsize 128}$,    
S.K.~Boutle$^\textrm{\scriptsize 55}$,    
A.~Boveia$^\textrm{\scriptsize 122}$,    
J.~Boyd$^\textrm{\scriptsize 35}$,    
D.~Boye$^\textrm{\scriptsize 32b}$,    
I.R.~Boyko$^\textrm{\scriptsize 77}$,    
A.J.~Bozson$^\textrm{\scriptsize 91}$,    
J.~Bracinik$^\textrm{\scriptsize 21}$,    
N.~Brahimi$^\textrm{\scriptsize 99}$,    
A.~Brandt$^\textrm{\scriptsize 8}$,    
G.~Brandt$^\textrm{\scriptsize 179}$,    
O.~Brandt$^\textrm{\scriptsize 59a}$,    
F.~Braren$^\textrm{\scriptsize 44}$,    
U.~Bratzler$^\textrm{\scriptsize 161}$,    
B.~Brau$^\textrm{\scriptsize 100}$,    
J.E.~Brau$^\textrm{\scriptsize 127}$,    
W.D.~Breaden~Madden$^\textrm{\scriptsize 55}$,    
K.~Brendlinger$^\textrm{\scriptsize 44}$,    
L.~Brenner$^\textrm{\scriptsize 44}$,    
R.~Brenner$^\textrm{\scriptsize 169}$,    
S.~Bressler$^\textrm{\scriptsize 177}$,    
B.~Brickwedde$^\textrm{\scriptsize 97}$,    
D.L.~Briglin$^\textrm{\scriptsize 21}$,    
D.~Britton$^\textrm{\scriptsize 55}$,    
D.~Britzger$^\textrm{\scriptsize 113}$,    
I.~Brock$^\textrm{\scriptsize 24}$,    
R.~Brock$^\textrm{\scriptsize 104}$,    
G.~Brooijmans$^\textrm{\scriptsize 38}$,    
T.~Brooks$^\textrm{\scriptsize 91}$,    
W.K.~Brooks$^\textrm{\scriptsize 144b}$,    
E.~Brost$^\textrm{\scriptsize 119}$,    
J.H~Broughton$^\textrm{\scriptsize 21}$,    
P.A.~Bruckman~de~Renstrom$^\textrm{\scriptsize 82}$,    
D.~Bruncko$^\textrm{\scriptsize 28b}$,    
A.~Bruni$^\textrm{\scriptsize 23b}$,    
G.~Bruni$^\textrm{\scriptsize 23b}$,    
L.S.~Bruni$^\textrm{\scriptsize 118}$,    
S.~Bruno$^\textrm{\scriptsize 71a,71b}$,    
B.H.~Brunt$^\textrm{\scriptsize 31}$,    
M.~Bruschi$^\textrm{\scriptsize 23b}$,    
N.~Bruscino$^\textrm{\scriptsize 135}$,    
P.~Bryant$^\textrm{\scriptsize 36}$,    
L.~Bryngemark$^\textrm{\scriptsize 94}$,    
T.~Buanes$^\textrm{\scriptsize 17}$,    
Q.~Buat$^\textrm{\scriptsize 35}$,    
P.~Buchholz$^\textrm{\scriptsize 148}$,    
A.G.~Buckley$^\textrm{\scriptsize 55}$,    
I.A.~Budagov$^\textrm{\scriptsize 77}$,    
F.~Buehrer$^\textrm{\scriptsize 50}$,    
M.K.~Bugge$^\textrm{\scriptsize 130}$,    
O.~Bulekov$^\textrm{\scriptsize 110}$,    
D.~Bullock$^\textrm{\scriptsize 8}$,    
T.J.~Burch$^\textrm{\scriptsize 119}$,    
S.~Burdin$^\textrm{\scriptsize 88}$,    
C.D.~Burgard$^\textrm{\scriptsize 118}$,    
A.M.~Burger$^\textrm{\scriptsize 5}$,    
B.~Burghgrave$^\textrm{\scriptsize 119}$,    
K.~Burka$^\textrm{\scriptsize 82}$,    
S.~Burke$^\textrm{\scriptsize 141}$,    
I.~Burmeister$^\textrm{\scriptsize 45}$,    
J.T.P.~Burr$^\textrm{\scriptsize 131}$,    
V.~B\"uscher$^\textrm{\scriptsize 97}$,    
E.~Buschmann$^\textrm{\scriptsize 51}$,    
P.~Bussey$^\textrm{\scriptsize 55}$,    
J.M.~Butler$^\textrm{\scriptsize 25}$,    
C.M.~Buttar$^\textrm{\scriptsize 55}$,    
J.M.~Butterworth$^\textrm{\scriptsize 92}$,    
P.~Butti$^\textrm{\scriptsize 35}$,    
W.~Buttinger$^\textrm{\scriptsize 35}$,    
A.~Buzatu$^\textrm{\scriptsize 155}$,    
A.R.~Buzykaev$^\textrm{\scriptsize 120b,120a}$,    
G.~Cabras$^\textrm{\scriptsize 23b,23a}$,    
S.~Cabrera~Urb\'an$^\textrm{\scriptsize 171}$,    
D.~Caforio$^\textrm{\scriptsize 138}$,    
H.~Cai$^\textrm{\scriptsize 170}$,    
V.M.M.~Cairo$^\textrm{\scriptsize 2}$,    
O.~Cakir$^\textrm{\scriptsize 4a}$,    
N.~Calace$^\textrm{\scriptsize 35}$,    
P.~Calafiura$^\textrm{\scriptsize 18}$,    
A.~Calandri$^\textrm{\scriptsize 99}$,    
G.~Calderini$^\textrm{\scriptsize 132}$,    
P.~Calfayan$^\textrm{\scriptsize 63}$,    
G.~Callea$^\textrm{\scriptsize 55}$,    
L.P.~Caloba$^\textrm{\scriptsize 78b}$,    
S.~Calvente~Lopez$^\textrm{\scriptsize 96}$,    
D.~Calvet$^\textrm{\scriptsize 37}$,    
S.~Calvet$^\textrm{\scriptsize 37}$,    
T.P.~Calvet$^\textrm{\scriptsize 152}$,    
M.~Calvetti$^\textrm{\scriptsize 69a,69b}$,    
R.~Camacho~Toro$^\textrm{\scriptsize 132}$,    
S.~Camarda$^\textrm{\scriptsize 35}$,    
D.~Camarero~Munoz$^\textrm{\scriptsize 96}$,    
P.~Camarri$^\textrm{\scriptsize 71a,71b}$,    
D.~Cameron$^\textrm{\scriptsize 130}$,    
R.~Caminal~Armadans$^\textrm{\scriptsize 100}$,    
C.~Camincher$^\textrm{\scriptsize 35}$,    
S.~Campana$^\textrm{\scriptsize 35}$,    
M.~Campanelli$^\textrm{\scriptsize 92}$,    
A.~Camplani$^\textrm{\scriptsize 39}$,    
A.~Campoverde$^\textrm{\scriptsize 148}$,    
V.~Canale$^\textrm{\scriptsize 67a,67b}$,    
M.~Cano~Bret$^\textrm{\scriptsize 58c}$,    
J.~Cantero$^\textrm{\scriptsize 125}$,    
T.~Cao$^\textrm{\scriptsize 158}$,    
Y.~Cao$^\textrm{\scriptsize 170}$,    
M.D.M.~Capeans~Garrido$^\textrm{\scriptsize 35}$,    
I.~Caprini$^\textrm{\scriptsize 27b}$,    
M.~Caprini$^\textrm{\scriptsize 27b}$,    
M.~Capua$^\textrm{\scriptsize 40b,40a}$,    
R.M.~Carbone$^\textrm{\scriptsize 38}$,    
R.~Cardarelli$^\textrm{\scriptsize 71a}$,    
F.C.~Cardillo$^\textrm{\scriptsize 146}$,    
I.~Carli$^\textrm{\scriptsize 139}$,    
T.~Carli$^\textrm{\scriptsize 35}$,    
G.~Carlino$^\textrm{\scriptsize 67a}$,    
B.T.~Carlson$^\textrm{\scriptsize 135}$,    
L.~Carminati$^\textrm{\scriptsize 66a,66b}$,    
R.M.D.~Carney$^\textrm{\scriptsize 43a,43b}$,    
S.~Caron$^\textrm{\scriptsize 117}$,    
E.~Carquin$^\textrm{\scriptsize 144b}$,    
S.~Carr\'a$^\textrm{\scriptsize 66a,66b}$,    
J.W.S.~Carter$^\textrm{\scriptsize 164}$,    
D.~Casadei$^\textrm{\scriptsize 32b}$,    
M.P.~Casado$^\textrm{\scriptsize 14,g}$,    
A.F.~Casha$^\textrm{\scriptsize 164}$,    
D.W.~Casper$^\textrm{\scriptsize 168}$,    
R.~Castelijn$^\textrm{\scriptsize 118}$,    
F.L.~Castillo$^\textrm{\scriptsize 171}$,    
V.~Castillo~Gimenez$^\textrm{\scriptsize 171}$,    
N.F.~Castro$^\textrm{\scriptsize 136a,136e}$,    
A.~Catinaccio$^\textrm{\scriptsize 35}$,    
J.R.~Catmore$^\textrm{\scriptsize 130}$,    
A.~Cattai$^\textrm{\scriptsize 35}$,    
J.~Caudron$^\textrm{\scriptsize 24}$,    
V.~Cavaliere$^\textrm{\scriptsize 29}$,    
E.~Cavallaro$^\textrm{\scriptsize 14}$,    
D.~Cavalli$^\textrm{\scriptsize 66a}$,    
M.~Cavalli-Sforza$^\textrm{\scriptsize 14}$,    
V.~Cavasinni$^\textrm{\scriptsize 69a,69b}$,    
E.~Celebi$^\textrm{\scriptsize 12b}$,    
F.~Ceradini$^\textrm{\scriptsize 72a,72b}$,    
L.~Cerda~Alberich$^\textrm{\scriptsize 171}$,    
A.S.~Cerqueira$^\textrm{\scriptsize 78a}$,    
A.~Cerri$^\textrm{\scriptsize 153}$,    
L.~Cerrito$^\textrm{\scriptsize 71a,71b}$,    
F.~Cerutti$^\textrm{\scriptsize 18}$,    
A.~Cervelli$^\textrm{\scriptsize 23b,23a}$,    
S.A.~Cetin$^\textrm{\scriptsize 12b}$,    
A.~Chafaq$^\textrm{\scriptsize 34a}$,    
D~Chakraborty$^\textrm{\scriptsize 119}$,    
S.K.~Chan$^\textrm{\scriptsize 57}$,    
W.S.~Chan$^\textrm{\scriptsize 118}$,    
Y.L.~Chan$^\textrm{\scriptsize 61a}$,    
J.D.~Chapman$^\textrm{\scriptsize 31}$,    
B.~Chargeishvili$^\textrm{\scriptsize 156b}$,    
D.G.~Charlton$^\textrm{\scriptsize 21}$,    
C.C.~Chau$^\textrm{\scriptsize 33}$,    
C.A.~Chavez~Barajas$^\textrm{\scriptsize 153}$,    
S.~Che$^\textrm{\scriptsize 122}$,    
A.~Chegwidden$^\textrm{\scriptsize 104}$,    
S.~Chekanov$^\textrm{\scriptsize 6}$,    
S.V.~Chekulaev$^\textrm{\scriptsize 165a}$,    
G.A.~Chelkov$^\textrm{\scriptsize 77,as}$,    
M.A.~Chelstowska$^\textrm{\scriptsize 35}$,    
C.~Chen$^\textrm{\scriptsize 58a}$,    
C.H.~Chen$^\textrm{\scriptsize 76}$,    
H.~Chen$^\textrm{\scriptsize 29}$,    
J.~Chen$^\textrm{\scriptsize 58a}$,    
J.~Chen$^\textrm{\scriptsize 38}$,    
S.~Chen$^\textrm{\scriptsize 133}$,    
S.J.~Chen$^\textrm{\scriptsize 15c}$,    
X.~Chen$^\textrm{\scriptsize 15b,ar}$,    
Y.~Chen$^\textrm{\scriptsize 80}$,    
Y-H.~Chen$^\textrm{\scriptsize 44}$,    
H.C.~Cheng$^\textrm{\scriptsize 103}$,    
H.J.~Cheng$^\textrm{\scriptsize 15d}$,    
A.~Cheplakov$^\textrm{\scriptsize 77}$,    
E.~Cheremushkina$^\textrm{\scriptsize 140}$,    
R.~Cherkaoui~El~Moursli$^\textrm{\scriptsize 34e}$,    
E.~Cheu$^\textrm{\scriptsize 7}$,    
K.~Cheung$^\textrm{\scriptsize 62}$,    
T.J.A.~Cheval\'erias$^\textrm{\scriptsize 142}$,    
L.~Chevalier$^\textrm{\scriptsize 142}$,    
V.~Chiarella$^\textrm{\scriptsize 49}$,    
G.~Chiarelli$^\textrm{\scriptsize 69a}$,    
G.~Chiodini$^\textrm{\scriptsize 65a}$,    
A.S.~Chisholm$^\textrm{\scriptsize 35,21}$,    
A.~Chitan$^\textrm{\scriptsize 27b}$,    
I.~Chiu$^\textrm{\scriptsize 160}$,    
Y.H.~Chiu$^\textrm{\scriptsize 173}$,    
M.V.~Chizhov$^\textrm{\scriptsize 77}$,    
K.~Choi$^\textrm{\scriptsize 63}$,    
A.R.~Chomont$^\textrm{\scriptsize 128}$,    
S.~Chouridou$^\textrm{\scriptsize 159}$,    
Y.S.~Chow$^\textrm{\scriptsize 118}$,    
V.~Christodoulou$^\textrm{\scriptsize 92}$,    
M.C.~Chu$^\textrm{\scriptsize 61a}$,    
J.~Chudoba$^\textrm{\scriptsize 137}$,    
A.J.~Chuinard$^\textrm{\scriptsize 101}$,    
J.J.~Chwastowski$^\textrm{\scriptsize 82}$,    
L.~Chytka$^\textrm{\scriptsize 126}$,    
D.~Cinca$^\textrm{\scriptsize 45}$,    
V.~Cindro$^\textrm{\scriptsize 89}$,    
I.A.~Cioar\u{a}$^\textrm{\scriptsize 24}$,    
A.~Ciocio$^\textrm{\scriptsize 18}$,    
F.~Cirotto$^\textrm{\scriptsize 67a,67b}$,    
Z.H.~Citron$^\textrm{\scriptsize 177}$,    
M.~Citterio$^\textrm{\scriptsize 66a}$,    
A.~Clark$^\textrm{\scriptsize 52}$,    
M.R.~Clark$^\textrm{\scriptsize 38}$,    
P.J.~Clark$^\textrm{\scriptsize 48}$,    
C.~Clement$^\textrm{\scriptsize 43a,43b}$,    
Y.~Coadou$^\textrm{\scriptsize 99}$,    
M.~Cobal$^\textrm{\scriptsize 64a,64c}$,    
A.~Coccaro$^\textrm{\scriptsize 53b,53a}$,    
J.~Cochran$^\textrm{\scriptsize 76}$,    
H.~Cohen$^\textrm{\scriptsize 158}$,    
A.E.C.~Coimbra$^\textrm{\scriptsize 177}$,    
L.~Colasurdo$^\textrm{\scriptsize 117}$,    
B.~Cole$^\textrm{\scriptsize 38}$,    
A.P.~Colijn$^\textrm{\scriptsize 118}$,    
J.~Collot$^\textrm{\scriptsize 56}$,    
P.~Conde~Mui\~no$^\textrm{\scriptsize 136a,136b}$,    
E.~Coniavitis$^\textrm{\scriptsize 50}$,    
S.H.~Connell$^\textrm{\scriptsize 32b}$,    
I.A.~Connelly$^\textrm{\scriptsize 98}$,    
S.~Constantinescu$^\textrm{\scriptsize 27b}$,    
F.~Conventi$^\textrm{\scriptsize 67a,au}$,    
A.M.~Cooper-Sarkar$^\textrm{\scriptsize 131}$,    
F.~Cormier$^\textrm{\scriptsize 172}$,    
K.J.R.~Cormier$^\textrm{\scriptsize 164}$,    
L.D.~Corpe$^\textrm{\scriptsize 92}$,    
M.~Corradi$^\textrm{\scriptsize 70a,70b}$,    
E.E.~Corrigan$^\textrm{\scriptsize 94}$,    
F.~Corriveau$^\textrm{\scriptsize 101,ac}$,    
A.~Cortes-Gonzalez$^\textrm{\scriptsize 35}$,    
M.J.~Costa$^\textrm{\scriptsize 171}$,    
F.~Costanza$^\textrm{\scriptsize 5}$,    
D.~Costanzo$^\textrm{\scriptsize 146}$,    
G.~Cottin$^\textrm{\scriptsize 31}$,    
G.~Cowan$^\textrm{\scriptsize 91}$,    
B.E.~Cox$^\textrm{\scriptsize 98}$,    
J.~Crane$^\textrm{\scriptsize 98}$,    
K.~Cranmer$^\textrm{\scriptsize 121}$,    
S.J.~Crawley$^\textrm{\scriptsize 55}$,    
R.A.~Creager$^\textrm{\scriptsize 133}$,    
G.~Cree$^\textrm{\scriptsize 33}$,    
S.~Cr\'ep\'e-Renaudin$^\textrm{\scriptsize 56}$,    
F.~Crescioli$^\textrm{\scriptsize 132}$,    
M.~Cristinziani$^\textrm{\scriptsize 24}$,    
V.~Croft$^\textrm{\scriptsize 121}$,    
G.~Crosetti$^\textrm{\scriptsize 40b,40a}$,    
A.~Cueto$^\textrm{\scriptsize 96}$,    
T.~Cuhadar~Donszelmann$^\textrm{\scriptsize 146}$,    
A.R.~Cukierman$^\textrm{\scriptsize 150}$,    
S.~Czekierda$^\textrm{\scriptsize 82}$,    
P.~Czodrowski$^\textrm{\scriptsize 35}$,    
M.J.~Da~Cunha~Sargedas~De~Sousa$^\textrm{\scriptsize 58b,136b}$,    
C.~Da~Via$^\textrm{\scriptsize 98}$,    
W.~Dabrowski$^\textrm{\scriptsize 81a}$,    
T.~Dado$^\textrm{\scriptsize 28a,x}$,    
S.~Dahbi$^\textrm{\scriptsize 34e}$,    
T.~Dai$^\textrm{\scriptsize 103}$,    
F.~Dallaire$^\textrm{\scriptsize 107}$,    
C.~Dallapiccola$^\textrm{\scriptsize 100}$,    
M.~Dam$^\textrm{\scriptsize 39}$,    
G.~D'amen$^\textrm{\scriptsize 23b,23a}$,    
J.~Damp$^\textrm{\scriptsize 97}$,    
J.R.~Dandoy$^\textrm{\scriptsize 133}$,    
M.F.~Daneri$^\textrm{\scriptsize 30}$,    
N.P.~Dang$^\textrm{\scriptsize 178,k}$,    
N.D~Dann$^\textrm{\scriptsize 98}$,    
M.~Danninger$^\textrm{\scriptsize 172}$,    
V.~Dao$^\textrm{\scriptsize 35}$,    
G.~Darbo$^\textrm{\scriptsize 53b}$,    
S.~Darmora$^\textrm{\scriptsize 8}$,    
O.~Dartsi$^\textrm{\scriptsize 5}$,    
A.~Dattagupta$^\textrm{\scriptsize 127}$,    
T.~Daubney$^\textrm{\scriptsize 44}$,    
S.~D'Auria$^\textrm{\scriptsize 66a,66b}$,    
W.~Davey$^\textrm{\scriptsize 24}$,    
C.~David$^\textrm{\scriptsize 44}$,    
T.~Davidek$^\textrm{\scriptsize 139}$,    
D.R.~Davis$^\textrm{\scriptsize 47}$,    
E.~Dawe$^\textrm{\scriptsize 102}$,    
I.~Dawson$^\textrm{\scriptsize 146}$,    
K.~De$^\textrm{\scriptsize 8}$,    
R.~De~Asmundis$^\textrm{\scriptsize 67a}$,    
A.~De~Benedetti$^\textrm{\scriptsize 124}$,    
M.~De~Beurs$^\textrm{\scriptsize 118}$,    
S.~De~Castro$^\textrm{\scriptsize 23b,23a}$,    
S.~De~Cecco$^\textrm{\scriptsize 70a,70b}$,    
N.~De~Groot$^\textrm{\scriptsize 117}$,    
P.~de~Jong$^\textrm{\scriptsize 118}$,    
H.~De~la~Torre$^\textrm{\scriptsize 104}$,    
F.~De~Lorenzi$^\textrm{\scriptsize 76}$,    
A.~De~Maria$^\textrm{\scriptsize 69a,69b}$,    
D.~De~Pedis$^\textrm{\scriptsize 70a}$,    
A.~De~Salvo$^\textrm{\scriptsize 70a}$,    
U.~De~Sanctis$^\textrm{\scriptsize 71a,71b}$,    
M.~De~Santis$^\textrm{\scriptsize 71a,71b}$,    
A.~De~Santo$^\textrm{\scriptsize 153}$,    
K.~De~Vasconcelos~Corga$^\textrm{\scriptsize 99}$,    
J.B.~De~Vivie~De~Regie$^\textrm{\scriptsize 128}$,    
C.~Debenedetti$^\textrm{\scriptsize 143}$,    
D.V.~Dedovich$^\textrm{\scriptsize 77}$,    
N.~Dehghanian$^\textrm{\scriptsize 3}$,    
M.~Del~Gaudio$^\textrm{\scriptsize 40b,40a}$,    
J.~Del~Peso$^\textrm{\scriptsize 96}$,    
Y.~Delabat~Diaz$^\textrm{\scriptsize 44}$,    
D.~Delgove$^\textrm{\scriptsize 128}$,    
F.~Deliot$^\textrm{\scriptsize 142}$,    
C.M.~Delitzsch$^\textrm{\scriptsize 7}$,    
M.~Della~Pietra$^\textrm{\scriptsize 67a,67b}$,    
D.~Della~Volpe$^\textrm{\scriptsize 52}$,    
A.~Dell'Acqua$^\textrm{\scriptsize 35}$,    
L.~Dell'Asta$^\textrm{\scriptsize 25}$,    
M.~Delmastro$^\textrm{\scriptsize 5}$,    
C.~Delporte$^\textrm{\scriptsize 128}$,    
P.A.~Delsart$^\textrm{\scriptsize 56}$,    
D.A.~DeMarco$^\textrm{\scriptsize 164}$,    
S.~Demers$^\textrm{\scriptsize 180}$,    
M.~Demichev$^\textrm{\scriptsize 77}$,    
S.P.~Denisov$^\textrm{\scriptsize 140}$,    
D.~Denysiuk$^\textrm{\scriptsize 118}$,    
L.~D'Eramo$^\textrm{\scriptsize 132}$,    
D.~Derendarz$^\textrm{\scriptsize 82}$,    
J.E.~Derkaoui$^\textrm{\scriptsize 34d}$,    
F.~Derue$^\textrm{\scriptsize 132}$,    
P.~Dervan$^\textrm{\scriptsize 88}$,    
K.~Desch$^\textrm{\scriptsize 24}$,    
C.~Deterre$^\textrm{\scriptsize 44}$,    
K.~Dette$^\textrm{\scriptsize 164}$,    
M.R.~Devesa$^\textrm{\scriptsize 30}$,    
P.O.~Deviveiros$^\textrm{\scriptsize 35}$,    
A.~Dewhurst$^\textrm{\scriptsize 141}$,    
S.~Dhaliwal$^\textrm{\scriptsize 26}$,    
F.A.~Di~Bello$^\textrm{\scriptsize 52}$,    
A.~Di~Ciaccio$^\textrm{\scriptsize 71a,71b}$,    
L.~Di~Ciaccio$^\textrm{\scriptsize 5}$,    
W.K.~Di~Clemente$^\textrm{\scriptsize 133}$,    
C.~Di~Donato$^\textrm{\scriptsize 67a,67b}$,    
A.~Di~Girolamo$^\textrm{\scriptsize 35}$,    
G.~Di~Gregorio$^\textrm{\scriptsize 69a,69b}$,    
B.~Di~Micco$^\textrm{\scriptsize 72a,72b}$,    
R.~Di~Nardo$^\textrm{\scriptsize 100}$,    
K.F.~Di~Petrillo$^\textrm{\scriptsize 57}$,    
R.~Di~Sipio$^\textrm{\scriptsize 164}$,    
D.~Di~Valentino$^\textrm{\scriptsize 33}$,    
C.~Diaconu$^\textrm{\scriptsize 99}$,    
M.~Diamond$^\textrm{\scriptsize 164}$,    
F.A.~Dias$^\textrm{\scriptsize 39}$,    
T.~Dias~Do~Vale$^\textrm{\scriptsize 136a}$,    
M.A.~Diaz$^\textrm{\scriptsize 144a}$,    
J.~Dickinson$^\textrm{\scriptsize 18}$,    
E.B.~Diehl$^\textrm{\scriptsize 103}$,    
J.~Dietrich$^\textrm{\scriptsize 19}$,    
S.~D\'iez~Cornell$^\textrm{\scriptsize 44}$,    
A.~Dimitrievska$^\textrm{\scriptsize 18}$,    
J.~Dingfelder$^\textrm{\scriptsize 24}$,    
F.~Dittus$^\textrm{\scriptsize 35}$,    
F.~Djama$^\textrm{\scriptsize 99}$,    
T.~Djobava$^\textrm{\scriptsize 156b}$,    
J.I.~Djuvsland$^\textrm{\scriptsize 17}$,    
M.A.B.~Do~Vale$^\textrm{\scriptsize 78c}$,    
M.~Dobre$^\textrm{\scriptsize 27b}$,    
D.~Dodsworth$^\textrm{\scriptsize 26}$,    
C.~Doglioni$^\textrm{\scriptsize 94}$,    
J.~Dolejsi$^\textrm{\scriptsize 139}$,    
Z.~Dolezal$^\textrm{\scriptsize 139}$,    
M.~Donadelli$^\textrm{\scriptsize 78d}$,    
J.~Donini$^\textrm{\scriptsize 37}$,    
A.~D'onofrio$^\textrm{\scriptsize 90}$,    
M.~D'Onofrio$^\textrm{\scriptsize 88}$,    
J.~Dopke$^\textrm{\scriptsize 141}$,    
A.~Doria$^\textrm{\scriptsize 67a}$,    
M.T.~Dova$^\textrm{\scriptsize 86}$,    
A.T.~Doyle$^\textrm{\scriptsize 55}$,    
E.~Drechsler$^\textrm{\scriptsize 149}$,    
E.~Dreyer$^\textrm{\scriptsize 149}$,    
T.~Dreyer$^\textrm{\scriptsize 51}$,    
Y.~Du$^\textrm{\scriptsize 58b}$,    
F.~Dubinin$^\textrm{\scriptsize 108}$,    
M.~Dubovsky$^\textrm{\scriptsize 28a}$,    
A.~Dubreuil$^\textrm{\scriptsize 52}$,    
E.~Duchovni$^\textrm{\scriptsize 177}$,    
G.~Duckeck$^\textrm{\scriptsize 112}$,    
A.~Ducourthial$^\textrm{\scriptsize 132}$,    
O.A.~Ducu$^\textrm{\scriptsize 107,w}$,    
D.~Duda$^\textrm{\scriptsize 113}$,    
A.~Dudarev$^\textrm{\scriptsize 35}$,    
A.C.~Dudder$^\textrm{\scriptsize 97}$,    
E.M.~Duffield$^\textrm{\scriptsize 18}$,    
L.~Duflot$^\textrm{\scriptsize 128}$,    
M.~D\"uhrssen$^\textrm{\scriptsize 35}$,    
C.~D{\"u}lsen$^\textrm{\scriptsize 179}$,    
M.~Dumancic$^\textrm{\scriptsize 177}$,    
A.E.~Dumitriu$^\textrm{\scriptsize 27b,e}$,    
A.K.~Duncan$^\textrm{\scriptsize 55}$,    
M.~Dunford$^\textrm{\scriptsize 59a}$,    
A.~Duperrin$^\textrm{\scriptsize 99}$,    
H.~Duran~Yildiz$^\textrm{\scriptsize 4a}$,    
M.~D\"uren$^\textrm{\scriptsize 54}$,    
A.~Durglishvili$^\textrm{\scriptsize 156b}$,    
D.~Duschinger$^\textrm{\scriptsize 46}$,    
B.~Dutta$^\textrm{\scriptsize 44}$,    
D.~Duvnjak$^\textrm{\scriptsize 1}$,    
M.~Dyndal$^\textrm{\scriptsize 44}$,    
S.~Dysch$^\textrm{\scriptsize 98}$,    
B.S.~Dziedzic$^\textrm{\scriptsize 82}$,    
K.M.~Ecker$^\textrm{\scriptsize 113}$,    
R.C.~Edgar$^\textrm{\scriptsize 103}$,    
T.~Eifert$^\textrm{\scriptsize 35}$,    
G.~Eigen$^\textrm{\scriptsize 17}$,    
K.~Einsweiler$^\textrm{\scriptsize 18}$,    
T.~Ekelof$^\textrm{\scriptsize 169}$,    
M.~El~Kacimi$^\textrm{\scriptsize 34c}$,    
R.~El~Kosseifi$^\textrm{\scriptsize 99}$,    
V.~Ellajosyula$^\textrm{\scriptsize 99}$,    
M.~Ellert$^\textrm{\scriptsize 169}$,    
F.~Ellinghaus$^\textrm{\scriptsize 179}$,    
A.A.~Elliot$^\textrm{\scriptsize 90}$,    
N.~Ellis$^\textrm{\scriptsize 35}$,    
J.~Elmsheuser$^\textrm{\scriptsize 29}$,    
M.~Elsing$^\textrm{\scriptsize 35}$,    
D.~Emeliyanov$^\textrm{\scriptsize 141}$,    
A.~Emerman$^\textrm{\scriptsize 38}$,    
Y.~Enari$^\textrm{\scriptsize 160}$,    
J.S.~Ennis$^\textrm{\scriptsize 175}$,    
M.B.~Epland$^\textrm{\scriptsize 47}$,    
J.~Erdmann$^\textrm{\scriptsize 45}$,    
A.~Ereditato$^\textrm{\scriptsize 20}$,    
S.~Errede$^\textrm{\scriptsize 170}$,    
M.~Escalier$^\textrm{\scriptsize 128}$,    
C.~Escobar$^\textrm{\scriptsize 171}$,    
O.~Estrada~Pastor$^\textrm{\scriptsize 171}$,    
A.I.~Etienvre$^\textrm{\scriptsize 142}$,    
E.~Etzion$^\textrm{\scriptsize 158}$,    
H.~Evans$^\textrm{\scriptsize 63}$,    
A.~Ezhilov$^\textrm{\scriptsize 134}$,    
M.~Ezzi$^\textrm{\scriptsize 34e}$,    
F.~Fabbri$^\textrm{\scriptsize 55}$,    
L.~Fabbri$^\textrm{\scriptsize 23b,23a}$,    
V.~Fabiani$^\textrm{\scriptsize 117}$,    
G.~Facini$^\textrm{\scriptsize 92}$,    
R.M.~Faisca~Rodrigues~Pereira$^\textrm{\scriptsize 136a}$,    
R.M.~Fakhrutdinov$^\textrm{\scriptsize 140}$,    
S.~Falciano$^\textrm{\scriptsize 70a}$,    
P.J.~Falke$^\textrm{\scriptsize 5}$,    
S.~Falke$^\textrm{\scriptsize 5}$,    
J.~Faltova$^\textrm{\scriptsize 139}$,    
Y.~Fang$^\textrm{\scriptsize 15a}$,    
M.~Fanti$^\textrm{\scriptsize 66a,66b}$,    
A.~Farbin$^\textrm{\scriptsize 8}$,    
A.~Farilla$^\textrm{\scriptsize 72a}$,    
E.M.~Farina$^\textrm{\scriptsize 68a,68b}$,    
T.~Farooque$^\textrm{\scriptsize 104}$,    
S.~Farrell$^\textrm{\scriptsize 18}$,    
S.M.~Farrington$^\textrm{\scriptsize 175}$,    
P.~Farthouat$^\textrm{\scriptsize 35}$,    
F.~Fassi$^\textrm{\scriptsize 34e}$,    
P.~Fassnacht$^\textrm{\scriptsize 35}$,    
D.~Fassouliotis$^\textrm{\scriptsize 9}$,    
M.~Faucci~Giannelli$^\textrm{\scriptsize 48}$,    
W.J.~Fawcett$^\textrm{\scriptsize 31}$,    
L.~Fayard$^\textrm{\scriptsize 128}$,    
O.L.~Fedin$^\textrm{\scriptsize 134,p}$,    
W.~Fedorko$^\textrm{\scriptsize 172}$,    
M.~Feickert$^\textrm{\scriptsize 41}$,    
S.~Feigl$^\textrm{\scriptsize 130}$,    
L.~Feligioni$^\textrm{\scriptsize 99}$,    
C.~Feng$^\textrm{\scriptsize 58b}$,    
E.J.~Feng$^\textrm{\scriptsize 35}$,    
M.~Feng$^\textrm{\scriptsize 47}$,    
M.J.~Fenton$^\textrm{\scriptsize 55}$,    
A.B.~Fenyuk$^\textrm{\scriptsize 140}$,    
J.~Ferrando$^\textrm{\scriptsize 44}$,    
A.~Ferrari$^\textrm{\scriptsize 169}$,    
P.~Ferrari$^\textrm{\scriptsize 118}$,    
R.~Ferrari$^\textrm{\scriptsize 68a}$,    
D.E.~Ferreira~de~Lima$^\textrm{\scriptsize 59b}$,    
A.~Ferrer$^\textrm{\scriptsize 171}$,    
D.~Ferrere$^\textrm{\scriptsize 52}$,    
C.~Ferretti$^\textrm{\scriptsize 103}$,    
F.~Fiedler$^\textrm{\scriptsize 97}$,    
A.~Filip\v{c}i\v{c}$^\textrm{\scriptsize 89}$,    
F.~Filthaut$^\textrm{\scriptsize 117}$,    
K.D.~Finelli$^\textrm{\scriptsize 25}$,    
M.C.N.~Fiolhais$^\textrm{\scriptsize 136a,136c,a}$,    
L.~Fiorini$^\textrm{\scriptsize 171}$,    
C.~Fischer$^\textrm{\scriptsize 14}$,    
W.C.~Fisher$^\textrm{\scriptsize 104}$,    
N.~Flaschel$^\textrm{\scriptsize 44}$,    
I.~Fleck$^\textrm{\scriptsize 148}$,    
P.~Fleischmann$^\textrm{\scriptsize 103}$,    
R.R.M.~Fletcher$^\textrm{\scriptsize 133}$,    
T.~Flick$^\textrm{\scriptsize 179}$,    
B.M.~Flierl$^\textrm{\scriptsize 112}$,    
L.M.~Flores$^\textrm{\scriptsize 133}$,    
L.R.~Flores~Castillo$^\textrm{\scriptsize 61a}$,    
F.M.~Follega$^\textrm{\scriptsize 73a,73b}$,    
N.~Fomin$^\textrm{\scriptsize 17}$,    
G.T.~Forcolin$^\textrm{\scriptsize 73a,73b}$,    
A.~Formica$^\textrm{\scriptsize 142}$,    
F.A.~F\"orster$^\textrm{\scriptsize 14}$,    
A.C.~Forti$^\textrm{\scriptsize 98}$,    
A.G.~Foster$^\textrm{\scriptsize 21}$,    
D.~Fournier$^\textrm{\scriptsize 128}$,    
H.~Fox$^\textrm{\scriptsize 87}$,    
S.~Fracchia$^\textrm{\scriptsize 146}$,    
P.~Francavilla$^\textrm{\scriptsize 69a,69b}$,    
M.~Franchini$^\textrm{\scriptsize 23b,23a}$,    
S.~Franchino$^\textrm{\scriptsize 59a}$,    
D.~Francis$^\textrm{\scriptsize 35}$,    
L.~Franconi$^\textrm{\scriptsize 143}$,    
M.~Franklin$^\textrm{\scriptsize 57}$,    
M.~Frate$^\textrm{\scriptsize 168}$,    
M.~Fraternali$^\textrm{\scriptsize 68a,68b}$,    
A.N.~Fray$^\textrm{\scriptsize 90}$,    
D.~Freeborn$^\textrm{\scriptsize 92}$,    
B.~Freund$^\textrm{\scriptsize 107}$,    
W.S.~Freund$^\textrm{\scriptsize 78b}$,    
E.M.~Freundlich$^\textrm{\scriptsize 45}$,    
D.C.~Frizzell$^\textrm{\scriptsize 124}$,    
D.~Froidevaux$^\textrm{\scriptsize 35}$,    
J.A.~Frost$^\textrm{\scriptsize 131}$,    
C.~Fukunaga$^\textrm{\scriptsize 161}$,    
E.~Fullana~Torregrosa$^\textrm{\scriptsize 171}$,    
E.~Fumagalli$^\textrm{\scriptsize 53b,53a}$,    
T.~Fusayasu$^\textrm{\scriptsize 114}$,    
J.~Fuster$^\textrm{\scriptsize 171}$,    
O.~Gabizon$^\textrm{\scriptsize 157}$,    
A.~Gabrielli$^\textrm{\scriptsize 23b,23a}$,    
A.~Gabrielli$^\textrm{\scriptsize 18}$,    
G.P.~Gach$^\textrm{\scriptsize 81a}$,    
S.~Gadatsch$^\textrm{\scriptsize 52}$,    
P.~Gadow$^\textrm{\scriptsize 113}$,    
G.~Gagliardi$^\textrm{\scriptsize 53b,53a}$,    
L.G.~Gagnon$^\textrm{\scriptsize 107}$,    
C.~Galea$^\textrm{\scriptsize 27b}$,    
B.~Galhardo$^\textrm{\scriptsize 136a,136c}$,    
E.J.~Gallas$^\textrm{\scriptsize 131}$,    
B.J.~Gallop$^\textrm{\scriptsize 141}$,    
P.~Gallus$^\textrm{\scriptsize 138}$,    
G.~Galster$^\textrm{\scriptsize 39}$,    
R.~Gamboa~Goni$^\textrm{\scriptsize 90}$,    
K.K.~Gan$^\textrm{\scriptsize 122}$,    
S.~Ganguly$^\textrm{\scriptsize 177}$,    
J.~Gao$^\textrm{\scriptsize 58a}$,    
Y.~Gao$^\textrm{\scriptsize 88}$,    
Y.S.~Gao$^\textrm{\scriptsize 150,m}$,    
C.~Garc\'ia$^\textrm{\scriptsize 171}$,    
J.E.~Garc\'ia~Navarro$^\textrm{\scriptsize 171}$,    
J.A.~Garc\'ia~Pascual$^\textrm{\scriptsize 15a}$,    
M.~Garcia-Sciveres$^\textrm{\scriptsize 18}$,    
R.W.~Gardner$^\textrm{\scriptsize 36}$,    
N.~Garelli$^\textrm{\scriptsize 150}$,    
S.~Gargiulo$^\textrm{\scriptsize 50}$,    
V.~Garonne$^\textrm{\scriptsize 130}$,    
K.~Gasnikova$^\textrm{\scriptsize 44}$,    
A.~Gaudiello$^\textrm{\scriptsize 53b,53a}$,    
G.~Gaudio$^\textrm{\scriptsize 68a}$,    
I.L.~Gavrilenko$^\textrm{\scriptsize 108}$,    
A.~Gavrilyuk$^\textrm{\scriptsize 109}$,    
C.~Gay$^\textrm{\scriptsize 172}$,    
G.~Gaycken$^\textrm{\scriptsize 24}$,    
E.N.~Gazis$^\textrm{\scriptsize 10}$,    
C.N.P.~Gee$^\textrm{\scriptsize 141}$,    
J.~Geisen$^\textrm{\scriptsize 51}$,    
M.~Geisen$^\textrm{\scriptsize 97}$,    
M.P.~Geisler$^\textrm{\scriptsize 59a}$,    
C.~Gemme$^\textrm{\scriptsize 53b}$,    
M.H.~Genest$^\textrm{\scriptsize 56}$,    
C.~Geng$^\textrm{\scriptsize 103}$,    
S.~Gentile$^\textrm{\scriptsize 70a,70b}$,    
S.~George$^\textrm{\scriptsize 91}$,    
D.~Gerbaudo$^\textrm{\scriptsize 14}$,    
G.~Gessner$^\textrm{\scriptsize 45}$,    
S.~Ghasemi$^\textrm{\scriptsize 148}$,    
M.~Ghasemi~Bostanabad$^\textrm{\scriptsize 173}$,    
M.~Ghneimat$^\textrm{\scriptsize 24}$,    
B.~Giacobbe$^\textrm{\scriptsize 23b}$,    
S.~Giagu$^\textrm{\scriptsize 70a,70b}$,    
N.~Giangiacomi$^\textrm{\scriptsize 23b,23a}$,    
P.~Giannetti$^\textrm{\scriptsize 69a}$,    
A.~Giannini$^\textrm{\scriptsize 67a,67b}$,    
S.M.~Gibson$^\textrm{\scriptsize 91}$,    
M.~Gignac$^\textrm{\scriptsize 143}$,    
D.~Gillberg$^\textrm{\scriptsize 33}$,    
G.~Gilles$^\textrm{\scriptsize 179}$,    
D.M.~Gingrich$^\textrm{\scriptsize 3,at}$,    
M.P.~Giordani$^\textrm{\scriptsize 64a,64c}$,    
F.M.~Giorgi$^\textrm{\scriptsize 23b}$,    
P.F.~Giraud$^\textrm{\scriptsize 142}$,    
P.~Giromini$^\textrm{\scriptsize 57}$,    
G.~Giugliarelli$^\textrm{\scriptsize 64a,64c}$,    
D.~Giugni$^\textrm{\scriptsize 66a}$,    
F.~Giuli$^\textrm{\scriptsize 131}$,    
M.~Giulini$^\textrm{\scriptsize 59b}$,    
S.~Gkaitatzis$^\textrm{\scriptsize 159}$,    
I.~Gkialas$^\textrm{\scriptsize 9,j}$,    
E.L.~Gkougkousis$^\textrm{\scriptsize 14}$,    
P.~Gkountoumis$^\textrm{\scriptsize 10}$,    
L.K.~Gladilin$^\textrm{\scriptsize 111}$,    
C.~Glasman$^\textrm{\scriptsize 96}$,    
J.~Glatzer$^\textrm{\scriptsize 14}$,    
P.C.F.~Glaysher$^\textrm{\scriptsize 44}$,    
A.~Glazov$^\textrm{\scriptsize 44}$,    
M.~Goblirsch-Kolb$^\textrm{\scriptsize 26}$,    
J.~Godlewski$^\textrm{\scriptsize 82}$,    
S.~Goldfarb$^\textrm{\scriptsize 102}$,    
T.~Golling$^\textrm{\scriptsize 52}$,    
D.~Golubkov$^\textrm{\scriptsize 140}$,    
A.~Gomes$^\textrm{\scriptsize 136a,136b,136d}$,    
R.~Goncalves~Gama$^\textrm{\scriptsize 51}$,    
R.~Gon\c{c}alo$^\textrm{\scriptsize 136a}$,    
G.~Gonella$^\textrm{\scriptsize 50}$,    
L.~Gonella$^\textrm{\scriptsize 21}$,    
A.~Gongadze$^\textrm{\scriptsize 77}$,    
F.~Gonnella$^\textrm{\scriptsize 21}$,    
J.L.~Gonski$^\textrm{\scriptsize 57}$,    
S.~Gonz\'alez~de~la~Hoz$^\textrm{\scriptsize 171}$,    
S.~Gonzalez-Sevilla$^\textrm{\scriptsize 52}$,    
L.~Goossens$^\textrm{\scriptsize 35}$,    
P.A.~Gorbounov$^\textrm{\scriptsize 109}$,    
H.A.~Gordon$^\textrm{\scriptsize 29}$,    
B.~Gorini$^\textrm{\scriptsize 35}$,    
E.~Gorini$^\textrm{\scriptsize 65a,65b}$,    
A.~Gori\v{s}ek$^\textrm{\scriptsize 89}$,    
A.T.~Goshaw$^\textrm{\scriptsize 47}$,    
C.~G\"ossling$^\textrm{\scriptsize 45}$,    
M.I.~Gostkin$^\textrm{\scriptsize 77}$,    
C.A.~Gottardo$^\textrm{\scriptsize 24}$,    
C.R.~Goudet$^\textrm{\scriptsize 128}$,    
D.~Goujdami$^\textrm{\scriptsize 34c}$,    
A.G.~Goussiou$^\textrm{\scriptsize 145}$,    
N.~Govender$^\textrm{\scriptsize 32b,c}$,    
C.~Goy$^\textrm{\scriptsize 5}$,    
E.~Gozani$^\textrm{\scriptsize 157}$,    
I.~Grabowska-Bold$^\textrm{\scriptsize 81a}$,    
P.O.J.~Gradin$^\textrm{\scriptsize 169}$,    
E.C.~Graham$^\textrm{\scriptsize 88}$,    
J.~Gramling$^\textrm{\scriptsize 168}$,    
E.~Gramstad$^\textrm{\scriptsize 130}$,    
S.~Grancagnolo$^\textrm{\scriptsize 19}$,    
V.~Gratchev$^\textrm{\scriptsize 134}$,    
P.M.~Gravila$^\textrm{\scriptsize 27f}$,    
F.G.~Gravili$^\textrm{\scriptsize 65a,65b}$,    
C.~Gray$^\textrm{\scriptsize 55}$,    
H.M.~Gray$^\textrm{\scriptsize 18}$,    
Z.D.~Greenwood$^\textrm{\scriptsize 93,aj}$,    
C.~Grefe$^\textrm{\scriptsize 24}$,    
K.~Gregersen$^\textrm{\scriptsize 94}$,    
I.M.~Gregor$^\textrm{\scriptsize 44}$,    
P.~Grenier$^\textrm{\scriptsize 150}$,    
K.~Grevtsov$^\textrm{\scriptsize 44}$,    
N.A.~Grieser$^\textrm{\scriptsize 124}$,    
J.~Griffiths$^\textrm{\scriptsize 8}$,    
A.A.~Grillo$^\textrm{\scriptsize 143}$,    
K.~Grimm$^\textrm{\scriptsize 150,b}$,    
S.~Grinstein$^\textrm{\scriptsize 14,y}$,    
Ph.~Gris$^\textrm{\scriptsize 37}$,    
J.-F.~Grivaz$^\textrm{\scriptsize 128}$,    
S.~Groh$^\textrm{\scriptsize 97}$,    
E.~Gross$^\textrm{\scriptsize 177}$,    
J.~Grosse-Knetter$^\textrm{\scriptsize 51}$,    
G.C.~Grossi$^\textrm{\scriptsize 93}$,    
Z.J.~Grout$^\textrm{\scriptsize 92}$,    
C.~Grud$^\textrm{\scriptsize 103}$,    
A.~Grummer$^\textrm{\scriptsize 116}$,    
L.~Guan$^\textrm{\scriptsize 103}$,    
W.~Guan$^\textrm{\scriptsize 178}$,    
J.~Guenther$^\textrm{\scriptsize 35}$,    
A.~Guerguichon$^\textrm{\scriptsize 128}$,    
F.~Guescini$^\textrm{\scriptsize 165a}$,    
D.~Guest$^\textrm{\scriptsize 168}$,    
R.~Gugel$^\textrm{\scriptsize 50}$,    
B.~Gui$^\textrm{\scriptsize 122}$,    
T.~Guillemin$^\textrm{\scriptsize 5}$,    
S.~Guindon$^\textrm{\scriptsize 35}$,    
U.~Gul$^\textrm{\scriptsize 55}$,    
J.~Guo$^\textrm{\scriptsize 58c}$,    
W.~Guo$^\textrm{\scriptsize 103}$,    
Y.~Guo$^\textrm{\scriptsize 58a,s}$,    
Z.~Guo$^\textrm{\scriptsize 99}$,    
R.~Gupta$^\textrm{\scriptsize 44}$,    
S.~Gurbuz$^\textrm{\scriptsize 12c}$,    
G.~Gustavino$^\textrm{\scriptsize 124}$,    
P.~Gutierrez$^\textrm{\scriptsize 124}$,    
C.~Gutschow$^\textrm{\scriptsize 92}$,    
C.~Guyot$^\textrm{\scriptsize 142}$,    
M.P.~Guzik$^\textrm{\scriptsize 81a}$,    
C.~Gwenlan$^\textrm{\scriptsize 131}$,    
C.B.~Gwilliam$^\textrm{\scriptsize 88}$,    
A.~Haas$^\textrm{\scriptsize 121}$,    
C.~Haber$^\textrm{\scriptsize 18}$,    
H.K.~Hadavand$^\textrm{\scriptsize 8}$,    
N.~Haddad$^\textrm{\scriptsize 34e}$,    
A.~Hadef$^\textrm{\scriptsize 58a}$,    
S.~Hageb\"ock$^\textrm{\scriptsize 24}$,    
M.~Hagihara$^\textrm{\scriptsize 166}$,    
M.~Haleem$^\textrm{\scriptsize 174}$,    
J.~Haley$^\textrm{\scriptsize 125}$,    
G.~Halladjian$^\textrm{\scriptsize 104}$,    
G.D.~Hallewell$^\textrm{\scriptsize 99}$,    
K.~Hamacher$^\textrm{\scriptsize 179}$,    
P.~Hamal$^\textrm{\scriptsize 126}$,    
K.~Hamano$^\textrm{\scriptsize 173}$,    
A.~Hamilton$^\textrm{\scriptsize 32a}$,    
G.N.~Hamity$^\textrm{\scriptsize 146}$,    
K.~Han$^\textrm{\scriptsize 58a,ai}$,    
L.~Han$^\textrm{\scriptsize 58a}$,    
S.~Han$^\textrm{\scriptsize 15d}$,    
K.~Hanagaki$^\textrm{\scriptsize 79,u}$,    
M.~Hance$^\textrm{\scriptsize 143}$,    
D.M.~Handl$^\textrm{\scriptsize 112}$,    
B.~Haney$^\textrm{\scriptsize 133}$,    
R.~Hankache$^\textrm{\scriptsize 132}$,    
P.~Hanke$^\textrm{\scriptsize 59a}$,    
E.~Hansen$^\textrm{\scriptsize 94}$,    
J.B.~Hansen$^\textrm{\scriptsize 39}$,    
J.D.~Hansen$^\textrm{\scriptsize 39}$,    
M.C.~Hansen$^\textrm{\scriptsize 24}$,    
P.H.~Hansen$^\textrm{\scriptsize 39}$,    
K.~Hara$^\textrm{\scriptsize 166}$,    
A.S.~Hard$^\textrm{\scriptsize 178}$,    
T.~Harenberg$^\textrm{\scriptsize 179}$,    
S.~Harkusha$^\textrm{\scriptsize 105}$,    
P.F.~Harrison$^\textrm{\scriptsize 175}$,    
N.M.~Hartmann$^\textrm{\scriptsize 112}$,    
Y.~Hasegawa$^\textrm{\scriptsize 147}$,    
A.~Hasib$^\textrm{\scriptsize 48}$,    
S.~Hassani$^\textrm{\scriptsize 142}$,    
S.~Haug$^\textrm{\scriptsize 20}$,    
R.~Hauser$^\textrm{\scriptsize 104}$,    
L.~Hauswald$^\textrm{\scriptsize 46}$,    
L.B.~Havener$^\textrm{\scriptsize 38}$,    
M.~Havranek$^\textrm{\scriptsize 138}$,    
C.M.~Hawkes$^\textrm{\scriptsize 21}$,    
R.J.~Hawkings$^\textrm{\scriptsize 35}$,    
D.~Hayden$^\textrm{\scriptsize 104}$,    
C.~Hayes$^\textrm{\scriptsize 152}$,    
C.P.~Hays$^\textrm{\scriptsize 131}$,    
J.M.~Hays$^\textrm{\scriptsize 90}$,    
H.S.~Hayward$^\textrm{\scriptsize 88}$,    
S.J.~Haywood$^\textrm{\scriptsize 141}$,    
F.~He$^\textrm{\scriptsize 58a}$,    
M.P.~Heath$^\textrm{\scriptsize 48}$,    
V.~Hedberg$^\textrm{\scriptsize 94}$,    
L.~Heelan$^\textrm{\scriptsize 8}$,    
S.~Heer$^\textrm{\scriptsize 24}$,    
K.K.~Heidegger$^\textrm{\scriptsize 50}$,    
J.~Heilman$^\textrm{\scriptsize 33}$,    
S.~Heim$^\textrm{\scriptsize 44}$,    
T.~Heim$^\textrm{\scriptsize 18}$,    
B.~Heinemann$^\textrm{\scriptsize 44,ao}$,    
J.J.~Heinrich$^\textrm{\scriptsize 112}$,    
L.~Heinrich$^\textrm{\scriptsize 121}$,    
C.~Heinz$^\textrm{\scriptsize 54}$,    
J.~Hejbal$^\textrm{\scriptsize 137}$,    
L.~Helary$^\textrm{\scriptsize 35}$,    
A.~Held$^\textrm{\scriptsize 172}$,    
S.~Hellesund$^\textrm{\scriptsize 130}$,    
C.M.~Helling$^\textrm{\scriptsize 143}$,    
S.~Hellman$^\textrm{\scriptsize 43a,43b}$,    
C.~Helsens$^\textrm{\scriptsize 35}$,    
R.C.W.~Henderson$^\textrm{\scriptsize 87}$,    
Y.~Heng$^\textrm{\scriptsize 178}$,    
S.~Henkelmann$^\textrm{\scriptsize 172}$,    
A.M.~Henriques~Correia$^\textrm{\scriptsize 35}$,    
G.H.~Herbert$^\textrm{\scriptsize 19}$,    
H.~Herde$^\textrm{\scriptsize 26}$,    
V.~Herget$^\textrm{\scriptsize 174}$,    
Y.~Hern\'andez~Jim\'enez$^\textrm{\scriptsize 32c}$,    
H.~Herr$^\textrm{\scriptsize 97}$,    
M.G.~Herrmann$^\textrm{\scriptsize 112}$,    
T.~Herrmann$^\textrm{\scriptsize 46}$,    
G.~Herten$^\textrm{\scriptsize 50}$,    
R.~Hertenberger$^\textrm{\scriptsize 112}$,    
L.~Hervas$^\textrm{\scriptsize 35}$,    
T.C.~Herwig$^\textrm{\scriptsize 133}$,    
G.G.~Hesketh$^\textrm{\scriptsize 92}$,    
N.P.~Hessey$^\textrm{\scriptsize 165a}$,    
A.~Higashida$^\textrm{\scriptsize 160}$,    
S.~Higashino$^\textrm{\scriptsize 79}$,    
E.~Hig\'on-Rodriguez$^\textrm{\scriptsize 171}$,    
K.~Hildebrand$^\textrm{\scriptsize 36}$,    
E.~Hill$^\textrm{\scriptsize 173}$,    
J.C.~Hill$^\textrm{\scriptsize 31}$,    
K.K.~Hill$^\textrm{\scriptsize 29}$,    
K.H.~Hiller$^\textrm{\scriptsize 44}$,    
S.J.~Hillier$^\textrm{\scriptsize 21}$,    
M.~Hils$^\textrm{\scriptsize 46}$,    
I.~Hinchliffe$^\textrm{\scriptsize 18}$,    
F.~Hinterkeuser$^\textrm{\scriptsize 24}$,    
M.~Hirose$^\textrm{\scriptsize 129}$,    
D.~Hirschbuehl$^\textrm{\scriptsize 179}$,    
B.~Hiti$^\textrm{\scriptsize 89}$,    
O.~Hladik$^\textrm{\scriptsize 137}$,    
D.R.~Hlaluku$^\textrm{\scriptsize 32c}$,    
X.~Hoad$^\textrm{\scriptsize 48}$,    
J.~Hobbs$^\textrm{\scriptsize 152}$,    
N.~Hod$^\textrm{\scriptsize 165a}$,    
M.C.~Hodgkinson$^\textrm{\scriptsize 146}$,    
A.~Hoecker$^\textrm{\scriptsize 35}$,    
M.R.~Hoeferkamp$^\textrm{\scriptsize 116}$,    
F.~Hoenig$^\textrm{\scriptsize 112}$,    
D.~Hohn$^\textrm{\scriptsize 50}$,    
D.~Hohov$^\textrm{\scriptsize 128}$,    
T.R.~Holmes$^\textrm{\scriptsize 36}$,    
M.~Holzbock$^\textrm{\scriptsize 112}$,    
M.~Homann$^\textrm{\scriptsize 45}$,    
B.H.~Hommels$^\textrm{\scriptsize 31}$,    
S.~Honda$^\textrm{\scriptsize 166}$,    
T.~Honda$^\textrm{\scriptsize 79}$,    
T.M.~Hong$^\textrm{\scriptsize 135}$,    
A.~H\"{o}nle$^\textrm{\scriptsize 113}$,    
B.H.~Hooberman$^\textrm{\scriptsize 170}$,    
W.H.~Hopkins$^\textrm{\scriptsize 127}$,    
Y.~Horii$^\textrm{\scriptsize 115}$,    
P.~Horn$^\textrm{\scriptsize 46}$,    
A.J.~Horton$^\textrm{\scriptsize 149}$,    
L.A.~Horyn$^\textrm{\scriptsize 36}$,    
J-Y.~Hostachy$^\textrm{\scriptsize 56}$,    
A.~Hostiuc$^\textrm{\scriptsize 145}$,    
S.~Hou$^\textrm{\scriptsize 155}$,    
A.~Hoummada$^\textrm{\scriptsize 34a}$,    
J.~Howarth$^\textrm{\scriptsize 98}$,    
J.~Hoya$^\textrm{\scriptsize 86}$,    
M.~Hrabovsky$^\textrm{\scriptsize 126}$,    
I.~Hristova$^\textrm{\scriptsize 19}$,    
J.~Hrivnac$^\textrm{\scriptsize 128}$,    
A.~Hrynevich$^\textrm{\scriptsize 106}$,    
T.~Hryn'ova$^\textrm{\scriptsize 5}$,    
P.J.~Hsu$^\textrm{\scriptsize 62}$,    
S.-C.~Hsu$^\textrm{\scriptsize 145}$,    
Q.~Hu$^\textrm{\scriptsize 29}$,    
S.~Hu$^\textrm{\scriptsize 58c}$,    
Y.~Huang$^\textrm{\scriptsize 15a}$,    
Z.~Hubacek$^\textrm{\scriptsize 138}$,    
F.~Hubaut$^\textrm{\scriptsize 99}$,    
M.~Huebner$^\textrm{\scriptsize 24}$,    
F.~Huegging$^\textrm{\scriptsize 24}$,    
T.B.~Huffman$^\textrm{\scriptsize 131}$,    
M.~Huhtinen$^\textrm{\scriptsize 35}$,    
R.F.H.~Hunter$^\textrm{\scriptsize 33}$,    
P.~Huo$^\textrm{\scriptsize 152}$,    
A.M.~Hupe$^\textrm{\scriptsize 33}$,    
N.~Huseynov$^\textrm{\scriptsize 77,ae}$,    
J.~Huston$^\textrm{\scriptsize 104}$,    
J.~Huth$^\textrm{\scriptsize 57}$,    
R.~Hyneman$^\textrm{\scriptsize 103}$,    
G.~Iacobucci$^\textrm{\scriptsize 52}$,    
G.~Iakovidis$^\textrm{\scriptsize 29}$,    
I.~Ibragimov$^\textrm{\scriptsize 148}$,    
L.~Iconomidou-Fayard$^\textrm{\scriptsize 128}$,    
Z.~Idrissi$^\textrm{\scriptsize 34e}$,    
P.~Iengo$^\textrm{\scriptsize 35}$,    
R.~Ignazzi$^\textrm{\scriptsize 39}$,    
O.~Igonkina$^\textrm{\scriptsize 118,aa}$,    
R.~Iguchi$^\textrm{\scriptsize 160}$,    
T.~Iizawa$^\textrm{\scriptsize 52}$,    
Y.~Ikegami$^\textrm{\scriptsize 79}$,    
M.~Ikeno$^\textrm{\scriptsize 79}$,    
D.~Iliadis$^\textrm{\scriptsize 159}$,    
N.~Ilic$^\textrm{\scriptsize 150}$,    
F.~Iltzsche$^\textrm{\scriptsize 46}$,    
G.~Introzzi$^\textrm{\scriptsize 68a,68b}$,    
M.~Iodice$^\textrm{\scriptsize 72a}$,    
K.~Iordanidou$^\textrm{\scriptsize 38}$,    
V.~Ippolito$^\textrm{\scriptsize 70a,70b}$,    
M.F.~Isacson$^\textrm{\scriptsize 169}$,    
N.~Ishijima$^\textrm{\scriptsize 129}$,    
M.~Ishino$^\textrm{\scriptsize 160}$,    
M.~Ishitsuka$^\textrm{\scriptsize 162}$,    
W.~Islam$^\textrm{\scriptsize 125}$,    
C.~Issever$^\textrm{\scriptsize 131}$,    
S.~Istin$^\textrm{\scriptsize 157}$,    
F.~Ito$^\textrm{\scriptsize 166}$,    
J.M.~Iturbe~Ponce$^\textrm{\scriptsize 61a}$,    
R.~Iuppa$^\textrm{\scriptsize 73a,73b}$,    
A.~Ivina$^\textrm{\scriptsize 177}$,    
H.~Iwasaki$^\textrm{\scriptsize 79}$,    
J.M.~Izen$^\textrm{\scriptsize 42}$,    
V.~Izzo$^\textrm{\scriptsize 67a}$,    
P.~Jacka$^\textrm{\scriptsize 137}$,    
P.~Jackson$^\textrm{\scriptsize 1}$,    
R.M.~Jacobs$^\textrm{\scriptsize 24}$,    
V.~Jain$^\textrm{\scriptsize 2}$,    
G.~J\"akel$^\textrm{\scriptsize 179}$,    
K.B.~Jakobi$^\textrm{\scriptsize 97}$,    
K.~Jakobs$^\textrm{\scriptsize 50}$,    
S.~Jakobsen$^\textrm{\scriptsize 74}$,    
T.~Jakoubek$^\textrm{\scriptsize 137}$,    
D.O.~Jamin$^\textrm{\scriptsize 125}$,    
R.~Jansky$^\textrm{\scriptsize 52}$,    
J.~Janssen$^\textrm{\scriptsize 24}$,    
M.~Janus$^\textrm{\scriptsize 51}$,    
P.A.~Janus$^\textrm{\scriptsize 81a}$,    
G.~Jarlskog$^\textrm{\scriptsize 94}$,    
N.~Javadov$^\textrm{\scriptsize 77,ae}$,    
T.~Jav\r{u}rek$^\textrm{\scriptsize 35}$,    
M.~Javurkova$^\textrm{\scriptsize 50}$,    
F.~Jeanneau$^\textrm{\scriptsize 142}$,    
L.~Jeanty$^\textrm{\scriptsize 18}$,    
J.~Jejelava$^\textrm{\scriptsize 156a,af}$,    
A.~Jelinskas$^\textrm{\scriptsize 175}$,    
P.~Jenni$^\textrm{\scriptsize 50,d}$,    
J.~Jeong$^\textrm{\scriptsize 44}$,    
N.~Jeong$^\textrm{\scriptsize 44}$,    
S.~J\'ez\'equel$^\textrm{\scriptsize 5}$,    
H.~Ji$^\textrm{\scriptsize 178}$,    
J.~Jia$^\textrm{\scriptsize 152}$,    
H.~Jiang$^\textrm{\scriptsize 76}$,    
Y.~Jiang$^\textrm{\scriptsize 58a}$,    
Z.~Jiang$^\textrm{\scriptsize 150,q}$,    
S.~Jiggins$^\textrm{\scriptsize 50}$,    
F.A.~Jimenez~Morales$^\textrm{\scriptsize 37}$,    
J.~Jimenez~Pena$^\textrm{\scriptsize 171}$,    
S.~Jin$^\textrm{\scriptsize 15c}$,    
A.~Jinaru$^\textrm{\scriptsize 27b}$,    
O.~Jinnouchi$^\textrm{\scriptsize 162}$,    
H.~Jivan$^\textrm{\scriptsize 32c}$,    
P.~Johansson$^\textrm{\scriptsize 146}$,    
K.A.~Johns$^\textrm{\scriptsize 7}$,    
C.A.~Johnson$^\textrm{\scriptsize 63}$,    
K.~Jon-And$^\textrm{\scriptsize 43a,43b}$,    
R.W.L.~Jones$^\textrm{\scriptsize 87}$,    
S.D.~Jones$^\textrm{\scriptsize 153}$,    
S.~Jones$^\textrm{\scriptsize 7}$,    
T.J.~Jones$^\textrm{\scriptsize 88}$,    
J.~Jongmanns$^\textrm{\scriptsize 59a}$,    
P.M.~Jorge$^\textrm{\scriptsize 136a,136b}$,    
J.~Jovicevic$^\textrm{\scriptsize 165a}$,    
X.~Ju$^\textrm{\scriptsize 18}$,    
J.J.~Junggeburth$^\textrm{\scriptsize 113}$,    
A.~Juste~Rozas$^\textrm{\scriptsize 14,y}$,    
A.~Kaczmarska$^\textrm{\scriptsize 82}$,    
M.~Kado$^\textrm{\scriptsize 128}$,    
H.~Kagan$^\textrm{\scriptsize 122}$,    
M.~Kagan$^\textrm{\scriptsize 150}$,    
T.~Kaji$^\textrm{\scriptsize 176}$,    
E.~Kajomovitz$^\textrm{\scriptsize 157}$,    
C.W.~Kalderon$^\textrm{\scriptsize 94}$,    
A.~Kaluza$^\textrm{\scriptsize 97}$,    
S.~Kama$^\textrm{\scriptsize 41}$,    
A.~Kamenshchikov$^\textrm{\scriptsize 140}$,    
L.~Kanjir$^\textrm{\scriptsize 89}$,    
Y.~Kano$^\textrm{\scriptsize 160}$,    
V.A.~Kantserov$^\textrm{\scriptsize 110}$,    
J.~Kanzaki$^\textrm{\scriptsize 79}$,    
L.S.~Kaplan$^\textrm{\scriptsize 178}$,    
D.~Kar$^\textrm{\scriptsize 32c}$,    
M.J.~Kareem$^\textrm{\scriptsize 165b}$,    
E.~Karentzos$^\textrm{\scriptsize 10}$,    
S.N.~Karpov$^\textrm{\scriptsize 77}$,    
Z.M.~Karpova$^\textrm{\scriptsize 77}$,    
V.~Kartvelishvili$^\textrm{\scriptsize 87}$,    
A.N.~Karyukhin$^\textrm{\scriptsize 140}$,    
L.~Kashif$^\textrm{\scriptsize 178}$,    
R.D.~Kass$^\textrm{\scriptsize 122}$,    
A.~Kastanas$^\textrm{\scriptsize 43a,43b}$,    
Y.~Kataoka$^\textrm{\scriptsize 160}$,    
C.~Kato$^\textrm{\scriptsize 58d,58c}$,    
J.~Katzy$^\textrm{\scriptsize 44}$,    
K.~Kawade$^\textrm{\scriptsize 80}$,    
K.~Kawagoe$^\textrm{\scriptsize 85}$,    
T.~Kawaguchi$^\textrm{\scriptsize 115}$,    
T.~Kawamoto$^\textrm{\scriptsize 160}$,    
G.~Kawamura$^\textrm{\scriptsize 51}$,    
E.F.~Kay$^\textrm{\scriptsize 88}$,    
V.F.~Kazanin$^\textrm{\scriptsize 120b,120a}$,    
R.~Keeler$^\textrm{\scriptsize 173}$,    
R.~Kehoe$^\textrm{\scriptsize 41}$,    
J.S.~Keller$^\textrm{\scriptsize 33}$,    
E.~Kellermann$^\textrm{\scriptsize 94}$,    
J.J.~Kempster$^\textrm{\scriptsize 21}$,    
J.~Kendrick$^\textrm{\scriptsize 21}$,    
O.~Kepka$^\textrm{\scriptsize 137}$,    
S.~Kersten$^\textrm{\scriptsize 179}$,    
B.P.~Ker\v{s}evan$^\textrm{\scriptsize 89}$,    
S.~Ketabchi~Haghighat$^\textrm{\scriptsize 164}$,    
R.A.~Keyes$^\textrm{\scriptsize 101}$,    
M.~Khader$^\textrm{\scriptsize 170}$,    
F.~Khalil-Zada$^\textrm{\scriptsize 13}$,    
A.~Khanov$^\textrm{\scriptsize 125}$,    
A.G.~Kharlamov$^\textrm{\scriptsize 120b,120a}$,    
T.~Kharlamova$^\textrm{\scriptsize 120b,120a}$,    
E.E.~Khoda$^\textrm{\scriptsize 172}$,    
A.~Khodinov$^\textrm{\scriptsize 163}$,    
T.J.~Khoo$^\textrm{\scriptsize 52}$,    
E.~Khramov$^\textrm{\scriptsize 77}$,    
J.~Khubua$^\textrm{\scriptsize 156b}$,    
S.~Kido$^\textrm{\scriptsize 80}$,    
M.~Kiehn$^\textrm{\scriptsize 52}$,    
C.R.~Kilby$^\textrm{\scriptsize 91}$,    
Y.K.~Kim$^\textrm{\scriptsize 36}$,    
N.~Kimura$^\textrm{\scriptsize 64a,64c}$,    
O.M.~Kind$^\textrm{\scriptsize 19}$,    
B.T.~King$^\textrm{\scriptsize 88}$,    
D.~Kirchmeier$^\textrm{\scriptsize 46}$,    
J.~Kirk$^\textrm{\scriptsize 141}$,    
A.E.~Kiryunin$^\textrm{\scriptsize 113}$,    
T.~Kishimoto$^\textrm{\scriptsize 160}$,    
D.~Kisielewska$^\textrm{\scriptsize 81a}$,    
V.~Kitali$^\textrm{\scriptsize 44}$,    
O.~Kivernyk$^\textrm{\scriptsize 5}$,    
E.~Kladiva$^\textrm{\scriptsize 28b,*}$,    
T.~Klapdor-Kleingrothaus$^\textrm{\scriptsize 50}$,    
M.H.~Klein$^\textrm{\scriptsize 103}$,    
M.~Klein$^\textrm{\scriptsize 88}$,    
U.~Klein$^\textrm{\scriptsize 88}$,    
K.~Kleinknecht$^\textrm{\scriptsize 97}$,    
P.~Klimek$^\textrm{\scriptsize 119}$,    
A.~Klimentov$^\textrm{\scriptsize 29}$,    
T.~Klingl$^\textrm{\scriptsize 24}$,    
T.~Klioutchnikova$^\textrm{\scriptsize 35}$,    
F.F.~Klitzner$^\textrm{\scriptsize 112}$,    
P.~Kluit$^\textrm{\scriptsize 118}$,    
S.~Kluth$^\textrm{\scriptsize 113}$,    
E.~Kneringer$^\textrm{\scriptsize 74}$,    
E.B.F.G.~Knoops$^\textrm{\scriptsize 99}$,    
A.~Knue$^\textrm{\scriptsize 50}$,    
A.~Kobayashi$^\textrm{\scriptsize 160}$,    
D.~Kobayashi$^\textrm{\scriptsize 85}$,    
T.~Kobayashi$^\textrm{\scriptsize 160}$,    
M.~Kobel$^\textrm{\scriptsize 46}$,    
M.~Kocian$^\textrm{\scriptsize 150}$,    
P.~Kodys$^\textrm{\scriptsize 139}$,    
P.T.~Koenig$^\textrm{\scriptsize 24}$,    
T.~Koffas$^\textrm{\scriptsize 33}$,    
E.~Koffeman$^\textrm{\scriptsize 118}$,    
N.M.~K\"ohler$^\textrm{\scriptsize 113}$,    
T.~Koi$^\textrm{\scriptsize 150}$,    
M.~Kolb$^\textrm{\scriptsize 59b}$,    
I.~Koletsou$^\textrm{\scriptsize 5}$,    
T.~Kondo$^\textrm{\scriptsize 79}$,    
N.~Kondrashova$^\textrm{\scriptsize 58c}$,    
K.~K\"oneke$^\textrm{\scriptsize 50}$,    
A.C.~K\"onig$^\textrm{\scriptsize 117}$,    
T.~Kono$^\textrm{\scriptsize 79}$,    
R.~Konoplich$^\textrm{\scriptsize 121,al}$,    
V.~Konstantinides$^\textrm{\scriptsize 92}$,    
N.~Konstantinidis$^\textrm{\scriptsize 92}$,    
B.~Konya$^\textrm{\scriptsize 94}$,    
R.~Kopeliansky$^\textrm{\scriptsize 63}$,    
S.~Koperny$^\textrm{\scriptsize 81a}$,    
K.~Korcyl$^\textrm{\scriptsize 82}$,    
K.~Kordas$^\textrm{\scriptsize 159}$,    
G.~Koren$^\textrm{\scriptsize 158}$,    
A.~Korn$^\textrm{\scriptsize 92}$,    
I.~Korolkov$^\textrm{\scriptsize 14}$,    
E.V.~Korolkova$^\textrm{\scriptsize 146}$,    
N.~Korotkova$^\textrm{\scriptsize 111}$,    
O.~Kortner$^\textrm{\scriptsize 113}$,    
S.~Kortner$^\textrm{\scriptsize 113}$,    
T.~Kosek$^\textrm{\scriptsize 139}$,    
V.V.~Kostyukhin$^\textrm{\scriptsize 24}$,    
A.~Kotwal$^\textrm{\scriptsize 47}$,    
A.~Koulouris$^\textrm{\scriptsize 10}$,    
A.~Kourkoumeli-Charalampidi$^\textrm{\scriptsize 68a,68b}$,    
C.~Kourkoumelis$^\textrm{\scriptsize 9}$,    
E.~Kourlitis$^\textrm{\scriptsize 146}$,    
V.~Kouskoura$^\textrm{\scriptsize 29}$,    
A.B.~Kowalewska$^\textrm{\scriptsize 82}$,    
R.~Kowalewski$^\textrm{\scriptsize 173}$,    
T.Z.~Kowalski$^\textrm{\scriptsize 81a}$,    
C.~Kozakai$^\textrm{\scriptsize 160}$,    
W.~Kozanecki$^\textrm{\scriptsize 142}$,    
A.S.~Kozhin$^\textrm{\scriptsize 140}$,    
V.A.~Kramarenko$^\textrm{\scriptsize 111}$,    
G.~Kramberger$^\textrm{\scriptsize 89}$,    
D.~Krasnopevtsev$^\textrm{\scriptsize 58a}$,    
M.W.~Krasny$^\textrm{\scriptsize 132}$,    
A.~Krasznahorkay$^\textrm{\scriptsize 35}$,    
D.~Krauss$^\textrm{\scriptsize 113}$,    
J.A.~Kremer$^\textrm{\scriptsize 81a}$,    
J.~Kretzschmar$^\textrm{\scriptsize 88}$,    
P.~Krieger$^\textrm{\scriptsize 164}$,    
K.~Krizka$^\textrm{\scriptsize 18}$,    
K.~Kroeninger$^\textrm{\scriptsize 45}$,    
H.~Kroha$^\textrm{\scriptsize 113}$,    
J.~Kroll$^\textrm{\scriptsize 137}$,    
J.~Kroll$^\textrm{\scriptsize 133}$,    
J.~Krstic$^\textrm{\scriptsize 16}$,    
U.~Kruchonak$^\textrm{\scriptsize 77}$,    
H.~Kr\"uger$^\textrm{\scriptsize 24}$,    
N.~Krumnack$^\textrm{\scriptsize 76}$,    
M.C.~Kruse$^\textrm{\scriptsize 47}$,    
T.~Kubota$^\textrm{\scriptsize 102}$,    
S.~Kuday$^\textrm{\scriptsize 4b}$,    
J.T.~Kuechler$^\textrm{\scriptsize 179}$,    
S.~Kuehn$^\textrm{\scriptsize 35}$,    
A.~Kugel$^\textrm{\scriptsize 59a}$,    
T.~Kuhl$^\textrm{\scriptsize 44}$,    
V.~Kukhtin$^\textrm{\scriptsize 77}$,    
R.~Kukla$^\textrm{\scriptsize 99}$,    
Y.~Kulchitsky$^\textrm{\scriptsize 105,ah}$,    
S.~Kuleshov$^\textrm{\scriptsize 144b}$,    
Y.P.~Kulinich$^\textrm{\scriptsize 170}$,    
M.~Kuna$^\textrm{\scriptsize 56}$,    
T.~Kunigo$^\textrm{\scriptsize 83}$,    
A.~Kupco$^\textrm{\scriptsize 137}$,    
T.~Kupfer$^\textrm{\scriptsize 45}$,    
O.~Kuprash$^\textrm{\scriptsize 158}$,    
H.~Kurashige$^\textrm{\scriptsize 80}$,    
L.L.~Kurchaninov$^\textrm{\scriptsize 165a}$,    
Y.A.~Kurochkin$^\textrm{\scriptsize 105}$,    
A.~Kurova$^\textrm{\scriptsize 110}$,    
M.G.~Kurth$^\textrm{\scriptsize 15d}$,    
E.S.~Kuwertz$^\textrm{\scriptsize 35}$,    
M.~Kuze$^\textrm{\scriptsize 162}$,    
J.~Kvita$^\textrm{\scriptsize 126}$,    
T.~Kwan$^\textrm{\scriptsize 101}$,    
A.~La~Rosa$^\textrm{\scriptsize 113}$,    
J.L.~La~Rosa~Navarro$^\textrm{\scriptsize 78d}$,    
L.~La~Rotonda$^\textrm{\scriptsize 40b,40a}$,    
F.~La~Ruffa$^\textrm{\scriptsize 40b,40a}$,    
C.~Lacasta$^\textrm{\scriptsize 171}$,    
F.~Lacava$^\textrm{\scriptsize 70a,70b}$,    
J.~Lacey$^\textrm{\scriptsize 44}$,    
D.P.J.~Lack$^\textrm{\scriptsize 98}$,    
H.~Lacker$^\textrm{\scriptsize 19}$,    
D.~Lacour$^\textrm{\scriptsize 132}$,    
E.~Ladygin$^\textrm{\scriptsize 77}$,    
R.~Lafaye$^\textrm{\scriptsize 5}$,    
B.~Laforge$^\textrm{\scriptsize 132}$,    
T.~Lagouri$^\textrm{\scriptsize 32c}$,    
S.~Lai$^\textrm{\scriptsize 51}$,    
S.~Lammers$^\textrm{\scriptsize 63}$,    
W.~Lampl$^\textrm{\scriptsize 7}$,    
E.~Lan\c{c}on$^\textrm{\scriptsize 29}$,    
U.~Landgraf$^\textrm{\scriptsize 50}$,    
M.P.J.~Landon$^\textrm{\scriptsize 90}$,    
M.C.~Lanfermann$^\textrm{\scriptsize 52}$,    
V.S.~Lang$^\textrm{\scriptsize 44}$,    
J.C.~Lange$^\textrm{\scriptsize 51}$,    
R.J.~Langenberg$^\textrm{\scriptsize 35}$,    
A.J.~Lankford$^\textrm{\scriptsize 168}$,    
F.~Lanni$^\textrm{\scriptsize 29}$,    
K.~Lantzsch$^\textrm{\scriptsize 24}$,    
A.~Lanza$^\textrm{\scriptsize 68a}$,    
A.~Lapertosa$^\textrm{\scriptsize 53b,53a}$,    
S.~Laplace$^\textrm{\scriptsize 132}$,    
J.F.~Laporte$^\textrm{\scriptsize 142}$,    
T.~Lari$^\textrm{\scriptsize 66a}$,    
F.~Lasagni~Manghi$^\textrm{\scriptsize 23b,23a}$,    
M.~Lassnig$^\textrm{\scriptsize 35}$,    
T.S.~Lau$^\textrm{\scriptsize 61a}$,    
A.~Laudrain$^\textrm{\scriptsize 128}$,    
M.~Lavorgna$^\textrm{\scriptsize 67a,67b}$,    
M.~Lazzaroni$^\textrm{\scriptsize 66a,66b}$,    
B.~Le$^\textrm{\scriptsize 102}$,    
O.~Le~Dortz$^\textrm{\scriptsize 132}$,    
E.~Le~Guirriec$^\textrm{\scriptsize 99}$,    
E.P.~Le~Quilleuc$^\textrm{\scriptsize 142}$,    
M.~LeBlanc$^\textrm{\scriptsize 7}$,    
T.~LeCompte$^\textrm{\scriptsize 6}$,    
F.~Ledroit-Guillon$^\textrm{\scriptsize 56}$,    
C.A.~Lee$^\textrm{\scriptsize 29}$,    
G.R.~Lee$^\textrm{\scriptsize 144a}$,    
L.~Lee$^\textrm{\scriptsize 57}$,    
S.C.~Lee$^\textrm{\scriptsize 155}$,    
B.~Lefebvre$^\textrm{\scriptsize 101}$,    
M.~Lefebvre$^\textrm{\scriptsize 173}$,    
F.~Legger$^\textrm{\scriptsize 112}$,    
C.~Leggett$^\textrm{\scriptsize 18}$,    
K.~Lehmann$^\textrm{\scriptsize 149}$,    
N.~Lehmann$^\textrm{\scriptsize 179}$,    
G.~Lehmann~Miotto$^\textrm{\scriptsize 35}$,    
W.A.~Leight$^\textrm{\scriptsize 44}$,    
A.~Leisos$^\textrm{\scriptsize 159,v}$,    
M.A.L.~Leite$^\textrm{\scriptsize 78d}$,    
R.~Leitner$^\textrm{\scriptsize 139}$,    
D.~Lellouch$^\textrm{\scriptsize 177}$,    
K.J.C.~Leney$^\textrm{\scriptsize 92}$,    
T.~Lenz$^\textrm{\scriptsize 24}$,    
B.~Lenzi$^\textrm{\scriptsize 35}$,    
R.~Leone$^\textrm{\scriptsize 7}$,    
S.~Leone$^\textrm{\scriptsize 69a}$,    
C.~Leonidopoulos$^\textrm{\scriptsize 48}$,    
G.~Lerner$^\textrm{\scriptsize 153}$,    
C.~Leroy$^\textrm{\scriptsize 107}$,    
R.~Les$^\textrm{\scriptsize 164}$,    
A.A.J.~Lesage$^\textrm{\scriptsize 142}$,    
C.G.~Lester$^\textrm{\scriptsize 31}$,    
M.~Levchenko$^\textrm{\scriptsize 134}$,    
J.~Lev\^eque$^\textrm{\scriptsize 5}$,    
D.~Levin$^\textrm{\scriptsize 103}$,    
L.J.~Levinson$^\textrm{\scriptsize 177}$,    
D.~Lewis$^\textrm{\scriptsize 90}$,    
B.~Li$^\textrm{\scriptsize 15b}$,    
B.~Li$^\textrm{\scriptsize 103}$,    
C-Q.~Li$^\textrm{\scriptsize 58a,ak}$,    
H.~Li$^\textrm{\scriptsize 58a}$,    
H.~Li$^\textrm{\scriptsize 58b}$,    
L.~Li$^\textrm{\scriptsize 58c}$,    
M.~Li$^\textrm{\scriptsize 15a}$,    
Q.~Li$^\textrm{\scriptsize 15d}$,    
Q.Y.~Li$^\textrm{\scriptsize 58a}$,    
S.~Li$^\textrm{\scriptsize 58d,58c}$,    
X.~Li$^\textrm{\scriptsize 58c}$,    
Y.~Li$^\textrm{\scriptsize 148}$,    
Z.~Liang$^\textrm{\scriptsize 15a}$,    
B.~Liberti$^\textrm{\scriptsize 71a}$,    
A.~Liblong$^\textrm{\scriptsize 164}$,    
K.~Lie$^\textrm{\scriptsize 61c}$,    
S.~Liem$^\textrm{\scriptsize 118}$,    
A.~Limosani$^\textrm{\scriptsize 154}$,    
C.Y.~Lin$^\textrm{\scriptsize 31}$,    
K.~Lin$^\textrm{\scriptsize 104}$,    
T.H.~Lin$^\textrm{\scriptsize 97}$,    
R.A.~Linck$^\textrm{\scriptsize 63}$,    
J.H.~Lindon$^\textrm{\scriptsize 21}$,    
B.E.~Lindquist$^\textrm{\scriptsize 152}$,    
A.L.~Lionti$^\textrm{\scriptsize 52}$,    
E.~Lipeles$^\textrm{\scriptsize 133}$,    
A.~Lipniacka$^\textrm{\scriptsize 17}$,    
M.~Lisovyi$^\textrm{\scriptsize 59b}$,    
T.M.~Liss$^\textrm{\scriptsize 170,aq}$,    
A.~Lister$^\textrm{\scriptsize 172}$,    
A.M.~Litke$^\textrm{\scriptsize 143}$,    
J.D.~Little$^\textrm{\scriptsize 8}$,    
B.~Liu$^\textrm{\scriptsize 76}$,    
B.L~Liu$^\textrm{\scriptsize 6}$,    
H.B.~Liu$^\textrm{\scriptsize 29}$,    
H.~Liu$^\textrm{\scriptsize 103}$,    
J.B.~Liu$^\textrm{\scriptsize 58a}$,    
J.K.K.~Liu$^\textrm{\scriptsize 131}$,    
K.~Liu$^\textrm{\scriptsize 132}$,    
M.~Liu$^\textrm{\scriptsize 58a}$,    
P.~Liu$^\textrm{\scriptsize 18}$,    
Y.~Liu$^\textrm{\scriptsize 15a}$,    
Y.L.~Liu$^\textrm{\scriptsize 58a}$,    
Y.W.~Liu$^\textrm{\scriptsize 58a}$,    
M.~Livan$^\textrm{\scriptsize 68a,68b}$,    
A.~Lleres$^\textrm{\scriptsize 56}$,    
J.~Llorente~Merino$^\textrm{\scriptsize 15a}$,    
S.L.~Lloyd$^\textrm{\scriptsize 90}$,    
C.Y.~Lo$^\textrm{\scriptsize 61b}$,    
F.~Lo~Sterzo$^\textrm{\scriptsize 41}$,    
E.M.~Lobodzinska$^\textrm{\scriptsize 44}$,    
P.~Loch$^\textrm{\scriptsize 7}$,    
T.~Lohse$^\textrm{\scriptsize 19}$,    
K.~Lohwasser$^\textrm{\scriptsize 146}$,    
M.~Lokajicek$^\textrm{\scriptsize 137}$,    
J.D.~Long$^\textrm{\scriptsize 170}$,    
R.E.~Long$^\textrm{\scriptsize 87}$,    
L.~Longo$^\textrm{\scriptsize 65a,65b}$,    
K.A.~Looper$^\textrm{\scriptsize 122}$,    
J.A.~Lopez$^\textrm{\scriptsize 144b}$,    
I.~Lopez~Paz$^\textrm{\scriptsize 98}$,    
A.~Lopez~Solis$^\textrm{\scriptsize 146}$,    
J.~Lorenz$^\textrm{\scriptsize 112}$,    
N.~Lorenzo~Martinez$^\textrm{\scriptsize 5}$,    
M.~Losada$^\textrm{\scriptsize 22}$,    
P.J.~L{\"o}sel$^\textrm{\scriptsize 112}$,    
A.~L\"osle$^\textrm{\scriptsize 50}$,    
X.~Lou$^\textrm{\scriptsize 44}$,    
X.~Lou$^\textrm{\scriptsize 15a}$,    
A.~Lounis$^\textrm{\scriptsize 128}$,    
J.~Love$^\textrm{\scriptsize 6}$,    
P.A.~Love$^\textrm{\scriptsize 87}$,    
J.J.~Lozano~Bahilo$^\textrm{\scriptsize 171}$,    
H.~Lu$^\textrm{\scriptsize 61a}$,    
M.~Lu$^\textrm{\scriptsize 58a}$,    
Y.J.~Lu$^\textrm{\scriptsize 62}$,    
H.J.~Lubatti$^\textrm{\scriptsize 145}$,    
C.~Luci$^\textrm{\scriptsize 70a,70b}$,    
A.~Lucotte$^\textrm{\scriptsize 56}$,    
C.~Luedtke$^\textrm{\scriptsize 50}$,    
F.~Luehring$^\textrm{\scriptsize 63}$,    
I.~Luise$^\textrm{\scriptsize 132}$,    
L.~Luminari$^\textrm{\scriptsize 70a}$,    
B.~Lund-Jensen$^\textrm{\scriptsize 151}$,    
M.S.~Lutz$^\textrm{\scriptsize 100}$,    
P.M.~Luzi$^\textrm{\scriptsize 132}$,    
D.~Lynn$^\textrm{\scriptsize 29}$,    
R.~Lysak$^\textrm{\scriptsize 137}$,    
E.~Lytken$^\textrm{\scriptsize 94}$,    
F.~Lyu$^\textrm{\scriptsize 15a}$,    
V.~Lyubushkin$^\textrm{\scriptsize 77}$,    
T.~Lyubushkina$^\textrm{\scriptsize 77}$,    
H.~Ma$^\textrm{\scriptsize 29}$,    
L.L.~Ma$^\textrm{\scriptsize 58b}$,    
Y.~Ma$^\textrm{\scriptsize 58b}$,    
G.~Maccarrone$^\textrm{\scriptsize 49}$,    
A.~Macchiolo$^\textrm{\scriptsize 113}$,    
C.M.~Macdonald$^\textrm{\scriptsize 146}$,    
J.~Machado~Miguens$^\textrm{\scriptsize 133,136b}$,    
D.~Madaffari$^\textrm{\scriptsize 171}$,    
R.~Madar$^\textrm{\scriptsize 37}$,    
W.F.~Mader$^\textrm{\scriptsize 46}$,    
N.~Madysa$^\textrm{\scriptsize 46}$,    
J.~Maeda$^\textrm{\scriptsize 80}$,    
K.~Maekawa$^\textrm{\scriptsize 160}$,    
S.~Maeland$^\textrm{\scriptsize 17}$,    
T.~Maeno$^\textrm{\scriptsize 29}$,    
M.~Maerker$^\textrm{\scriptsize 46}$,    
A.S.~Maevskiy$^\textrm{\scriptsize 111}$,    
V.~Magerl$^\textrm{\scriptsize 50}$,    
D.J.~Mahon$^\textrm{\scriptsize 38}$,    
C.~Maidantchik$^\textrm{\scriptsize 78b}$,    
T.~Maier$^\textrm{\scriptsize 112}$,    
A.~Maio$^\textrm{\scriptsize 136a,136b,136d}$,    
O.~Majersky$^\textrm{\scriptsize 28a}$,    
S.~Majewski$^\textrm{\scriptsize 127}$,    
Y.~Makida$^\textrm{\scriptsize 79}$,    
N.~Makovec$^\textrm{\scriptsize 128}$,    
B.~Malaescu$^\textrm{\scriptsize 132}$,    
Pa.~Malecki$^\textrm{\scriptsize 82}$,    
V.P.~Maleev$^\textrm{\scriptsize 134}$,    
F.~Malek$^\textrm{\scriptsize 56}$,    
U.~Mallik$^\textrm{\scriptsize 75}$,    
D.~Malon$^\textrm{\scriptsize 6}$,    
C.~Malone$^\textrm{\scriptsize 31}$,    
S.~Maltezos$^\textrm{\scriptsize 10}$,    
S.~Malyukov$^\textrm{\scriptsize 35}$,    
J.~Mamuzic$^\textrm{\scriptsize 171}$,    
G.~Mancini$^\textrm{\scriptsize 49}$,    
I.~Mandi\'{c}$^\textrm{\scriptsize 89}$,    
J.~Maneira$^\textrm{\scriptsize 136a}$,    
L.~Manhaes~de~Andrade~Filho$^\textrm{\scriptsize 78a}$,    
J.~Manjarres~Ramos$^\textrm{\scriptsize 46}$,    
K.H.~Mankinen$^\textrm{\scriptsize 94}$,    
A.~Mann$^\textrm{\scriptsize 112}$,    
A.~Manousos$^\textrm{\scriptsize 74}$,    
B.~Mansoulie$^\textrm{\scriptsize 142}$,    
S.~Manzoni$^\textrm{\scriptsize 66a,66b}$,    
A.~Marantis$^\textrm{\scriptsize 159}$,    
G.~Marceca$^\textrm{\scriptsize 30}$,    
L.~March$^\textrm{\scriptsize 52}$,    
L.~Marchese$^\textrm{\scriptsize 131}$,    
G.~Marchiori$^\textrm{\scriptsize 132}$,    
M.~Marcisovsky$^\textrm{\scriptsize 137}$,    
C.~Marcon$^\textrm{\scriptsize 94}$,    
C.A.~Marin~Tobon$^\textrm{\scriptsize 35}$,    
M.~Marjanovic$^\textrm{\scriptsize 37}$,    
F.~Marroquim$^\textrm{\scriptsize 78b}$,    
Z.~Marshall$^\textrm{\scriptsize 18}$,    
M.U.F~Martensson$^\textrm{\scriptsize 169}$,    
S.~Marti-Garcia$^\textrm{\scriptsize 171}$,    
C.B.~Martin$^\textrm{\scriptsize 122}$,    
T.A.~Martin$^\textrm{\scriptsize 175}$,    
V.J.~Martin$^\textrm{\scriptsize 48}$,    
B.~Martin~dit~Latour$^\textrm{\scriptsize 17}$,    
M.~Martinez$^\textrm{\scriptsize 14,y}$,    
V.I.~Martinez~Outschoorn$^\textrm{\scriptsize 100}$,    
S.~Martin-Haugh$^\textrm{\scriptsize 141}$,    
V.S.~Martoiu$^\textrm{\scriptsize 27b}$,    
A.C.~Martyniuk$^\textrm{\scriptsize 92}$,    
A.~Marzin$^\textrm{\scriptsize 35}$,    
L.~Masetti$^\textrm{\scriptsize 97}$,    
T.~Mashimo$^\textrm{\scriptsize 160}$,    
R.~Mashinistov$^\textrm{\scriptsize 108}$,    
J.~Masik$^\textrm{\scriptsize 98}$,    
A.L.~Maslennikov$^\textrm{\scriptsize 120b,120a}$,    
L.H.~Mason$^\textrm{\scriptsize 102}$,    
L.~Massa$^\textrm{\scriptsize 71a,71b}$,    
P.~Massarotti$^\textrm{\scriptsize 67a,67b}$,    
P.~Mastrandrea$^\textrm{\scriptsize 152}$,    
A.~Mastroberardino$^\textrm{\scriptsize 40b,40a}$,    
T.~Masubuchi$^\textrm{\scriptsize 160}$,    
P.~M\"attig$^\textrm{\scriptsize 24}$,    
J.~Maurer$^\textrm{\scriptsize 27b}$,    
B.~Ma\v{c}ek$^\textrm{\scriptsize 89}$,    
S.J.~Maxfield$^\textrm{\scriptsize 88}$,    
D.A.~Maximov$^\textrm{\scriptsize 120b,120a}$,    
R.~Mazini$^\textrm{\scriptsize 155}$,    
I.~Maznas$^\textrm{\scriptsize 159}$,    
S.M.~Mazza$^\textrm{\scriptsize 143}$,    
S.P.~Mc~Kee$^\textrm{\scriptsize 103}$,    
A.~McCarn$^\textrm{\scriptsize 41}$,    
T.G.~McCarthy$^\textrm{\scriptsize 113}$,    
L.I.~McClymont$^\textrm{\scriptsize 92}$,    
W.P.~McCormack$^\textrm{\scriptsize 18}$,    
E.F.~McDonald$^\textrm{\scriptsize 102}$,    
J.A.~Mcfayden$^\textrm{\scriptsize 35}$,    
G.~Mchedlidze$^\textrm{\scriptsize 51}$,    
M.A.~McKay$^\textrm{\scriptsize 41}$,    
K.D.~McLean$^\textrm{\scriptsize 173}$,    
S.J.~McMahon$^\textrm{\scriptsize 141}$,    
P.C.~McNamara$^\textrm{\scriptsize 102}$,    
C.J.~McNicol$^\textrm{\scriptsize 175}$,    
R.A.~McPherson$^\textrm{\scriptsize 173,ac}$,    
J.E.~Mdhluli$^\textrm{\scriptsize 32c}$,    
Z.A.~Meadows$^\textrm{\scriptsize 100}$,    
S.~Meehan$^\textrm{\scriptsize 145}$,    
T.M.~Megy$^\textrm{\scriptsize 50}$,    
S.~Mehlhase$^\textrm{\scriptsize 112}$,    
A.~Mehta$^\textrm{\scriptsize 88}$,    
T.~Meideck$^\textrm{\scriptsize 56}$,    
B.~Meirose$^\textrm{\scriptsize 42}$,    
D.~Melini$^\textrm{\scriptsize 171,h}$,    
B.R.~Mellado~Garcia$^\textrm{\scriptsize 32c}$,    
J.D.~Mellenthin$^\textrm{\scriptsize 51}$,    
M.~Melo$^\textrm{\scriptsize 28a}$,    
F.~Meloni$^\textrm{\scriptsize 44}$,    
A.~Melzer$^\textrm{\scriptsize 24}$,    
S.B.~Menary$^\textrm{\scriptsize 98}$,    
E.D.~Mendes~Gouveia$^\textrm{\scriptsize 136a}$,    
L.~Meng$^\textrm{\scriptsize 88}$,    
X.T.~Meng$^\textrm{\scriptsize 103}$,    
S.~Menke$^\textrm{\scriptsize 113}$,    
E.~Meoni$^\textrm{\scriptsize 40b,40a}$,    
S.~Mergelmeyer$^\textrm{\scriptsize 19}$,    
S.A.M.~Merkt$^\textrm{\scriptsize 135}$,    
C.~Merlassino$^\textrm{\scriptsize 20}$,    
P.~Mermod$^\textrm{\scriptsize 52}$,    
L.~Merola$^\textrm{\scriptsize 67a,67b}$,    
C.~Meroni$^\textrm{\scriptsize 66a}$,    
F.S.~Merritt$^\textrm{\scriptsize 36}$,    
A.~Messina$^\textrm{\scriptsize 70a,70b}$,    
J.~Metcalfe$^\textrm{\scriptsize 6}$,    
A.S.~Mete$^\textrm{\scriptsize 168}$,    
C.~Meyer$^\textrm{\scriptsize 63}$,    
J.~Meyer$^\textrm{\scriptsize 157}$,    
J-P.~Meyer$^\textrm{\scriptsize 142}$,    
H.~Meyer~Zu~Theenhausen$^\textrm{\scriptsize 59a}$,    
F.~Miano$^\textrm{\scriptsize 153}$,    
R.P.~Middleton$^\textrm{\scriptsize 141}$,    
L.~Mijovi\'{c}$^\textrm{\scriptsize 48}$,    
G.~Mikenberg$^\textrm{\scriptsize 177}$,    
M.~Mikestikova$^\textrm{\scriptsize 137}$,    
M.~Miku\v{z}$^\textrm{\scriptsize 89}$,    
M.~Milesi$^\textrm{\scriptsize 102}$,    
A.~Milic$^\textrm{\scriptsize 164}$,    
D.A.~Millar$^\textrm{\scriptsize 90}$,    
D.W.~Miller$^\textrm{\scriptsize 36}$,    
A.~Milov$^\textrm{\scriptsize 177}$,    
D.A.~Milstead$^\textrm{\scriptsize 43a,43b}$,    
R.A.~Mina$^\textrm{\scriptsize 150,q}$,    
A.A.~Minaenko$^\textrm{\scriptsize 140}$,    
M.~Mi\~nano~Moya$^\textrm{\scriptsize 171}$,    
I.A.~Minashvili$^\textrm{\scriptsize 156b}$,    
A.I.~Mincer$^\textrm{\scriptsize 121}$,    
B.~Mindur$^\textrm{\scriptsize 81a}$,    
M.~Mineev$^\textrm{\scriptsize 77}$,    
Y.~Minegishi$^\textrm{\scriptsize 160}$,    
Y.~Ming$^\textrm{\scriptsize 178}$,    
L.M.~Mir$^\textrm{\scriptsize 14}$,    
A.~Mirto$^\textrm{\scriptsize 65a,65b}$,    
K.P.~Mistry$^\textrm{\scriptsize 133}$,    
T.~Mitani$^\textrm{\scriptsize 176}$,    
J.~Mitrevski$^\textrm{\scriptsize 112}$,    
V.A.~Mitsou$^\textrm{\scriptsize 171}$,    
M.~Mittal$^\textrm{\scriptsize 58c}$,    
A.~Miucci$^\textrm{\scriptsize 20}$,    
P.S.~Miyagawa$^\textrm{\scriptsize 146}$,    
A.~Mizukami$^\textrm{\scriptsize 79}$,    
J.U.~Mj\"ornmark$^\textrm{\scriptsize 94}$,    
T.~Mkrtchyan$^\textrm{\scriptsize 181}$,    
M.~Mlynarikova$^\textrm{\scriptsize 139}$,    
T.~Moa$^\textrm{\scriptsize 43a,43b}$,    
K.~Mochizuki$^\textrm{\scriptsize 107}$,    
P.~Mogg$^\textrm{\scriptsize 50}$,    
S.~Mohapatra$^\textrm{\scriptsize 38}$,    
S.~Molander$^\textrm{\scriptsize 43a,43b}$,    
R.~Moles-Valls$^\textrm{\scriptsize 24}$,    
M.C.~Mondragon$^\textrm{\scriptsize 104}$,    
K.~M\"onig$^\textrm{\scriptsize 44}$,    
J.~Monk$^\textrm{\scriptsize 39}$,    
E.~Monnier$^\textrm{\scriptsize 99}$,    
A.~Montalbano$^\textrm{\scriptsize 149}$,    
J.~Montejo~Berlingen$^\textrm{\scriptsize 35}$,    
F.~Monticelli$^\textrm{\scriptsize 86}$,    
S.~Monzani$^\textrm{\scriptsize 66a}$,    
N.~Morange$^\textrm{\scriptsize 128}$,    
D.~Moreno$^\textrm{\scriptsize 22}$,    
M.~Moreno~Ll\'acer$^\textrm{\scriptsize 35}$,    
P.~Morettini$^\textrm{\scriptsize 53b}$,    
M.~Morgenstern$^\textrm{\scriptsize 118}$,    
S.~Morgenstern$^\textrm{\scriptsize 46}$,    
D.~Mori$^\textrm{\scriptsize 149}$,    
M.~Morii$^\textrm{\scriptsize 57}$,    
M.~Morinaga$^\textrm{\scriptsize 176}$,    
V.~Morisbak$^\textrm{\scriptsize 130}$,    
A.K.~Morley$^\textrm{\scriptsize 35}$,    
G.~Mornacchi$^\textrm{\scriptsize 35}$,    
A.P.~Morris$^\textrm{\scriptsize 92}$,    
J.D.~Morris$^\textrm{\scriptsize 90}$,    
L.~Morvaj$^\textrm{\scriptsize 152}$,    
P.~Moschovakos$^\textrm{\scriptsize 10}$,    
M.~Mosidze$^\textrm{\scriptsize 156b}$,    
H.J.~Moss$^\textrm{\scriptsize 146}$,    
J.~Moss$^\textrm{\scriptsize 150,n}$,    
K.~Motohashi$^\textrm{\scriptsize 162}$,    
R.~Mount$^\textrm{\scriptsize 150}$,    
E.~Mountricha$^\textrm{\scriptsize 35}$,    
E.J.W.~Moyse$^\textrm{\scriptsize 100}$,    
S.~Muanza$^\textrm{\scriptsize 99}$,    
F.~Mueller$^\textrm{\scriptsize 113}$,    
J.~Mueller$^\textrm{\scriptsize 135}$,    
R.S.P.~Mueller$^\textrm{\scriptsize 112}$,    
D.~Muenstermann$^\textrm{\scriptsize 87}$,    
G.A.~Mullier$^\textrm{\scriptsize 94}$,    
F.J.~Munoz~Sanchez$^\textrm{\scriptsize 98}$,    
P.~Murin$^\textrm{\scriptsize 28b}$,    
W.J.~Murray$^\textrm{\scriptsize 175,141}$,    
A.~Murrone$^\textrm{\scriptsize 66a,66b}$,    
M.~Mu\v{s}kinja$^\textrm{\scriptsize 89}$,    
C.~Mwewa$^\textrm{\scriptsize 32a}$,    
A.G.~Myagkov$^\textrm{\scriptsize 140,am}$,    
J.~Myers$^\textrm{\scriptsize 127}$,    
M.~Myska$^\textrm{\scriptsize 138}$,    
B.P.~Nachman$^\textrm{\scriptsize 18}$,    
O.~Nackenhorst$^\textrm{\scriptsize 45}$,    
K.~Nagai$^\textrm{\scriptsize 131}$,    
K.~Nagano$^\textrm{\scriptsize 79}$,    
Y.~Nagasaka$^\textrm{\scriptsize 60}$,    
M.~Nagel$^\textrm{\scriptsize 50}$,    
E.~Nagy$^\textrm{\scriptsize 99}$,    
A.M.~Nairz$^\textrm{\scriptsize 35}$,    
Y.~Nakahama$^\textrm{\scriptsize 115}$,    
K.~Nakamura$^\textrm{\scriptsize 79}$,    
T.~Nakamura$^\textrm{\scriptsize 160}$,    
I.~Nakano$^\textrm{\scriptsize 123}$,    
H.~Nanjo$^\textrm{\scriptsize 129}$,    
F.~Napolitano$^\textrm{\scriptsize 59a}$,    
R.F.~Naranjo~Garcia$^\textrm{\scriptsize 44}$,    
R.~Narayan$^\textrm{\scriptsize 11}$,    
D.I.~Narrias~Villar$^\textrm{\scriptsize 59a}$,    
I.~Naryshkin$^\textrm{\scriptsize 134}$,    
T.~Naumann$^\textrm{\scriptsize 44}$,    
G.~Navarro$^\textrm{\scriptsize 22}$,    
R.~Nayyar$^\textrm{\scriptsize 7}$,    
H.A.~Neal$^\textrm{\scriptsize 103,*}$,    
P.Y.~Nechaeva$^\textrm{\scriptsize 108}$,    
T.J.~Neep$^\textrm{\scriptsize 142}$,    
A.~Negri$^\textrm{\scriptsize 68a,68b}$,    
M.~Negrini$^\textrm{\scriptsize 23b}$,    
S.~Nektarijevic$^\textrm{\scriptsize 117}$,    
C.~Nellist$^\textrm{\scriptsize 51}$,    
M.E.~Nelson$^\textrm{\scriptsize 131}$,    
S.~Nemecek$^\textrm{\scriptsize 137}$,    
P.~Nemethy$^\textrm{\scriptsize 121}$,    
M.~Nessi$^\textrm{\scriptsize 35,f}$,    
M.S.~Neubauer$^\textrm{\scriptsize 170}$,    
M.~Neumann$^\textrm{\scriptsize 179}$,    
P.R.~Newman$^\textrm{\scriptsize 21}$,    
T.Y.~Ng$^\textrm{\scriptsize 61c}$,    
Y.S.~Ng$^\textrm{\scriptsize 19}$,    
Y.W.Y.~Ng$^\textrm{\scriptsize 168}$,    
H.D.N.~Nguyen$^\textrm{\scriptsize 99}$,    
T.~Nguyen~Manh$^\textrm{\scriptsize 107}$,    
E.~Nibigira$^\textrm{\scriptsize 37}$,    
R.B.~Nickerson$^\textrm{\scriptsize 131}$,    
R.~Nicolaidou$^\textrm{\scriptsize 142}$,    
D.S.~Nielsen$^\textrm{\scriptsize 39}$,    
J.~Nielsen$^\textrm{\scriptsize 143}$,    
N.~Nikiforou$^\textrm{\scriptsize 11}$,    
V.~Nikolaenko$^\textrm{\scriptsize 140,am}$,    
I.~Nikolic-Audit$^\textrm{\scriptsize 132}$,    
K.~Nikolopoulos$^\textrm{\scriptsize 21}$,    
P.~Nilsson$^\textrm{\scriptsize 29}$,    
Y.~Ninomiya$^\textrm{\scriptsize 79}$,    
A.~Nisati$^\textrm{\scriptsize 70a}$,    
N.~Nishu$^\textrm{\scriptsize 58c}$,    
R.~Nisius$^\textrm{\scriptsize 113}$,    
I.~Nitsche$^\textrm{\scriptsize 45}$,    
T.~Nitta$^\textrm{\scriptsize 176}$,    
T.~Nobe$^\textrm{\scriptsize 160}$,    
Y.~Noguchi$^\textrm{\scriptsize 83}$,    
M.~Nomachi$^\textrm{\scriptsize 129}$,    
I.~Nomidis$^\textrm{\scriptsize 132}$,    
M.A.~Nomura$^\textrm{\scriptsize 29}$,    
T.~Nooney$^\textrm{\scriptsize 90}$,    
M.~Nordberg$^\textrm{\scriptsize 35}$,    
N.~Norjoharuddeen$^\textrm{\scriptsize 131}$,    
T.~Novak$^\textrm{\scriptsize 89}$,    
O.~Novgorodova$^\textrm{\scriptsize 46}$,    
R.~Novotny$^\textrm{\scriptsize 138}$,    
L.~Nozka$^\textrm{\scriptsize 126}$,    
K.~Ntekas$^\textrm{\scriptsize 168}$,    
E.~Nurse$^\textrm{\scriptsize 92}$,    
F.~Nuti$^\textrm{\scriptsize 102}$,    
F.G.~Oakham$^\textrm{\scriptsize 33,at}$,    
H.~Oberlack$^\textrm{\scriptsize 113}$,    
J.~Ocariz$^\textrm{\scriptsize 132}$,    
A.~Ochi$^\textrm{\scriptsize 80}$,    
I.~Ochoa$^\textrm{\scriptsize 38}$,    
J.P.~Ochoa-Ricoux$^\textrm{\scriptsize 144a}$,    
K.~O'Connor$^\textrm{\scriptsize 26}$,    
S.~Oda$^\textrm{\scriptsize 85}$,    
S.~Odaka$^\textrm{\scriptsize 79}$,    
S.~Oerdek$^\textrm{\scriptsize 51}$,    
A.~Oh$^\textrm{\scriptsize 98}$,    
S.H.~Oh$^\textrm{\scriptsize 47}$,    
C.C.~Ohm$^\textrm{\scriptsize 151}$,    
H.~Oide$^\textrm{\scriptsize 53b,53a}$,    
M.L.~Ojeda$^\textrm{\scriptsize 164}$,    
H.~Okawa$^\textrm{\scriptsize 166}$,    
Y.~Okazaki$^\textrm{\scriptsize 83}$,    
Y.~Okumura$^\textrm{\scriptsize 160}$,    
T.~Okuyama$^\textrm{\scriptsize 79}$,    
A.~Olariu$^\textrm{\scriptsize 27b}$,    
L.F.~Oleiro~Seabra$^\textrm{\scriptsize 136a}$,    
S.A.~Olivares~Pino$^\textrm{\scriptsize 144a}$,    
D.~Oliveira~Damazio$^\textrm{\scriptsize 29}$,    
J.L.~Oliver$^\textrm{\scriptsize 1}$,    
M.J.R.~Olsson$^\textrm{\scriptsize 36}$,    
A.~Olszewski$^\textrm{\scriptsize 82}$,    
J.~Olszowska$^\textrm{\scriptsize 82}$,    
D.C.~O'Neil$^\textrm{\scriptsize 149}$,    
A.~Onofre$^\textrm{\scriptsize 136a,136e}$,    
K.~Onogi$^\textrm{\scriptsize 115}$,    
P.U.E.~Onyisi$^\textrm{\scriptsize 11}$,    
H.~Oppen$^\textrm{\scriptsize 130}$,    
M.J.~Oreglia$^\textrm{\scriptsize 36}$,    
G.E.~Orellana$^\textrm{\scriptsize 86}$,    
Y.~Oren$^\textrm{\scriptsize 158}$,    
D.~Orestano$^\textrm{\scriptsize 72a,72b}$,    
E.C.~Orgill$^\textrm{\scriptsize 98}$,    
N.~Orlando$^\textrm{\scriptsize 61b}$,    
A.A.~O'Rourke$^\textrm{\scriptsize 44}$,    
R.S.~Orr$^\textrm{\scriptsize 164}$,    
B.~Osculati$^\textrm{\scriptsize 53b,53a,*}$,    
V.~O'Shea$^\textrm{\scriptsize 55}$,    
R.~Ospanov$^\textrm{\scriptsize 58a}$,    
G.~Otero~y~Garzon$^\textrm{\scriptsize 30}$,    
H.~Otono$^\textrm{\scriptsize 85}$,    
M.~Ouchrif$^\textrm{\scriptsize 34d}$,    
F.~Ould-Saada$^\textrm{\scriptsize 130}$,    
A.~Ouraou$^\textrm{\scriptsize 142}$,    
Q.~Ouyang$^\textrm{\scriptsize 15a}$,    
M.~Owen$^\textrm{\scriptsize 55}$,    
R.E.~Owen$^\textrm{\scriptsize 21}$,    
V.E.~Ozcan$^\textrm{\scriptsize 12c}$,    
N.~Ozturk$^\textrm{\scriptsize 8}$,    
J.~Pacalt$^\textrm{\scriptsize 126}$,    
H.A.~Pacey$^\textrm{\scriptsize 31}$,    
K.~Pachal$^\textrm{\scriptsize 149}$,    
A.~Pacheco~Pages$^\textrm{\scriptsize 14}$,    
L.~Pacheco~Rodriguez$^\textrm{\scriptsize 142}$,    
C.~Padilla~Aranda$^\textrm{\scriptsize 14}$,    
S.~Pagan~Griso$^\textrm{\scriptsize 18}$,    
M.~Paganini$^\textrm{\scriptsize 180}$,    
G.~Palacino$^\textrm{\scriptsize 63}$,    
S.~Palazzo$^\textrm{\scriptsize 48}$,    
S.~Palestini$^\textrm{\scriptsize 35}$,    
M.~Palka$^\textrm{\scriptsize 81b}$,    
D.~Pallin$^\textrm{\scriptsize 37}$,    
I.~Panagoulias$^\textrm{\scriptsize 10}$,    
C.E.~Pandini$^\textrm{\scriptsize 35}$,    
J.G.~Panduro~Vazquez$^\textrm{\scriptsize 91}$,    
P.~Pani$^\textrm{\scriptsize 35}$,    
G.~Panizzo$^\textrm{\scriptsize 64a,64c}$,    
L.~Paolozzi$^\textrm{\scriptsize 52}$,    
T.D.~Papadopoulou$^\textrm{\scriptsize 10}$,    
K.~Papageorgiou$^\textrm{\scriptsize 9,j}$,    
A.~Paramonov$^\textrm{\scriptsize 6}$,    
D.~Paredes~Hernandez$^\textrm{\scriptsize 61b}$,    
S.R.~Paredes~Saenz$^\textrm{\scriptsize 131}$,    
B.~Parida$^\textrm{\scriptsize 163}$,    
A.J.~Parker$^\textrm{\scriptsize 87}$,    
K.A.~Parker$^\textrm{\scriptsize 44}$,    
M.A.~Parker$^\textrm{\scriptsize 31}$,    
F.~Parodi$^\textrm{\scriptsize 53b,53a}$,    
J.A.~Parsons$^\textrm{\scriptsize 38}$,    
U.~Parzefall$^\textrm{\scriptsize 50}$,    
V.R.~Pascuzzi$^\textrm{\scriptsize 164}$,    
J.M.P.~Pasner$^\textrm{\scriptsize 143}$,    
E.~Pasqualucci$^\textrm{\scriptsize 70a}$,    
S.~Passaggio$^\textrm{\scriptsize 53b}$,    
F.~Pastore$^\textrm{\scriptsize 91}$,    
P.~Pasuwan$^\textrm{\scriptsize 43a,43b}$,    
S.~Pataraia$^\textrm{\scriptsize 97}$,    
J.R.~Pater$^\textrm{\scriptsize 98}$,    
A.~Pathak$^\textrm{\scriptsize 178,k}$,    
T.~Pauly$^\textrm{\scriptsize 35}$,    
B.~Pearson$^\textrm{\scriptsize 113}$,    
M.~Pedersen$^\textrm{\scriptsize 130}$,    
L.~Pedraza~Diaz$^\textrm{\scriptsize 117}$,    
R.~Pedro$^\textrm{\scriptsize 136a,136b}$,    
S.V.~Peleganchuk$^\textrm{\scriptsize 120b,120a}$,    
O.~Penc$^\textrm{\scriptsize 137}$,    
C.~Peng$^\textrm{\scriptsize 15d}$,    
H.~Peng$^\textrm{\scriptsize 58a}$,    
B.S.~Peralva$^\textrm{\scriptsize 78a}$,    
M.M.~Perego$^\textrm{\scriptsize 128}$,    
A.P.~Pereira~Peixoto$^\textrm{\scriptsize 136a}$,    
D.V.~Perepelitsa$^\textrm{\scriptsize 29}$,    
F.~Peri$^\textrm{\scriptsize 19}$,    
L.~Perini$^\textrm{\scriptsize 66a,66b}$,    
H.~Pernegger$^\textrm{\scriptsize 35}$,    
S.~Perrella$^\textrm{\scriptsize 67a,67b}$,    
V.D.~Peshekhonov$^\textrm{\scriptsize 77,*}$,    
K.~Peters$^\textrm{\scriptsize 44}$,    
R.F.Y.~Peters$^\textrm{\scriptsize 98}$,    
B.A.~Petersen$^\textrm{\scriptsize 35}$,    
T.C.~Petersen$^\textrm{\scriptsize 39}$,    
E.~Petit$^\textrm{\scriptsize 56}$,    
A.~Petridis$^\textrm{\scriptsize 1}$,    
C.~Petridou$^\textrm{\scriptsize 159}$,    
P.~Petroff$^\textrm{\scriptsize 128}$,    
M.~Petrov$^\textrm{\scriptsize 131}$,    
F.~Petrucci$^\textrm{\scriptsize 72a,72b}$,    
M.~Pettee$^\textrm{\scriptsize 180}$,    
N.E.~Pettersson$^\textrm{\scriptsize 100}$,    
A.~Peyaud$^\textrm{\scriptsize 142}$,    
R.~Pezoa$^\textrm{\scriptsize 144b}$,    
T.~Pham$^\textrm{\scriptsize 102}$,    
F.H.~Phillips$^\textrm{\scriptsize 104}$,    
P.W.~Phillips$^\textrm{\scriptsize 141}$,    
M.W.~Phipps$^\textrm{\scriptsize 170}$,    
G.~Piacquadio$^\textrm{\scriptsize 152}$,    
E.~Pianori$^\textrm{\scriptsize 18}$,    
A.~Picazio$^\textrm{\scriptsize 100}$,    
R.H.~Pickles$^\textrm{\scriptsize 98}$,    
R.~Piegaia$^\textrm{\scriptsize 30}$,    
J.E.~Pilcher$^\textrm{\scriptsize 36}$,    
A.D.~Pilkington$^\textrm{\scriptsize 98}$,    
M.~Pinamonti$^\textrm{\scriptsize 71a,71b}$,    
J.L.~Pinfold$^\textrm{\scriptsize 3}$,    
M.~Pitt$^\textrm{\scriptsize 177}$,    
L.~Pizzimento$^\textrm{\scriptsize 71a,71b}$,    
M-A.~Pleier$^\textrm{\scriptsize 29}$,    
V.~Pleskot$^\textrm{\scriptsize 139}$,    
E.~Plotnikova$^\textrm{\scriptsize 77}$,    
D.~Pluth$^\textrm{\scriptsize 76}$,    
P.~Podberezko$^\textrm{\scriptsize 120b,120a}$,    
R.~Poettgen$^\textrm{\scriptsize 94}$,    
R.~Poggi$^\textrm{\scriptsize 52}$,    
L.~Poggioli$^\textrm{\scriptsize 128}$,    
I.~Pogrebnyak$^\textrm{\scriptsize 104}$,    
D.~Pohl$^\textrm{\scriptsize 24}$,    
I.~Pokharel$^\textrm{\scriptsize 51}$,    
G.~Polesello$^\textrm{\scriptsize 68a}$,    
A.~Poley$^\textrm{\scriptsize 18}$,    
A.~Policicchio$^\textrm{\scriptsize 70a,70b}$,    
R.~Polifka$^\textrm{\scriptsize 35}$,    
A.~Polini$^\textrm{\scriptsize 23b}$,    
C.S.~Pollard$^\textrm{\scriptsize 44}$,    
V.~Polychronakos$^\textrm{\scriptsize 29}$,    
D.~Ponomarenko$^\textrm{\scriptsize 110}$,    
L.~Pontecorvo$^\textrm{\scriptsize 70a}$,    
G.A.~Popeneciu$^\textrm{\scriptsize 27d}$,    
D.M.~Portillo~Quintero$^\textrm{\scriptsize 132}$,    
S.~Pospisil$^\textrm{\scriptsize 138}$,    
K.~Potamianos$^\textrm{\scriptsize 44}$,    
I.N.~Potrap$^\textrm{\scriptsize 77}$,    
C.J.~Potter$^\textrm{\scriptsize 31}$,    
H.~Potti$^\textrm{\scriptsize 11}$,    
T.~Poulsen$^\textrm{\scriptsize 94}$,    
J.~Poveda$^\textrm{\scriptsize 35}$,    
T.D.~Powell$^\textrm{\scriptsize 146}$,    
M.E.~Pozo~Astigarraga$^\textrm{\scriptsize 35}$,    
P.~Pralavorio$^\textrm{\scriptsize 99}$,    
S.~Prell$^\textrm{\scriptsize 76}$,    
D.~Price$^\textrm{\scriptsize 98}$,    
M.~Primavera$^\textrm{\scriptsize 65a}$,    
S.~Prince$^\textrm{\scriptsize 101}$,    
N.~Proklova$^\textrm{\scriptsize 110}$,    
K.~Prokofiev$^\textrm{\scriptsize 61c}$,    
F.~Prokoshin$^\textrm{\scriptsize 144b}$,    
S.~Protopopescu$^\textrm{\scriptsize 29}$,    
J.~Proudfoot$^\textrm{\scriptsize 6}$,    
M.~Przybycien$^\textrm{\scriptsize 81a}$,    
A.~Puri$^\textrm{\scriptsize 170}$,    
P.~Puzo$^\textrm{\scriptsize 128}$,    
J.~Qian$^\textrm{\scriptsize 103}$,    
Y.~Qin$^\textrm{\scriptsize 98}$,    
A.~Quadt$^\textrm{\scriptsize 51}$,    
M.~Queitsch-Maitland$^\textrm{\scriptsize 44}$,    
A.~Qureshi$^\textrm{\scriptsize 1}$,    
P.~Rados$^\textrm{\scriptsize 102}$,    
F.~Ragusa$^\textrm{\scriptsize 66a,66b}$,    
G.~Rahal$^\textrm{\scriptsize 95}$,    
J.A.~Raine$^\textrm{\scriptsize 52}$,    
S.~Rajagopalan$^\textrm{\scriptsize 29}$,    
A.~Ramirez~Morales$^\textrm{\scriptsize 90}$,    
K.~Ran$^\textrm{\scriptsize 15a}$,    
T.~Rashid$^\textrm{\scriptsize 128}$,    
S.~Raspopov$^\textrm{\scriptsize 5}$,    
M.G.~Ratti$^\textrm{\scriptsize 66a,66b}$,    
D.M.~Rauch$^\textrm{\scriptsize 44}$,    
F.~Rauscher$^\textrm{\scriptsize 112}$,    
S.~Rave$^\textrm{\scriptsize 97}$,    
B.~Ravina$^\textrm{\scriptsize 146}$,    
I.~Ravinovich$^\textrm{\scriptsize 177}$,    
J.H.~Rawling$^\textrm{\scriptsize 98}$,    
M.~Raymond$^\textrm{\scriptsize 35}$,    
A.L.~Read$^\textrm{\scriptsize 130}$,    
N.P.~Readioff$^\textrm{\scriptsize 56}$,    
M.~Reale$^\textrm{\scriptsize 65a,65b}$,    
D.M.~Rebuzzi$^\textrm{\scriptsize 68a,68b}$,    
A.~Redelbach$^\textrm{\scriptsize 174}$,    
G.~Redlinger$^\textrm{\scriptsize 29}$,    
R.~Reece$^\textrm{\scriptsize 143}$,    
R.G.~Reed$^\textrm{\scriptsize 32c}$,    
K.~Reeves$^\textrm{\scriptsize 42}$,    
L.~Rehnisch$^\textrm{\scriptsize 19}$,    
J.~Reichert$^\textrm{\scriptsize 133}$,    
D.~Reikher$^\textrm{\scriptsize 158}$,    
A.~Reiss$^\textrm{\scriptsize 97}$,    
C.~Rembser$^\textrm{\scriptsize 35}$,    
H.~Ren$^\textrm{\scriptsize 15d}$,    
M.~Rescigno$^\textrm{\scriptsize 70a}$,    
S.~Resconi$^\textrm{\scriptsize 66a}$,    
E.D.~Resseguie$^\textrm{\scriptsize 133}$,    
S.~Rettie$^\textrm{\scriptsize 172}$,    
E.~Reynolds$^\textrm{\scriptsize 21}$,    
O.L.~Rezanova$^\textrm{\scriptsize 120b,120a}$,    
P.~Reznicek$^\textrm{\scriptsize 139}$,    
E.~Ricci$^\textrm{\scriptsize 73a,73b}$,    
R.~Richter$^\textrm{\scriptsize 113}$,    
S.~Richter$^\textrm{\scriptsize 44}$,    
E.~Richter-Was$^\textrm{\scriptsize 81b}$,    
O.~Ricken$^\textrm{\scriptsize 24}$,    
M.~Ridel$^\textrm{\scriptsize 132}$,    
P.~Rieck$^\textrm{\scriptsize 113}$,    
C.J.~Riegel$^\textrm{\scriptsize 179}$,    
O.~Rifki$^\textrm{\scriptsize 44}$,    
M.~Rijssenbeek$^\textrm{\scriptsize 152}$,    
A.~Rimoldi$^\textrm{\scriptsize 68a,68b}$,    
M.~Rimoldi$^\textrm{\scriptsize 20}$,    
L.~Rinaldi$^\textrm{\scriptsize 23b}$,    
G.~Ripellino$^\textrm{\scriptsize 151}$,    
B.~Risti\'{c}$^\textrm{\scriptsize 87}$,    
E.~Ritsch$^\textrm{\scriptsize 35}$,    
I.~Riu$^\textrm{\scriptsize 14}$,    
J.C.~Rivera~Vergara$^\textrm{\scriptsize 144a}$,    
F.~Rizatdinova$^\textrm{\scriptsize 125}$,    
E.~Rizvi$^\textrm{\scriptsize 90}$,    
C.~Rizzi$^\textrm{\scriptsize 14}$,    
R.T.~Roberts$^\textrm{\scriptsize 98}$,    
S.H.~Robertson$^\textrm{\scriptsize 101,ac}$,    
D.~Robinson$^\textrm{\scriptsize 31}$,    
J.E.M.~Robinson$^\textrm{\scriptsize 44}$,    
A.~Robson$^\textrm{\scriptsize 55}$,    
E.~Rocco$^\textrm{\scriptsize 97}$,    
C.~Roda$^\textrm{\scriptsize 69a,69b}$,    
Y.~Rodina$^\textrm{\scriptsize 99}$,    
S.~Rodriguez~Bosca$^\textrm{\scriptsize 171}$,    
A.~Rodriguez~Perez$^\textrm{\scriptsize 14}$,    
D.~Rodriguez~Rodriguez$^\textrm{\scriptsize 171}$,    
A.M.~Rodr\'iguez~Vera$^\textrm{\scriptsize 165b}$,    
S.~Roe$^\textrm{\scriptsize 35}$,    
C.S.~Rogan$^\textrm{\scriptsize 57}$,    
O.~R{\o}hne$^\textrm{\scriptsize 130}$,    
R.~R\"ohrig$^\textrm{\scriptsize 113}$,    
C.P.A.~Roland$^\textrm{\scriptsize 63}$,    
J.~Roloff$^\textrm{\scriptsize 57}$,    
A.~Romaniouk$^\textrm{\scriptsize 110}$,    
M.~Romano$^\textrm{\scriptsize 23b,23a}$,    
N.~Rompotis$^\textrm{\scriptsize 88}$,    
M.~Ronzani$^\textrm{\scriptsize 121}$,    
L.~Roos$^\textrm{\scriptsize 132}$,    
S.~Rosati$^\textrm{\scriptsize 70a}$,    
K.~Rosbach$^\textrm{\scriptsize 50}$,    
N-A.~Rosien$^\textrm{\scriptsize 51}$,    
B.J.~Rosser$^\textrm{\scriptsize 133}$,    
E.~Rossi$^\textrm{\scriptsize 44}$,    
E.~Rossi$^\textrm{\scriptsize 72a,72b}$,    
E.~Rossi$^\textrm{\scriptsize 67a,67b}$,    
L.P.~Rossi$^\textrm{\scriptsize 53b}$,    
L.~Rossini$^\textrm{\scriptsize 66a,66b}$,    
J.H.N.~Rosten$^\textrm{\scriptsize 31}$,    
R.~Rosten$^\textrm{\scriptsize 14}$,    
M.~Rotaru$^\textrm{\scriptsize 27b}$,    
J.~Rothberg$^\textrm{\scriptsize 145}$,    
D.~Rousseau$^\textrm{\scriptsize 128}$,    
D.~Roy$^\textrm{\scriptsize 32c}$,    
A.~Rozanov$^\textrm{\scriptsize 99}$,    
Y.~Rozen$^\textrm{\scriptsize 157}$,    
X.~Ruan$^\textrm{\scriptsize 32c}$,    
F.~Rubbo$^\textrm{\scriptsize 150}$,    
F.~R\"uhr$^\textrm{\scriptsize 50}$,    
A.~Ruiz-Martinez$^\textrm{\scriptsize 171}$,    
Z.~Rurikova$^\textrm{\scriptsize 50}$,    
N.A.~Rusakovich$^\textrm{\scriptsize 77}$,    
H.L.~Russell$^\textrm{\scriptsize 101}$,    
J.P.~Rutherfoord$^\textrm{\scriptsize 7}$,    
E.M.~R{\"u}ttinger$^\textrm{\scriptsize 44,l}$,    
Y.F.~Ryabov$^\textrm{\scriptsize 134}$,    
M.~Rybar$^\textrm{\scriptsize 170}$,    
G.~Rybkin$^\textrm{\scriptsize 128}$,    
S.~Ryu$^\textrm{\scriptsize 6}$,    
A.~Ryzhov$^\textrm{\scriptsize 140}$,    
G.F.~Rzehorz$^\textrm{\scriptsize 51}$,    
P.~Sabatini$^\textrm{\scriptsize 51}$,    
G.~Sabato$^\textrm{\scriptsize 118}$,    
S.~Sacerdoti$^\textrm{\scriptsize 128}$,    
H.F-W.~Sadrozinski$^\textrm{\scriptsize 143}$,    
R.~Sadykov$^\textrm{\scriptsize 77}$,    
F.~Safai~Tehrani$^\textrm{\scriptsize 70a}$,    
P.~Saha$^\textrm{\scriptsize 119}$,    
M.~Sahinsoy$^\textrm{\scriptsize 59a}$,    
A.~Sahu$^\textrm{\scriptsize 179}$,    
M.~Saimpert$^\textrm{\scriptsize 44}$,    
M.~Saito$^\textrm{\scriptsize 160}$,    
T.~Saito$^\textrm{\scriptsize 160}$,    
H.~Sakamoto$^\textrm{\scriptsize 160}$,    
A.~Sakharov$^\textrm{\scriptsize 121,al}$,    
D.~Salamani$^\textrm{\scriptsize 52}$,    
G.~Salamanna$^\textrm{\scriptsize 72a,72b}$,    
J.E.~Salazar~Loyola$^\textrm{\scriptsize 144b}$,    
P.H.~Sales~De~Bruin$^\textrm{\scriptsize 169}$,    
D.~Salihagic$^\textrm{\scriptsize 113}$,    
A.~Salnikov$^\textrm{\scriptsize 150}$,    
J.~Salt$^\textrm{\scriptsize 171}$,    
D.~Salvatore$^\textrm{\scriptsize 40b,40a}$,    
F.~Salvatore$^\textrm{\scriptsize 153}$,    
A.~Salvucci$^\textrm{\scriptsize 61a,61b,61c}$,    
A.~Salzburger$^\textrm{\scriptsize 35}$,    
J.~Samarati$^\textrm{\scriptsize 35}$,    
D.~Sammel$^\textrm{\scriptsize 50}$,    
D.~Sampsonidis$^\textrm{\scriptsize 159}$,    
D.~Sampsonidou$^\textrm{\scriptsize 159}$,    
J.~S\'anchez$^\textrm{\scriptsize 171}$,    
A.~Sanchez~Pineda$^\textrm{\scriptsize 64a,64c}$,    
H.~Sandaker$^\textrm{\scriptsize 130}$,    
C.O.~Sander$^\textrm{\scriptsize 44}$,    
M.~Sandhoff$^\textrm{\scriptsize 179}$,    
C.~Sandoval$^\textrm{\scriptsize 22}$,    
D.P.C.~Sankey$^\textrm{\scriptsize 141}$,    
M.~Sannino$^\textrm{\scriptsize 53b,53a}$,    
Y.~Sano$^\textrm{\scriptsize 115}$,    
A.~Sansoni$^\textrm{\scriptsize 49}$,    
C.~Santoni$^\textrm{\scriptsize 37}$,    
H.~Santos$^\textrm{\scriptsize 136a}$,    
I.~Santoyo~Castillo$^\textrm{\scriptsize 153}$,    
A.~Santra$^\textrm{\scriptsize 171}$,    
A.~Sapronov$^\textrm{\scriptsize 77}$,    
J.G.~Saraiva$^\textrm{\scriptsize 136a,136d}$,    
O.~Sasaki$^\textrm{\scriptsize 79}$,    
K.~Sato$^\textrm{\scriptsize 166}$,    
E.~Sauvan$^\textrm{\scriptsize 5}$,    
P.~Savard$^\textrm{\scriptsize 164,at}$,    
N.~Savic$^\textrm{\scriptsize 113}$,    
R.~Sawada$^\textrm{\scriptsize 160}$,    
C.~Sawyer$^\textrm{\scriptsize 141}$,    
L.~Sawyer$^\textrm{\scriptsize 93,aj}$,    
C.~Sbarra$^\textrm{\scriptsize 23b}$,    
A.~Sbrizzi$^\textrm{\scriptsize 23b,23a}$,    
T.~Scanlon$^\textrm{\scriptsize 92}$,    
J.~Schaarschmidt$^\textrm{\scriptsize 145}$,    
P.~Schacht$^\textrm{\scriptsize 113}$,    
B.M.~Schachtner$^\textrm{\scriptsize 112}$,    
D.~Schaefer$^\textrm{\scriptsize 36}$,    
L.~Schaefer$^\textrm{\scriptsize 133}$,    
J.~Schaeffer$^\textrm{\scriptsize 97}$,    
S.~Schaepe$^\textrm{\scriptsize 35}$,    
U.~Sch\"afer$^\textrm{\scriptsize 97}$,    
A.C.~Schaffer$^\textrm{\scriptsize 128}$,    
D.~Schaile$^\textrm{\scriptsize 112}$,    
R.D.~Schamberger$^\textrm{\scriptsize 152}$,    
N.~Scharmberg$^\textrm{\scriptsize 98}$,    
V.A.~Schegelsky$^\textrm{\scriptsize 134}$,    
D.~Scheirich$^\textrm{\scriptsize 139}$,    
F.~Schenck$^\textrm{\scriptsize 19}$,    
M.~Schernau$^\textrm{\scriptsize 168}$,    
C.~Schiavi$^\textrm{\scriptsize 53b,53a}$,    
S.~Schier$^\textrm{\scriptsize 143}$,    
L.K.~Schildgen$^\textrm{\scriptsize 24}$,    
Z.M.~Schillaci$^\textrm{\scriptsize 26}$,    
E.J.~Schioppa$^\textrm{\scriptsize 35}$,    
M.~Schioppa$^\textrm{\scriptsize 40b,40a}$,    
K.E.~Schleicher$^\textrm{\scriptsize 50}$,    
S.~Schlenker$^\textrm{\scriptsize 35}$,    
K.R.~Schmidt-Sommerfeld$^\textrm{\scriptsize 113}$,    
K.~Schmieden$^\textrm{\scriptsize 35}$,    
C.~Schmitt$^\textrm{\scriptsize 97}$,    
S.~Schmitt$^\textrm{\scriptsize 44}$,    
S.~Schmitz$^\textrm{\scriptsize 97}$,    
J.C.~Schmoeckel$^\textrm{\scriptsize 44}$,    
U.~Schnoor$^\textrm{\scriptsize 50}$,    
L.~Schoeffel$^\textrm{\scriptsize 142}$,    
A.~Schoening$^\textrm{\scriptsize 59b}$,    
E.~Schopf$^\textrm{\scriptsize 131}$,    
M.~Schott$^\textrm{\scriptsize 97}$,    
J.F.P.~Schouwenberg$^\textrm{\scriptsize 117}$,    
J.~Schovancova$^\textrm{\scriptsize 35}$,    
S.~Schramm$^\textrm{\scriptsize 52}$,    
A.~Schulte$^\textrm{\scriptsize 97}$,    
H-C.~Schultz-Coulon$^\textrm{\scriptsize 59a}$,    
M.~Schumacher$^\textrm{\scriptsize 50}$,    
B.A.~Schumm$^\textrm{\scriptsize 143}$,    
Ph.~Schune$^\textrm{\scriptsize 142}$,    
A.~Schwartzman$^\textrm{\scriptsize 150}$,    
T.A.~Schwarz$^\textrm{\scriptsize 103}$,    
Ph.~Schwemling$^\textrm{\scriptsize 142}$,    
R.~Schwienhorst$^\textrm{\scriptsize 104}$,    
A.~Sciandra$^\textrm{\scriptsize 24}$,    
G.~Sciolla$^\textrm{\scriptsize 26}$,    
M.~Scornajenghi$^\textrm{\scriptsize 40b,40a}$,    
F.~Scuri$^\textrm{\scriptsize 69a}$,    
F.~Scutti$^\textrm{\scriptsize 102}$,    
L.M.~Scyboz$^\textrm{\scriptsize 113}$,    
C.D.~Sebastiani$^\textrm{\scriptsize 70a,70b}$,    
P.~Seema$^\textrm{\scriptsize 19}$,    
S.C.~Seidel$^\textrm{\scriptsize 116}$,    
A.~Seiden$^\textrm{\scriptsize 143}$,    
T.~Seiss$^\textrm{\scriptsize 36}$,    
J.M.~Seixas$^\textrm{\scriptsize 78b}$,    
G.~Sekhniaidze$^\textrm{\scriptsize 67a}$,    
K.~Sekhon$^\textrm{\scriptsize 103}$,    
S.J.~Sekula$^\textrm{\scriptsize 41}$,    
N.~Semprini-Cesari$^\textrm{\scriptsize 23b,23a}$,    
S.~Sen$^\textrm{\scriptsize 47}$,    
S.~Senkin$^\textrm{\scriptsize 37}$,    
C.~Serfon$^\textrm{\scriptsize 130}$,    
L.~Serin$^\textrm{\scriptsize 128}$,    
L.~Serkin$^\textrm{\scriptsize 64a,64b}$,    
M.~Sessa$^\textrm{\scriptsize 58a}$,    
H.~Severini$^\textrm{\scriptsize 124}$,    
F.~Sforza$^\textrm{\scriptsize 167}$,    
A.~Sfyrla$^\textrm{\scriptsize 52}$,    
E.~Shabalina$^\textrm{\scriptsize 51}$,    
J.D.~Shahinian$^\textrm{\scriptsize 143}$,    
N.W.~Shaikh$^\textrm{\scriptsize 43a,43b}$,    
D.~Shaked~Renous$^\textrm{\scriptsize 177}$,    
L.Y.~Shan$^\textrm{\scriptsize 15a}$,    
R.~Shang$^\textrm{\scriptsize 170}$,    
J.T.~Shank$^\textrm{\scriptsize 25}$,    
M.~Shapiro$^\textrm{\scriptsize 18}$,    
A.S.~Sharma$^\textrm{\scriptsize 1}$,    
A.~Sharma$^\textrm{\scriptsize 131}$,    
P.B.~Shatalov$^\textrm{\scriptsize 109}$,    
K.~Shaw$^\textrm{\scriptsize 153}$,    
S.M.~Shaw$^\textrm{\scriptsize 98}$,    
A.~Shcherbakova$^\textrm{\scriptsize 134}$,    
Y.~Shen$^\textrm{\scriptsize 124}$,    
N.~Sherafati$^\textrm{\scriptsize 33}$,    
A.D.~Sherman$^\textrm{\scriptsize 25}$,    
P.~Sherwood$^\textrm{\scriptsize 92}$,    
L.~Shi$^\textrm{\scriptsize 155,ap}$,    
S.~Shimizu$^\textrm{\scriptsize 79}$,    
C.O.~Shimmin$^\textrm{\scriptsize 180}$,    
Y.~Shimogama$^\textrm{\scriptsize 176}$,    
M.~Shimojima$^\textrm{\scriptsize 114}$,    
I.P.J.~Shipsey$^\textrm{\scriptsize 131}$,    
S.~Shirabe$^\textrm{\scriptsize 85}$,    
M.~Shiyakova$^\textrm{\scriptsize 77}$,    
J.~Shlomi$^\textrm{\scriptsize 177}$,    
A.~Shmeleva$^\textrm{\scriptsize 108}$,    
D.~Shoaleh~Saadi$^\textrm{\scriptsize 107}$,    
M.J.~Shochet$^\textrm{\scriptsize 36}$,    
S.~Shojaii$^\textrm{\scriptsize 102}$,    
D.R.~Shope$^\textrm{\scriptsize 124}$,    
S.~Shrestha$^\textrm{\scriptsize 122}$,    
E.~Shulga$^\textrm{\scriptsize 110}$,    
P.~Sicho$^\textrm{\scriptsize 137}$,    
A.M.~Sickles$^\textrm{\scriptsize 170}$,    
P.E.~Sidebo$^\textrm{\scriptsize 151}$,    
E.~Sideras~Haddad$^\textrm{\scriptsize 32c}$,    
O.~Sidiropoulou$^\textrm{\scriptsize 35}$,    
A.~Sidoti$^\textrm{\scriptsize 23b,23a}$,    
F.~Siegert$^\textrm{\scriptsize 46}$,    
Dj.~Sijacki$^\textrm{\scriptsize 16}$,    
J.~Silva$^\textrm{\scriptsize 136a}$,    
M.~Silva~Jr.$^\textrm{\scriptsize 178}$,    
M.V.~Silva~Oliveira$^\textrm{\scriptsize 78a}$,    
S.B.~Silverstein$^\textrm{\scriptsize 43a}$,    
S.~Simion$^\textrm{\scriptsize 128}$,    
E.~Simioni$^\textrm{\scriptsize 97}$,    
M.~Simon$^\textrm{\scriptsize 97}$,    
R.~Simoniello$^\textrm{\scriptsize 97}$,    
P.~Sinervo$^\textrm{\scriptsize 164}$,    
N.B.~Sinev$^\textrm{\scriptsize 127}$,    
M.~Sioli$^\textrm{\scriptsize 23b,23a}$,    
I.~Siral$^\textrm{\scriptsize 103}$,    
S.Yu.~Sivoklokov$^\textrm{\scriptsize 111}$,    
J.~Sj\"{o}lin$^\textrm{\scriptsize 43a,43b}$,    
P.~Skubic$^\textrm{\scriptsize 124}$,    
M.~Slater$^\textrm{\scriptsize 21}$,    
T.~Slavicek$^\textrm{\scriptsize 138}$,    
M.~Slawinska$^\textrm{\scriptsize 82}$,    
K.~Sliwa$^\textrm{\scriptsize 167}$,    
R.~Slovak$^\textrm{\scriptsize 139}$,    
V.~Smakhtin$^\textrm{\scriptsize 177}$,    
B.H.~Smart$^\textrm{\scriptsize 5}$,    
J.~Smiesko$^\textrm{\scriptsize 28a}$,    
N.~Smirnov$^\textrm{\scriptsize 110}$,    
S.Yu.~Smirnov$^\textrm{\scriptsize 110}$,    
Y.~Smirnov$^\textrm{\scriptsize 110}$,    
L.N.~Smirnova$^\textrm{\scriptsize 111}$,    
O.~Smirnova$^\textrm{\scriptsize 94}$,    
J.W.~Smith$^\textrm{\scriptsize 51}$,    
M.~Smizanska$^\textrm{\scriptsize 87}$,    
K.~Smolek$^\textrm{\scriptsize 138}$,    
A.~Smykiewicz$^\textrm{\scriptsize 82}$,    
A.A.~Snesarev$^\textrm{\scriptsize 108}$,    
I.M.~Snyder$^\textrm{\scriptsize 127}$,    
S.~Snyder$^\textrm{\scriptsize 29}$,    
R.~Sobie$^\textrm{\scriptsize 173,ac}$,    
A.M.~Soffa$^\textrm{\scriptsize 168}$,    
A.~Soffer$^\textrm{\scriptsize 158}$,    
A.~S{\o}gaard$^\textrm{\scriptsize 48}$,    
F.~Sohns$^\textrm{\scriptsize 51}$,    
G.~Sokhrannyi$^\textrm{\scriptsize 89}$,    
C.A.~Solans~Sanchez$^\textrm{\scriptsize 35}$,    
M.~Solar$^\textrm{\scriptsize 138}$,    
E.Yu.~Soldatov$^\textrm{\scriptsize 110}$,    
U.~Soldevila$^\textrm{\scriptsize 171}$,    
A.A.~Solodkov$^\textrm{\scriptsize 140}$,    
A.~Soloshenko$^\textrm{\scriptsize 77}$,    
O.V.~Solovyanov$^\textrm{\scriptsize 140}$,    
V.~Solovyev$^\textrm{\scriptsize 134}$,    
P.~Sommer$^\textrm{\scriptsize 146}$,    
H.~Son$^\textrm{\scriptsize 167}$,    
W.~Song$^\textrm{\scriptsize 141}$,    
W.Y.~Song$^\textrm{\scriptsize 165b}$,    
A.~Sopczak$^\textrm{\scriptsize 138}$,    
F.~Sopkova$^\textrm{\scriptsize 28b}$,    
C.L.~Sotiropoulou$^\textrm{\scriptsize 69a,69b}$,    
S.~Sottocornola$^\textrm{\scriptsize 68a,68b}$,    
R.~Soualah$^\textrm{\scriptsize 64a,64c,i}$,    
A.M.~Soukharev$^\textrm{\scriptsize 120b,120a}$,    
D.~South$^\textrm{\scriptsize 44}$,    
S.~Spagnolo$^\textrm{\scriptsize 65a,65b}$,    
M.~Spalla$^\textrm{\scriptsize 113}$,    
M.~Spangenberg$^\textrm{\scriptsize 175}$,    
F.~Span\`o$^\textrm{\scriptsize 91}$,    
D.~Sperlich$^\textrm{\scriptsize 19}$,    
T.M.~Spieker$^\textrm{\scriptsize 59a}$,    
R.~Spighi$^\textrm{\scriptsize 23b}$,    
G.~Spigo$^\textrm{\scriptsize 35}$,    
L.A.~Spiller$^\textrm{\scriptsize 102}$,    
D.P.~Spiteri$^\textrm{\scriptsize 55}$,    
M.~Spousta$^\textrm{\scriptsize 139}$,    
A.~Stabile$^\textrm{\scriptsize 66a,66b}$,    
R.~Stamen$^\textrm{\scriptsize 59a}$,    
S.~Stamm$^\textrm{\scriptsize 19}$,    
E.~Stanecka$^\textrm{\scriptsize 82}$,    
R.W.~Stanek$^\textrm{\scriptsize 6}$,    
C.~Stanescu$^\textrm{\scriptsize 72a}$,    
B.~Stanislaus$^\textrm{\scriptsize 131}$,    
M.M.~Stanitzki$^\textrm{\scriptsize 44}$,    
B.~Stapf$^\textrm{\scriptsize 118}$,    
S.~Stapnes$^\textrm{\scriptsize 130}$,    
E.A.~Starchenko$^\textrm{\scriptsize 140}$,    
G.H.~Stark$^\textrm{\scriptsize 36}$,    
J.~Stark$^\textrm{\scriptsize 56}$,    
S.H~Stark$^\textrm{\scriptsize 39}$,    
P.~Staroba$^\textrm{\scriptsize 137}$,    
P.~Starovoitov$^\textrm{\scriptsize 59a}$,    
S.~St\"arz$^\textrm{\scriptsize 101}$,    
R.~Staszewski$^\textrm{\scriptsize 82}$,    
M.~Stegler$^\textrm{\scriptsize 44}$,    
P.~Steinberg$^\textrm{\scriptsize 29}$,    
B.~Stelzer$^\textrm{\scriptsize 149}$,    
H.J.~Stelzer$^\textrm{\scriptsize 35}$,    
O.~Stelzer-Chilton$^\textrm{\scriptsize 165a}$,    
H.~Stenzel$^\textrm{\scriptsize 54}$,    
T.J.~Stevenson$^\textrm{\scriptsize 90}$,    
G.A.~Stewart$^\textrm{\scriptsize 55}$,    
M.C.~Stockton$^\textrm{\scriptsize 35}$,    
G.~Stoicea$^\textrm{\scriptsize 27b}$,    
P.~Stolte$^\textrm{\scriptsize 51}$,    
S.~Stonjek$^\textrm{\scriptsize 113}$,    
A.~Straessner$^\textrm{\scriptsize 46}$,    
J.~Strandberg$^\textrm{\scriptsize 151}$,    
S.~Strandberg$^\textrm{\scriptsize 43a,43b}$,    
M.~Strauss$^\textrm{\scriptsize 124}$,    
P.~Strizenec$^\textrm{\scriptsize 28b}$,    
R.~Str\"ohmer$^\textrm{\scriptsize 174}$,    
D.M.~Strom$^\textrm{\scriptsize 127}$,    
R.~Stroynowski$^\textrm{\scriptsize 41}$,    
A.~Strubig$^\textrm{\scriptsize 48}$,    
S.A.~Stucci$^\textrm{\scriptsize 29}$,    
B.~Stugu$^\textrm{\scriptsize 17}$,    
J.~Stupak$^\textrm{\scriptsize 124}$,    
N.A.~Styles$^\textrm{\scriptsize 44}$,    
D.~Su$^\textrm{\scriptsize 150}$,    
J.~Su$^\textrm{\scriptsize 135}$,    
S.~Suchek$^\textrm{\scriptsize 59a}$,    
Y.~Sugaya$^\textrm{\scriptsize 129}$,    
M.~Suk$^\textrm{\scriptsize 138}$,    
V.V.~Sulin$^\textrm{\scriptsize 108}$,    
M.J.~Sullivan$^\textrm{\scriptsize 88}$,    
D.M.S.~Sultan$^\textrm{\scriptsize 52}$,    
S.~Sultansoy$^\textrm{\scriptsize 4c}$,    
T.~Sumida$^\textrm{\scriptsize 83}$,    
S.~Sun$^\textrm{\scriptsize 103}$,    
X.~Sun$^\textrm{\scriptsize 3}$,    
K.~Suruliz$^\textrm{\scriptsize 153}$,    
C.J.E.~Suster$^\textrm{\scriptsize 154}$,    
M.R.~Sutton$^\textrm{\scriptsize 153}$,    
S.~Suzuki$^\textrm{\scriptsize 79}$,    
M.~Svatos$^\textrm{\scriptsize 137}$,    
M.~Swiatlowski$^\textrm{\scriptsize 36}$,    
S.P.~Swift$^\textrm{\scriptsize 2}$,    
A.~Sydorenko$^\textrm{\scriptsize 97}$,    
I.~Sykora$^\textrm{\scriptsize 28a}$,    
T.~Sykora$^\textrm{\scriptsize 139}$,    
D.~Ta$^\textrm{\scriptsize 97}$,    
K.~Tackmann$^\textrm{\scriptsize 44,z}$,    
J.~Taenzer$^\textrm{\scriptsize 158}$,    
A.~Taffard$^\textrm{\scriptsize 168}$,    
R.~Tafirout$^\textrm{\scriptsize 165a}$,    
E.~Tahirovic$^\textrm{\scriptsize 90}$,    
N.~Taiblum$^\textrm{\scriptsize 158}$,    
H.~Takai$^\textrm{\scriptsize 29}$,    
R.~Takashima$^\textrm{\scriptsize 84}$,    
E.H.~Takasugi$^\textrm{\scriptsize 113}$,    
K.~Takeda$^\textrm{\scriptsize 80}$,    
T.~Takeshita$^\textrm{\scriptsize 147}$,    
Y.~Takubo$^\textrm{\scriptsize 79}$,    
M.~Talby$^\textrm{\scriptsize 99}$,    
A.A.~Talyshev$^\textrm{\scriptsize 120b,120a}$,    
J.~Tanaka$^\textrm{\scriptsize 160}$,    
M.~Tanaka$^\textrm{\scriptsize 162}$,    
R.~Tanaka$^\textrm{\scriptsize 128}$,    
B.B.~Tannenwald$^\textrm{\scriptsize 122}$,    
S.~Tapia~Araya$^\textrm{\scriptsize 144b}$,    
S.~Tapprogge$^\textrm{\scriptsize 97}$,    
A.~Tarek~Abouelfadl~Mohamed$^\textrm{\scriptsize 132}$,    
S.~Tarem$^\textrm{\scriptsize 157}$,    
G.~Tarna$^\textrm{\scriptsize 27b,e}$,    
G.F.~Tartarelli$^\textrm{\scriptsize 66a}$,    
P.~Tas$^\textrm{\scriptsize 139}$,    
M.~Tasevsky$^\textrm{\scriptsize 137}$,    
T.~Tashiro$^\textrm{\scriptsize 83}$,    
E.~Tassi$^\textrm{\scriptsize 40b,40a}$,    
A.~Tavares~Delgado$^\textrm{\scriptsize 136a,136b}$,    
Y.~Tayalati$^\textrm{\scriptsize 34e}$,    
A.J.~Taylor$^\textrm{\scriptsize 48}$,    
G.N.~Taylor$^\textrm{\scriptsize 102}$,    
P.T.E.~Taylor$^\textrm{\scriptsize 102}$,    
W.~Taylor$^\textrm{\scriptsize 165b}$,    
A.S.~Tee$^\textrm{\scriptsize 87}$,    
P.~Teixeira-Dias$^\textrm{\scriptsize 91}$,    
H.~Ten~Kate$^\textrm{\scriptsize 35}$,    
J.J.~Teoh$^\textrm{\scriptsize 118}$,    
S.~Terada$^\textrm{\scriptsize 79}$,    
K.~Terashi$^\textrm{\scriptsize 160}$,    
J.~Terron$^\textrm{\scriptsize 96}$,    
S.~Terzo$^\textrm{\scriptsize 14}$,    
M.~Testa$^\textrm{\scriptsize 49}$,    
R.J.~Teuscher$^\textrm{\scriptsize 164,ac}$,    
S.J.~Thais$^\textrm{\scriptsize 180}$,    
T.~Theveneaux-Pelzer$^\textrm{\scriptsize 44}$,    
F.~Thiele$^\textrm{\scriptsize 39}$,    
D.W.~Thomas$^\textrm{\scriptsize 91}$,    
J.P.~Thomas$^\textrm{\scriptsize 21}$,    
A.S.~Thompson$^\textrm{\scriptsize 55}$,    
P.D.~Thompson$^\textrm{\scriptsize 21}$,    
L.A.~Thomsen$^\textrm{\scriptsize 180}$,    
E.~Thomson$^\textrm{\scriptsize 133}$,    
Y.~Tian$^\textrm{\scriptsize 38}$,    
R.E.~Ticse~Torres$^\textrm{\scriptsize 51}$,    
V.O.~Tikhomirov$^\textrm{\scriptsize 108,an}$,    
Yu.A.~Tikhonov$^\textrm{\scriptsize 120b,120a}$,    
S.~Timoshenko$^\textrm{\scriptsize 110}$,    
P.~Tipton$^\textrm{\scriptsize 180}$,    
S.~Tisserant$^\textrm{\scriptsize 99}$,    
K.~Todome$^\textrm{\scriptsize 162}$,    
S.~Todorova-Nova$^\textrm{\scriptsize 5}$,    
S.~Todt$^\textrm{\scriptsize 46}$,    
J.~Tojo$^\textrm{\scriptsize 85}$,    
S.~Tok\'ar$^\textrm{\scriptsize 28a}$,    
K.~Tokushuku$^\textrm{\scriptsize 79}$,    
E.~Tolley$^\textrm{\scriptsize 122}$,    
K.G.~Tomiwa$^\textrm{\scriptsize 32c}$,    
M.~Tomoto$^\textrm{\scriptsize 115}$,    
L.~Tompkins$^\textrm{\scriptsize 150,q}$,    
K.~Toms$^\textrm{\scriptsize 116}$,    
B.~Tong$^\textrm{\scriptsize 57}$,    
P.~Tornambe$^\textrm{\scriptsize 50}$,    
E.~Torrence$^\textrm{\scriptsize 127}$,    
H.~Torres$^\textrm{\scriptsize 46}$,    
E.~Torr\'o~Pastor$^\textrm{\scriptsize 145}$,    
C.~Tosciri$^\textrm{\scriptsize 131}$,    
J.~Toth$^\textrm{\scriptsize 99,ab}$,    
F.~Touchard$^\textrm{\scriptsize 99}$,    
D.R.~Tovey$^\textrm{\scriptsize 146}$,    
C.J.~Treado$^\textrm{\scriptsize 121}$,    
T.~Trefzger$^\textrm{\scriptsize 174}$,    
F.~Tresoldi$^\textrm{\scriptsize 153}$,    
A.~Tricoli$^\textrm{\scriptsize 29}$,    
I.M.~Trigger$^\textrm{\scriptsize 165a}$,    
S.~Trincaz-Duvoid$^\textrm{\scriptsize 132}$,    
W.~Trischuk$^\textrm{\scriptsize 164}$,    
B.~Trocm\'e$^\textrm{\scriptsize 56}$,    
A.~Trofymov$^\textrm{\scriptsize 128}$,    
C.~Troncon$^\textrm{\scriptsize 66a}$,    
M.~Trovatelli$^\textrm{\scriptsize 173}$,    
F.~Trovato$^\textrm{\scriptsize 153}$,    
L.~Truong$^\textrm{\scriptsize 32b}$,    
M.~Trzebinski$^\textrm{\scriptsize 82}$,    
A.~Trzupek$^\textrm{\scriptsize 82}$,    
F.~Tsai$^\textrm{\scriptsize 44}$,    
J.C-L.~Tseng$^\textrm{\scriptsize 131}$,    
P.V.~Tsiareshka$^\textrm{\scriptsize 105,ah}$,    
A.~Tsirigotis$^\textrm{\scriptsize 159}$,    
N.~Tsirintanis$^\textrm{\scriptsize 9}$,    
V.~Tsiskaridze$^\textrm{\scriptsize 152}$,    
E.G.~Tskhadadze$^\textrm{\scriptsize 156a}$,    
I.I.~Tsukerman$^\textrm{\scriptsize 109}$,    
V.~Tsulaia$^\textrm{\scriptsize 18}$,    
S.~Tsuno$^\textrm{\scriptsize 79}$,    
D.~Tsybychev$^\textrm{\scriptsize 152,163}$,    
Y.~Tu$^\textrm{\scriptsize 61b}$,    
A.~Tudorache$^\textrm{\scriptsize 27b}$,    
V.~Tudorache$^\textrm{\scriptsize 27b}$,    
T.T.~Tulbure$^\textrm{\scriptsize 27a}$,    
A.N.~Tuna$^\textrm{\scriptsize 57}$,    
S.~Turchikhin$^\textrm{\scriptsize 77}$,    
D.~Turgeman$^\textrm{\scriptsize 177}$,    
I.~Turk~Cakir$^\textrm{\scriptsize 4b,t}$,    
R.~Turra$^\textrm{\scriptsize 66a}$,    
P.M.~Tuts$^\textrm{\scriptsize 38}$,    
S~Tzamarias$^\textrm{\scriptsize 159}$,    
E.~Tzovara$^\textrm{\scriptsize 97}$,    
G.~Ucchielli$^\textrm{\scriptsize 45}$,    
I.~Ueda$^\textrm{\scriptsize 79}$,    
M.~Ughetto$^\textrm{\scriptsize 43a,43b}$,    
F.~Ukegawa$^\textrm{\scriptsize 166}$,    
G.~Unal$^\textrm{\scriptsize 35}$,    
A.~Undrus$^\textrm{\scriptsize 29}$,    
G.~Unel$^\textrm{\scriptsize 168}$,    
F.C.~Ungaro$^\textrm{\scriptsize 102}$,    
Y.~Unno$^\textrm{\scriptsize 79}$,    
K.~Uno$^\textrm{\scriptsize 160}$,    
J.~Urban$^\textrm{\scriptsize 28b}$,    
P.~Urquijo$^\textrm{\scriptsize 102}$,    
G.~Usai$^\textrm{\scriptsize 8}$,    
J.~Usui$^\textrm{\scriptsize 79}$,    
L.~Vacavant$^\textrm{\scriptsize 99}$,    
V.~Vacek$^\textrm{\scriptsize 138}$,    
B.~Vachon$^\textrm{\scriptsize 101}$,    
K.O.H.~Vadla$^\textrm{\scriptsize 130}$,    
A.~Vaidya$^\textrm{\scriptsize 92}$,    
C.~Valderanis$^\textrm{\scriptsize 112}$,    
E.~Valdes~Santurio$^\textrm{\scriptsize 43a,43b}$,    
M.~Valente$^\textrm{\scriptsize 52}$,    
S.~Valentinetti$^\textrm{\scriptsize 23b,23a}$,    
A.~Valero$^\textrm{\scriptsize 171}$,    
L.~Val\'ery$^\textrm{\scriptsize 44}$,    
R.A.~Vallance$^\textrm{\scriptsize 21}$,    
A.~Vallier$^\textrm{\scriptsize 5}$,    
J.A.~Valls~Ferrer$^\textrm{\scriptsize 171}$,    
T.R.~Van~Daalen$^\textrm{\scriptsize 14}$,    
H.~Van~der~Graaf$^\textrm{\scriptsize 118}$,    
P.~Van~Gemmeren$^\textrm{\scriptsize 6}$,    
I.~Van~Vulpen$^\textrm{\scriptsize 118}$,    
M.~Vanadia$^\textrm{\scriptsize 71a,71b}$,    
W.~Vandelli$^\textrm{\scriptsize 35}$,    
A.~Vaniachine$^\textrm{\scriptsize 163}$,    
P.~Vankov$^\textrm{\scriptsize 118}$,    
R.~Vari$^\textrm{\scriptsize 70a}$,    
E.W.~Varnes$^\textrm{\scriptsize 7}$,    
C.~Varni$^\textrm{\scriptsize 53b,53a}$,    
T.~Varol$^\textrm{\scriptsize 41}$,    
D.~Varouchas$^\textrm{\scriptsize 128}$,    
K.E.~Varvell$^\textrm{\scriptsize 154}$,    
G.A.~Vasquez$^\textrm{\scriptsize 144b}$,    
J.G.~Vasquez$^\textrm{\scriptsize 180}$,    
F.~Vazeille$^\textrm{\scriptsize 37}$,    
D.~Vazquez~Furelos$^\textrm{\scriptsize 14}$,    
T.~Vazquez~Schroeder$^\textrm{\scriptsize 35}$,    
J.~Veatch$^\textrm{\scriptsize 51}$,    
V.~Vecchio$^\textrm{\scriptsize 72a,72b}$,    
L.M.~Veloce$^\textrm{\scriptsize 164}$,    
F.~Veloso$^\textrm{\scriptsize 136a,136c}$,    
S.~Veneziano$^\textrm{\scriptsize 70a}$,    
A.~Ventura$^\textrm{\scriptsize 65a,65b}$,    
N.~Venturi$^\textrm{\scriptsize 35}$,    
V.~Vercesi$^\textrm{\scriptsize 68a}$,    
M.~Verducci$^\textrm{\scriptsize 72a,72b}$,    
C.M.~Vergel~Infante$^\textrm{\scriptsize 76}$,    
C.~Vergis$^\textrm{\scriptsize 24}$,    
W.~Verkerke$^\textrm{\scriptsize 118}$,    
A.T.~Vermeulen$^\textrm{\scriptsize 118}$,    
J.C.~Vermeulen$^\textrm{\scriptsize 118}$,    
M.C.~Vetterli$^\textrm{\scriptsize 149,at}$,    
N.~Viaux~Maira$^\textrm{\scriptsize 144b}$,    
M.~Vicente~Barreto~Pinto$^\textrm{\scriptsize 52}$,    
I.~Vichou$^\textrm{\scriptsize 170,*}$,    
T.~Vickey$^\textrm{\scriptsize 146}$,    
O.E.~Vickey~Boeriu$^\textrm{\scriptsize 146}$,    
G.H.A.~Viehhauser$^\textrm{\scriptsize 131}$,    
S.~Viel$^\textrm{\scriptsize 18}$,    
L.~Vigani$^\textrm{\scriptsize 131}$,    
M.~Villa$^\textrm{\scriptsize 23b,23a}$,    
M.~Villaplana~Perez$^\textrm{\scriptsize 66a,66b}$,    
E.~Vilucchi$^\textrm{\scriptsize 49}$,    
M.G.~Vincter$^\textrm{\scriptsize 33}$,    
V.B.~Vinogradov$^\textrm{\scriptsize 77}$,    
A.~Vishwakarma$^\textrm{\scriptsize 44}$,    
C.~Vittori$^\textrm{\scriptsize 23b,23a}$,    
I.~Vivarelli$^\textrm{\scriptsize 153}$,    
S.~Vlachos$^\textrm{\scriptsize 10}$,    
M.~Vogel$^\textrm{\scriptsize 179}$,    
P.~Vokac$^\textrm{\scriptsize 138}$,    
G.~Volpi$^\textrm{\scriptsize 14}$,    
S.E.~von~Buddenbrock$^\textrm{\scriptsize 32c}$,    
E.~Von~Toerne$^\textrm{\scriptsize 24}$,    
V.~Vorobel$^\textrm{\scriptsize 139}$,    
K.~Vorobev$^\textrm{\scriptsize 110}$,    
M.~Vos$^\textrm{\scriptsize 171}$,    
J.H.~Vossebeld$^\textrm{\scriptsize 88}$,    
N.~Vranjes$^\textrm{\scriptsize 16}$,    
M.~Vranjes~Milosavljevic$^\textrm{\scriptsize 16}$,    
V.~Vrba$^\textrm{\scriptsize 138}$,    
M.~Vreeswijk$^\textrm{\scriptsize 118}$,    
T.~\v{S}filigoj$^\textrm{\scriptsize 89}$,    
R.~Vuillermet$^\textrm{\scriptsize 35}$,    
I.~Vukotic$^\textrm{\scriptsize 36}$,    
T.~\v{Z}eni\v{s}$^\textrm{\scriptsize 28a}$,    
L.~\v{Z}ivkovi\'{c}$^\textrm{\scriptsize 16}$,    
P.~Wagner$^\textrm{\scriptsize 24}$,    
W.~Wagner$^\textrm{\scriptsize 179}$,    
J.~Wagner-Kuhr$^\textrm{\scriptsize 112}$,    
H.~Wahlberg$^\textrm{\scriptsize 86}$,    
S.~Wahrmund$^\textrm{\scriptsize 46}$,    
K.~Wakamiya$^\textrm{\scriptsize 80}$,    
V.M.~Walbrecht$^\textrm{\scriptsize 113}$,    
J.~Walder$^\textrm{\scriptsize 87}$,    
R.~Walker$^\textrm{\scriptsize 112}$,    
S.D.~Walker$^\textrm{\scriptsize 91}$,    
W.~Walkowiak$^\textrm{\scriptsize 148}$,    
V.~Wallangen$^\textrm{\scriptsize 43a,43b}$,    
A.M.~Wang$^\textrm{\scriptsize 57}$,    
C.~Wang$^\textrm{\scriptsize 58b}$,    
F.~Wang$^\textrm{\scriptsize 178}$,    
H.~Wang$^\textrm{\scriptsize 18}$,    
H.~Wang$^\textrm{\scriptsize 3}$,    
J.~Wang$^\textrm{\scriptsize 154}$,    
J.~Wang$^\textrm{\scriptsize 59b}$,    
P.~Wang$^\textrm{\scriptsize 41}$,    
Q.~Wang$^\textrm{\scriptsize 124}$,    
R.-J.~Wang$^\textrm{\scriptsize 132}$,    
R.~Wang$^\textrm{\scriptsize 58a}$,    
R.~Wang$^\textrm{\scriptsize 6}$,    
S.M.~Wang$^\textrm{\scriptsize 155}$,    
W.T.~Wang$^\textrm{\scriptsize 58a}$,    
W.~Wang$^\textrm{\scriptsize 15c,ad}$,    
W.X.~Wang$^\textrm{\scriptsize 58a,ad}$,    
Y.~Wang$^\textrm{\scriptsize 58a,ak}$,    
Z.~Wang$^\textrm{\scriptsize 58c}$,    
C.~Wanotayaroj$^\textrm{\scriptsize 44}$,    
A.~Warburton$^\textrm{\scriptsize 101}$,    
C.P.~Ward$^\textrm{\scriptsize 31}$,    
D.R.~Wardrope$^\textrm{\scriptsize 92}$,    
A.~Washbrook$^\textrm{\scriptsize 48}$,    
P.M.~Watkins$^\textrm{\scriptsize 21}$,    
A.T.~Watson$^\textrm{\scriptsize 21}$,    
M.F.~Watson$^\textrm{\scriptsize 21}$,    
G.~Watts$^\textrm{\scriptsize 145}$,    
S.~Watts$^\textrm{\scriptsize 98}$,    
B.M.~Waugh$^\textrm{\scriptsize 92}$,    
A.F.~Webb$^\textrm{\scriptsize 11}$,    
S.~Webb$^\textrm{\scriptsize 97}$,    
C.~Weber$^\textrm{\scriptsize 180}$,    
M.S.~Weber$^\textrm{\scriptsize 20}$,    
S.A.~Weber$^\textrm{\scriptsize 33}$,    
S.M.~Weber$^\textrm{\scriptsize 59a}$,    
A.R.~Weidberg$^\textrm{\scriptsize 131}$,    
J.~Weingarten$^\textrm{\scriptsize 45}$,    
M.~Weirich$^\textrm{\scriptsize 97}$,    
C.~Weiser$^\textrm{\scriptsize 50}$,    
P.S.~Wells$^\textrm{\scriptsize 35}$,    
T.~Wenaus$^\textrm{\scriptsize 29}$,    
T.~Wengler$^\textrm{\scriptsize 35}$,    
S.~Wenig$^\textrm{\scriptsize 35}$,    
N.~Wermes$^\textrm{\scriptsize 24}$,    
M.D.~Werner$^\textrm{\scriptsize 76}$,    
P.~Werner$^\textrm{\scriptsize 35}$,    
M.~Wessels$^\textrm{\scriptsize 59a}$,    
T.D.~Weston$^\textrm{\scriptsize 20}$,    
K.~Whalen$^\textrm{\scriptsize 127}$,    
N.L.~Whallon$^\textrm{\scriptsize 145}$,    
A.M.~Wharton$^\textrm{\scriptsize 87}$,    
A.S.~White$^\textrm{\scriptsize 103}$,    
A.~White$^\textrm{\scriptsize 8}$,    
M.J.~White$^\textrm{\scriptsize 1}$,    
R.~White$^\textrm{\scriptsize 144b}$,    
D.~Whiteson$^\textrm{\scriptsize 168}$,    
B.W.~Whitmore$^\textrm{\scriptsize 87}$,    
F.J.~Wickens$^\textrm{\scriptsize 141}$,    
W.~Wiedenmann$^\textrm{\scriptsize 178}$,    
M.~Wielers$^\textrm{\scriptsize 141}$,    
C.~Wiglesworth$^\textrm{\scriptsize 39}$,    
L.A.M.~Wiik-Fuchs$^\textrm{\scriptsize 50}$,    
F.~Wilk$^\textrm{\scriptsize 98}$,    
H.G.~Wilkens$^\textrm{\scriptsize 35}$,    
L.J.~Wilkins$^\textrm{\scriptsize 91}$,    
H.H.~Williams$^\textrm{\scriptsize 133}$,    
S.~Williams$^\textrm{\scriptsize 31}$,    
C.~Willis$^\textrm{\scriptsize 104}$,    
S.~Willocq$^\textrm{\scriptsize 100}$,    
J.A.~Wilson$^\textrm{\scriptsize 21}$,    
I.~Wingerter-Seez$^\textrm{\scriptsize 5}$,    
E.~Winkels$^\textrm{\scriptsize 153}$,    
F.~Winklmeier$^\textrm{\scriptsize 127}$,    
O.J.~Winston$^\textrm{\scriptsize 153}$,    
B.T.~Winter$^\textrm{\scriptsize 50}$,    
M.~Wittgen$^\textrm{\scriptsize 150}$,    
M.~Wobisch$^\textrm{\scriptsize 93}$,    
A.~Wolf$^\textrm{\scriptsize 97}$,    
T.M.H.~Wolf$^\textrm{\scriptsize 118}$,    
R.~Wolff$^\textrm{\scriptsize 99}$,    
M.W.~Wolter$^\textrm{\scriptsize 82}$,    
H.~Wolters$^\textrm{\scriptsize 136a,136c}$,    
V.W.S.~Wong$^\textrm{\scriptsize 172}$,    
N.L.~Woods$^\textrm{\scriptsize 143}$,    
S.D.~Worm$^\textrm{\scriptsize 21}$,    
B.K.~Wosiek$^\textrm{\scriptsize 82}$,    
K.W.~Wo\'{z}niak$^\textrm{\scriptsize 82}$,    
K.~Wraight$^\textrm{\scriptsize 55}$,    
M.~Wu$^\textrm{\scriptsize 36}$,    
S.L.~Wu$^\textrm{\scriptsize 178}$,    
X.~Wu$^\textrm{\scriptsize 52}$,    
Y.~Wu$^\textrm{\scriptsize 58a}$,    
T.R.~Wyatt$^\textrm{\scriptsize 98}$,    
B.M.~Wynne$^\textrm{\scriptsize 48}$,    
S.~Xella$^\textrm{\scriptsize 39}$,    
Z.~Xi$^\textrm{\scriptsize 103}$,    
L.~Xia$^\textrm{\scriptsize 175}$,    
D.~Xu$^\textrm{\scriptsize 15a}$,    
H.~Xu$^\textrm{\scriptsize 58a,e}$,    
L.~Xu$^\textrm{\scriptsize 29}$,    
T.~Xu$^\textrm{\scriptsize 142}$,    
W.~Xu$^\textrm{\scriptsize 103}$,    
Z.~Xu$^\textrm{\scriptsize 150}$,    
B.~Yabsley$^\textrm{\scriptsize 154}$,    
S.~Yacoob$^\textrm{\scriptsize 32a}$,    
K.~Yajima$^\textrm{\scriptsize 129}$,    
D.P.~Yallup$^\textrm{\scriptsize 92}$,    
D.~Yamaguchi$^\textrm{\scriptsize 162}$,    
Y.~Yamaguchi$^\textrm{\scriptsize 162}$,    
A.~Yamamoto$^\textrm{\scriptsize 79}$,    
T.~Yamanaka$^\textrm{\scriptsize 160}$,    
F.~Yamane$^\textrm{\scriptsize 80}$,    
M.~Yamatani$^\textrm{\scriptsize 160}$,    
T.~Yamazaki$^\textrm{\scriptsize 160}$,    
Y.~Yamazaki$^\textrm{\scriptsize 80}$,    
Z.~Yan$^\textrm{\scriptsize 25}$,    
H.J.~Yang$^\textrm{\scriptsize 58c,58d}$,    
H.T.~Yang$^\textrm{\scriptsize 18}$,    
S.~Yang$^\textrm{\scriptsize 75}$,    
Y.~Yang$^\textrm{\scriptsize 160}$,    
Z.~Yang$^\textrm{\scriptsize 17}$,    
W-M.~Yao$^\textrm{\scriptsize 18}$,    
Y.C.~Yap$^\textrm{\scriptsize 44}$,    
Y.~Yasu$^\textrm{\scriptsize 79}$,    
E.~Yatsenko$^\textrm{\scriptsize 58c,58d}$,    
J.~Ye$^\textrm{\scriptsize 41}$,    
S.~Ye$^\textrm{\scriptsize 29}$,    
I.~Yeletskikh$^\textrm{\scriptsize 77}$,    
E.~Yigitbasi$^\textrm{\scriptsize 25}$,    
E.~Yildirim$^\textrm{\scriptsize 97}$,    
K.~Yorita$^\textrm{\scriptsize 176}$,    
K.~Yoshihara$^\textrm{\scriptsize 133}$,    
C.J.S.~Young$^\textrm{\scriptsize 35}$,    
C.~Young$^\textrm{\scriptsize 150}$,    
J.~Yu$^\textrm{\scriptsize 8}$,    
J.~Yu$^\textrm{\scriptsize 76}$,    
X.~Yue$^\textrm{\scriptsize 59a}$,    
S.P.Y.~Yuen$^\textrm{\scriptsize 24}$,    
B.~Zabinski$^\textrm{\scriptsize 82}$,    
G.~Zacharis$^\textrm{\scriptsize 10}$,    
E.~Zaffaroni$^\textrm{\scriptsize 52}$,    
R.~Zaidan$^\textrm{\scriptsize 14}$,    
A.M.~Zaitsev$^\textrm{\scriptsize 140,am}$,    
T.~Zakareishvili$^\textrm{\scriptsize 156b}$,    
N.~Zakharchuk$^\textrm{\scriptsize 33}$,    
S.~Zambito$^\textrm{\scriptsize 57}$,    
D.~Zanzi$^\textrm{\scriptsize 35}$,    
D.R.~Zaripovas$^\textrm{\scriptsize 55}$,    
S.V.~Zei{\ss}ner$^\textrm{\scriptsize 45}$,    
C.~Zeitnitz$^\textrm{\scriptsize 179}$,    
G.~Zemaityte$^\textrm{\scriptsize 131}$,    
J.C.~Zeng$^\textrm{\scriptsize 170}$,    
Q.~Zeng$^\textrm{\scriptsize 150}$,    
O.~Zenin$^\textrm{\scriptsize 140}$,    
D.~Zerwas$^\textrm{\scriptsize 128}$,    
M.~Zgubi\v{c}$^\textrm{\scriptsize 131}$,    
D.F.~Zhang$^\textrm{\scriptsize 58b}$,    
D.~Zhang$^\textrm{\scriptsize 103}$,    
F.~Zhang$^\textrm{\scriptsize 178}$,    
G.~Zhang$^\textrm{\scriptsize 58a}$,    
G.~Zhang$^\textrm{\scriptsize 15b}$,    
H.~Zhang$^\textrm{\scriptsize 15c}$,    
J.~Zhang$^\textrm{\scriptsize 6}$,    
L.~Zhang$^\textrm{\scriptsize 15c}$,    
L.~Zhang$^\textrm{\scriptsize 58a}$,    
M.~Zhang$^\textrm{\scriptsize 170}$,    
P.~Zhang$^\textrm{\scriptsize 15c}$,    
R.~Zhang$^\textrm{\scriptsize 58a}$,    
R.~Zhang$^\textrm{\scriptsize 24}$,    
X.~Zhang$^\textrm{\scriptsize 58b}$,    
Y.~Zhang$^\textrm{\scriptsize 15d}$,    
Z.~Zhang$^\textrm{\scriptsize 128}$,    
P.~Zhao$^\textrm{\scriptsize 47}$,    
Y.~Zhao$^\textrm{\scriptsize 58b,128,ai}$,    
Z.~Zhao$^\textrm{\scriptsize 58a}$,    
A.~Zhemchugov$^\textrm{\scriptsize 77}$,    
Z.~Zheng$^\textrm{\scriptsize 103}$,    
D.~Zhong$^\textrm{\scriptsize 170}$,    
B.~Zhou$^\textrm{\scriptsize 103}$,    
C.~Zhou$^\textrm{\scriptsize 178}$,    
M.S.~Zhou$^\textrm{\scriptsize 15d}$,    
M.~Zhou$^\textrm{\scriptsize 152}$,    
N.~Zhou$^\textrm{\scriptsize 58c}$,    
Y.~Zhou$^\textrm{\scriptsize 7}$,    
C.G.~Zhu$^\textrm{\scriptsize 58b}$,    
H.L.~Zhu$^\textrm{\scriptsize 58a}$,    
H.~Zhu$^\textrm{\scriptsize 15a}$,    
J.~Zhu$^\textrm{\scriptsize 103}$,    
Y.~Zhu$^\textrm{\scriptsize 58a}$,    
X.~Zhuang$^\textrm{\scriptsize 15a}$,    
K.~Zhukov$^\textrm{\scriptsize 108}$,    
V.~Zhulanov$^\textrm{\scriptsize 120b,120a}$,    
A.~Zibell$^\textrm{\scriptsize 174}$,    
D.~Zieminska$^\textrm{\scriptsize 63}$,    
N.I.~Zimine$^\textrm{\scriptsize 77}$,    
S.~Zimmermann$^\textrm{\scriptsize 50}$,    
Z.~Zinonos$^\textrm{\scriptsize 113}$,    
M.~Ziolkowski$^\textrm{\scriptsize 148}$,    
G.~Zobernig$^\textrm{\scriptsize 178}$,    
A.~Zoccoli$^\textrm{\scriptsize 23b,23a}$,    
K.~Zoch$^\textrm{\scriptsize 51}$,    
T.G.~Zorbas$^\textrm{\scriptsize 146}$,    
R.~Zou$^\textrm{\scriptsize 36}$,    
M.~Zur~Nedden$^\textrm{\scriptsize 19}$,    
L.~Zwalinski$^\textrm{\scriptsize 35}$.    
\bigskip
\\

$^{1}$Department of Physics, University of Adelaide, Adelaide; Australia.\\
$^{2}$Physics Department, SUNY Albany, Albany NY; United States of America.\\
$^{3}$Department of Physics, University of Alberta, Edmonton AB; Canada.\\
$^{4}$$^{(a)}$Department of Physics, Ankara University, Ankara;$^{(b)}$Istanbul Aydin University, Istanbul;$^{(c)}$Division of Physics, TOBB University of Economics and Technology, Ankara; Turkey.\\
$^{5}$LAPP, Universit\'e Grenoble Alpes, Universit\'e Savoie Mont Blanc, CNRS/IN2P3, Annecy; France.\\
$^{6}$High Energy Physics Division, Argonne National Laboratory, Argonne IL; United States of America.\\
$^{7}$Department of Physics, University of Arizona, Tucson AZ; United States of America.\\
$^{8}$Department of Physics, University of Texas at Arlington, Arlington TX; United States of America.\\
$^{9}$Physics Department, National and Kapodistrian University of Athens, Athens; Greece.\\
$^{10}$Physics Department, National Technical University of Athens, Zografou; Greece.\\
$^{11}$Department of Physics, University of Texas at Austin, Austin TX; United States of America.\\
$^{12}$$^{(a)}$Bahcesehir University, Faculty of Engineering and Natural Sciences, Istanbul;$^{(b)}$Istanbul Bilgi University, Faculty of Engineering and Natural Sciences, Istanbul;$^{(c)}$Department of Physics, Bogazici University, Istanbul;$^{(d)}$Department of Physics Engineering, Gaziantep University, Gaziantep; Turkey.\\
$^{13}$Institute of Physics, Azerbaijan Academy of Sciences, Baku; Azerbaijan.\\
$^{14}$Institut de F\'isica d'Altes Energies (IFAE), Barcelona Institute of Science and Technology, Barcelona; Spain.\\
$^{15}$$^{(a)}$Institute of High Energy Physics, Chinese Academy of Sciences, Beijing;$^{(b)}$Physics Department, Tsinghua University, Beijing;$^{(c)}$Department of Physics, Nanjing University, Nanjing;$^{(d)}$University of Chinese Academy of Science (UCAS), Beijing; China.\\
$^{16}$Institute of Physics, University of Belgrade, Belgrade; Serbia.\\
$^{17}$Department for Physics and Technology, University of Bergen, Bergen; Norway.\\
$^{18}$Physics Division, Lawrence Berkeley National Laboratory and University of California, Berkeley CA; United States of America.\\
$^{19}$Institut f\"{u}r Physik, Humboldt Universit\"{a}t zu Berlin, Berlin; Germany.\\
$^{20}$Albert Einstein Center for Fundamental Physics and Laboratory for High Energy Physics, University of Bern, Bern; Switzerland.\\
$^{21}$School of Physics and Astronomy, University of Birmingham, Birmingham; United Kingdom.\\
$^{22}$Centro de Investigaci\'ones, Universidad Antonio Nari\~no, Bogota; Colombia.\\
$^{23}$$^{(a)}$Dipartimento di Fisica e Astronomia, Universit\`a di Bologna, Bologna;$^{(b)}$INFN Sezione di Bologna; Italy.\\
$^{24}$Physikalisches Institut, Universit\"{a}t Bonn, Bonn; Germany.\\
$^{25}$Department of Physics, Boston University, Boston MA; United States of America.\\
$^{26}$Department of Physics, Brandeis University, Waltham MA; United States of America.\\
$^{27}$$^{(a)}$Transilvania University of Brasov, Brasov;$^{(b)}$Horia Hulubei National Institute of Physics and Nuclear Engineering, Bucharest;$^{(c)}$Department of Physics, Alexandru Ioan Cuza University of Iasi, Iasi;$^{(d)}$National Institute for Research and Development of Isotopic and Molecular Technologies, Physics Department, Cluj-Napoca;$^{(e)}$University Politehnica Bucharest, Bucharest;$^{(f)}$West University in Timisoara, Timisoara; Romania.\\
$^{28}$$^{(a)}$Faculty of Mathematics, Physics and Informatics, Comenius University, Bratislava;$^{(b)}$Department of Subnuclear Physics, Institute of Experimental Physics of the Slovak Academy of Sciences, Kosice; Slovak Republic.\\
$^{29}$Physics Department, Brookhaven National Laboratory, Upton NY; United States of America.\\
$^{30}$Departamento de F\'isica, Universidad de Buenos Aires, Buenos Aires; Argentina.\\
$^{31}$Cavendish Laboratory, University of Cambridge, Cambridge; United Kingdom.\\
$^{32}$$^{(a)}$Department of Physics, University of Cape Town, Cape Town;$^{(b)}$Department of Mechanical Engineering Science, University of Johannesburg, Johannesburg;$^{(c)}$School of Physics, University of the Witwatersrand, Johannesburg; South Africa.\\
$^{33}$Department of Physics, Carleton University, Ottawa ON; Canada.\\
$^{34}$$^{(a)}$Facult\'e des Sciences Ain Chock, R\'eseau Universitaire de Physique des Hautes Energies - Universit\'e Hassan II, Casablanca;$^{(b)}$Centre National de l'Energie des Sciences Techniques Nucleaires (CNESTEN), Rabat;$^{(c)}$Facult\'e des Sciences Semlalia, Universit\'e Cadi Ayyad, LPHEA-Marrakech;$^{(d)}$Facult\'e des Sciences, Universit\'e Mohamed Premier and LPTPM, Oujda;$^{(e)}$Facult\'e des sciences, Universit\'e Mohammed V, Rabat; Morocco.\\
$^{35}$CERN, Geneva; Switzerland.\\
$^{36}$Enrico Fermi Institute, University of Chicago, Chicago IL; United States of America.\\
$^{37}$LPC, Universit\'e Clermont Auvergne, CNRS/IN2P3, Clermont-Ferrand; France.\\
$^{38}$Nevis Laboratory, Columbia University, Irvington NY; United States of America.\\
$^{39}$Niels Bohr Institute, University of Copenhagen, Copenhagen; Denmark.\\
$^{40}$$^{(a)}$Dipartimento di Fisica, Universit\`a della Calabria, Rende;$^{(b)}$INFN Gruppo Collegato di Cosenza, Laboratori Nazionali di Frascati; Italy.\\
$^{41}$Physics Department, Southern Methodist University, Dallas TX; United States of America.\\
$^{42}$Physics Department, University of Texas at Dallas, Richardson TX; United States of America.\\
$^{43}$$^{(a)}$Department of Physics, Stockholm University;$^{(b)}$Oskar Klein Centre, Stockholm; Sweden.\\
$^{44}$Deutsches Elektronen-Synchrotron DESY, Hamburg and Zeuthen; Germany.\\
$^{45}$Lehrstuhl f{\"u}r Experimentelle Physik IV, Technische Universit{\"a}t Dortmund, Dortmund; Germany.\\
$^{46}$Institut f\"{u}r Kern-~und Teilchenphysik, Technische Universit\"{a}t Dresden, Dresden; Germany.\\
$^{47}$Department of Physics, Duke University, Durham NC; United States of America.\\
$^{48}$SUPA - School of Physics and Astronomy, University of Edinburgh, Edinburgh; United Kingdom.\\
$^{49}$INFN e Laboratori Nazionali di Frascati, Frascati; Italy.\\
$^{50}$Physikalisches Institut, Albert-Ludwigs-Universit\"{a}t Freiburg, Freiburg; Germany.\\
$^{51}$II. Physikalisches Institut, Georg-August-Universit\"{a}t G\"ottingen, G\"ottingen; Germany.\\
$^{52}$D\'epartement de Physique Nucl\'eaire et Corpusculaire, Universit\'e de Gen\`eve, Gen\`eve; Switzerland.\\
$^{53}$$^{(a)}$Dipartimento di Fisica, Universit\`a di Genova, Genova;$^{(b)}$INFN Sezione di Genova; Italy.\\
$^{54}$II. Physikalisches Institut, Justus-Liebig-Universit{\"a}t Giessen, Giessen; Germany.\\
$^{55}$SUPA - School of Physics and Astronomy, University of Glasgow, Glasgow; United Kingdom.\\
$^{56}$LPSC, Universit\'e Grenoble Alpes, CNRS/IN2P3, Grenoble INP, Grenoble; France.\\
$^{57}$Laboratory for Particle Physics and Cosmology, Harvard University, Cambridge MA; United States of America.\\
$^{58}$$^{(a)}$Department of Modern Physics and State Key Laboratory of Particle Detection and Electronics, University of Science and Technology of China, Hefei;$^{(b)}$Institute of Frontier and Interdisciplinary Science and Key Laboratory of Particle Physics and Particle Irradiation (MOE), Shandong University, Qingdao;$^{(c)}$School of Physics and Astronomy, Shanghai Jiao Tong University, KLPPAC-MoE, SKLPPC, Shanghai;$^{(d)}$Tsung-Dao Lee Institute, Shanghai; China.\\
$^{59}$$^{(a)}$Kirchhoff-Institut f\"{u}r Physik, Ruprecht-Karls-Universit\"{a}t Heidelberg, Heidelberg;$^{(b)}$Physikalisches Institut, Ruprecht-Karls-Universit\"{a}t Heidelberg, Heidelberg; Germany.\\
$^{60}$Faculty of Applied Information Science, Hiroshima Institute of Technology, Hiroshima; Japan.\\
$^{61}$$^{(a)}$Department of Physics, Chinese University of Hong Kong, Shatin, N.T., Hong Kong;$^{(b)}$Department of Physics, University of Hong Kong, Hong Kong;$^{(c)}$Department of Physics and Institute for Advanced Study, Hong Kong University of Science and Technology, Clear Water Bay, Kowloon, Hong Kong; China.\\
$^{62}$Department of Physics, National Tsing Hua University, Hsinchu; Taiwan.\\
$^{63}$Department of Physics, Indiana University, Bloomington IN; United States of America.\\
$^{64}$$^{(a)}$INFN Gruppo Collegato di Udine, Sezione di Trieste, Udine;$^{(b)}$ICTP, Trieste;$^{(c)}$Dipartimento di Chimica, Fisica e Ambiente, Universit\`a di Udine, Udine; Italy.\\
$^{65}$$^{(a)}$INFN Sezione di Lecce;$^{(b)}$Dipartimento di Matematica e Fisica, Universit\`a del Salento, Lecce; Italy.\\
$^{66}$$^{(a)}$INFN Sezione di Milano;$^{(b)}$Dipartimento di Fisica, Universit\`a di Milano, Milano; Italy.\\
$^{67}$$^{(a)}$INFN Sezione di Napoli;$^{(b)}$Dipartimento di Fisica, Universit\`a di Napoli, Napoli; Italy.\\
$^{68}$$^{(a)}$INFN Sezione di Pavia;$^{(b)}$Dipartimento di Fisica, Universit\`a di Pavia, Pavia; Italy.\\
$^{69}$$^{(a)}$INFN Sezione di Pisa;$^{(b)}$Dipartimento di Fisica E. Fermi, Universit\`a di Pisa, Pisa; Italy.\\
$^{70}$$^{(a)}$INFN Sezione di Roma;$^{(b)}$Dipartimento di Fisica, Sapienza Universit\`a di Roma, Roma; Italy.\\
$^{71}$$^{(a)}$INFN Sezione di Roma Tor Vergata;$^{(b)}$Dipartimento di Fisica, Universit\`a di Roma Tor Vergata, Roma; Italy.\\
$^{72}$$^{(a)}$INFN Sezione di Roma Tre;$^{(b)}$Dipartimento di Matematica e Fisica, Universit\`a Roma Tre, Roma; Italy.\\
$^{73}$$^{(a)}$INFN-TIFPA;$^{(b)}$Universit\`a degli Studi di Trento, Trento; Italy.\\
$^{74}$Institut f\"{u}r Astro-~und Teilchenphysik, Leopold-Franzens-Universit\"{a}t, Innsbruck; Austria.\\
$^{75}$University of Iowa, Iowa City IA; United States of America.\\
$^{76}$Department of Physics and Astronomy, Iowa State University, Ames IA; United States of America.\\
$^{77}$Joint Institute for Nuclear Research, Dubna; Russia.\\
$^{78}$$^{(a)}$Departamento de Engenharia El\'etrica, Universidade Federal de Juiz de Fora (UFJF), Juiz de Fora;$^{(b)}$Universidade Federal do Rio De Janeiro COPPE/EE/IF, Rio de Janeiro;$^{(c)}$Universidade Federal de S\~ao Jo\~ao del Rei (UFSJ), S\~ao Jo\~ao del Rei;$^{(d)}$Instituto de F\'isica, Universidade de S\~ao Paulo, S\~ao Paulo; Brazil.\\
$^{79}$KEK, High Energy Accelerator Research Organization, Tsukuba; Japan.\\
$^{80}$Graduate School of Science, Kobe University, Kobe; Japan.\\
$^{81}$$^{(a)}$AGH University of Science and Technology, Faculty of Physics and Applied Computer Science, Krakow;$^{(b)}$Marian Smoluchowski Institute of Physics, Jagiellonian University, Krakow; Poland.\\
$^{82}$Institute of Nuclear Physics Polish Academy of Sciences, Krakow; Poland.\\
$^{83}$Faculty of Science, Kyoto University, Kyoto; Japan.\\
$^{84}$Kyoto University of Education, Kyoto; Japan.\\
$^{85}$Research Center for Advanced Particle Physics and Department of Physics, Kyushu University, Fukuoka ; Japan.\\
$^{86}$Instituto de F\'{i}sica La Plata, Universidad Nacional de La Plata and CONICET, La Plata; Argentina.\\
$^{87}$Physics Department, Lancaster University, Lancaster; United Kingdom.\\
$^{88}$Oliver Lodge Laboratory, University of Liverpool, Liverpool; United Kingdom.\\
$^{89}$Department of Experimental Particle Physics, Jo\v{z}ef Stefan Institute and Department of Physics, University of Ljubljana, Ljubljana; Slovenia.\\
$^{90}$School of Physics and Astronomy, Queen Mary University of London, London; United Kingdom.\\
$^{91}$Department of Physics, Royal Holloway University of London, Egham; United Kingdom.\\
$^{92}$Department of Physics and Astronomy, University College London, London; United Kingdom.\\
$^{93}$Louisiana Tech University, Ruston LA; United States of America.\\
$^{94}$Fysiska institutionen, Lunds universitet, Lund; Sweden.\\
$^{95}$Centre de Calcul de l'Institut National de Physique Nucl\'eaire et de Physique des Particules (IN2P3), Villeurbanne; France.\\
$^{96}$Departamento de F\'isica Teorica C-15 and CIAFF, Universidad Aut\'onoma de Madrid, Madrid; Spain.\\
$^{97}$Institut f\"{u}r Physik, Universit\"{a}t Mainz, Mainz; Germany.\\
$^{98}$School of Physics and Astronomy, University of Manchester, Manchester; United Kingdom.\\
$^{99}$CPPM, Aix-Marseille Universit\'e, CNRS/IN2P3, Marseille; France.\\
$^{100}$Department of Physics, University of Massachusetts, Amherst MA; United States of America.\\
$^{101}$Department of Physics, McGill University, Montreal QC; Canada.\\
$^{102}$School of Physics, University of Melbourne, Victoria; Australia.\\
$^{103}$Department of Physics, University of Michigan, Ann Arbor MI; United States of America.\\
$^{104}$Department of Physics and Astronomy, Michigan State University, East Lansing MI; United States of America.\\
$^{105}$B.I. Stepanov Institute of Physics, National Academy of Sciences of Belarus, Minsk; Belarus.\\
$^{106}$Research Institute for Nuclear Problems of Byelorussian State University, Minsk; Belarus.\\
$^{107}$Group of Particle Physics, University of Montreal, Montreal QC; Canada.\\
$^{108}$P.N. Lebedev Physical Institute of the Russian Academy of Sciences, Moscow; Russia.\\
$^{109}$Institute for Theoretical and Experimental Physics (ITEP), Moscow; Russia.\\
$^{110}$National Research Nuclear University MEPhI, Moscow; Russia.\\
$^{111}$D.V. Skobeltsyn Institute of Nuclear Physics, M.V. Lomonosov Moscow State University, Moscow; Russia.\\
$^{112}$Fakult\"at f\"ur Physik, Ludwig-Maximilians-Universit\"at M\"unchen, M\"unchen; Germany.\\
$^{113}$Max-Planck-Institut f\"ur Physik (Werner-Heisenberg-Institut), M\"unchen; Germany.\\
$^{114}$Nagasaki Institute of Applied Science, Nagasaki; Japan.\\
$^{115}$Graduate School of Science and Kobayashi-Maskawa Institute, Nagoya University, Nagoya; Japan.\\
$^{116}$Department of Physics and Astronomy, University of New Mexico, Albuquerque NM; United States of America.\\
$^{117}$Institute for Mathematics, Astrophysics and Particle Physics, Radboud University Nijmegen/Nikhef, Nijmegen; Netherlands.\\
$^{118}$Nikhef National Institute for Subatomic Physics and University of Amsterdam, Amsterdam; Netherlands.\\
$^{119}$Department of Physics, Northern Illinois University, DeKalb IL; United States of America.\\
$^{120}$$^{(a)}$Budker Institute of Nuclear Physics, SB RAS, Novosibirsk;$^{(b)}$Novosibirsk State University Novosibirsk; Russia.\\
$^{121}$Department of Physics, New York University, New York NY; United States of America.\\
$^{122}$Ohio State University, Columbus OH; United States of America.\\
$^{123}$Faculty of Science, Okayama University, Okayama; Japan.\\
$^{124}$Homer L. Dodge Department of Physics and Astronomy, University of Oklahoma, Norman OK; United States of America.\\
$^{125}$Department of Physics, Oklahoma State University, Stillwater OK; United States of America.\\
$^{126}$Palack\'y University, RCPTM, Joint Laboratory of Optics, Olomouc; Czech Republic.\\
$^{127}$Center for High Energy Physics, University of Oregon, Eugene OR; United States of America.\\
$^{128}$LAL, Universit\'e Paris-Sud, CNRS/IN2P3, Universit\'e Paris-Saclay, Orsay; France.\\
$^{129}$Graduate School of Science, Osaka University, Osaka; Japan.\\
$^{130}$Department of Physics, University of Oslo, Oslo; Norway.\\
$^{131}$Department of Physics, Oxford University, Oxford; United Kingdom.\\
$^{132}$LPNHE, Sorbonne Universit\'e, Paris Diderot Sorbonne Paris Cit\'e, CNRS/IN2P3, Paris; France.\\
$^{133}$Department of Physics, University of Pennsylvania, Philadelphia PA; United States of America.\\
$^{134}$Konstantinov Nuclear Physics Institute of National Research Centre "Kurchatov Institute", PNPI, St. Petersburg; Russia.\\
$^{135}$Department of Physics and Astronomy, University of Pittsburgh, Pittsburgh PA; United States of America.\\
$^{136}$$^{(a)}$Laborat\'orio de Instrumenta\c{c}\~ao e F\'isica Experimental de Part\'iculas - LIP;$^{(b)}$Departamento de F\'isica, Faculdade de Ci\^{e}ncias, Universidade de Lisboa, Lisboa;$^{(c)}$Departamento de F\'isica, Universidade de Coimbra, Coimbra;$^{(d)}$Centro de F\'isica Nuclear da Universidade de Lisboa, Lisboa;$^{(e)}$Departamento de F\'isica, Universidade do Minho, Braga;$^{(f)}$Departamento de F\'isica Teorica y del Cosmos, Universidad de Granada, Granada (Spain);$^{(g)}$Dep F\'isica and CEFITEC of Faculdade de Ci\^{e}ncias e Tecnologia, Universidade Nova de Lisboa, Caparica; Portugal.\\
$^{137}$Institute of Physics, Academy of Sciences of the Czech Republic, Prague; Czech Republic.\\
$^{138}$Czech Technical University in Prague, Prague; Czech Republic.\\
$^{139}$Charles University, Faculty of Mathematics and Physics, Prague; Czech Republic.\\
$^{140}$State Research Center Institute for High Energy Physics, NRC KI, Protvino; Russia.\\
$^{141}$Particle Physics Department, Rutherford Appleton Laboratory, Didcot; United Kingdom.\\
$^{142}$IRFU, CEA, Universit\'e Paris-Saclay, Gif-sur-Yvette; France.\\
$^{143}$Santa Cruz Institute for Particle Physics, University of California Santa Cruz, Santa Cruz CA; United States of America.\\
$^{144}$$^{(a)}$Departamento de F\'isica, Pontificia Universidad Cat\'olica de Chile, Santiago;$^{(b)}$Departamento de F\'isica, Universidad T\'ecnica Federico Santa Mar\'ia, Valpara\'iso; Chile.\\
$^{145}$Department of Physics, University of Washington, Seattle WA; United States of America.\\
$^{146}$Department of Physics and Astronomy, University of Sheffield, Sheffield; United Kingdom.\\
$^{147}$Department of Physics, Shinshu University, Nagano; Japan.\\
$^{148}$Department Physik, Universit\"{a}t Siegen, Siegen; Germany.\\
$^{149}$Department of Physics, Simon Fraser University, Burnaby BC; Canada.\\
$^{150}$SLAC National Accelerator Laboratory, Stanford CA; United States of America.\\
$^{151}$Physics Department, Royal Institute of Technology, Stockholm; Sweden.\\
$^{152}$Departments of Physics and Astronomy, Stony Brook University, Stony Brook NY; United States of America.\\
$^{153}$Department of Physics and Astronomy, University of Sussex, Brighton; United Kingdom.\\
$^{154}$School of Physics, University of Sydney, Sydney; Australia.\\
$^{155}$Institute of Physics, Academia Sinica, Taipei; Taiwan.\\
$^{156}$$^{(a)}$E. Andronikashvili Institute of Physics, Iv. Javakhishvili Tbilisi State University, Tbilisi;$^{(b)}$High Energy Physics Institute, Tbilisi State University, Tbilisi; Georgia.\\
$^{157}$Department of Physics, Technion, Israel Institute of Technology, Haifa; Israel.\\
$^{158}$Raymond and Beverly Sackler School of Physics and Astronomy, Tel Aviv University, Tel Aviv; Israel.\\
$^{159}$Department of Physics, Aristotle University of Thessaloniki, Thessaloniki; Greece.\\
$^{160}$International Center for Elementary Particle Physics and Department of Physics, University of Tokyo, Tokyo; Japan.\\
$^{161}$Graduate School of Science and Technology, Tokyo Metropolitan University, Tokyo; Japan.\\
$^{162}$Department of Physics, Tokyo Institute of Technology, Tokyo; Japan.\\
$^{163}$Tomsk State University, Tomsk; Russia.\\
$^{164}$Department of Physics, University of Toronto, Toronto ON; Canada.\\
$^{165}$$^{(a)}$TRIUMF, Vancouver BC;$^{(b)}$Department of Physics and Astronomy, York University, Toronto ON; Canada.\\
$^{166}$Division of Physics and Tomonaga Center for the History of the Universe, Faculty of Pure and Applied Sciences, University of Tsukuba, Tsukuba; Japan.\\
$^{167}$Department of Physics and Astronomy, Tufts University, Medford MA; United States of America.\\
$^{168}$Department of Physics and Astronomy, University of California Irvine, Irvine CA; United States of America.\\
$^{169}$Department of Physics and Astronomy, University of Uppsala, Uppsala; Sweden.\\
$^{170}$Department of Physics, University of Illinois, Urbana IL; United States of America.\\
$^{171}$Instituto de F\'isica Corpuscular (IFIC), Centro Mixto Universidad de Valencia - CSIC, Valencia; Spain.\\
$^{172}$Department of Physics, University of British Columbia, Vancouver BC; Canada.\\
$^{173}$Department of Physics and Astronomy, University of Victoria, Victoria BC; Canada.\\
$^{174}$Fakult\"at f\"ur Physik und Astronomie, Julius-Maximilians-Universit\"at W\"urzburg, W\"urzburg; Germany.\\
$^{175}$Department of Physics, University of Warwick, Coventry; United Kingdom.\\
$^{176}$Waseda University, Tokyo; Japan.\\
$^{177}$Department of Particle Physics, Weizmann Institute of Science, Rehovot; Israel.\\
$^{178}$Department of Physics, University of Wisconsin, Madison WI; United States of America.\\
$^{179}$Fakult{\"a}t f{\"u}r Mathematik und Naturwissenschaften, Fachgruppe Physik, Bergische Universit\"{a}t Wuppertal, Wuppertal; Germany.\\
$^{180}$Department of Physics, Yale University, New Haven CT; United States of America.\\
$^{181}$Yerevan Physics Institute, Yerevan; Armenia.\\

$^{a}$ Also at Borough of Manhattan Community College, City University of New York, NY; United States of America.\\
$^{b}$ Also at California State University, East Bay; United States of America.\\
$^{c}$ Also at Centre for High Performance Computing, CSIR Campus, Rosebank, Cape Town; South Africa.\\
$^{d}$ Also at CERN, Geneva; Switzerland.\\
$^{e}$ Also at CPPM, Aix-Marseille Universit\'e, CNRS/IN2P3, Marseille; France.\\
$^{f}$ Also at D\'epartement de Physique Nucl\'eaire et Corpusculaire, Universit\'e de Gen\`eve, Gen\`eve; Switzerland.\\
$^{g}$ Also at Departament de Fisica de la Universitat Autonoma de Barcelona, Barcelona; Spain.\\
$^{h}$ Also at Departamento de F\'isica Teorica y del Cosmos, Universidad de Granada, Granada (Spain); Spain.\\
$^{i}$ Also at Department of Applied Physics and Astronomy, University of Sharjah, Sharjah; United Arab Emirates.\\
$^{j}$ Also at Department of Financial and Management Engineering, University of the Aegean, Chios; Greece.\\
$^{k}$ Also at Department of Physics and Astronomy, University of Louisville, Louisville, KY; United States of America.\\
$^{l}$ Also at Department of Physics and Astronomy, University of Sheffield, Sheffield; United Kingdom.\\
$^{m}$ Also at Department of Physics, California State University, Fresno CA; United States of America.\\
$^{n}$ Also at Department of Physics, California State University, Sacramento CA; United States of America.\\
$^{o}$ Also at Department of Physics, King's College London, London; United Kingdom.\\
$^{p}$ Also at Department of Physics, St. Petersburg State Polytechnical University, St. Petersburg; Russia.\\
$^{q}$ Also at Department of Physics, Stanford University; United States of America.\\
$^{r}$ Also at Department of Physics, University of Fribourg, Fribourg; Switzerland.\\
$^{s}$ Also at Department of Physics, University of Michigan, Ann Arbor MI; United States of America.\\
$^{t}$ Also at Giresun University, Faculty of Engineering, Giresun; Turkey.\\
$^{u}$ Also at Graduate School of Science, Osaka University, Osaka; Japan.\\
$^{v}$ Also at Hellenic Open University, Patras; Greece.\\
$^{w}$ Also at Horia Hulubei National Institute of Physics and Nuclear Engineering, Bucharest; Romania.\\
$^{x}$ Also at II. Physikalisches Institut, Georg-August-Universit\"{a}t G\"ottingen, G\"ottingen; Germany.\\
$^{y}$ Also at Institucio Catalana de Recerca i Estudis Avancats, ICREA, Barcelona; Spain.\\
$^{z}$ Also at Institut f\"{u}r Experimentalphysik, Universit\"{a}t Hamburg, Hamburg; Germany.\\
$^{aa}$ Also at Institute for Mathematics, Astrophysics and Particle Physics, Radboud University Nijmegen/Nikhef, Nijmegen; Netherlands.\\
$^{ab}$ Also at Institute for Particle and Nuclear Physics, Wigner Research Centre for Physics, Budapest; Hungary.\\
$^{ac}$ Also at Institute of Particle Physics (IPP); Canada.\\
$^{ad}$ Also at Institute of Physics, Academia Sinica, Taipei; Taiwan.\\
$^{ae}$ Also at Institute of Physics, Azerbaijan Academy of Sciences, Baku; Azerbaijan.\\
$^{af}$ Also at Institute of Theoretical Physics, Ilia State University, Tbilisi; Georgia.\\
$^{ag}$ Also at Istanbul University, Dept. of Physics, Istanbul; Turkey.\\
$^{ah}$ Also at Joint Institute for Nuclear Research, Dubna; Russia.\\
$^{ai}$ Also at LAL, Universit\'e Paris-Sud, CNRS/IN2P3, Universit\'e Paris-Saclay, Orsay; France.\\
$^{aj}$ Also at Louisiana Tech University, Ruston LA; United States of America.\\
$^{ak}$ Also at LPNHE, Sorbonne Universit\'e, Paris Diderot Sorbonne Paris Cit\'e, CNRS/IN2P3, Paris; France.\\
$^{al}$ Also at Manhattan College, New York NY; United States of America.\\
$^{am}$ Also at Moscow Institute of Physics and Technology State University, Dolgoprudny; Russia.\\
$^{an}$ Also at National Research Nuclear University MEPhI, Moscow; Russia.\\
$^{ao}$ Also at Physikalisches Institut, Albert-Ludwigs-Universit\"{a}t Freiburg, Freiburg; Germany.\\
$^{ap}$ Also at School of Physics, Sun Yat-sen University, Guangzhou; China.\\
$^{aq}$ Also at The City College of New York, New York NY; United States of America.\\
$^{ar}$ Also at The Collaborative Innovation Center of Quantum Matter (CICQM), Beijing; China.\\
$^{as}$ Also at Tomsk State University, Tomsk, and Moscow Institute of Physics and Technology State University, Dolgoprudny; Russia.\\
$^{at}$ Also at TRIUMF, Vancouver BC; Canada.\\
$^{au}$ Also at Universita di Napoli Parthenope, Napoli; Italy.\\
$^{*}$ Deceased

\end{flushleft}

% Created with Glance <Atlas.Glance@cern.ch>
 

\end{document}

