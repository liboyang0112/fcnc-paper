%-------------------------------------------------------------------------------
% This file provides template EXOT group object descriptions and cuts.
% \pdfinclusioncopyfonts=1
% This command may be needed in order to get \ell in PDF plots to appear. Found in
% https://tex.stackexchange.com/questions/322010/pdflatex-glyph-undefined-symbols-disappear-from-included-pdf
%-------------------------------------------------------------------------------
% Specify where ATLAS LaTeX style files can be found.
\newcommand*{\ATLASLATEXPATH}{../../latex/}
% Use this variant if the files are in a central location, e.g. $HOME/texmf.
% \newcommand*{\ATLASLATEXPATH}{}
%-------------------------------------------------------------------------------
\documentclass[NOTE, atlasdraft=true, texlive=2016, USenglish]{\ATLASLATEXPATH atlasdoc}
% The language of the document must be set: usually UKenglish or USenglish.
% british and american also work!
% Commonly used options:
%  atlasdraft=true|false This document is an ATLAS draft.
%  texlive=YYYY          Specify TeX Live version (2016 is default).
%  txfonts=true|false    Use txfonts rather than the default newtx
%  paper=a4|letter       Set paper size to A4 (default) or letter.

%-------------------------------------------------------------------------------
% Extra packages:
\usepackage{\ATLASLATEXPATH atlaspackage}
% Commonly used options:
%  subfigure|subfig|subcaption  to use one of these packages for figures in figures.
%-------------------------------------------------------------------------------
\usepackage{multirow}

% Useful macros
\usepackage[jetetmiss]{\ATLASLATEXPATH atlasphysics}
% See doc/atlas_physics.pdf for a list of the defined symbols.
% Default options are:
%   true:  journal, misc, particle, unit, xref
%   false: BSM, heppparticle, hepprocess, hion, jetetmiss, math, process,
%          other, snippets, texmf
% See the package for details on the options.

% Add you own definitions here (file atlas-document-defs.sty).
% \usepackage{atlas-document-defs}

% Paths for figures - do not forget the / at the end of the directory name.
\graphicspath{{\ATLASLATEXPATH ../logos/}{figures/}}

%-------------------------------------------------------------------------------
% Generic document information
%-------------------------------------------------------------------------------

\AtlasTitle{EXOT group text snippets for INT notes}
\AtlasVersion{0.1}
\author{ATLAS EXOT Group}
\AtlasRefCode{EXOT-2018-XX}
\AtlasAbstract{%
  This note contains text snippets and tables that should be included in supporting notes
  from the EXOT group.

  The templates are in American English.
  If wanted, some adaption to British English could be made. 

  This document was generated using version \ATPackageVersion\ of the ATLAS \LaTeX\ package.

  \emph{2018-10-23: This file is a work in progress (WIP) and will probably be updated.
  Backwards incompatible changes may be made as the examples develop.}
}
% Author and title for the PDF file
\hypersetup{pdftitle={ATLAS EXOT supporting note},pdfauthor={ATLAS EXOT group}}

%-------------------------------------------------------------------------------
% Main document
%-------------------------------------------------------------------------------
\begin{document}

\maketitle

\tableofcontents

\subsection{Target} 

\(\mathcal{O}\)(1 paragraph)
Is this a new analysis? If not, what are the main improvements expected with respect to the previous version?
What is the target publication date / conference?

\subsection{Context and Motivation} 

Motivate this analysis in 1 paragraph: why is this signature interesting? Which kind of models are you probing?

How is the analysis done is 1 paragraph: what are the main BG processes and how do you estimate them (are they MC- or data-driven,
what is the general idea of the control regions, \ldots), general characteristics of the PL fit (which distribution, binned?, \ldots)

\subsection{Milestones}

 Table giving a factual list of who is working on what and what else they do; the idea is to show how the team can / does progress. 
 Including dates for completion of these milestones will help further demonstrate that you are ready for the collaboration
 review, in the form of an editorial board.

%Example : 
The following table summarizes the tasks to be worked on by analysis team.
This is not a complete analysis outline but only an overview of the further steps to be taken as of the time of writing.
Details are not provided here but in the dedicated sections throughout this note.
Tasks which are based on established techniques and straightforward to achieve are marked green in the table.
Tasks which require new work are marked red.
Concerning the involved people, the responsible student supervisors and analysis coordinators are already mentioned in the list of contributions above,
which shall not be repeated here.
A fair overview of all single tasks including past work and of all relevant team members is only given in the list of contributions above!
It is also worth noting that some of the tasks listed below are being worked on in parallel. 

\begin{table}[ht]
  \caption{Milestones in the analysis.}%
  \label{tab:Miles_Ahead} 
  % \resizebox{\textwidth}{!}{
\begin{tabular}{llll} 
  \toprule
  \textbf{Task} & \textbf{Analyzer} & \textbf{Role} & \textbf{Other responsibilities} \\
  \midrule
  \multicolumn{4}{p{\textwidth}}{\textbf{Describe a first milestone.}} \\
  \midrule
  \textcolor{green}{A straightforward task}       & Name         & PhD student, PostDoc/Prof/\ldots & thesis writing \\
  &&& / teaching \\
  &&& / name some CP work \ldots \\ 
  \textcolor{red}{A more involved task}      &    &    &  \\ 
  \bottomrule
  
  \multicolumn{4}{l}{\textbf{Describe a second milestone}} \\
  \midrule
  First task \ldots      &          &  &  \\ 
  \bottomrule
\end{tabular}
%}
\end{table}




\section{Object selection}
The supporting notes should now include the following standardized tables of properties: each analysis should simply fill them
in by writing / replacing the value with the appropriate number or by choosing the appropriate option.
The idea of these tables is to harmonize some sections of the supporting notes as to make review and analysis comparisons simpler.

If you use non-standard selections which do not fit in these tables, this should of course be noted and discussed in more detail in the text.
 
\subsection{Electron selection}

\begin{table}[ht]
  \caption{Electron selection criteria.}%
  \label{tab:object:electron}
  \centering
  % \resizebox{\textwidth}{!}{
  \begin{tabular}{ll}
    \toprule
    Feature & \multicolumn{1}{c}{Criterion} \\
    \midrule
    Pseudorapidity range & \(|\eta| <\) X\\
    Energy calibration & \texttt{es2017\_R21\_PRE} (ESModel)\\
    Energy & \(E > \SI[parse-numbers=false]{XX}{\GeV}\) \\
    Transverse energy & \(\ET > \SI[parse-numbers=false]{XX}{\GeV}\) \\
    Transverse momentum & \(\pT > \SI[parse-numbers=false]{XX}{\GeV}\) \\
    \midrule
    \multirow{2}{*}{Object quality} & Not from a bad calorimeter cluster (\texttt{BADCLUSELECTRON})\\ %\cline{2-2}
      & Remove clusters from regions with EMEC bad HV (2016 data only) \\
    \midrule
    \multirow{2}{*}{Track to vertex association} & \(|d_{0}^{\text{BL}}(\sigma)| < X\) \\ %\cline{2-2}
    & \(|\Delta z_{0}^{\text{BL}} \sin{\theta}| < \SI[parse-numbers=false]{X}{\mm}\) \\
    \midrule
    Identification & (\texttt{Loose/Medium/Tight}) \\
    Isolation & \texttt{LooseTrackOnly / Loose / Tight / Gradient / \ldots} \\
      \bottomrule
  \end{tabular}
  % }
\end{table}

Notes:
\begin{itemize}
\item Pseudorapidity: when the calorimeter crack is not excluded, the range can be indicated simply as \enquote{\(|\eta| < 2.47\)}, when the crack is excluded: \enquote{\((|\eta| < 1.37) \quad || \quad (1.52 < |\eta| < 2.47)\)}.
\item Usually only one among \enquote{Energy}, \enquote{Transverse energy} and \enquote{Transverse momentum} criteria is applied --- the \SI{30}{\GeV} value is just an example.
  In special cases energy (i.e.\ calorimeter-based measurement) and momentum (i.e.\ tracking-based measurement) criteria can be required in order to constraint different aspects of the reconstruction.
\item Electron ID\@: 3 working points (Loose/Medium/Tight) are evaluated using the Likelihood-based (LH) method, by the
  \href{https://twiki.cern.ch/twiki/bin/view/AtlasProtected/EGammaIdentificationRun2}{ElectronPhotonSelectorTools}.
\item Energy calibration of electrons is implemented in the\\
  \href{https://twiki.cern.ch/twiki/bin/view/AtlasProtected/ElectronPhotonFourMomentumCorrection}{ElectronPhotonFourMomentumCorrection} tool.
\item Scale Factors for efficiencies for electrons are implemented in the\\
  \href{https://twiki.cern.ch/twiki/bin/view/AtlasProtected/XAODElectronEfficiencyCorrectionTool}{ElectronEfficiencyCorrection} tool.
\item Updated configurations for the EGamma CP tools can be found on this \href{https://twiki.cern.ch/twiki/bin/view/AtlasProtected/EGammaRecommendationsR21}{TWiki} page.
\end{itemize}

\newpage

\subsection{Photon selection}

\begin{table}[ht]
  \caption{Photon selection criteria.}%
  \label{tab:object:photon} 
  \centering
  % \resizebox{\textwidth}{!}{
  \begin{tabular}{ll}
    \toprule
    Feature & \multicolumn{1}{c}{Criterion} \\
    \midrule
    Pseudorapidity range & \(|\eta| <\) X\\
    Energy calibration & \texttt{es2017\_R21\_PRE} (ESModel)\\
    Energy & \(E > \SI[parse-numbers=false]{XX}{\GeV}\) \\
    Transverse energy & \(\ET > \SI[parse-numbers=false]{XX}{\GeV}\) \\
    \midrule
    \multirow{2}{*}{Object quality} & Not from a bad calorimeter cluster (\texttt{BADCLUSELECTRON})\\ %\cline{2-2}
      & Remove clusters from regions with EMEC bad HV (2016 data only) \\
    \midrule
    Photon cleaning & \texttt{passOQquality} \\
    Fudging & Applied for Full sim / not for AtlFastII \\
    \midrule
    Identification & (\texttt{Loose/Tight}) \\
    Isolation &  \texttt{FixedCutTightCaloOnly / FixedCutTight / FixedCutLoose} \\
    \bottomrule
  \end{tabular}
  %  }
\end{table}

Notes:
\begin{itemize}
\item Pseudorapidity: please note that the maximum value for \(|\eta|\) for photon candidates (2.37) is smaller than for electron candidates (2.47). 
  If crack excluded: \enquote{\((|\eta| < 1.37) \quad || \quad (1.52 < |\eta| < 2.37)\)}.
\item Usually only one between \enquote{Energy} and \enquote{Transverse energy} criteria is applied --- the \SI{30}{\GeV} value is just an example.
\item Photon cleaning: a new Photon helper is available to apply the photon cleaning cut 
  (from the \texttt{ElectronPhotonSelectorTools}, tag \(\ge\) 00-02-92-21, release \(\ge\) 2.4.30).
\item Photon ID\@: 2 working points (Loose/Tight) are evaluated using a cut-based method, by the
  \href{https://twiki.cern.ch/twiki/bin/view/AtlasProtected/EGammaIdentificationRun2}{ElectronPhotonSelectorTools}.
\item Energy calibration of photons is implemented in the\\
  \href{https://twiki.cern.ch/twiki/bin/view/AtlasProtected/ElectronPhotonFourMomentumCorrection}{ElectronPhotonFourMomentumCorrection} tool.
\item Scale Factors for efficiencies for photons are implemented in the\\
  \href{https://twiki.cern.ch/twiki/bin/view/AtlasProtected/XAODElectronEfficiencyCorrectionTool}{ElectronEfficiencyCorrection} tool.
\item Updated configurations for the EGamma CP tools can be found on this \href{https://twiki.cern.ch/twiki/bin/view/AtlasProtected/EGammaRecommendationsR21}{TWiki} page.
\end{itemize}

\subsection{Muon selection}

\begin{table}[ht]
  \caption{Muon selection criteria.}%
  \label{tab:object:muon}
  \centering
  % \resizebox{\textwidth}{!}{
  \begin{tabular}[ht]{ll}
    \toprule
    Feature & Criterion \\
    \midrule
    Selection working point & \texttt{Loose/Medium/Tight /High-pT} \\
    Isolation working point & \texttt{LooseTrackOnly/Loose/Tight/Gradient/\ldots}\\
    Momentum calibration & Sagitta correction [used/not used] \\
    \pT Cut & \SI[parse-numbers=false]{X}{\GeV} \\
    \(|\eta|\) cut & \(< X\) \\
    \dzero significance cut & X \\
    \(z_{0}\) cut & \SI[parse-numbers=false]{X}{\mm} \\
    \bottomrule
  \end{tabular}
  % }
\end{table}

The selection criteria are implemented in the \texttt{MuonSelectorTools-XX-XX-XX}\\
with \texttt{MuonMomentumCorrections-XX-XX-XX}, 
isolation in \texttt{IsolationSelection-XX-XX-XX} and \dzero and \(z_{0}\) cuts in \texttt{xAODTracking-XX-XX-XX}.
The muon recommendations can be found in 
\href{https://twiki.cern.ch/twiki/bin/view/AtlasProtected/MCPAnalysisGuidelinesMC16}{MCPAnalysisGuidelinesMC16}.

\subsection{Tau selection}

\begin{table}[ht]
  \caption{Tau selection criteria.}%
  \label{tab:object:tau}
  \centering
  % \resizebox{\textwidth}{!}{
  \begin{tabular}{ll}
  \toprule
  Feature & Criterion \\
  \midrule
  Pseudorapidity range & \(|\eta| < X\) \\
  Track selection & 1 or 3 tracks \\\
  Charge & \(|Q| = 1\) \\
  Tau energy scale & \texttt{MVA TES}\\
  Transverse momentum & \(\pT > \SI[parse-numbers=false]{XX}{\GeV}\) \\
  Jet rejection & BDT-based (\texttt{Loose/Medium/Tight}) \\
  Electron rejection & BDT-based\\
  Muon rejection & Via overlap removal in \(\Delta R < 0.2\) and \(\pT > \SI{2}{\GeV}\).
    Muons must not be Calo-tagged\\
  \bottomrule
  \end{tabular}
  % }
\end{table}

If the crack is excluded: \((|\eta| < 1.37) || (1.52 < |\eta| < 2.5)\)

The selection criteria are all implemented in the \texttt{TauSelectionTool} as part of the \texttt{TauAnalysisTools}.
Documentation can be found in the \href{https://gitlab.cern.ch/atlas/athena/blob/21.2/PhysicsAnalysis/TauID/TauAnalysisTools/doc/README-TauSelectionTool.rst}{README-TauSelectionTool.rst}.

\subsection{Small-\(R\) jet selection}

If you want to use variables such as \verb|\fcut| you need to add the option
\texttt{jetetmiss} to \texttt{atlaspackage}.

\begin{table}[ht]
  \caption{Jet reconstruction criteria.}%
  \label{tab:object:jet1}
  \centering
  % \resizebox{\textwidth}{!}{
  \begin{tabular}{ll}
  \toprule
  Feature & Criterion \\
  \midrule
  Algorithm & \Antikt  \\
  \(R\)-parameter & 0.4 \\
  Input constituent & EMTopo \\
  Analysis release number & 21.2.10 \\
  %Calibration tag & JetCalibTools-00-04-76 \\
  \texttt{CalibArea} tag & 00-04-81 \\
  Calibration configuration & \texttt{JES\_data2017\_2016\_2015\_Recommendation\_Feb2018\_rel21.config} \\
  Calibration sequence (Data) & \texttt{JetArea\_Residual\_EtaJES\_GSC\_Insitu} \\
  Calibration sequence (MC) & \texttt{JetArea\_Residual\_EtaJES\_GSC} \\
  %Calibration sequence (AFII) & \texttt{JetArea\_Residual\_EtaJES\_GSC} \\
  \midrule
  \multicolumn{2}{c}{Selection requirements} \\
  \midrule
  Observable & Requirement \\
  \midrule
  Jet cleaning & \texttt{LooseBad} \\
  BatMan cleaning & No \\
  \pT & \(> \SI[parse-numbers=false]{XX}{\GeV}\) \\
  \(|\eta|\) & \(< X\) \\
  JVT & (\emph{Update if needed}) \(>0.59\) for \(\pT < \SI{60}{\GeV}\), \(|\eta| < 0.4\)\\
  \bottomrule
  \end{tabular}
  % }
\end{table}


\clearpage
\subsection{Large-\(R\) jet selection}

\begin{table}[ht]
  \caption{Large-\(R\) jet reconstruction criteria.}%
  \label{tab:object:jet2}
  \centering
  % \resizebox{\textwidth}{!}{
  \begin{tabular}{ll}
    \toprule
    Feature & Criterion \\ 
    \midrule
    Algorithm & \antikt  \\
    R-parameter & 1.0 \\
    Input constituent & \texttt{LCTopo} \\
    Grooming algorithm & Trimming \\ 
    \fcut & 0.05 \\
    \(R_{\text{trim}}\) & 0.2 \\
    Analysis release number & 21.2.10 \\
    %Calibration tag & JetCalibTools-00-04-76 \\
    \texttt{CalibArea} tag & 00-04-81 \\
    Calibration configuration & \texttt{JES\_MC16recommendation\_FatJet\_JMS\_comb\_19Jan2018.config} \\
    Calibration sequence (Data) & \texttt{EtaJES\_JMS\_Insitu} \\
    Calibration sequence (MC) & \texttt{EtaJES\_JMS} \\
    \bottomrule
    \multicolumn{2}{c}{Selection requirements} \\
    \midrule
    Observable & Requirement \\
    \midrule
    \pT  & \(> \SI[parse-numbers=false]{XX}{\GeV}\) \\
    \(|\eta|\) & \(< X\) \\
    Mass & \(> \SI[parse-numbers=false]{XX}{\GeV}\) \\
    \bottomrule
    \multicolumn{2}{c}{Boosted object tagger} \\
    \midrule
    Object  & Working point \\
    \midrule
    \(W\) / \(Z\) / top & 50\% / 80\% \\
    \(X\rightarrow bb\) & single/double \btag with/without loose/tight mass \\
    \bottomrule
  \end{tabular}
  % }
\end{table}


\subsection{\MET selection}

\begin{table}[ht]
  \caption{\MET reconstruction criteria.}%
  \label{tab:object:met}
  \centering
  \begin{tabular}{ll}
    \toprule
    Parameter & Value \\ 
    \midrule
    Algorithm & Calo-based \\
    Soft term & Track-based (TST) \\ 
    MET operating point & \texttt{Tight} \\
    Analysis release & 21.2.16 \\
    Calibration tag & \texttt{METUtilities-00-02-46} \\
    \bottomrule
    \multicolumn{2}{c}{Selection requirements} \\
    \midrule
    Observable & Requirement \\
    \midrule
    \MET & \(> \SI[parse-numbers=false]{XX}{\GeV}\) \\
    \(\sum{\ET} / \MET\)  & \(< X\) \\
    Object-based \MET significance & \(> X\) \\
    \bottomrule
  \end{tabular}
\end{table}



\subsection{Jet flavor tagging selection}

\begin{table}[ht]
  \caption{\btag selection criteria.}%
  \label{tab:object:btag}
  \centering
  % \resizebox{\textwidth}{!}{
  \begin{tabular}{ll}
    \toprule
    Feature & Criterion \\ 
    \midrule
    & EM Topo Jets / Track jets / VR jets \\
    \midrule
    Jet collection  & \texttt{AntiKt4EMTopo/AntiKt2PV0/AntiKtVR30Rmax4Rmin02} \\
    Jet selection   & \(\pT > \SI[parse-numbers=false]{XX}{\GeV}\) \\
    	            	& \(|\eta| < X\) \\				
                    & JVT cut if applicable \\
    \midrule
    Algorithm 		  & \texttt{MV2c10/MV2c10mu/MV2c10rnn/DL1/DL1mu/DL1rnn} \\
    \midrule
    Operating point & Hybrid /  Fixed \\
                    & Eff = 60 / 70 / 77 / 85 \\
    CDI             & \texttt{2017-21-13TeV-MC16-CDI-2017-12-22\_v1} \\
  \bottomrule
  \end{tabular}
  % }
\end{table}%

\subsection{Track selection}

If you use tracks as particular objects on which you cut in your analysis.

\begin{table}[ht]
  \caption{\texttt{TrackParticle} object selection criteria.}%
  \label{tab:object:track}
  \centering
  % \resizebox{\textwidth}{!}{
  \begin{tabular}{ll}
    \toprule
    Tracking algorithm								    & Primary / Large Radius Tracking / Custom \\
    Track quality selection (official)    & \texttt{Loose/Tight} \\
    \pT                                   & \(> \SI[parse-numbers=false]{XX}{\GeV}\) \\
    \(|\eta|\)                            & \(< X\) \\
    Track-vertex association criteria     & \texttt{Loose/Tight} \\
    Track-to-tet association method       & Ghost Matched / \(\Delta R\) \\
    \bottomrule
  \end{tabular}
  % }
\end{table}

\subsection{Overlap removal}

The reconstruction of the same energy deposits as multiple objects is resolved using the standard overlap removal tools, \texttt{AssociationUtils}, documented \href{https://gitlab.cern.ch/atlas/athena/blob/21.2/PhysicsAnalysis/AnalysisCommon/AssociationUtils/README.rst}{here}

The (Standard/Heavy-flavor/Boosted/Boosted+Heavy-flavor/lepton-favored) working point is used corresponding to:

\begin{table}[ht]
  % \resizebox{\textwidth}{!}{
  \begin{tabular}{lll}
    \toprule
    Reject & Against & Criteria \\
    \midrule
    Electron & Electron & shared track, \(\pTX[][1] < \pTX[][2]\) \\
    Tau      & Electron & \(\Delta R <\) 0.2 \\
    Tau      & Muon     & \(\Delta R <\) 0.2 \\
    Muon     & Electron & is Calo-Muon and shared ID track \\
    Electron & Muon     & shared ID track \\
    Photon   & Electron & \(\Delta R < 0.4\) \\
    Photon   & Muon     & \(\Delta R < 0.4\) \\
    Jet      & Electron & [\(\Delta R < 0.2\) / Not a \bjet and \(\Delta R <\) 0.2] \\
    Electron & Jet      & [\(\Delta R < 0.4\) / \(\Delta R < \min(0.4, 0.04 + \SI{10}{\GeV}/\pT(e))\)/None] \\
    Jet      & Muon     & [\(\texttt{NumTrack} < 3\) and (ghost-associated or \(\Delta R < 0.2\)) / \\
                       && not a \bjet and \(\texttt{NumTrack} < 3\) and (ghost-associated or \(\Delta R < 0.2\))] \\
    Muon     & Jet      & [\(\Delta R < 0.4\) / \(\Delta R < \min(0.4, 0.04 + \SI{10}{\GeV}/\pT(\mu))\)/None] \\
    Jet      & Tau      & \(\Delta R < 0.2\) \\
    Photon   & Jet      & \(\Delta R < 0.4\) \\
    Fat-jet  & Electron & \(\Delta R < 1.0\) \\
    Jet      & Fat-jet  & \(\Delta R < 1.0\) \\
    \bottomrule
  \end{tabular}
  % }
\end{table}

\(\Delta R\) is calculated using rapidity by default.





\section{Event selection}
The following items should also be filled in for the event selection.

\subsection{Event cleaning}
Following the \href{https://twiki.cern.ch/twiki/bin/viewauth/Atlas/DataPreparationCheckListForPhysicsAnalysis}{recommendations of the DataPrep group}, the following event-level requirements are made.

We use the official GRL\@:
 \begin{verbatim} FILL IN HERE \end{verbatim}
 
The following event-level vetos are made to reject bad / corrupt events:
 \begin{itemize}
  \item LAr noise burst and data corruption (\verb|xAOD::EventInfo::LAr|),
  \item Tile corrupted events (\verb|xAOD::EventInfo::Tile|),
  \item events affected by the SCT recovery procedure for single event upsets (\verb|xAOD::EventInfo::SCT|),
  \item incomplete events (\verb|xAOD::EventInfo::Core|).
 \end{itemize}
 
 Debug stream events [have/have not] been included.
 
 Checks [have/have not] been done to remove duplicate events.
 
Events are required to have a primary vertex with at least two associated tracks.
The primary vertex is selected as the one with the largest \(\Sigma \pT^2\),
where the sum is over all tracks with transverse momentum \(\pT > \SI{0.4}{\GeV}\) that are associated with the vertex.
 
 

\end{document}
