%-------------------------------------------------------------------------------
% This file provides a skeleton ATLAS paper.
%-------------------------------------------------------------------------------
% \pdfoutput=1
% The \pdfoutput command is needed by arXiv/JHEP/JINST to ensure use of pdflatex.
% It should be included in the first 5 lines of the file.
\pdfinclusioncopyfonts=1
% This command may be needed in order to get \ell in PDF plots to appear. Found in
% https://tex.stackexchange.com/questions/322010/pdflatex-glyph-undefined-symbols-disappear-from-included-pdf
%-------------------------------------------------------------------------------
% Specify where ATLAS LaTeX style files can be found.
\RequirePackage{latex/atlaslatexpath}
% Comment out the above line if the files are in a central location, e.g. $HOME/texmf.
%-------------------------------------------------------------------------------
\documentclass[PAPER, atlasdraft=true, texlive=2020, UKenglish]{atlasdoc}
% The language of the document must be set: usually UKenglish or USenglish.
% british and american also work!
% Commonly used options:
%  atlasdraft=true|false This document is an ATLAS draft.
%  texlive=YYYY          Specify TeX Live version (2020 is default).
%  coverpage             Create ATLAS draft cover page for collaboration circulation.
%                        See atlas-draft-cover.tex for a list of variables that should be defined.
%  cernpreprint          Create front page for a CERN preprint.
%                        See atlas-preprint-cover.tex for a list of variables that should be defined.
%  NOTE                  The document is an ATLAS note (draft).
%  PAPER                 The document is an ATLAS paper (draft).
%  CONF                  The document is a CONF note (draft).
%  PUB                   The document is a PUB note (draft).
%  BOOK                  The document is of book form, like an LOI or TDR (draft).
%  txfonts=true|false    Use txfonts rather than the default newtx.
%  paper=a4|letter       Set paper size to A4 (default) or letter.

%-------------------------------------------------------------------------------
% Extra packages:
\usepackage{atlaspackage}
% Commonly used options:
%  backend=bibtex        Use the bibtex backend rather than biber.
%  subfigure|subfig|subcaption  to use one of these packages for figures in figures.
%  minimal               Minimal set of packages.
%  default               Standard set of packages.
%  full                  Full set of packages.
%-------------------------------------------------------------------------------
% Style file with biblatex options for ATLAS documents.
\usepackage{atlasbiblatex}

% Useful macros
\usepackage{atlasphysics}
% See doc/atlas_physics.pdf for a list of the defined symbols.
% Default options are:
%   true:  journal, misc, particle, unit, xref
%   false: BSM, hepparticle, hepprocess, hion, jetetmiss, math, process,
%          other, snippets, texmf
% See the package for details on the options.

% Macro to add to-do notes (for several authors). Uses the todonotes package.
% \ATLnote{JS}{Jane}{green!20}{green!50!black!60}
% add macros \JSnote and \JSinote for notes in the margin and inline.
% The first colour is for the body and the second for the border of the note.
% Set output=false in order not to print out the notes.
% Set shift=false to avoid adjustment of margins.
% This package should not be used in the final version of the paper.
\ifthenelse{\boolean{AtlasDraft}}{\usepackage[output=true, shift=true]{atlastodo}}{}

% Files with references for use with biblatex.
% Note that biber gives an error if it finds empty bib files.
% \addbibresource{atlas-document.bib}
\addbibresource{bib/ATLAS.bib}
\addbibresource{bib/CMS.bib}
\addbibresource{bib/ConfNotes.bib}
\addbibresource{bib/PubNotes.bib}

% Paths for figures - do not forget the / at the end of the directory name.
\graphicspath{{logos/}{figures/}}

% Add your own definitions here (file atlas-document-defs.sty).
\usepackage{atlas-document-defs}

%-------------------------------------------------------------------------------
% Generic document information.
%-------------------------------------------------------------------------------

% Title, abstract and document.
%-------------------------------------------------------------------------------
% This file contains the title, author and abstract.
% It also contains all relevant document numbers used by the different cover pages.
%-------------------------------------------------------------------------------

%% Title
\AtlasTitle{A simple ATLAS paper draft}
%
% Author - this does not work with revtex
% \author{The ATLAS Collaboration}
%
% Date: if not given, uses current date
% \date{\today}

% Draft version:
% Should be 1.0 for the first circulation, and 2.0 for the second circulation.
% If given, adds draft version on front page, a 'DRAFT' box on top of each other page, 
% and line numbers.
% Comment or remove in final version.
\AtlasVersion{1.0}

% ATLAS reference code, to help ATLAS members to locate the paper
\AtlasRefCode{GROUP-2014-XX}

% ATLAS note number. Can be an COM, INT, PUB or CONF note
% \AtlasNote{ATLAS-CONF-2014-XXX}
% \AtlasNote{ATLAS-PUB-2014-XXX}
\AtlasNote{ATLAS-COM-PHYS-2014-XXX}

% Submission journal
\AtlasJournal{Phys.\ Lett.\ B.}

% Abstract - % directly after { is important for correct indentation
\AtlasAbstract{%
	This is a bare bones ATLAS document used to demonstrate how to use journal templates.

	This document is typeset using the \DocumentStyle.
}

%-------------------------------------------------------------------------------
% The following information is needed for the cover page. The commands are only defined
% if you use the coverpage option in atlasdoc or use the atlascover package
%-------------------------------------------------------------------------------

% Explicit list of supporting notes  (leave as null \AtlasCoverSupportingNote{} if you want to skip this option)
\AtlasCoverSupportingNote{A supporting note}{http://cds.cern.ch/record/XXXXXX}
\AtlasCoverSupportingNote{Another supporting note}{http://cds.cern.ch/record/YYYYYY}
%
% OR (the 2nd option is deprecated, especially for CONF and PUB notes)
%
% Supporting material TWiki page  (leave as null \AtlasCoverTwikiURL{} if you want to skip this option)
% \AtlasCoverTwikiURL{https://twiki.cern.ch/twiki/bin/view/Atlas/WebHome}

% Comment deadline
\AtlasCoverCommentsDeadline{DD Month 2014}

% Analysis team members - contact editors should no longer be specified
% as there is a generic email list name for the editors
\AtlasCoverAnalysisTeam{Peter Analyser, Susan Editor1, Jenny Editor2, Alphonse Physicien}

% Editorial Board Members - indicate the Chair by a (chair) after his/her name
\AtlasCoverEdBoardMember{EdBoard~Chair~(chair), EB~Member~1, EB~Member~2, EB~Member~3}

% Editors egroup
\AtlasCoverEgroupEditors{atlas-GROUP-2014-XX-editors@cern.ch}

% EdBoard egroup
\AtlasCoverEgroupEdBoard{atlas-GROUP-2014-XX-editorial-board@cern.ch}

% CERN preprintnumber
\PreprintIdNumber{CERN-EP-2014-XX}

% Author and title for the PDF file.
\hypersetup{pdftitle={ATLAS document},pdfauthor={The ATLAS Collaboration}}

%-------------------------------------------------------------------------------
% Content
%-------------------------------------------------------------------------------
\begin{document}

\maketitle

\tableofcontents

% List of to-do notes.
% \listoftodos

%-------------------------------------------------------------------------------
\section{Introduction}
\label{sec:intro}
%-------------------------------------------------------------------------------

Place your introduction here.

%-------------------------------------------------------------------------------
\section{ATLAS detector}
\label{sec:detector}
%-------------------------------------------------------------------------------

The ATLAS detector~\cite{PERF-2007-01} ...
% % Footnote with ATLAS coordinate system
\newcommand{\AtlasCoordFootnote}{%
ATLAS uses a right-handed coordinate system with its origin at the nominal interaction point (IP)
in the centre of the detector and the $z$-axis along the beam pipe.
The $x$-axis points from the IP to the centre of the LHC ring,
and the $y$-axis points upwards.
Cylindrical coordinates $(r,\phi)$ are used in the transverse plane, 
$\phi$ being the azimuthal angle around the $z$-axis.
The pseudorapidity is defined in terms of the polar angle $\theta$ as $\eta = -\ln \tan(\theta/2)$.
Angular distance is measured in units of $\Delta R \equiv \sqrt{(\Delta\eta)^{2} + (\Delta\phi)^{2}}$.}

%-------------------------------------------------------------------------------
\subsection{Run 1 ATLAS detector example for a Letter}
\label{sec:atlas1}
%-------------------------------------------------------------------------------

The ATLAS experiment~\cite{PERF-2007-01} at the LHC is a multi-purpose particle detector
with a forward-backward symmetric cylindrical geometry and a near $4\pi$ coverage in 
solid angle.\footnote{\AtlasCoordFootnote}
It consists of an inner tracking detector surrounded by a thin superconducting solenoid
providing a \SI{2}{\tesla} axial magnetic field, electromagnetic and hadron calorimeters, and a muon spectrometer.
The inner tracking detector covers the pseudorapidity range $|\eta| < 2.5$.
It consists of silicon pixel, silicon micro-strip, and transition radiation tracking detectors.
Lead/liquid-argon (LAr) sampling calorimeters provide electromagnetic (EM) energy measurements
with high granularity.
A hadron (steel/scintillator-tile) calorimeter covers the central pseudorapidity range ($|\eta| < 1.7$).
The end-cap and forward regions are instrumented with LAr calorimeters
for both EM and hadronic energy measurements up to $|\eta| = 4.9$.
The muon spectrometer surrounds the calorimeters and is based on
three large air-core toroidal superconducting magnets with eight coils each.
The field integral of the toroids ranges between \num{2.0} and \SI{6.0}{\tesla\metre}
across most of the detector. 
The muon spectrometer includes a system of precision tracking chambers and fast detectors for triggering.
A three-level trigger system is used to select events.
The first-level trigger is implemented in hardware and uses a subset of the detector information
to reduce the accepted rate to at most \SI{75}{\kilo\hertz}.
This is followed by two software-based trigger levels that
together reduce the accepted event rate to \SI{400}{\hertz} on average
depending on the data-taking conditions during 2012.


%-------------------------------------------------------------------------------
\subsection{Run 1 ATLAS detector example for a paper making use of the whole detector}
\label{sec:atlas2}
%-------------------------------------------------------------------------------

The ATLAS detector~\cite{PERF-2007-01} at the LHC covers nearly the entire solid angle around the collision point.
It consists of an inner tracking detector surrounded by a thin superconducting solenoid, electromagnetic and hadronic calorimeters,
and a muon spectrometer incorporating three large superconducting toroidal magnets.
The inner-detector system (ID) is immersed in a \SI{2}{\tesla} axial magnetic field 
and provides charged particle tracking in the range $|\eta| < 2.5$.

The high-granularity silicon pixel detector covers the vertex region and typically provides three measurements per track, 
the first hit being normally in the innermost layer.
It is followed by the silicon microstrip tracker which usually provides four two-dimensional measurement points per track.
These silicon detectors are complemented by the transition radiation tracker,
which enables radially extended track reconstruction up to $|\eta| = 2.0$. 
The transition radiation tracker also provides electron identification information 
based on the fraction of hits (typically 30 in total) above a higher energy deposit threshold corresponding to transition radiation.

The calorimeter system covers the pseudorapidity range $|\eta| < 4.9$.
Within the region $|\eta|< 3.2$, electromagnetic calorimetry is provided by barrel and 
endcap high-granularity lead/liquid-argon (LAr) electromagnetic calorimeters,
with an additional thin LAr presampler covering $|\eta| < 1.8$,
to correct for energy loss in material upstream of the calorimeters.
Hadronic calorimetry is provided by the steel/scintillating-tile calorimeter,
segmented into three barrel structures within $|\eta| < 1.7$, and two copper/LAr hadronic endcap calorimeters.
The solid angle coverage is completed with forward copper/LAr and tungsten/LAr calorimeter modules
optimised for electromagnetic and hadronic measurements respectively.

The muon spectrometer (MS) comprises separate trigger and
high-precision tracking chambers measuring the deflection of muons in a magnetic field generated by superconducting air-core toroids.
The field integral of the toroids ranges between \num{2.0} and \SI{6.0}{\tesla\metre}
across most of the detector. 
A set of precision chambers covers the region $|\eta| < 2.7$ with three layers of monitored drift tubes,
complemented by cathode strip chambers in the forward region, where the background is highest.
The muon trigger system covers the range $|\eta| < 2.4$ with resistive plate chambers in the barrel, and thin gap chambers in the endcap regions.
A three-level trigger system is used to select interesting events~\cite{PERF-2011-02}.
The Level-1 trigger is implemented in hardware and uses a subset of detector information
to reduce the event rate to a design value of at most \SI{75}{\kHz}.
This is followed by two software-based trigger levels which together reduce the event rate to about \SI{200}{\Hz}.



%-------------------------------------------------------------------------------
\section{Analysis}
\label{sec:analysis}
%-------------------------------------------------------------------------------

You can find some text snippets that can be used in papers in \texttt{latex/atlassnippets.sty}.
To use them, provide the \texttt{snippets} option to \texttt{atlasphysics}.

%-------------------------------------------------------------------------------
\section{Results}
\label{sec:result}
%-------------------------------------------------------------------------------

Place your results here.

% All figures and tables should appear before the summary and conclusion.
% The package placeins provides the macro \FloatBarrier to achieve this.
% \FloatBarrier

%-------------------------------------------------------------------------------
\section{Conclusion}
\label{sec:conclusion}
%-------------------------------------------------------------------------------

Place your conclusion here.

%-------------------------------------------------------------------------------
\section*{Acknowledgements}
%-------------------------------------------------------------------------------

% % Acknowledgements for papers with collision data
% Version 24-Oct-2018

% Standard acknowledgements start here
%----------------------------------------------
We thank CERN for the very successful operation of the LHC, as well as the
support staff from our institutions without whom ATLAS could not be
operated efficiently.

We acknowledge the support of ANPCyT, Argentina; YerPhI, Armenia; ARC, Australia; BMWFW and FWF, Austria; ANAS, Azerbaijan; SSTC, Belarus; CNPq and FAPESP, Brazil; NSERC, NRC and CFI, Canada; CERN; CONICYT, Chile; CAS, MOST and NSFC, China; COLCIENCIAS, Colombia; MSMT CR, MPO CR and VSC CR, Czech Republic; DNRF and DNSRC, Denmark; IN2P3-CNRS, CEA-DRF/IRFU, France; SRNSFG, Georgia; BMBF, HGF, and MPG, Germany; GSRT, Greece; RGC, Hong Kong SAR, China; ISF and Benoziyo Center, Israel; INFN, Italy; MEXT and JSPS, Japan; CNRST, Morocco; NWO, Netherlands; RCN, Norway; MNiSW and NCN, Poland; FCT, Portugal; MNE/IFA, Romania; MES of Russia and NRC KI, Russian Federation; JINR; MESTD, Serbia; MSSR, Slovakia; ARRS and MIZ\v{S}, Slovenia; DST/NRF, South Africa; MINECO, Spain; SRC and Wallenberg Foundation, Sweden; SERI, SNSF and Cantons of Bern and Geneva, Switzerland; MOST, Taiwan; TAEK, Turkey; STFC, United Kingdom; DOE and NSF, United States of America. In addition, individual groups and members have received support from BCKDF, CANARIE, CRC and Compute Canada, Canada; COST, ERC, ERDF, Horizon 2020, and Marie Sk{\l}odowska-Curie Actions, European Union; Investissements d' Avenir Labex and Idex, ANR, France; DFG and AvH Foundation, Germany; Herakleitos, Thales and Aristeia programmes co-financed by EU-ESF and the Greek NSRF, Greece; BSF-NSF and GIF, Israel; CERCA Programme Generalitat de Catalunya, Spain; The Royal Society and Leverhulme Trust, United Kingdom. 

The crucial computing support from all WLCG partners is acknowledged gratefully, in particular from CERN, the ATLAS Tier-1 facilities at TRIUMF (Canada), NDGF (Denmark, Norway, Sweden), CC-IN2P3 (France), KIT/GridKA (Germany), INFN-CNAF (Italy), NL-T1 (Netherlands), PIC (Spain), ASGC (Taiwan), RAL (UK) and BNL (USA), the Tier-2 facilities worldwide and large non-WLCG resource providers. Major contributors of computing resources are listed in Ref.~\cite{ATL-GEN-PUB-2016-002}.
%----------------------------------------------



The \texttt{atlaslatex} package contains the acknowledgements that were valid
at the time of the release you are using.
These can be found in the \texttt{acknowledgements} subdirectory.
When your ATLAS paper or PUB/CONF note is ready to be published,
the acknowledgements should be automatically updated by PO-GitLab.
The acknowledgements can be inspected on the web page:
\url{https://atlas-glance.cern.ch/atlas/membership/funding_agencies/history}.

%-------------------------------------------------------------------------------
\clearpage
\appendix
\part*{Appendix}
\addcontentsline{toc}{part}{Appendix}
%-------------------------------------------------------------------------------

In a paper, an appendix is used for technical details that would otherwise disturb the flow of the paper.
Such an appendix should be printed before the Bibliography.

%-------------------------------------------------------------------------------
% If you use biblatex and either biber or bibtex to process the bibliography
% just say \printbibliography here.
\printbibliography
% If you want to use the traditional BibTeX you need to use the syntax below.
% \bibliographystyle{obsolete/bst/atlasBibStyleWoTitle}
% \bibliography{atlas-document,bib/ATLAS,bib/CMS,bib/ConfNotes,bib/PubNotes}
%-------------------------------------------------------------------------------

%-------------------------------------------------------------------------------
% Author list - comment in these two lines when you are ready to include it.
% \clearpage
% % ATLAS Collaboration author list
% Reference date of TOPQ-2017-07 is 2018-08-20
% Author list last updated on date 29-OCT-18
% Data extracted on 29-Oct-2018 for paper reference TOPQ-2017-07
% at 11:37am
 
\begin{flushleft}
{\Large The ATLAS Collaboration}

\bigskip

M.~Aaboud$^\textrm{\scriptsize 34d}$,    
G.~Aad$^\textrm{\scriptsize 99}$,    
B.~Abbott$^\textrm{\scriptsize 125}$,    
D.C.~Abbott$^\textrm{\scriptsize 100}$,    
O.~Abdinov$^\textrm{\scriptsize 13,*}$,    
B.~Abeloos$^\textrm{\scriptsize 129}$,    
D.K.~Abhayasinghe$^\textrm{\scriptsize 91}$,    
S.H.~Abidi$^\textrm{\scriptsize 164}$,    
O.S.~AbouZeid$^\textrm{\scriptsize 39}$,    
N.L.~Abraham$^\textrm{\scriptsize 153}$,    
H.~Abramowicz$^\textrm{\scriptsize 158}$,    
H.~Abreu$^\textrm{\scriptsize 157}$,    
Y.~Abulaiti$^\textrm{\scriptsize 6}$,    
B.S.~Acharya$^\textrm{\scriptsize 64a,64b,p}$,    
S.~Adachi$^\textrm{\scriptsize 160}$,    
L.~Adam$^\textrm{\scriptsize 97}$,    
L.~Adamczyk$^\textrm{\scriptsize 81a}$,    
L.~Adamek$^\textrm{\scriptsize 164}$,    
J.~Adelman$^\textrm{\scriptsize 119}$,    
M.~Adersberger$^\textrm{\scriptsize 112}$,    
A.~Adiguzel$^\textrm{\scriptsize 12c,ai}$,    
T.~Adye$^\textrm{\scriptsize 141}$,    
A.A.~Affolder$^\textrm{\scriptsize 143}$,    
Y.~Afik$^\textrm{\scriptsize 157}$,    
C.~Agheorghiesei$^\textrm{\scriptsize 27c}$,    
J.A.~Aguilar-Saavedra$^\textrm{\scriptsize 137f,137a,ah}$,    
F.~Ahmadov$^\textrm{\scriptsize 77,af}$,    
G.~Aielli$^\textrm{\scriptsize 71a,71b}$,    
S.~Akatsuka$^\textrm{\scriptsize 83}$,    
T.P.A.~{\AA}kesson$^\textrm{\scriptsize 94}$,    
E.~Akilli$^\textrm{\scriptsize 52}$,    
A.V.~Akimov$^\textrm{\scriptsize 108}$,    
G.L.~Alberghi$^\textrm{\scriptsize 23b,23a}$,    
J.~Albert$^\textrm{\scriptsize 173}$,    
P.~Albicocco$^\textrm{\scriptsize 49}$,    
M.J.~Alconada~Verzini$^\textrm{\scriptsize 86}$,    
S.~Alderweireldt$^\textrm{\scriptsize 117}$,    
M.~Aleksa$^\textrm{\scriptsize 35}$,    
I.N.~Aleksandrov$^\textrm{\scriptsize 77}$,    
C.~Alexa$^\textrm{\scriptsize 27b}$,    
D.~Alexandre$^\textrm{\scriptsize 19}$,    
T.~Alexopoulos$^\textrm{\scriptsize 10}$,    
M.~Alhroob$^\textrm{\scriptsize 125}$,    
B.~Ali$^\textrm{\scriptsize 139}$,    
G.~Alimonti$^\textrm{\scriptsize 66a}$,    
J.~Alison$^\textrm{\scriptsize 36}$,    
S.P.~Alkire$^\textrm{\scriptsize 145}$,    
C.~Allaire$^\textrm{\scriptsize 129}$,    
B.M.M.~Allbrooke$^\textrm{\scriptsize 153}$,    
B.W.~Allen$^\textrm{\scriptsize 128}$,    
P.P.~Allport$^\textrm{\scriptsize 21}$,    
A.~Aloisio$^\textrm{\scriptsize 67a,67b}$,    
A.~Alonso$^\textrm{\scriptsize 39}$,    
F.~Alonso$^\textrm{\scriptsize 86}$,    
C.~Alpigiani$^\textrm{\scriptsize 145}$,    
A.A.~Alshehri$^\textrm{\scriptsize 55}$,    
M.I.~Alstaty$^\textrm{\scriptsize 99}$,    
B.~Alvarez~Gonzalez$^\textrm{\scriptsize 35}$,    
D.~\'{A}lvarez~Piqueras$^\textrm{\scriptsize 171}$,    
M.G.~Alviggi$^\textrm{\scriptsize 67a,67b}$,    
B.T.~Amadio$^\textrm{\scriptsize 18}$,    
Y.~Amaral~Coutinho$^\textrm{\scriptsize 78b}$,    
A.~Ambler$^\textrm{\scriptsize 101}$,    
L.~Ambroz$^\textrm{\scriptsize 132}$,    
C.~Amelung$^\textrm{\scriptsize 26}$,    
D.~Amidei$^\textrm{\scriptsize 103}$,    
S.P.~Amor~Dos~Santos$^\textrm{\scriptsize 137a,137c}$,    
S.~Amoroso$^\textrm{\scriptsize 44}$,    
C.S.~Amrouche$^\textrm{\scriptsize 52}$,    
F.~An$^\textrm{\scriptsize 76}$,    
C.~Anastopoulos$^\textrm{\scriptsize 146}$,    
L.S.~Ancu$^\textrm{\scriptsize 52}$,    
N.~Andari$^\textrm{\scriptsize 142}$,    
T.~Andeen$^\textrm{\scriptsize 11}$,    
C.F.~Anders$^\textrm{\scriptsize 59b}$,    
J.K.~Anders$^\textrm{\scriptsize 20}$,    
K.J.~Anderson$^\textrm{\scriptsize 36}$,    
A.~Andreazza$^\textrm{\scriptsize 66a,66b}$,    
V.~Andrei$^\textrm{\scriptsize 59a}$,    
C.R.~Anelli$^\textrm{\scriptsize 173}$,    
S.~Angelidakis$^\textrm{\scriptsize 37}$,    
I.~Angelozzi$^\textrm{\scriptsize 118}$,    
A.~Angerami$^\textrm{\scriptsize 38}$,    
A.V.~Anisenkov$^\textrm{\scriptsize 120b,120a}$,    
A.~Annovi$^\textrm{\scriptsize 69a}$,    
C.~Antel$^\textrm{\scriptsize 59a}$,    
M.T.~Anthony$^\textrm{\scriptsize 146}$,    
M.~Antonelli$^\textrm{\scriptsize 49}$,    
D.J.A.~Antrim$^\textrm{\scriptsize 168}$,    
F.~Anulli$^\textrm{\scriptsize 70a}$,    
M.~Aoki$^\textrm{\scriptsize 79}$,    
J.A.~Aparisi~Pozo$^\textrm{\scriptsize 171}$,    
L.~Aperio~Bella$^\textrm{\scriptsize 35}$,    
G.~Arabidze$^\textrm{\scriptsize 104}$,    
J.P.~Araque$^\textrm{\scriptsize 137a}$,    
V.~Araujo~Ferraz$^\textrm{\scriptsize 78b}$,    
R.~Araujo~Pereira$^\textrm{\scriptsize 78b}$,    
A.T.H.~Arce$^\textrm{\scriptsize 47}$,    
R.E.~Ardell$^\textrm{\scriptsize 91}$,    
F.A.~Arduh$^\textrm{\scriptsize 86}$,    
J-F.~Arguin$^\textrm{\scriptsize 107}$,    
S.~Argyropoulos$^\textrm{\scriptsize 75}$,    
J.-H.~Arling$^\textrm{\scriptsize 44}$,    
A.J.~Armbruster$^\textrm{\scriptsize 35}$,    
L.J.~Armitage$^\textrm{\scriptsize 90}$,    
A.~Armstrong$^\textrm{\scriptsize 168}$,    
O.~Arnaez$^\textrm{\scriptsize 164}$,    
H.~Arnold$^\textrm{\scriptsize 118}$,    
M.~Arratia$^\textrm{\scriptsize 31}$,    
O.~Arslan$^\textrm{\scriptsize 24}$,    
A.~Artamonov$^\textrm{\scriptsize 109,*}$,    
G.~Artoni$^\textrm{\scriptsize 132}$,    
S.~Artz$^\textrm{\scriptsize 97}$,    
S.~Asai$^\textrm{\scriptsize 160}$,    
N.~Asbah$^\textrm{\scriptsize 57}$,    
E.M.~Asimakopoulou$^\textrm{\scriptsize 169}$,    
L.~Asquith$^\textrm{\scriptsize 153}$,    
K.~Assamagan$^\textrm{\scriptsize 29}$,    
R.~Astalos$^\textrm{\scriptsize 28a}$,    
R.J.~Atkin$^\textrm{\scriptsize 32a}$,    
M.~Atkinson$^\textrm{\scriptsize 170}$,    
N.B.~Atlay$^\textrm{\scriptsize 148}$,    
K.~Augsten$^\textrm{\scriptsize 139}$,    
G.~Avolio$^\textrm{\scriptsize 35}$,    
R.~Avramidou$^\textrm{\scriptsize 58a}$,    
M.K.~Ayoub$^\textrm{\scriptsize 15a}$,    
A.M.~Azoulay$^\textrm{\scriptsize 165b}$,    
G.~Azuelos$^\textrm{\scriptsize 107,av}$,    
A.E.~Baas$^\textrm{\scriptsize 59a}$,    
M.J.~Baca$^\textrm{\scriptsize 21}$,    
H.~Bachacou$^\textrm{\scriptsize 142}$,    
K.~Bachas$^\textrm{\scriptsize 65a,65b}$,    
M.~Backes$^\textrm{\scriptsize 132}$,    
P.~Bagnaia$^\textrm{\scriptsize 70a,70b}$,    
M.~Bahmani$^\textrm{\scriptsize 82}$,    
H.~Bahrasemani$^\textrm{\scriptsize 149}$,    
A.J.~Bailey$^\textrm{\scriptsize 171}$,    
V.R.~Bailey$^\textrm{\scriptsize 170}$,    
J.T.~Baines$^\textrm{\scriptsize 141}$,    
M.~Bajic$^\textrm{\scriptsize 39}$,    
C.~Bakalis$^\textrm{\scriptsize 10}$,    
O.K.~Baker$^\textrm{\scriptsize 180}$,    
P.J.~Bakker$^\textrm{\scriptsize 118}$,    
D.~Bakshi~Gupta$^\textrm{\scriptsize 8}$,    
S.~Balaji$^\textrm{\scriptsize 154}$,    
E.M.~Baldin$^\textrm{\scriptsize 120b,120a}$,    
P.~Balek$^\textrm{\scriptsize 177}$,    
F.~Balli$^\textrm{\scriptsize 142}$,    
W.K.~Balunas$^\textrm{\scriptsize 134}$,    
J.~Balz$^\textrm{\scriptsize 97}$,    
E.~Banas$^\textrm{\scriptsize 82}$,    
A.~Bandyopadhyay$^\textrm{\scriptsize 24}$,    
S.~Banerjee$^\textrm{\scriptsize 178,l}$,    
A.A.E.~Bannoura$^\textrm{\scriptsize 179}$,    
L.~Barak$^\textrm{\scriptsize 158}$,    
W.M.~Barbe$^\textrm{\scriptsize 37}$,    
E.L.~Barberio$^\textrm{\scriptsize 102}$,    
D.~Barberis$^\textrm{\scriptsize 53b,53a}$,    
M.~Barbero$^\textrm{\scriptsize 99}$,    
T.~Barillari$^\textrm{\scriptsize 113}$,    
M-S.~Barisits$^\textrm{\scriptsize 35}$,    
J.~Barkeloo$^\textrm{\scriptsize 128}$,    
T.~Barklow$^\textrm{\scriptsize 150}$,    
R.~Barnea$^\textrm{\scriptsize 157}$,    
S.L.~Barnes$^\textrm{\scriptsize 58c}$,    
B.M.~Barnett$^\textrm{\scriptsize 141}$,    
R.M.~Barnett$^\textrm{\scriptsize 18}$,    
Z.~Barnovska-Blenessy$^\textrm{\scriptsize 58a}$,    
A.~Baroncelli$^\textrm{\scriptsize 72a}$,    
G.~Barone$^\textrm{\scriptsize 29}$,    
A.J.~Barr$^\textrm{\scriptsize 132}$,    
L.~Barranco~Navarro$^\textrm{\scriptsize 171}$,    
F.~Barreiro$^\textrm{\scriptsize 96}$,    
J.~Barreiro~Guimar\~{a}es~da~Costa$^\textrm{\scriptsize 15a}$,    
R.~Bartoldus$^\textrm{\scriptsize 150}$,    
A.E.~Barton$^\textrm{\scriptsize 87}$,    
P.~Bartos$^\textrm{\scriptsize 28a}$,    
A.~Basalaev$^\textrm{\scriptsize 135}$,    
A.~Bassalat$^\textrm{\scriptsize 129}$,    
R.L.~Bates$^\textrm{\scriptsize 55}$,    
S.J.~Batista$^\textrm{\scriptsize 164}$,    
S.~Batlamous$^\textrm{\scriptsize 34e}$,    
J.R.~Batley$^\textrm{\scriptsize 31}$,    
M.~Battaglia$^\textrm{\scriptsize 143}$,    
M.~Bauce$^\textrm{\scriptsize 70a,70b}$,    
F.~Bauer$^\textrm{\scriptsize 142}$,    
K.T.~Bauer$^\textrm{\scriptsize 168}$,    
H.S.~Bawa$^\textrm{\scriptsize 150}$,    
J.B.~Beacham$^\textrm{\scriptsize 123}$,    
T.~Beau$^\textrm{\scriptsize 133}$,    
P.H.~Beauchemin$^\textrm{\scriptsize 167}$,    
P.~Bechtle$^\textrm{\scriptsize 24}$,    
H.C.~Beck$^\textrm{\scriptsize 51}$,    
H.P.~Beck$^\textrm{\scriptsize 20,s}$,    
K.~Becker$^\textrm{\scriptsize 50}$,    
M.~Becker$^\textrm{\scriptsize 97}$,    
C.~Becot$^\textrm{\scriptsize 44}$,    
A.~Beddall$^\textrm{\scriptsize 12d}$,    
A.J.~Beddall$^\textrm{\scriptsize 12a}$,    
V.A.~Bednyakov$^\textrm{\scriptsize 77}$,    
M.~Bedognetti$^\textrm{\scriptsize 118}$,    
C.P.~Bee$^\textrm{\scriptsize 152}$,    
T.A.~Beermann$^\textrm{\scriptsize 74}$,    
M.~Begalli$^\textrm{\scriptsize 78b}$,    
M.~Begel$^\textrm{\scriptsize 29}$,    
A.~Behera$^\textrm{\scriptsize 152}$,    
J.K.~Behr$^\textrm{\scriptsize 44}$,    
F.~Beisiegel$^\textrm{\scriptsize 24}$,    
A.S.~Bell$^\textrm{\scriptsize 92}$,    
G.~Bella$^\textrm{\scriptsize 158}$,    
L.~Bellagamba$^\textrm{\scriptsize 23b}$,    
A.~Bellerive$^\textrm{\scriptsize 33}$,    
M.~Bellomo$^\textrm{\scriptsize 157}$,    
P.~Bellos$^\textrm{\scriptsize 9}$,    
K.~Belotskiy$^\textrm{\scriptsize 110}$,    
N.L.~Belyaev$^\textrm{\scriptsize 110}$,    
O.~Benary$^\textrm{\scriptsize 158,*}$,    
D.~Benchekroun$^\textrm{\scriptsize 34a}$,    
M.~Bender$^\textrm{\scriptsize 112}$,    
N.~Benekos$^\textrm{\scriptsize 10}$,    
Y.~Benhammou$^\textrm{\scriptsize 158}$,    
E.~Benhar~Noccioli$^\textrm{\scriptsize 180}$,    
J.~Benitez$^\textrm{\scriptsize 75}$,    
D.P.~Benjamin$^\textrm{\scriptsize 6}$,    
M.~Benoit$^\textrm{\scriptsize 52}$,    
J.R.~Bensinger$^\textrm{\scriptsize 26}$,    
S.~Bentvelsen$^\textrm{\scriptsize 118}$,    
L.~Beresford$^\textrm{\scriptsize 132}$,    
M.~Beretta$^\textrm{\scriptsize 49}$,    
D.~Berge$^\textrm{\scriptsize 44}$,    
E.~Bergeaas~Kuutmann$^\textrm{\scriptsize 169}$,    
N.~Berger$^\textrm{\scriptsize 5}$,    
B.~Bergmann$^\textrm{\scriptsize 139}$,    
L.J.~Bergsten$^\textrm{\scriptsize 26}$,    
J.~Beringer$^\textrm{\scriptsize 18}$,    
S.~Berlendis$^\textrm{\scriptsize 7}$,    
N.R.~Bernard$^\textrm{\scriptsize 100}$,    
G.~Bernardi$^\textrm{\scriptsize 133}$,    
C.~Bernius$^\textrm{\scriptsize 150}$,    
F.U.~Bernlochner$^\textrm{\scriptsize 24}$,    
T.~Berry$^\textrm{\scriptsize 91}$,    
P.~Berta$^\textrm{\scriptsize 97}$,    
C.~Bertella$^\textrm{\scriptsize 15a}$,    
G.~Bertoli$^\textrm{\scriptsize 43a,43b}$,    
I.A.~Bertram$^\textrm{\scriptsize 87}$,    
G.J.~Besjes$^\textrm{\scriptsize 39}$,    
O.~Bessidskaia~Bylund$^\textrm{\scriptsize 179}$,    
M.~Bessner$^\textrm{\scriptsize 44}$,    
N.~Besson$^\textrm{\scriptsize 142}$,    
A.~Bethani$^\textrm{\scriptsize 98}$,    
S.~Bethke$^\textrm{\scriptsize 113}$,    
A.~Betti$^\textrm{\scriptsize 24}$,    
A.J.~Bevan$^\textrm{\scriptsize 90}$,    
J.~Beyer$^\textrm{\scriptsize 113}$,    
R.~Bi$^\textrm{\scriptsize 136}$,    
R.M.~Bianchi$^\textrm{\scriptsize 136}$,    
O.~Biebel$^\textrm{\scriptsize 112}$,    
D.~Biedermann$^\textrm{\scriptsize 19}$,    
R.~Bielski$^\textrm{\scriptsize 35}$,    
K.~Bierwagen$^\textrm{\scriptsize 97}$,    
N.V.~Biesuz$^\textrm{\scriptsize 69a,69b}$,    
M.~Biglietti$^\textrm{\scriptsize 72a}$,    
T.R.V.~Billoud$^\textrm{\scriptsize 107}$,    
M.~Bindi$^\textrm{\scriptsize 51}$,    
A.~Bingul$^\textrm{\scriptsize 12d}$,    
C.~Bini$^\textrm{\scriptsize 70a,70b}$,    
S.~Biondi$^\textrm{\scriptsize 23b,23a}$,    
M.~Birman$^\textrm{\scriptsize 177}$,    
T.~Bisanz$^\textrm{\scriptsize 51}$,    
J.P.~Biswal$^\textrm{\scriptsize 158}$,    
C.~Bittrich$^\textrm{\scriptsize 46}$,    
D.M.~Bjergaard$^\textrm{\scriptsize 47}$,    
J.E.~Black$^\textrm{\scriptsize 150}$,    
K.M.~Black$^\textrm{\scriptsize 25}$,    
T.~Blazek$^\textrm{\scriptsize 28a}$,    
I.~Bloch$^\textrm{\scriptsize 44}$,    
C.~Blocker$^\textrm{\scriptsize 26}$,    
A.~Blue$^\textrm{\scriptsize 55}$,    
U.~Blumenschein$^\textrm{\scriptsize 90}$,    
Dr.~Blunier$^\textrm{\scriptsize 144a}$,    
G.J.~Bobbink$^\textrm{\scriptsize 118}$,    
V.S.~Bobrovnikov$^\textrm{\scriptsize 120b,120a}$,    
S.S.~Bocchetta$^\textrm{\scriptsize 94}$,    
A.~Bocci$^\textrm{\scriptsize 47}$,    
D.~Boerner$^\textrm{\scriptsize 179}$,    
D.~Bogavac$^\textrm{\scriptsize 112}$,    
A.G.~Bogdanchikov$^\textrm{\scriptsize 120b,120a}$,    
C.~Bohm$^\textrm{\scriptsize 43a}$,    
V.~Boisvert$^\textrm{\scriptsize 91}$,    
P.~Bokan$^\textrm{\scriptsize 51,169}$,    
T.~Bold$^\textrm{\scriptsize 81a}$,    
A.S.~Boldyrev$^\textrm{\scriptsize 111}$,    
A.E.~Bolz$^\textrm{\scriptsize 59b}$,    
M.~Bomben$^\textrm{\scriptsize 133}$,    
M.~Bona$^\textrm{\scriptsize 90}$,    
J.S.~Bonilla$^\textrm{\scriptsize 128}$,    
M.~Boonekamp$^\textrm{\scriptsize 142}$,    
H.M.~Borecka-Bielska$^\textrm{\scriptsize 88}$,    
A.~Borisov$^\textrm{\scriptsize 121}$,    
G.~Borissov$^\textrm{\scriptsize 87}$,    
J.~Bortfeldt$^\textrm{\scriptsize 35}$,    
D.~Bortoletto$^\textrm{\scriptsize 132}$,    
V.~Bortolotto$^\textrm{\scriptsize 71a,71b}$,    
D.~Boscherini$^\textrm{\scriptsize 23b}$,    
M.~Bosman$^\textrm{\scriptsize 14}$,    
J.D.~Bossio~Sola$^\textrm{\scriptsize 30}$,    
K.~Bouaouda$^\textrm{\scriptsize 34a}$,    
J.~Boudreau$^\textrm{\scriptsize 136}$,    
E.V.~Bouhova-Thacker$^\textrm{\scriptsize 87}$,    
D.~Boumediene$^\textrm{\scriptsize 37}$,    
C.~Bourdarios$^\textrm{\scriptsize 129}$,    
S.K.~Boutle$^\textrm{\scriptsize 55}$,    
A.~Boveia$^\textrm{\scriptsize 123}$,    
J.~Boyd$^\textrm{\scriptsize 35}$,    
D.~Boye$^\textrm{\scriptsize 32b}$,    
I.R.~Boyko$^\textrm{\scriptsize 77}$,    
A.J.~Bozson$^\textrm{\scriptsize 91}$,    
J.~Bracinik$^\textrm{\scriptsize 21}$,    
N.~Brahimi$^\textrm{\scriptsize 99}$,    
A.~Brandt$^\textrm{\scriptsize 8}$,    
G.~Brandt$^\textrm{\scriptsize 179}$,    
O.~Brandt$^\textrm{\scriptsize 59a}$,    
F.~Braren$^\textrm{\scriptsize 44}$,    
U.~Bratzler$^\textrm{\scriptsize 161}$,    
B.~Brau$^\textrm{\scriptsize 100}$,    
J.E.~Brau$^\textrm{\scriptsize 128}$,    
W.D.~Breaden~Madden$^\textrm{\scriptsize 55}$,    
K.~Brendlinger$^\textrm{\scriptsize 44}$,    
L.~Brenner$^\textrm{\scriptsize 44}$,    
R.~Brenner$^\textrm{\scriptsize 169}$,    
S.~Bressler$^\textrm{\scriptsize 177}$,    
B.~Brickwedde$^\textrm{\scriptsize 97}$,    
D.L.~Briglin$^\textrm{\scriptsize 21}$,    
D.~Britton$^\textrm{\scriptsize 55}$,    
D.~Britzger$^\textrm{\scriptsize 113}$,    
I.~Brock$^\textrm{\scriptsize 24}$,    
R.~Brock$^\textrm{\scriptsize 104}$,    
G.~Brooijmans$^\textrm{\scriptsize 38}$,    
T.~Brooks$^\textrm{\scriptsize 91}$,    
W.K.~Brooks$^\textrm{\scriptsize 144b}$,    
E.~Brost$^\textrm{\scriptsize 119}$,    
J.H~Broughton$^\textrm{\scriptsize 21}$,    
P.A.~Bruckman~de~Renstrom$^\textrm{\scriptsize 82}$,    
D.~Bruncko$^\textrm{\scriptsize 28b}$,    
A.~Bruni$^\textrm{\scriptsize 23b}$,    
G.~Bruni$^\textrm{\scriptsize 23b}$,    
L.S.~Bruni$^\textrm{\scriptsize 118}$,    
S.~Bruno$^\textrm{\scriptsize 71a,71b}$,    
B.H.~Brunt$^\textrm{\scriptsize 31}$,    
M.~Bruschi$^\textrm{\scriptsize 23b}$,    
N.~Bruscino$^\textrm{\scriptsize 136}$,    
P.~Bryant$^\textrm{\scriptsize 36}$,    
L.~Bryngemark$^\textrm{\scriptsize 94}$,    
T.~Buanes$^\textrm{\scriptsize 17}$,    
Q.~Buat$^\textrm{\scriptsize 35}$,    
P.~Buchholz$^\textrm{\scriptsize 148}$,    
A.G.~Buckley$^\textrm{\scriptsize 55}$,    
I.A.~Budagov$^\textrm{\scriptsize 77}$,    
M.K.~Bugge$^\textrm{\scriptsize 131}$,    
F.~B\"uhrer$^\textrm{\scriptsize 50}$,    
O.~Bulekov$^\textrm{\scriptsize 110}$,    
D.~Bullock$^\textrm{\scriptsize 8}$,    
T.J.~Burch$^\textrm{\scriptsize 119}$,    
S.~Burdin$^\textrm{\scriptsize 88}$,    
C.D.~Burgard$^\textrm{\scriptsize 118}$,    
A.M.~Burger$^\textrm{\scriptsize 5}$,    
B.~Burghgrave$^\textrm{\scriptsize 119}$,    
K.~Burka$^\textrm{\scriptsize 82}$,    
S.~Burke$^\textrm{\scriptsize 141}$,    
I.~Burmeister$^\textrm{\scriptsize 45}$,    
J.T.P.~Burr$^\textrm{\scriptsize 132}$,    
V.~B\"uscher$^\textrm{\scriptsize 97}$,    
E.~Buschmann$^\textrm{\scriptsize 51}$,    
P.~Bussey$^\textrm{\scriptsize 55}$,    
J.M.~Butler$^\textrm{\scriptsize 25}$,    
C.M.~Buttar$^\textrm{\scriptsize 55}$,    
J.M.~Butterworth$^\textrm{\scriptsize 92}$,    
P.~Butti$^\textrm{\scriptsize 35}$,    
W.~Buttinger$^\textrm{\scriptsize 35}$,    
A.~Buzatu$^\textrm{\scriptsize 155}$,    
A.R.~Buzykaev$^\textrm{\scriptsize 120b,120a}$,    
G.~Cabras$^\textrm{\scriptsize 23b,23a}$,    
S.~Cabrera~Urb\'an$^\textrm{\scriptsize 171}$,    
D.~Caforio$^\textrm{\scriptsize 139}$,    
H.~Cai$^\textrm{\scriptsize 170}$,    
V.M.M.~Cairo$^\textrm{\scriptsize 2}$,    
O.~Cakir$^\textrm{\scriptsize 4a}$,    
N.~Calace$^\textrm{\scriptsize 35}$,    
P.~Calafiura$^\textrm{\scriptsize 18}$,    
A.~Calandri$^\textrm{\scriptsize 99}$,    
G.~Calderini$^\textrm{\scriptsize 133}$,    
P.~Calfayan$^\textrm{\scriptsize 63}$,    
G.~Callea$^\textrm{\scriptsize 55}$,    
L.P.~Caloba$^\textrm{\scriptsize 78b}$,    
S.~Calvente~Lopez$^\textrm{\scriptsize 96}$,    
D.~Calvet$^\textrm{\scriptsize 37}$,    
S.~Calvet$^\textrm{\scriptsize 37}$,    
T.P.~Calvet$^\textrm{\scriptsize 152}$,    
M.~Calvetti$^\textrm{\scriptsize 69a,69b}$,    
R.~Camacho~Toro$^\textrm{\scriptsize 133}$,    
S.~Camarda$^\textrm{\scriptsize 35}$,    
D.~Camarero~Munoz$^\textrm{\scriptsize 96}$,    
P.~Camarri$^\textrm{\scriptsize 71a,71b}$,    
D.~Cameron$^\textrm{\scriptsize 131}$,    
R.~Caminal~Armadans$^\textrm{\scriptsize 100}$,    
C.~Camincher$^\textrm{\scriptsize 35}$,    
S.~Campana$^\textrm{\scriptsize 35}$,    
M.~Campanelli$^\textrm{\scriptsize 92}$,    
A.~Camplani$^\textrm{\scriptsize 39}$,    
A.~Campoverde$^\textrm{\scriptsize 148}$,    
V.~Canale$^\textrm{\scriptsize 67a,67b}$,    
M.~Cano~Bret$^\textrm{\scriptsize 58c}$,    
J.~Cantero$^\textrm{\scriptsize 126}$,    
T.~Cao$^\textrm{\scriptsize 158}$,    
Y.~Cao$^\textrm{\scriptsize 170}$,    
M.D.M.~Capeans~Garrido$^\textrm{\scriptsize 35}$,    
I.~Caprini$^\textrm{\scriptsize 27b}$,    
M.~Caprini$^\textrm{\scriptsize 27b}$,    
M.~Capua$^\textrm{\scriptsize 40b,40a}$,    
R.M.~Carbone$^\textrm{\scriptsize 38}$,    
R.~Cardarelli$^\textrm{\scriptsize 71a}$,    
F.C.~Cardillo$^\textrm{\scriptsize 146}$,    
I.~Carli$^\textrm{\scriptsize 140}$,    
T.~Carli$^\textrm{\scriptsize 35}$,    
G.~Carlino$^\textrm{\scriptsize 67a}$,    
B.T.~Carlson$^\textrm{\scriptsize 136}$,    
L.~Carminati$^\textrm{\scriptsize 66a,66b}$,    
R.M.D.~Carney$^\textrm{\scriptsize 43a,43b}$,    
S.~Caron$^\textrm{\scriptsize 117}$,    
E.~Carquin$^\textrm{\scriptsize 144b}$,    
S.~Carr\'a$^\textrm{\scriptsize 66a,66b}$,    
J.W.S.~Carter$^\textrm{\scriptsize 164}$,    
D.~Casadei$^\textrm{\scriptsize 32b}$,    
M.P.~Casado$^\textrm{\scriptsize 14,g}$,    
A.F.~Casha$^\textrm{\scriptsize 164}$,    
D.W.~Casper$^\textrm{\scriptsize 168}$,    
R.~Castelijn$^\textrm{\scriptsize 118}$,    
F.L.~Castillo$^\textrm{\scriptsize 171}$,    
V.~Castillo~Gimenez$^\textrm{\scriptsize 171}$,    
N.F.~Castro$^\textrm{\scriptsize 137a,137e}$,    
A.~Catinaccio$^\textrm{\scriptsize 35}$,    
J.R.~Catmore$^\textrm{\scriptsize 131}$,    
A.~Cattai$^\textrm{\scriptsize 35}$,    
J.~Caudron$^\textrm{\scriptsize 24}$,    
V.~Cavaliere$^\textrm{\scriptsize 29}$,    
E.~Cavallaro$^\textrm{\scriptsize 14}$,    
D.~Cavalli$^\textrm{\scriptsize 66a}$,    
M.~Cavalli-Sforza$^\textrm{\scriptsize 14}$,    
V.~Cavasinni$^\textrm{\scriptsize 69a,69b}$,    
E.~Celebi$^\textrm{\scriptsize 12b}$,    
F.~Ceradini$^\textrm{\scriptsize 72a,72b}$,    
L.~Cerda~Alberich$^\textrm{\scriptsize 171}$,    
A.S.~Cerqueira$^\textrm{\scriptsize 78a}$,    
A.~Cerri$^\textrm{\scriptsize 153}$,    
L.~Cerrito$^\textrm{\scriptsize 71a,71b}$,    
F.~Cerutti$^\textrm{\scriptsize 18}$,    
A.~Cervelli$^\textrm{\scriptsize 23b,23a}$,    
S.A.~Cetin$^\textrm{\scriptsize 12b}$,    
A.~Chafaq$^\textrm{\scriptsize 34a}$,    
D.~Chakraborty$^\textrm{\scriptsize 119}$,    
S.K.~Chan$^\textrm{\scriptsize 57}$,    
W.S.~Chan$^\textrm{\scriptsize 118}$,    
W.Y.~Chan$^\textrm{\scriptsize 88}$,    
J.D.~Chapman$^\textrm{\scriptsize 31}$,    
B.~Chargeishvili$^\textrm{\scriptsize 156b}$,    
D.G.~Charlton$^\textrm{\scriptsize 21}$,    
C.C.~Chau$^\textrm{\scriptsize 33}$,    
C.A.~Chavez~Barajas$^\textrm{\scriptsize 153}$,    
S.~Che$^\textrm{\scriptsize 123}$,    
A.~Chegwidden$^\textrm{\scriptsize 104}$,    
S.~Chekanov$^\textrm{\scriptsize 6}$,    
S.V.~Chekulaev$^\textrm{\scriptsize 165a}$,    
G.A.~Chelkov$^\textrm{\scriptsize 77,au}$,    
M.A.~Chelstowska$^\textrm{\scriptsize 35}$,    
B.~Chen$^\textrm{\scriptsize 76}$,    
C.~Chen$^\textrm{\scriptsize 58a}$,    
C.H.~Chen$^\textrm{\scriptsize 76}$,    
H.~Chen$^\textrm{\scriptsize 29}$,    
J.~Chen$^\textrm{\scriptsize 58a}$,    
J.~Chen$^\textrm{\scriptsize 38}$,    
S.~Chen$^\textrm{\scriptsize 134}$,    
S.J.~Chen$^\textrm{\scriptsize 15c}$,    
X.~Chen$^\textrm{\scriptsize 15b,at}$,    
Y.~Chen$^\textrm{\scriptsize 80}$,    
Y-H.~Chen$^\textrm{\scriptsize 44}$,    
H.C.~Cheng$^\textrm{\scriptsize 61a}$,    
H.J.~Cheng$^\textrm{\scriptsize 15d}$,    
A.~Cheplakov$^\textrm{\scriptsize 77}$,    
E.~Cheremushkina$^\textrm{\scriptsize 121}$,    
R.~Cherkaoui~El~Moursli$^\textrm{\scriptsize 34e}$,    
E.~Cheu$^\textrm{\scriptsize 7}$,    
K.~Cheung$^\textrm{\scriptsize 62}$,    
T.J.A.~Cheval\'erias$^\textrm{\scriptsize 142}$,    
L.~Chevalier$^\textrm{\scriptsize 142}$,    
V.~Chiarella$^\textrm{\scriptsize 49}$,    
G.~Chiarelli$^\textrm{\scriptsize 69a}$,    
G.~Chiodini$^\textrm{\scriptsize 65a}$,    
A.S.~Chisholm$^\textrm{\scriptsize 35,21}$,    
A.~Chitan$^\textrm{\scriptsize 27b}$,    
I.~Chiu$^\textrm{\scriptsize 160}$,    
Y.H.~Chiu$^\textrm{\scriptsize 173}$,    
M.V.~Chizhov$^\textrm{\scriptsize 77}$,    
K.~Choi$^\textrm{\scriptsize 63}$,    
A.R.~Chomont$^\textrm{\scriptsize 129}$,    
S.~Chouridou$^\textrm{\scriptsize 159}$,    
Y.S.~Chow$^\textrm{\scriptsize 118}$,    
V.~Christodoulou$^\textrm{\scriptsize 92}$,    
M.C.~Chu$^\textrm{\scriptsize 61a}$,    
J.~Chudoba$^\textrm{\scriptsize 138}$,    
A.J.~Chuinard$^\textrm{\scriptsize 101}$,    
J.J.~Chwastowski$^\textrm{\scriptsize 82}$,    
L.~Chytka$^\textrm{\scriptsize 127}$,    
D.~Cinca$^\textrm{\scriptsize 45}$,    
V.~Cindro$^\textrm{\scriptsize 89}$,    
I.A.~Cioar\u{a}$^\textrm{\scriptsize 24}$,    
A.~Ciocio$^\textrm{\scriptsize 18}$,    
F.~Cirotto$^\textrm{\scriptsize 67a,67b}$,    
Z.H.~Citron$^\textrm{\scriptsize 177}$,    
M.~Citterio$^\textrm{\scriptsize 66a}$,    
A.~Clark$^\textrm{\scriptsize 52}$,    
M.R.~Clark$^\textrm{\scriptsize 38}$,    
P.J.~Clark$^\textrm{\scriptsize 48}$,    
C.~Clement$^\textrm{\scriptsize 43a,43b}$,    
Y.~Coadou$^\textrm{\scriptsize 99}$,    
M.~Cobal$^\textrm{\scriptsize 64a,64c}$,    
A.~Coccaro$^\textrm{\scriptsize 53b,53a}$,    
J.~Cochran$^\textrm{\scriptsize 76}$,    
H.~Cohen$^\textrm{\scriptsize 158}$,    
A.E.C.~Coimbra$^\textrm{\scriptsize 177}$,    
L.~Colasurdo$^\textrm{\scriptsize 117}$,    
B.~Cole$^\textrm{\scriptsize 38}$,    
A.P.~Colijn$^\textrm{\scriptsize 118}$,    
J.~Collot$^\textrm{\scriptsize 56}$,    
P.~Conde~Mui\~no$^\textrm{\scriptsize 137a,i}$,    
E.~Coniavitis$^\textrm{\scriptsize 50}$,    
S.H.~Connell$^\textrm{\scriptsize 32b}$,    
I.A.~Connelly$^\textrm{\scriptsize 98}$,    
S.~Constantinescu$^\textrm{\scriptsize 27b}$,    
F.~Conventi$^\textrm{\scriptsize 67a,aw}$,    
A.M.~Cooper-Sarkar$^\textrm{\scriptsize 132}$,    
F.~Cormier$^\textrm{\scriptsize 172}$,    
K.J.R.~Cormier$^\textrm{\scriptsize 164}$,    
L.D.~Corpe$^\textrm{\scriptsize 92}$,    
M.~Corradi$^\textrm{\scriptsize 70a,70b}$,    
E.E.~Corrigan$^\textrm{\scriptsize 94}$,    
F.~Corriveau$^\textrm{\scriptsize 101,ad}$,    
A.~Cortes-Gonzalez$^\textrm{\scriptsize 35}$,    
M.J.~Costa$^\textrm{\scriptsize 171}$,    
F.~Costanza$^\textrm{\scriptsize 5}$,    
D.~Costanzo$^\textrm{\scriptsize 146}$,    
G.~Cottin$^\textrm{\scriptsize 31}$,    
G.~Cowan$^\textrm{\scriptsize 91}$,    
J.W.~Cowley$^\textrm{\scriptsize 31}$,    
B.E.~Cox$^\textrm{\scriptsize 98}$,    
J.~Crane$^\textrm{\scriptsize 98}$,    
K.~Cranmer$^\textrm{\scriptsize 122}$,    
S.J.~Crawley$^\textrm{\scriptsize 55}$,    
R.A.~Creager$^\textrm{\scriptsize 134}$,    
G.~Cree$^\textrm{\scriptsize 33}$,    
S.~Cr\'ep\'e-Renaudin$^\textrm{\scriptsize 56}$,    
F.~Crescioli$^\textrm{\scriptsize 133}$,    
M.~Cristinziani$^\textrm{\scriptsize 24}$,    
V.~Croft$^\textrm{\scriptsize 122}$,    
G.~Crosetti$^\textrm{\scriptsize 40b,40a}$,    
A.~Cueto$^\textrm{\scriptsize 96}$,    
T.~Cuhadar~Donszelmann$^\textrm{\scriptsize 146}$,    
A.R.~Cukierman$^\textrm{\scriptsize 150}$,    
S.~Czekierda$^\textrm{\scriptsize 82}$,    
P.~Czodrowski$^\textrm{\scriptsize 35}$,    
M.J.~Da~Cunha~Sargedas~De~Sousa$^\textrm{\scriptsize 58b}$,    
C.~Da~Via$^\textrm{\scriptsize 98}$,    
W.~Dabrowski$^\textrm{\scriptsize 81a}$,    
T.~Dado$^\textrm{\scriptsize 28a,y}$,    
S.~Dahbi$^\textrm{\scriptsize 34e}$,    
T.~Dai$^\textrm{\scriptsize 103}$,    
F.~Dallaire$^\textrm{\scriptsize 107}$,    
C.~Dallapiccola$^\textrm{\scriptsize 100}$,    
M.~Dam$^\textrm{\scriptsize 39}$,    
G.~D'amen$^\textrm{\scriptsize 23b,23a}$,    
J.~Damp$^\textrm{\scriptsize 97}$,    
J.R.~Dandoy$^\textrm{\scriptsize 134}$,    
M.F.~Daneri$^\textrm{\scriptsize 30}$,    
N.P.~Dang$^\textrm{\scriptsize 178,l}$,    
N.D~Dann$^\textrm{\scriptsize 98}$,    
M.~Danninger$^\textrm{\scriptsize 172}$,    
V.~Dao$^\textrm{\scriptsize 35}$,    
G.~Darbo$^\textrm{\scriptsize 53b}$,    
S.~Darmora$^\textrm{\scriptsize 8}$,    
O.~Dartsi$^\textrm{\scriptsize 5}$,    
A.~Dattagupta$^\textrm{\scriptsize 128}$,    
T.~Daubney$^\textrm{\scriptsize 44}$,    
S.~D'Auria$^\textrm{\scriptsize 66a,66b}$,    
W.~Davey$^\textrm{\scriptsize 24}$,    
C.~David$^\textrm{\scriptsize 44}$,    
T.~Davidek$^\textrm{\scriptsize 140}$,    
D.R.~Davis$^\textrm{\scriptsize 47}$,    
E.~Dawe$^\textrm{\scriptsize 102}$,    
I.~Dawson$^\textrm{\scriptsize 146}$,    
K.~De$^\textrm{\scriptsize 8}$,    
R.~De~Asmundis$^\textrm{\scriptsize 67a}$,    
A.~De~Benedetti$^\textrm{\scriptsize 125}$,    
M.~De~Beurs$^\textrm{\scriptsize 118}$,    
S.~De~Castro$^\textrm{\scriptsize 23b,23a}$,    
S.~De~Cecco$^\textrm{\scriptsize 70a,70b}$,    
N.~De~Groot$^\textrm{\scriptsize 117}$,    
P.~de~Jong$^\textrm{\scriptsize 118}$,    
H.~De~la~Torre$^\textrm{\scriptsize 104}$,    
F.~De~Lorenzi$^\textrm{\scriptsize 76}$,    
A.~De~Maria$^\textrm{\scriptsize 69a,69b}$,    
D.~De~Pedis$^\textrm{\scriptsize 70a}$,    
A.~De~Salvo$^\textrm{\scriptsize 70a}$,    
U.~De~Sanctis$^\textrm{\scriptsize 71a,71b}$,    
M.~De~Santis$^\textrm{\scriptsize 71a,71b}$,    
A.~De~Santo$^\textrm{\scriptsize 153}$,    
K.~De~Vasconcelos~Corga$^\textrm{\scriptsize 99}$,    
J.B.~De~Vivie~De~Regie$^\textrm{\scriptsize 129}$,    
C.~Debenedetti$^\textrm{\scriptsize 143}$,    
D.V.~Dedovich$^\textrm{\scriptsize 77}$,    
N.~Dehghanian$^\textrm{\scriptsize 3}$,    
M.~Del~Gaudio$^\textrm{\scriptsize 40b,40a}$,    
J.~Del~Peso$^\textrm{\scriptsize 96}$,    
Y.~Delabat~Diaz$^\textrm{\scriptsize 44}$,    
D.~Delgove$^\textrm{\scriptsize 129}$,    
F.~Deliot$^\textrm{\scriptsize 142}$,    
C.M.~Delitzsch$^\textrm{\scriptsize 7}$,    
M.~Della~Pietra$^\textrm{\scriptsize 67a,67b}$,    
D.~Della~Volpe$^\textrm{\scriptsize 52}$,    
A.~Dell'Acqua$^\textrm{\scriptsize 35}$,    
L.~Dell'Asta$^\textrm{\scriptsize 25}$,    
M.~Delmastro$^\textrm{\scriptsize 5}$,    
C.~Delporte$^\textrm{\scriptsize 129}$,    
P.A.~Delsart$^\textrm{\scriptsize 56}$,    
D.A.~DeMarco$^\textrm{\scriptsize 164}$,    
S.~Demers$^\textrm{\scriptsize 180}$,    
M.~Demichev$^\textrm{\scriptsize 77}$,    
S.P.~Denisov$^\textrm{\scriptsize 121}$,    
D.~Denysiuk$^\textrm{\scriptsize 118}$,    
L.~D'Eramo$^\textrm{\scriptsize 133}$,    
D.~Derendarz$^\textrm{\scriptsize 82}$,    
J.E.~Derkaoui$^\textrm{\scriptsize 34d}$,    
F.~Derue$^\textrm{\scriptsize 133}$,    
P.~Dervan$^\textrm{\scriptsize 88}$,    
K.~Desch$^\textrm{\scriptsize 24}$,    
C.~Deterre$^\textrm{\scriptsize 44}$,    
K.~Dette$^\textrm{\scriptsize 164}$,    
M.R.~Devesa$^\textrm{\scriptsize 30}$,    
P.O.~Deviveiros$^\textrm{\scriptsize 35}$,    
A.~Dewhurst$^\textrm{\scriptsize 141}$,    
S.~Dhaliwal$^\textrm{\scriptsize 26}$,    
F.A.~Di~Bello$^\textrm{\scriptsize 52}$,    
A.~Di~Ciaccio$^\textrm{\scriptsize 71a,71b}$,    
L.~Di~Ciaccio$^\textrm{\scriptsize 5}$,    
W.K.~Di~Clemente$^\textrm{\scriptsize 134}$,    
C.~Di~Donato$^\textrm{\scriptsize 67a,67b}$,    
A.~Di~Girolamo$^\textrm{\scriptsize 35}$,    
G.~Di~Gregorio$^\textrm{\scriptsize 69a,69b}$,    
B.~Di~Micco$^\textrm{\scriptsize 72a,72b}$,    
R.~Di~Nardo$^\textrm{\scriptsize 100}$,    
K.F.~Di~Petrillo$^\textrm{\scriptsize 57}$,    
R.~Di~Sipio$^\textrm{\scriptsize 164}$,    
D.~Di~Valentino$^\textrm{\scriptsize 33}$,    
C.~Diaconu$^\textrm{\scriptsize 99}$,    
M.~Diamond$^\textrm{\scriptsize 164}$,    
F.A.~Dias$^\textrm{\scriptsize 39}$,    
T.~Dias~Do~Vale$^\textrm{\scriptsize 137a}$,    
M.A.~Diaz$^\textrm{\scriptsize 144a}$,    
J.~Dickinson$^\textrm{\scriptsize 18}$,    
E.B.~Diehl$^\textrm{\scriptsize 103}$,    
J.~Dietrich$^\textrm{\scriptsize 19}$,    
S.~D\'iez~Cornell$^\textrm{\scriptsize 44}$,    
A.~Dimitrievska$^\textrm{\scriptsize 18}$,    
J.~Dingfelder$^\textrm{\scriptsize 24}$,    
F.~Dittus$^\textrm{\scriptsize 35}$,    
F.~Djama$^\textrm{\scriptsize 99}$,    
T.~Djobava$^\textrm{\scriptsize 156b}$,    
J.I.~Djuvsland$^\textrm{\scriptsize 17}$,    
M.A.B.~Do~Vale$^\textrm{\scriptsize 78c}$,    
M.~Dobre$^\textrm{\scriptsize 27b}$,    
D.~Dodsworth$^\textrm{\scriptsize 26}$,    
C.~Doglioni$^\textrm{\scriptsize 94}$,    
J.~Dolejsi$^\textrm{\scriptsize 140}$,    
Z.~Dolezal$^\textrm{\scriptsize 140}$,    
M.~Donadelli$^\textrm{\scriptsize 78d}$,    
J.~Donini$^\textrm{\scriptsize 37}$,    
A.~D'onofrio$^\textrm{\scriptsize 90}$,    
M.~D'Onofrio$^\textrm{\scriptsize 88}$,    
J.~Dopke$^\textrm{\scriptsize 141}$,    
A.~Doria$^\textrm{\scriptsize 67a}$,    
M.T.~Dova$^\textrm{\scriptsize 86}$,    
A.T.~Doyle$^\textrm{\scriptsize 55}$,    
E.~Drechsler$^\textrm{\scriptsize 149}$,    
E.~Dreyer$^\textrm{\scriptsize 149}$,    
T.~Dreyer$^\textrm{\scriptsize 51}$,    
Y.~Du$^\textrm{\scriptsize 58b}$,    
F.~Dubinin$^\textrm{\scriptsize 108}$,    
M.~Dubovsky$^\textrm{\scriptsize 28a}$,    
A.~Dubreuil$^\textrm{\scriptsize 52}$,    
E.~Duchovni$^\textrm{\scriptsize 177}$,    
G.~Duckeck$^\textrm{\scriptsize 112}$,    
A.~Ducourthial$^\textrm{\scriptsize 133}$,    
O.A.~Ducu$^\textrm{\scriptsize 107,x}$,    
D.~Duda$^\textrm{\scriptsize 113}$,    
A.~Dudarev$^\textrm{\scriptsize 35}$,    
A.C.~Dudder$^\textrm{\scriptsize 97}$,    
E.M.~Duffield$^\textrm{\scriptsize 18}$,    
L.~Duflot$^\textrm{\scriptsize 129}$,    
M.~D\"uhrssen$^\textrm{\scriptsize 35}$,    
C.~D{\"u}lsen$^\textrm{\scriptsize 179}$,    
M.~Dumancic$^\textrm{\scriptsize 177}$,    
A.E.~Dumitriu$^\textrm{\scriptsize 27b,e}$,    
A.K.~Duncan$^\textrm{\scriptsize 55}$,    
M.~Dunford$^\textrm{\scriptsize 59a}$,    
A.~Duperrin$^\textrm{\scriptsize 99}$,    
H.~Duran~Yildiz$^\textrm{\scriptsize 4a}$,    
M.~D\"uren$^\textrm{\scriptsize 54}$,    
A.~Durglishvili$^\textrm{\scriptsize 156b}$,    
D.~Duschinger$^\textrm{\scriptsize 46}$,    
B.~Dutta$^\textrm{\scriptsize 44}$,    
D.~Duvnjak$^\textrm{\scriptsize 1}$,    
M.~Dyndal$^\textrm{\scriptsize 44}$,    
S.~Dysch$^\textrm{\scriptsize 98}$,    
B.S.~Dziedzic$^\textrm{\scriptsize 82}$,    
K.M.~Ecker$^\textrm{\scriptsize 113}$,    
R.C.~Edgar$^\textrm{\scriptsize 103}$,    
T.~Eifert$^\textrm{\scriptsize 35}$,    
G.~Eigen$^\textrm{\scriptsize 17}$,    
K.~Einsweiler$^\textrm{\scriptsize 18}$,    
T.~Ekelof$^\textrm{\scriptsize 169}$,    
M.~El~Kacimi$^\textrm{\scriptsize 34c}$,    
R.~El~Kosseifi$^\textrm{\scriptsize 99}$,    
V.~Ellajosyula$^\textrm{\scriptsize 99}$,    
M.~Ellert$^\textrm{\scriptsize 169}$,    
F.~Ellinghaus$^\textrm{\scriptsize 179}$,    
A.A.~Elliot$^\textrm{\scriptsize 90}$,    
N.~Ellis$^\textrm{\scriptsize 35}$,    
J.~Elmsheuser$^\textrm{\scriptsize 29}$,    
M.~Elsing$^\textrm{\scriptsize 35}$,    
D.~Emeliyanov$^\textrm{\scriptsize 141}$,    
A.~Emerman$^\textrm{\scriptsize 38}$,    
Y.~Enari$^\textrm{\scriptsize 160}$,    
J.S.~Ennis$^\textrm{\scriptsize 175}$,    
M.B.~Epland$^\textrm{\scriptsize 47}$,    
J.~Erdmann$^\textrm{\scriptsize 45}$,    
A.~Ereditato$^\textrm{\scriptsize 20}$,    
S.~Errede$^\textrm{\scriptsize 170}$,    
M.~Escalier$^\textrm{\scriptsize 129}$,    
C.~Escobar$^\textrm{\scriptsize 171}$,    
O.~Estrada~Pastor$^\textrm{\scriptsize 171}$,    
A.I.~Etienvre$^\textrm{\scriptsize 142}$,    
E.~Etzion$^\textrm{\scriptsize 158}$,    
H.~Evans$^\textrm{\scriptsize 63}$,    
A.~Ezhilov$^\textrm{\scriptsize 135}$,    
M.~Ezzi$^\textrm{\scriptsize 34e}$,    
F.~Fabbri$^\textrm{\scriptsize 55}$,    
L.~Fabbri$^\textrm{\scriptsize 23b,23a}$,    
V.~Fabiani$^\textrm{\scriptsize 117}$,    
G.~Facini$^\textrm{\scriptsize 92}$,    
R.M.~Faisca~Rodrigues~Pereira$^\textrm{\scriptsize 137a}$,    
R.M.~Fakhrutdinov$^\textrm{\scriptsize 121}$,    
S.~Falciano$^\textrm{\scriptsize 70a}$,    
P.J.~Falke$^\textrm{\scriptsize 5}$,    
S.~Falke$^\textrm{\scriptsize 5}$,    
J.~Faltova$^\textrm{\scriptsize 140}$,    
Y.~Fang$^\textrm{\scriptsize 15a}$,    
M.~Fanti$^\textrm{\scriptsize 66a,66b}$,    
A.~Farbin$^\textrm{\scriptsize 8}$,    
A.~Farilla$^\textrm{\scriptsize 72a}$,    
E.M.~Farina$^\textrm{\scriptsize 68a,68b}$,    
T.~Farooque$^\textrm{\scriptsize 104}$,    
S.~Farrell$^\textrm{\scriptsize 18}$,    
S.M.~Farrington$^\textrm{\scriptsize 175}$,    
P.~Farthouat$^\textrm{\scriptsize 35}$,    
F.~Fassi$^\textrm{\scriptsize 34e}$,    
P.~Fassnacht$^\textrm{\scriptsize 35}$,    
D.~Fassouliotis$^\textrm{\scriptsize 9}$,    
M.~Faucci~Giannelli$^\textrm{\scriptsize 48}$,    
W.J.~Fawcett$^\textrm{\scriptsize 31}$,    
L.~Fayard$^\textrm{\scriptsize 129}$,    
O.L.~Fedin$^\textrm{\scriptsize 135,q}$,    
W.~Fedorko$^\textrm{\scriptsize 172}$,    
M.~Feickert$^\textrm{\scriptsize 41}$,    
S.~Feigl$^\textrm{\scriptsize 131}$,    
L.~Feligioni$^\textrm{\scriptsize 99}$,    
C.~Feng$^\textrm{\scriptsize 58b}$,    
E.J.~Feng$^\textrm{\scriptsize 35}$,    
M.~Feng$^\textrm{\scriptsize 47}$,    
M.J.~Fenton$^\textrm{\scriptsize 55}$,    
A.B.~Fenyuk$^\textrm{\scriptsize 121}$,    
J.~Ferrando$^\textrm{\scriptsize 44}$,    
A.~Ferrari$^\textrm{\scriptsize 169}$,    
P.~Ferrari$^\textrm{\scriptsize 118}$,    
R.~Ferrari$^\textrm{\scriptsize 68a}$,    
D.E.~Ferreira~de~Lima$^\textrm{\scriptsize 59b}$,    
A.~Ferrer$^\textrm{\scriptsize 171}$,    
D.~Ferrere$^\textrm{\scriptsize 52}$,    
C.~Ferretti$^\textrm{\scriptsize 103}$,    
F.~Fiedler$^\textrm{\scriptsize 97}$,    
A.~Filip\v{c}i\v{c}$^\textrm{\scriptsize 89}$,    
F.~Filthaut$^\textrm{\scriptsize 117}$,    
K.D.~Finelli$^\textrm{\scriptsize 25}$,    
M.C.N.~Fiolhais$^\textrm{\scriptsize 137a,137c,a}$,    
L.~Fiorini$^\textrm{\scriptsize 171}$,    
C.~Fischer$^\textrm{\scriptsize 14}$,    
W.C.~Fisher$^\textrm{\scriptsize 104}$,    
N.~Flaschel$^\textrm{\scriptsize 44}$,    
I.~Fleck$^\textrm{\scriptsize 148}$,    
P.~Fleischmann$^\textrm{\scriptsize 103}$,    
R.R.M.~Fletcher$^\textrm{\scriptsize 134}$,    
T.~Flick$^\textrm{\scriptsize 179}$,    
B.M.~Flierl$^\textrm{\scriptsize 112}$,    
L.M.~Flores$^\textrm{\scriptsize 134}$,    
L.R.~Flores~Castillo$^\textrm{\scriptsize 61a}$,    
F.M.~Follega$^\textrm{\scriptsize 73a,73b}$,    
N.~Fomin$^\textrm{\scriptsize 17}$,    
G.T.~Forcolin$^\textrm{\scriptsize 73a,73b}$,    
A.~Formica$^\textrm{\scriptsize 142}$,    
F.A.~F\"orster$^\textrm{\scriptsize 14}$,    
A.C.~Forti$^\textrm{\scriptsize 98}$,    
A.G.~Foster$^\textrm{\scriptsize 21}$,    
D.~Fournier$^\textrm{\scriptsize 129}$,    
H.~Fox$^\textrm{\scriptsize 87}$,    
S.~Fracchia$^\textrm{\scriptsize 146}$,    
P.~Francavilla$^\textrm{\scriptsize 69a,69b}$,    
M.~Franchini$^\textrm{\scriptsize 23b,23a}$,    
S.~Franchino$^\textrm{\scriptsize 59a}$,    
D.~Francis$^\textrm{\scriptsize 35}$,    
L.~Franconi$^\textrm{\scriptsize 143}$,    
M.~Franklin$^\textrm{\scriptsize 57}$,    
M.~Frate$^\textrm{\scriptsize 168}$,    
M.~Fraternali$^\textrm{\scriptsize 68a,68b}$,    
A.N.~Fray$^\textrm{\scriptsize 90}$,    
D.~Freeborn$^\textrm{\scriptsize 92}$,    
B.~Freund$^\textrm{\scriptsize 107}$,    
W.S.~Freund$^\textrm{\scriptsize 78b}$,    
E.M.~Freundlich$^\textrm{\scriptsize 45}$,    
D.C.~Frizzell$^\textrm{\scriptsize 125}$,    
D.~Froidevaux$^\textrm{\scriptsize 35}$,    
J.A.~Frost$^\textrm{\scriptsize 132}$,    
C.~Fukunaga$^\textrm{\scriptsize 161}$,    
E.~Fullana~Torregrosa$^\textrm{\scriptsize 171}$,    
E.~Fumagalli$^\textrm{\scriptsize 53b,53a}$,    
T.~Fusayasu$^\textrm{\scriptsize 114}$,    
J.~Fuster$^\textrm{\scriptsize 171}$,    
O.~Gabizon$^\textrm{\scriptsize 157}$,    
A.~Gabrielli$^\textrm{\scriptsize 23b,23a}$,    
A.~Gabrielli$^\textrm{\scriptsize 18}$,    
G.P.~Gach$^\textrm{\scriptsize 81a}$,    
S.~Gadatsch$^\textrm{\scriptsize 52}$,    
P.~Gadow$^\textrm{\scriptsize 113}$,    
G.~Gagliardi$^\textrm{\scriptsize 53b,53a}$,    
L.G.~Gagnon$^\textrm{\scriptsize 107}$,    
C.~Galea$^\textrm{\scriptsize 27b}$,    
B.~Galhardo$^\textrm{\scriptsize 137a,137c}$,    
E.J.~Gallas$^\textrm{\scriptsize 132}$,    
B.J.~Gallop$^\textrm{\scriptsize 141}$,    
P.~Gallus$^\textrm{\scriptsize 139}$,    
G.~Galster$^\textrm{\scriptsize 39}$,    
R.~Gamboa~Goni$^\textrm{\scriptsize 90}$,    
K.K.~Gan$^\textrm{\scriptsize 123}$,    
S.~Ganguly$^\textrm{\scriptsize 177}$,    
J.~Gao$^\textrm{\scriptsize 58a}$,    
Y.~Gao$^\textrm{\scriptsize 88}$,    
Y.S.~Gao$^\textrm{\scriptsize 150,n}$,    
C.~Garc\'ia$^\textrm{\scriptsize 171}$,    
J.E.~Garc\'ia~Navarro$^\textrm{\scriptsize 171}$,    
J.A.~Garc\'ia~Pascual$^\textrm{\scriptsize 15a}$,    
C.~Garcia-Argos$^\textrm{\scriptsize 50}$,    
M.~Garcia-Sciveres$^\textrm{\scriptsize 18}$,    
R.W.~Gardner$^\textrm{\scriptsize 36}$,    
N.~Garelli$^\textrm{\scriptsize 150}$,    
S.~Gargiulo$^\textrm{\scriptsize 50}$,    
V.~Garonne$^\textrm{\scriptsize 131}$,    
K.~Gasnikova$^\textrm{\scriptsize 44}$,    
A.~Gaudiello$^\textrm{\scriptsize 53b,53a}$,    
G.~Gaudio$^\textrm{\scriptsize 68a}$,    
I.L.~Gavrilenko$^\textrm{\scriptsize 108}$,    
A.~Gavrilyuk$^\textrm{\scriptsize 109}$,    
C.~Gay$^\textrm{\scriptsize 172}$,    
G.~Gaycken$^\textrm{\scriptsize 24}$,    
E.N.~Gazis$^\textrm{\scriptsize 10}$,    
C.N.P.~Gee$^\textrm{\scriptsize 141}$,    
J.~Geisen$^\textrm{\scriptsize 51}$,    
M.~Geisen$^\textrm{\scriptsize 97}$,    
M.P.~Geisler$^\textrm{\scriptsize 59a}$,    
C.~Gemme$^\textrm{\scriptsize 53b}$,    
M.H.~Genest$^\textrm{\scriptsize 56}$,    
C.~Geng$^\textrm{\scriptsize 103}$,    
S.~Gentile$^\textrm{\scriptsize 70a,70b}$,    
S.~George$^\textrm{\scriptsize 91}$,    
D.~Gerbaudo$^\textrm{\scriptsize 14}$,    
G.~Gessner$^\textrm{\scriptsize 45}$,    
S.~Ghasemi$^\textrm{\scriptsize 148}$,    
M.~Ghasemi~Bostanabad$^\textrm{\scriptsize 173}$,    
M.~Ghneimat$^\textrm{\scriptsize 24}$,    
B.~Giacobbe$^\textrm{\scriptsize 23b}$,    
S.~Giagu$^\textrm{\scriptsize 70a,70b}$,    
N.~Giangiacomi$^\textrm{\scriptsize 23b,23a}$,    
P.~Giannetti$^\textrm{\scriptsize 69a}$,    
A.~Giannini$^\textrm{\scriptsize 67a,67b}$,    
S.M.~Gibson$^\textrm{\scriptsize 91}$,    
M.~Gignac$^\textrm{\scriptsize 143}$,    
D.~Gillberg$^\textrm{\scriptsize 33}$,    
G.~Gilles$^\textrm{\scriptsize 179}$,    
D.M.~Gingrich$^\textrm{\scriptsize 3,av}$,    
M.P.~Giordani$^\textrm{\scriptsize 64a,64c}$,    
F.M.~Giorgi$^\textrm{\scriptsize 23b}$,    
P.F.~Giraud$^\textrm{\scriptsize 142}$,    
P.~Giromini$^\textrm{\scriptsize 57}$,    
G.~Giugliarelli$^\textrm{\scriptsize 64a,64c}$,    
D.~Giugni$^\textrm{\scriptsize 66a}$,    
F.~Giuli$^\textrm{\scriptsize 132}$,    
M.~Giulini$^\textrm{\scriptsize 59b}$,    
S.~Gkaitatzis$^\textrm{\scriptsize 159}$,    
I.~Gkialas$^\textrm{\scriptsize 9,k}$,    
E.L.~Gkougkousis$^\textrm{\scriptsize 14}$,    
P.~Gkountoumis$^\textrm{\scriptsize 10}$,    
L.K.~Gladilin$^\textrm{\scriptsize 111}$,    
C.~Glasman$^\textrm{\scriptsize 96}$,    
J.~Glatzer$^\textrm{\scriptsize 14}$,    
P.C.F.~Glaysher$^\textrm{\scriptsize 44}$,    
A.~Glazov$^\textrm{\scriptsize 44}$,    
M.~Goblirsch-Kolb$^\textrm{\scriptsize 26}$,    
J.~Godlewski$^\textrm{\scriptsize 82}$,    
S.~Goldfarb$^\textrm{\scriptsize 102}$,    
T.~Golling$^\textrm{\scriptsize 52}$,    
D.~Golubkov$^\textrm{\scriptsize 121}$,    
A.~Gomes$^\textrm{\scriptsize 137a,137b}$,    
R.~Goncalves~Gama$^\textrm{\scriptsize 51}$,    
R.~Gon\c{c}alo$^\textrm{\scriptsize 137a}$,    
G.~Gonella$^\textrm{\scriptsize 50}$,    
L.~Gonella$^\textrm{\scriptsize 21}$,    
A.~Gongadze$^\textrm{\scriptsize 77}$,    
F.~Gonnella$^\textrm{\scriptsize 21}$,    
J.L.~Gonski$^\textrm{\scriptsize 57}$,    
S.~Gonz\'alez~de~la~Hoz$^\textrm{\scriptsize 171}$,    
S.~Gonzalez-Sevilla$^\textrm{\scriptsize 52}$,    
L.~Goossens$^\textrm{\scriptsize 35}$,    
P.A.~Gorbounov$^\textrm{\scriptsize 109}$,    
H.A.~Gordon$^\textrm{\scriptsize 29}$,    
B.~Gorini$^\textrm{\scriptsize 35}$,    
E.~Gorini$^\textrm{\scriptsize 65a,65b}$,    
A.~Gori\v{s}ek$^\textrm{\scriptsize 89}$,    
A.T.~Goshaw$^\textrm{\scriptsize 47}$,    
C.~G\"ossling$^\textrm{\scriptsize 45}$,    
M.I.~Gostkin$^\textrm{\scriptsize 77}$,    
C.A.~Gottardo$^\textrm{\scriptsize 24}$,    
C.R.~Goudet$^\textrm{\scriptsize 129}$,    
D.~Goujdami$^\textrm{\scriptsize 34c}$,    
A.G.~Goussiou$^\textrm{\scriptsize 145}$,    
N.~Govender$^\textrm{\scriptsize 32b,c}$,    
C.~Goy$^\textrm{\scriptsize 5}$,    
E.~Gozani$^\textrm{\scriptsize 157}$,    
I.~Grabowska-Bold$^\textrm{\scriptsize 81a}$,    
P.O.J.~Gradin$^\textrm{\scriptsize 169}$,    
E.C.~Graham$^\textrm{\scriptsize 88}$,    
J.~Gramling$^\textrm{\scriptsize 168}$,    
E.~Gramstad$^\textrm{\scriptsize 131}$,    
S.~Grancagnolo$^\textrm{\scriptsize 19}$,    
V.~Gratchev$^\textrm{\scriptsize 135}$,    
P.M.~Gravila$^\textrm{\scriptsize 27f}$,    
F.G.~Gravili$^\textrm{\scriptsize 65a,65b}$,    
C.~Gray$^\textrm{\scriptsize 55}$,    
H.M.~Gray$^\textrm{\scriptsize 18}$,    
Z.D.~Greenwood$^\textrm{\scriptsize 93,al}$,    
C.~Grefe$^\textrm{\scriptsize 24}$,    
K.~Gregersen$^\textrm{\scriptsize 94}$,    
I.M.~Gregor$^\textrm{\scriptsize 44}$,    
P.~Grenier$^\textrm{\scriptsize 150}$,    
K.~Grevtsov$^\textrm{\scriptsize 44}$,    
N.A.~Grieser$^\textrm{\scriptsize 125}$,    
J.~Griffiths$^\textrm{\scriptsize 8}$,    
A.A.~Grillo$^\textrm{\scriptsize 143}$,    
K.~Grimm$^\textrm{\scriptsize 150,b}$,    
S.~Grinstein$^\textrm{\scriptsize 14,z}$,    
Ph.~Gris$^\textrm{\scriptsize 37}$,    
J.-F.~Grivaz$^\textrm{\scriptsize 129}$,    
S.~Groh$^\textrm{\scriptsize 97}$,    
E.~Gross$^\textrm{\scriptsize 177}$,    
J.~Grosse-Knetter$^\textrm{\scriptsize 51}$,    
G.C.~Grossi$^\textrm{\scriptsize 93}$,    
Z.J.~Grout$^\textrm{\scriptsize 92}$,    
C.~Grud$^\textrm{\scriptsize 103}$,    
A.~Grummer$^\textrm{\scriptsize 116}$,    
L.~Guan$^\textrm{\scriptsize 103}$,    
W.~Guan$^\textrm{\scriptsize 178}$,    
J.~Guenther$^\textrm{\scriptsize 35}$,    
A.~Guerguichon$^\textrm{\scriptsize 129}$,    
F.~Guescini$^\textrm{\scriptsize 165a}$,    
D.~Guest$^\textrm{\scriptsize 168}$,    
R.~Gugel$^\textrm{\scriptsize 50}$,    
B.~Gui$^\textrm{\scriptsize 123}$,    
T.~Guillemin$^\textrm{\scriptsize 5}$,    
S.~Guindon$^\textrm{\scriptsize 35}$,    
U.~Gul$^\textrm{\scriptsize 55}$,    
J.~Guo$^\textrm{\scriptsize 58c}$,    
W.~Guo$^\textrm{\scriptsize 103}$,    
Y.~Guo$^\textrm{\scriptsize 58a,t}$,    
Z.~Guo$^\textrm{\scriptsize 99}$,    
R.~Gupta$^\textrm{\scriptsize 44}$,    
S.~Gurbuz$^\textrm{\scriptsize 12c}$,    
G.~Gustavino$^\textrm{\scriptsize 125}$,    
P.~Gutierrez$^\textrm{\scriptsize 125}$,    
C.~Gutschow$^\textrm{\scriptsize 92}$,    
C.~Guyot$^\textrm{\scriptsize 142}$,    
M.P.~Guzik$^\textrm{\scriptsize 81a}$,    
C.~Gwenlan$^\textrm{\scriptsize 132}$,    
C.B.~Gwilliam$^\textrm{\scriptsize 88}$,    
A.~Haas$^\textrm{\scriptsize 122}$,    
C.~Haber$^\textrm{\scriptsize 18}$,    
H.K.~Hadavand$^\textrm{\scriptsize 8}$,    
N.~Haddad$^\textrm{\scriptsize 34e}$,    
A.~Hadef$^\textrm{\scriptsize 58a}$,    
S.~Hageb\"ock$^\textrm{\scriptsize 24}$,    
M.~Hagihara$^\textrm{\scriptsize 166}$,    
M.~Haleem$^\textrm{\scriptsize 174}$,    
J.~Haley$^\textrm{\scriptsize 126}$,    
G.~Halladjian$^\textrm{\scriptsize 104}$,    
G.D.~Hallewell$^\textrm{\scriptsize 99}$,    
K.~Hamacher$^\textrm{\scriptsize 179}$,    
P.~Hamal$^\textrm{\scriptsize 127}$,    
K.~Hamano$^\textrm{\scriptsize 173}$,    
A.~Hamilton$^\textrm{\scriptsize 32a}$,    
G.N.~Hamity$^\textrm{\scriptsize 146}$,    
K.~Han$^\textrm{\scriptsize 58a,ak}$,    
L.~Han$^\textrm{\scriptsize 58a}$,    
S.~Han$^\textrm{\scriptsize 15d}$,    
K.~Hanagaki$^\textrm{\scriptsize 79,v}$,    
M.~Hance$^\textrm{\scriptsize 143}$,    
D.M.~Handl$^\textrm{\scriptsize 112}$,    
B.~Haney$^\textrm{\scriptsize 134}$,    
R.~Hankache$^\textrm{\scriptsize 133}$,    
P.~Hanke$^\textrm{\scriptsize 59a}$,    
E.~Hansen$^\textrm{\scriptsize 94}$,    
J.B.~Hansen$^\textrm{\scriptsize 39}$,    
J.D.~Hansen$^\textrm{\scriptsize 39}$,    
M.C.~Hansen$^\textrm{\scriptsize 24}$,    
P.H.~Hansen$^\textrm{\scriptsize 39}$,    
K.~Hara$^\textrm{\scriptsize 166}$,    
A.S.~Hard$^\textrm{\scriptsize 178}$,    
T.~Harenberg$^\textrm{\scriptsize 179}$,    
S.~Harkusha$^\textrm{\scriptsize 105}$,    
P.F.~Harrison$^\textrm{\scriptsize 175}$,    
N.M.~Hartmann$^\textrm{\scriptsize 112}$,    
Y.~Hasegawa$^\textrm{\scriptsize 147}$,    
A.~Hasib$^\textrm{\scriptsize 48}$,    
S.~Hassani$^\textrm{\scriptsize 142}$,    
S.~Haug$^\textrm{\scriptsize 20}$,    
R.~Hauser$^\textrm{\scriptsize 104}$,    
L.~Hauswald$^\textrm{\scriptsize 46}$,    
L.B.~Havener$^\textrm{\scriptsize 38}$,    
M.~Havranek$^\textrm{\scriptsize 139}$,    
C.M.~Hawkes$^\textrm{\scriptsize 21}$,    
R.J.~Hawkings$^\textrm{\scriptsize 35}$,    
D.~Hayden$^\textrm{\scriptsize 104}$,    
C.~Hayes$^\textrm{\scriptsize 152}$,    
C.P.~Hays$^\textrm{\scriptsize 132}$,    
J.M.~Hays$^\textrm{\scriptsize 90}$,    
H.S.~Hayward$^\textrm{\scriptsize 88}$,    
S.J.~Haywood$^\textrm{\scriptsize 141}$,    
F.~He$^\textrm{\scriptsize 58a}$,    
M.P.~Heath$^\textrm{\scriptsize 48}$,    
V.~Hedberg$^\textrm{\scriptsize 94}$,    
L.~Heelan$^\textrm{\scriptsize 8}$,    
S.~Heer$^\textrm{\scriptsize 24}$,    
K.K.~Heidegger$^\textrm{\scriptsize 50}$,    
J.~Heilman$^\textrm{\scriptsize 33}$,    
S.~Heim$^\textrm{\scriptsize 44}$,    
T.~Heim$^\textrm{\scriptsize 18}$,    
B.~Heinemann$^\textrm{\scriptsize 44,aq}$,    
J.J.~Heinrich$^\textrm{\scriptsize 112}$,    
L.~Heinrich$^\textrm{\scriptsize 122}$,    
C.~Heinz$^\textrm{\scriptsize 54}$,    
J.~Hejbal$^\textrm{\scriptsize 138}$,    
L.~Helary$^\textrm{\scriptsize 35}$,    
A.~Held$^\textrm{\scriptsize 172}$,    
S.~Hellesund$^\textrm{\scriptsize 131}$,    
C.M.~Helling$^\textrm{\scriptsize 143}$,    
S.~Hellman$^\textrm{\scriptsize 43a,43b}$,    
C.~Helsens$^\textrm{\scriptsize 35}$,    
R.C.W.~Henderson$^\textrm{\scriptsize 87}$,    
Y.~Heng$^\textrm{\scriptsize 178}$,    
S.~Henkelmann$^\textrm{\scriptsize 172}$,    
A.M.~Henriques~Correia$^\textrm{\scriptsize 35}$,    
G.H.~Herbert$^\textrm{\scriptsize 19}$,    
H.~Herde$^\textrm{\scriptsize 26}$,    
V.~Herget$^\textrm{\scriptsize 174}$,    
Y.~Hern\'andez~Jim\'enez$^\textrm{\scriptsize 32c}$,    
H.~Herr$^\textrm{\scriptsize 97}$,    
M.G.~Herrmann$^\textrm{\scriptsize 112}$,    
T.~Herrmann$^\textrm{\scriptsize 46}$,    
G.~Herten$^\textrm{\scriptsize 50}$,    
R.~Hertenberger$^\textrm{\scriptsize 112}$,    
L.~Hervas$^\textrm{\scriptsize 35}$,    
T.C.~Herwig$^\textrm{\scriptsize 134}$,    
G.G.~Hesketh$^\textrm{\scriptsize 92}$,    
N.P.~Hessey$^\textrm{\scriptsize 165a}$,    
A.~Higashida$^\textrm{\scriptsize 160}$,    
S.~Higashino$^\textrm{\scriptsize 79}$,    
E.~Hig\'on-Rodriguez$^\textrm{\scriptsize 171}$,    
K.~Hildebrand$^\textrm{\scriptsize 36}$,    
E.~Hill$^\textrm{\scriptsize 173}$,    
J.C.~Hill$^\textrm{\scriptsize 31}$,    
K.K.~Hill$^\textrm{\scriptsize 29}$,    
K.H.~Hiller$^\textrm{\scriptsize 44}$,    
S.J.~Hillier$^\textrm{\scriptsize 21}$,    
M.~Hils$^\textrm{\scriptsize 46}$,    
I.~Hinchliffe$^\textrm{\scriptsize 18}$,    
F.~Hinterkeuser$^\textrm{\scriptsize 24}$,    
M.~Hirose$^\textrm{\scriptsize 130}$,    
D.~Hirschbuehl$^\textrm{\scriptsize 179}$,    
B.~Hiti$^\textrm{\scriptsize 89}$,    
O.~Hladik$^\textrm{\scriptsize 138}$,    
D.R.~Hlaluku$^\textrm{\scriptsize 32c}$,    
X.~Hoad$^\textrm{\scriptsize 48}$,    
J.~Hobbs$^\textrm{\scriptsize 152}$,    
N.~Hod$^\textrm{\scriptsize 165a}$,    
M.C.~Hodgkinson$^\textrm{\scriptsize 146}$,    
A.~Hoecker$^\textrm{\scriptsize 35}$,    
M.R.~Hoeferkamp$^\textrm{\scriptsize 116}$,    
F.~Hoenig$^\textrm{\scriptsize 112}$,    
D.~Hohn$^\textrm{\scriptsize 50}$,    
D.~Hohov$^\textrm{\scriptsize 129}$,    
T.R.~Holmes$^\textrm{\scriptsize 36}$,    
M.~Holzbock$^\textrm{\scriptsize 112}$,    
M.~Homann$^\textrm{\scriptsize 45}$,    
B.H.~Hommels$^\textrm{\scriptsize 31}$,    
S.~Honda$^\textrm{\scriptsize 166}$,    
T.~Honda$^\textrm{\scriptsize 79}$,    
T.M.~Hong$^\textrm{\scriptsize 136}$,    
A.~H\"{o}nle$^\textrm{\scriptsize 113}$,    
B.H.~Hooberman$^\textrm{\scriptsize 170}$,    
W.H.~Hopkins$^\textrm{\scriptsize 128}$,    
Y.~Horii$^\textrm{\scriptsize 115}$,    
P.~Horn$^\textrm{\scriptsize 46}$,    
A.J.~Horton$^\textrm{\scriptsize 149}$,    
L.A.~Horyn$^\textrm{\scriptsize 36}$,    
J-Y.~Hostachy$^\textrm{\scriptsize 56}$,    
A.~Hostiuc$^\textrm{\scriptsize 145}$,    
S.~Hou$^\textrm{\scriptsize 155}$,    
A.~Hoummada$^\textrm{\scriptsize 34a}$,    
J.~Howarth$^\textrm{\scriptsize 98}$,    
J.~Hoya$^\textrm{\scriptsize 86}$,    
M.~Hrabovsky$^\textrm{\scriptsize 127}$,    
I.~Hristova$^\textrm{\scriptsize 19}$,    
J.~Hrivnac$^\textrm{\scriptsize 129}$,    
A.~Hrynevich$^\textrm{\scriptsize 106}$,    
T.~Hryn'ova$^\textrm{\scriptsize 5}$,    
P.J.~Hsu$^\textrm{\scriptsize 62}$,    
S.-C.~Hsu$^\textrm{\scriptsize 145}$,    
Q.~Hu$^\textrm{\scriptsize 29}$,    
S.~Hu$^\textrm{\scriptsize 58c}$,    
Y.~Huang$^\textrm{\scriptsize 15a}$,    
Z.~Hubacek$^\textrm{\scriptsize 139}$,    
F.~Hubaut$^\textrm{\scriptsize 99}$,    
M.~Huebner$^\textrm{\scriptsize 24}$,    
F.~Huegging$^\textrm{\scriptsize 24}$,    
T.B.~Huffman$^\textrm{\scriptsize 132}$,    
M.~Huhtinen$^\textrm{\scriptsize 35}$,    
R.F.H.~Hunter$^\textrm{\scriptsize 33}$,    
P.~Huo$^\textrm{\scriptsize 152}$,    
A.M.~Hupe$^\textrm{\scriptsize 33}$,    
N.~Huseynov$^\textrm{\scriptsize 77,af}$,    
J.~Huston$^\textrm{\scriptsize 104}$,    
J.~Huth$^\textrm{\scriptsize 57}$,    
R.~Hyneman$^\textrm{\scriptsize 103}$,    
G.~Iacobucci$^\textrm{\scriptsize 52}$,    
G.~Iakovidis$^\textrm{\scriptsize 29}$,    
I.~Ibragimov$^\textrm{\scriptsize 148}$,    
L.~Iconomidou-Fayard$^\textrm{\scriptsize 129}$,    
Z.~Idrissi$^\textrm{\scriptsize 34e}$,    
P.~Iengo$^\textrm{\scriptsize 35}$,    
R.~Ignazzi$^\textrm{\scriptsize 39}$,    
O.~Igonkina$^\textrm{\scriptsize 118,ab}$,    
R.~Iguchi$^\textrm{\scriptsize 160}$,    
T.~Iizawa$^\textrm{\scriptsize 52}$,    
Y.~Ikegami$^\textrm{\scriptsize 79}$,    
M.~Ikeno$^\textrm{\scriptsize 79}$,    
D.~Iliadis$^\textrm{\scriptsize 159}$,    
N.~Ilic$^\textrm{\scriptsize 117}$,    
F.~Iltzsche$^\textrm{\scriptsize 46}$,    
G.~Introzzi$^\textrm{\scriptsize 68a,68b}$,    
M.~Iodice$^\textrm{\scriptsize 72a}$,    
K.~Iordanidou$^\textrm{\scriptsize 38}$,    
V.~Ippolito$^\textrm{\scriptsize 70a,70b}$,    
M.F.~Isacson$^\textrm{\scriptsize 169}$,    
N.~Ishijima$^\textrm{\scriptsize 130}$,    
M.~Ishino$^\textrm{\scriptsize 160}$,    
M.~Ishitsuka$^\textrm{\scriptsize 162}$,    
W.~Islam$^\textrm{\scriptsize 126}$,    
C.~Issever$^\textrm{\scriptsize 132}$,    
S.~Istin$^\textrm{\scriptsize 157}$,    
F.~Ito$^\textrm{\scriptsize 166}$,    
J.M.~Iturbe~Ponce$^\textrm{\scriptsize 61a}$,    
R.~Iuppa$^\textrm{\scriptsize 73a,73b}$,    
A.~Ivina$^\textrm{\scriptsize 177}$,    
H.~Iwasaki$^\textrm{\scriptsize 79}$,    
J.M.~Izen$^\textrm{\scriptsize 42}$,    
V.~Izzo$^\textrm{\scriptsize 67a}$,    
P.~Jacka$^\textrm{\scriptsize 138}$,    
P.~Jackson$^\textrm{\scriptsize 1}$,    
R.M.~Jacobs$^\textrm{\scriptsize 24}$,    
V.~Jain$^\textrm{\scriptsize 2}$,    
G.~J\"akel$^\textrm{\scriptsize 179}$,    
K.B.~Jakobi$^\textrm{\scriptsize 97}$,    
K.~Jakobs$^\textrm{\scriptsize 50}$,    
S.~Jakobsen$^\textrm{\scriptsize 74}$,    
T.~Jakoubek$^\textrm{\scriptsize 138}$,    
D.O.~Jamin$^\textrm{\scriptsize 126}$,    
R.~Jansky$^\textrm{\scriptsize 52}$,    
J.~Janssen$^\textrm{\scriptsize 24}$,    
M.~Janus$^\textrm{\scriptsize 51}$,    
P.A.~Janus$^\textrm{\scriptsize 81a}$,    
G.~Jarlskog$^\textrm{\scriptsize 94}$,    
N.~Javadov$^\textrm{\scriptsize 77,af}$,    
T.~Jav\r{u}rek$^\textrm{\scriptsize 35}$,    
M.~Javurkova$^\textrm{\scriptsize 50}$,    
F.~Jeanneau$^\textrm{\scriptsize 142}$,    
L.~Jeanty$^\textrm{\scriptsize 18}$,    
J.~Jejelava$^\textrm{\scriptsize 156a,ag}$,    
A.~Jelinskas$^\textrm{\scriptsize 175}$,    
P.~Jenni$^\textrm{\scriptsize 50,d}$,    
J.~Jeong$^\textrm{\scriptsize 44}$,    
N.~Jeong$^\textrm{\scriptsize 44}$,    
S.~J\'ez\'equel$^\textrm{\scriptsize 5}$,    
H.~Ji$^\textrm{\scriptsize 178}$,    
J.~Jia$^\textrm{\scriptsize 152}$,    
H.~Jiang$^\textrm{\scriptsize 76}$,    
Y.~Jiang$^\textrm{\scriptsize 58a}$,    
Z.~Jiang$^\textrm{\scriptsize 150,r}$,    
S.~Jiggins$^\textrm{\scriptsize 50}$,    
F.A.~Jimenez~Morales$^\textrm{\scriptsize 37}$,    
J.~Jimenez~Pena$^\textrm{\scriptsize 171}$,    
S.~Jin$^\textrm{\scriptsize 15c}$,    
A.~Jinaru$^\textrm{\scriptsize 27b}$,    
O.~Jinnouchi$^\textrm{\scriptsize 162}$,    
H.~Jivan$^\textrm{\scriptsize 32c}$,    
P.~Johansson$^\textrm{\scriptsize 146}$,    
K.A.~Johns$^\textrm{\scriptsize 7}$,    
C.A.~Johnson$^\textrm{\scriptsize 63}$,    
K.~Jon-And$^\textrm{\scriptsize 43a,43b}$,    
R.W.L.~Jones$^\textrm{\scriptsize 87}$,    
S.D.~Jones$^\textrm{\scriptsize 153}$,    
S.~Jones$^\textrm{\scriptsize 7}$,    
T.J.~Jones$^\textrm{\scriptsize 88}$,    
J.~Jongmanns$^\textrm{\scriptsize 59a}$,    
P.M.~Jorge$^\textrm{\scriptsize 137a,137b}$,    
J.~Jovicevic$^\textrm{\scriptsize 165a}$,    
X.~Ju$^\textrm{\scriptsize 18}$,    
J.J.~Junggeburth$^\textrm{\scriptsize 113}$,    
A.~Juste~Rozas$^\textrm{\scriptsize 14,z}$,    
A.~Kaczmarska$^\textrm{\scriptsize 82}$,    
M.~Kado$^\textrm{\scriptsize 129}$,    
H.~Kagan$^\textrm{\scriptsize 123}$,    
M.~Kagan$^\textrm{\scriptsize 150}$,    
T.~Kaji$^\textrm{\scriptsize 176}$,    
E.~Kajomovitz$^\textrm{\scriptsize 157}$,    
C.W.~Kalderon$^\textrm{\scriptsize 94}$,    
A.~Kaluza$^\textrm{\scriptsize 97}$,    
S.~Kama$^\textrm{\scriptsize 41}$,    
A.~Kamenshchikov$^\textrm{\scriptsize 121}$,    
L.~Kanjir$^\textrm{\scriptsize 89}$,    
Y.~Kano$^\textrm{\scriptsize 160}$,    
V.A.~Kantserov$^\textrm{\scriptsize 110}$,    
J.~Kanzaki$^\textrm{\scriptsize 79}$,    
L.S.~Kaplan$^\textrm{\scriptsize 178}$,    
D.~Kar$^\textrm{\scriptsize 32c}$,    
M.J.~Kareem$^\textrm{\scriptsize 165b}$,    
E.~Karentzos$^\textrm{\scriptsize 10}$,    
S.N.~Karpov$^\textrm{\scriptsize 77}$,    
Z.M.~Karpova$^\textrm{\scriptsize 77}$,    
V.~Kartvelishvili$^\textrm{\scriptsize 87}$,    
A.N.~Karyukhin$^\textrm{\scriptsize 121}$,    
L.~Kashif$^\textrm{\scriptsize 178}$,    
R.D.~Kass$^\textrm{\scriptsize 123}$,    
A.~Kastanas$^\textrm{\scriptsize 43a,43b}$,    
Y.~Kataoka$^\textrm{\scriptsize 160}$,    
C.~Kato$^\textrm{\scriptsize 58d,58c}$,    
J.~Katzy$^\textrm{\scriptsize 44}$,    
K.~Kawade$^\textrm{\scriptsize 80}$,    
K.~Kawagoe$^\textrm{\scriptsize 85}$,    
T.~Kawaguchi$^\textrm{\scriptsize 115}$,    
T.~Kawamoto$^\textrm{\scriptsize 160}$,    
G.~Kawamura$^\textrm{\scriptsize 51}$,    
E.F.~Kay$^\textrm{\scriptsize 88}$,    
V.F.~Kazanin$^\textrm{\scriptsize 120b,120a}$,    
R.~Keeler$^\textrm{\scriptsize 173}$,    
R.~Kehoe$^\textrm{\scriptsize 41}$,    
J.S.~Keller$^\textrm{\scriptsize 33}$,    
E.~Kellermann$^\textrm{\scriptsize 94}$,    
J.J.~Kempster$^\textrm{\scriptsize 21}$,    
J.~Kendrick$^\textrm{\scriptsize 21}$,    
O.~Kepka$^\textrm{\scriptsize 138}$,    
S.~Kersten$^\textrm{\scriptsize 179}$,    
B.P.~Ker\v{s}evan$^\textrm{\scriptsize 89}$,    
S.~Ketabchi~Haghighat$^\textrm{\scriptsize 164}$,    
R.A.~Keyes$^\textrm{\scriptsize 101}$,    
M.~Khader$^\textrm{\scriptsize 170}$,    
F.~Khalil-Zada$^\textrm{\scriptsize 13}$,    
A.~Khanov$^\textrm{\scriptsize 126}$,    
A.G.~Kharlamov$^\textrm{\scriptsize 120b,120a}$,    
T.~Kharlamova$^\textrm{\scriptsize 120b,120a}$,    
E.E.~Khoda$^\textrm{\scriptsize 172}$,    
A.~Khodinov$^\textrm{\scriptsize 163}$,    
T.J.~Khoo$^\textrm{\scriptsize 52}$,    
E.~Khramov$^\textrm{\scriptsize 77}$,    
J.~Khubua$^\textrm{\scriptsize 156b}$,    
S.~Kido$^\textrm{\scriptsize 80}$,    
M.~Kiehn$^\textrm{\scriptsize 52}$,    
C.R.~Kilby$^\textrm{\scriptsize 91}$,    
Y.K.~Kim$^\textrm{\scriptsize 36}$,    
N.~Kimura$^\textrm{\scriptsize 64a,64c}$,    
O.M.~Kind$^\textrm{\scriptsize 19}$,    
B.T.~King$^\textrm{\scriptsize 88}$,    
D.~Kirchmeier$^\textrm{\scriptsize 46}$,    
J.~Kirk$^\textrm{\scriptsize 141}$,    
A.E.~Kiryunin$^\textrm{\scriptsize 113}$,    
T.~Kishimoto$^\textrm{\scriptsize 160}$,    
D.~Kisielewska$^\textrm{\scriptsize 81a}$,    
V.~Kitali$^\textrm{\scriptsize 44}$,    
O.~Kivernyk$^\textrm{\scriptsize 5}$,    
E.~Kladiva$^\textrm{\scriptsize 28b,*}$,    
T.~Klapdor-Kleingrothaus$^\textrm{\scriptsize 50}$,    
M.H.~Klein$^\textrm{\scriptsize 103}$,    
M.~Klein$^\textrm{\scriptsize 88}$,    
U.~Klein$^\textrm{\scriptsize 88}$,    
K.~Kleinknecht$^\textrm{\scriptsize 97}$,    
P.~Klimek$^\textrm{\scriptsize 119}$,    
A.~Klimentov$^\textrm{\scriptsize 29}$,    
T.~Klingl$^\textrm{\scriptsize 24}$,    
T.~Klioutchnikova$^\textrm{\scriptsize 35}$,    
F.F.~Klitzner$^\textrm{\scriptsize 112}$,    
P.~Kluit$^\textrm{\scriptsize 118}$,    
S.~Kluth$^\textrm{\scriptsize 113}$,    
E.~Kneringer$^\textrm{\scriptsize 74}$,    
E.B.F.G.~Knoops$^\textrm{\scriptsize 99}$,    
A.~Knue$^\textrm{\scriptsize 50}$,    
A.~Kobayashi$^\textrm{\scriptsize 160}$,    
D.~Kobayashi$^\textrm{\scriptsize 85}$,    
T.~Kobayashi$^\textrm{\scriptsize 160}$,    
M.~Kobel$^\textrm{\scriptsize 46}$,    
M.~Kocian$^\textrm{\scriptsize 150}$,    
P.~Kodys$^\textrm{\scriptsize 140}$,    
P.T.~Koenig$^\textrm{\scriptsize 24}$,    
T.~Koffas$^\textrm{\scriptsize 33}$,    
E.~Koffeman$^\textrm{\scriptsize 118}$,    
N.M.~K\"ohler$^\textrm{\scriptsize 113}$,    
T.~Koi$^\textrm{\scriptsize 150}$,    
M.~Kolb$^\textrm{\scriptsize 59b}$,    
I.~Koletsou$^\textrm{\scriptsize 5}$,    
T.~Kondo$^\textrm{\scriptsize 79}$,    
N.~Kondrashova$^\textrm{\scriptsize 58c}$,    
K.~K\"oneke$^\textrm{\scriptsize 50}$,    
A.C.~K\"onig$^\textrm{\scriptsize 117}$,    
T.~Kono$^\textrm{\scriptsize 79}$,    
R.~Konoplich$^\textrm{\scriptsize 122,an}$,    
V.~Konstantinides$^\textrm{\scriptsize 92}$,    
N.~Konstantinidis$^\textrm{\scriptsize 92}$,    
B.~Konya$^\textrm{\scriptsize 94}$,    
R.~Kopeliansky$^\textrm{\scriptsize 63}$,    
S.~Koperny$^\textrm{\scriptsize 81a}$,    
K.~Korcyl$^\textrm{\scriptsize 82}$,    
K.~Kordas$^\textrm{\scriptsize 159}$,    
G.~Koren$^\textrm{\scriptsize 158}$,    
A.~Korn$^\textrm{\scriptsize 92}$,    
I.~Korolkov$^\textrm{\scriptsize 14}$,    
E.V.~Korolkova$^\textrm{\scriptsize 146}$,    
N.~Korotkova$^\textrm{\scriptsize 111}$,    
O.~Kortner$^\textrm{\scriptsize 113}$,    
S.~Kortner$^\textrm{\scriptsize 113}$,    
T.~Kosek$^\textrm{\scriptsize 140}$,    
V.V.~Kostyukhin$^\textrm{\scriptsize 24}$,    
A.~Kotwal$^\textrm{\scriptsize 47}$,    
A.~Koulouris$^\textrm{\scriptsize 10}$,    
A.~Kourkoumeli-Charalampidi$^\textrm{\scriptsize 68a,68b}$,    
C.~Kourkoumelis$^\textrm{\scriptsize 9}$,    
E.~Kourlitis$^\textrm{\scriptsize 146}$,    
V.~Kouskoura$^\textrm{\scriptsize 29}$,    
A.B.~Kowalewska$^\textrm{\scriptsize 82}$,    
R.~Kowalewski$^\textrm{\scriptsize 173}$,    
T.Z.~Kowalski$^\textrm{\scriptsize 81a}$,    
C.~Kozakai$^\textrm{\scriptsize 160}$,    
W.~Kozanecki$^\textrm{\scriptsize 142}$,    
A.S.~Kozhin$^\textrm{\scriptsize 121}$,    
V.A.~Kramarenko$^\textrm{\scriptsize 111}$,    
G.~Kramberger$^\textrm{\scriptsize 89}$,    
D.~Krasnopevtsev$^\textrm{\scriptsize 58a}$,    
M.W.~Krasny$^\textrm{\scriptsize 133}$,    
A.~Krasznahorkay$^\textrm{\scriptsize 35}$,    
D.~Krauss$^\textrm{\scriptsize 113}$,    
J.A.~Kremer$^\textrm{\scriptsize 81a}$,    
J.~Kretzschmar$^\textrm{\scriptsize 88}$,    
P.~Krieger$^\textrm{\scriptsize 164}$,    
K.~Krizka$^\textrm{\scriptsize 18}$,    
K.~Kroeninger$^\textrm{\scriptsize 45}$,    
H.~Kroha$^\textrm{\scriptsize 113}$,    
J.~Kroll$^\textrm{\scriptsize 138}$,    
J.~Kroll$^\textrm{\scriptsize 134}$,    
J.~Krstic$^\textrm{\scriptsize 16}$,    
U.~Kruchonak$^\textrm{\scriptsize 77}$,    
H.~Kr\"uger$^\textrm{\scriptsize 24}$,    
N.~Krumnack$^\textrm{\scriptsize 76}$,    
M.C.~Kruse$^\textrm{\scriptsize 47}$,    
T.~Kubota$^\textrm{\scriptsize 102}$,    
S.~Kuday$^\textrm{\scriptsize 4b}$,    
J.T.~Kuechler$^\textrm{\scriptsize 179}$,    
S.~Kuehn$^\textrm{\scriptsize 35}$,    
A.~Kugel$^\textrm{\scriptsize 59a}$,    
T.~Kuhl$^\textrm{\scriptsize 44}$,    
V.~Kukhtin$^\textrm{\scriptsize 77}$,    
R.~Kukla$^\textrm{\scriptsize 99}$,    
Y.~Kulchitsky$^\textrm{\scriptsize 105,aj}$,    
S.~Kuleshov$^\textrm{\scriptsize 144b}$,    
Y.P.~Kulinich$^\textrm{\scriptsize 170}$,    
M.~Kuna$^\textrm{\scriptsize 56}$,    
T.~Kunigo$^\textrm{\scriptsize 83}$,    
A.~Kupco$^\textrm{\scriptsize 138}$,    
T.~Kupfer$^\textrm{\scriptsize 45}$,    
O.~Kuprash$^\textrm{\scriptsize 158}$,    
H.~Kurashige$^\textrm{\scriptsize 80}$,    
L.L.~Kurchaninov$^\textrm{\scriptsize 165a}$,    
Y.A.~Kurochkin$^\textrm{\scriptsize 105}$,    
A.~Kurova$^\textrm{\scriptsize 110}$,    
M.G.~Kurth$^\textrm{\scriptsize 15d}$,    
E.S.~Kuwertz$^\textrm{\scriptsize 35}$,    
M.~Kuze$^\textrm{\scriptsize 162}$,    
J.~Kvita$^\textrm{\scriptsize 127}$,    
T.~Kwan$^\textrm{\scriptsize 101}$,    
A.~La~Rosa$^\textrm{\scriptsize 113}$,    
J.L.~La~Rosa~Navarro$^\textrm{\scriptsize 78d}$,    
L.~La~Rotonda$^\textrm{\scriptsize 40b,40a}$,    
F.~La~Ruffa$^\textrm{\scriptsize 40b,40a}$,    
C.~Lacasta$^\textrm{\scriptsize 171}$,    
F.~Lacava$^\textrm{\scriptsize 70a,70b}$,    
J.~Lacey$^\textrm{\scriptsize 44}$,    
D.P.J.~Lack$^\textrm{\scriptsize 98}$,    
H.~Lacker$^\textrm{\scriptsize 19}$,    
D.~Lacour$^\textrm{\scriptsize 133}$,    
E.~Ladygin$^\textrm{\scriptsize 77}$,    
R.~Lafaye$^\textrm{\scriptsize 5}$,    
B.~Laforge$^\textrm{\scriptsize 133}$,    
T.~Lagouri$^\textrm{\scriptsize 32c}$,    
S.~Lai$^\textrm{\scriptsize 51}$,    
S.~Lammers$^\textrm{\scriptsize 63}$,    
W.~Lampl$^\textrm{\scriptsize 7}$,    
E.~Lan\c{c}on$^\textrm{\scriptsize 29}$,    
U.~Landgraf$^\textrm{\scriptsize 50}$,    
M.P.J.~Landon$^\textrm{\scriptsize 90}$,    
M.C.~Lanfermann$^\textrm{\scriptsize 52}$,    
V.S.~Lang$^\textrm{\scriptsize 44}$,    
J.C.~Lange$^\textrm{\scriptsize 51}$,    
R.J.~Langenberg$^\textrm{\scriptsize 35}$,    
A.J.~Lankford$^\textrm{\scriptsize 168}$,    
F.~Lanni$^\textrm{\scriptsize 29}$,    
K.~Lantzsch$^\textrm{\scriptsize 24}$,    
A.~Lanza$^\textrm{\scriptsize 68a}$,    
A.~Lapertosa$^\textrm{\scriptsize 53b,53a}$,    
S.~Laplace$^\textrm{\scriptsize 133}$,    
J.F.~Laporte$^\textrm{\scriptsize 142}$,    
T.~Lari$^\textrm{\scriptsize 66a}$,    
F.~Lasagni~Manghi$^\textrm{\scriptsize 23b,23a}$,    
M.~Lassnig$^\textrm{\scriptsize 35}$,    
T.S.~Lau$^\textrm{\scriptsize 61a}$,    
A.~Laudrain$^\textrm{\scriptsize 129}$,    
M.~Lavorgna$^\textrm{\scriptsize 67a,67b}$,    
M.~Lazzaroni$^\textrm{\scriptsize 66a,66b}$,    
B.~Le$^\textrm{\scriptsize 102}$,    
O.~Le~Dortz$^\textrm{\scriptsize 133}$,    
E.~Le~Guirriec$^\textrm{\scriptsize 99}$,    
E.P.~Le~Quilleuc$^\textrm{\scriptsize 142}$,    
M.~LeBlanc$^\textrm{\scriptsize 7}$,    
T.~LeCompte$^\textrm{\scriptsize 6}$,    
F.~Ledroit-Guillon$^\textrm{\scriptsize 56}$,    
C.A.~Lee$^\textrm{\scriptsize 29}$,    
G.R.~Lee$^\textrm{\scriptsize 144a}$,    
L.~Lee$^\textrm{\scriptsize 57}$,    
S.C.~Lee$^\textrm{\scriptsize 155}$,    
B.~Lefebvre$^\textrm{\scriptsize 101}$,    
M.~Lefebvre$^\textrm{\scriptsize 173}$,    
F.~Legger$^\textrm{\scriptsize 112}$,    
C.~Leggett$^\textrm{\scriptsize 18}$,    
K.~Lehmann$^\textrm{\scriptsize 149}$,    
N.~Lehmann$^\textrm{\scriptsize 179}$,    
G.~Lehmann~Miotto$^\textrm{\scriptsize 35}$,    
W.A.~Leight$^\textrm{\scriptsize 44}$,    
A.~Leisos$^\textrm{\scriptsize 159,w}$,    
M.A.L.~Leite$^\textrm{\scriptsize 78d}$,    
R.~Leitner$^\textrm{\scriptsize 140}$,    
D.~Lellouch$^\textrm{\scriptsize 177}$,    
K.J.C.~Leney$^\textrm{\scriptsize 92}$,    
T.~Lenz$^\textrm{\scriptsize 24}$,    
B.~Lenzi$^\textrm{\scriptsize 35}$,    
R.~Leone$^\textrm{\scriptsize 7}$,    
S.~Leone$^\textrm{\scriptsize 69a}$,    
C.~Leonidopoulos$^\textrm{\scriptsize 48}$,    
G.~Lerner$^\textrm{\scriptsize 153}$,    
C.~Leroy$^\textrm{\scriptsize 107}$,    
R.~Les$^\textrm{\scriptsize 164}$,    
A.A.J.~Lesage$^\textrm{\scriptsize 142}$,    
C.G.~Lester$^\textrm{\scriptsize 31}$,    
M.~Levchenko$^\textrm{\scriptsize 135}$,    
J.~Lev\^eque$^\textrm{\scriptsize 5}$,    
D.~Levin$^\textrm{\scriptsize 103}$,    
L.J.~Levinson$^\textrm{\scriptsize 177}$,    
D.~Lewis$^\textrm{\scriptsize 90}$,    
B.~Li$^\textrm{\scriptsize 15b}$,    
B.~Li$^\textrm{\scriptsize 103}$,    
C-Q.~Li$^\textrm{\scriptsize 58a,am}$,    
H.~Li$^\textrm{\scriptsize 58a}$,    
H.~Li$^\textrm{\scriptsize 58b}$,    
L.~Li$^\textrm{\scriptsize 58c}$,    
M.~Li$^\textrm{\scriptsize 15a}$,    
Q.~Li$^\textrm{\scriptsize 15d}$,    
Q.Y.~Li$^\textrm{\scriptsize 58a}$,    
S.~Li$^\textrm{\scriptsize 58d,58c}$,    
X.~Li$^\textrm{\scriptsize 58c}$,    
Y.~Li$^\textrm{\scriptsize 148}$,    
Z.~Liang$^\textrm{\scriptsize 15a}$,    
B.~Liberti$^\textrm{\scriptsize 71a}$,    
A.~Liblong$^\textrm{\scriptsize 164}$,    
K.~Lie$^\textrm{\scriptsize 61c}$,    
S.~Liem$^\textrm{\scriptsize 118}$,    
A.~Limosani$^\textrm{\scriptsize 154}$,    
C.Y.~Lin$^\textrm{\scriptsize 31}$,    
K.~Lin$^\textrm{\scriptsize 104}$,    
T.H.~Lin$^\textrm{\scriptsize 97}$,    
R.A.~Linck$^\textrm{\scriptsize 63}$,    
J.H.~Lindon$^\textrm{\scriptsize 21}$,    
B.E.~Lindquist$^\textrm{\scriptsize 152}$,    
A.L.~Lionti$^\textrm{\scriptsize 52}$,    
E.~Lipeles$^\textrm{\scriptsize 134}$,    
A.~Lipniacka$^\textrm{\scriptsize 17}$,    
M.~Lisovyi$^\textrm{\scriptsize 59b}$,    
T.M.~Liss$^\textrm{\scriptsize 170,as}$,    
A.~Lister$^\textrm{\scriptsize 172}$,    
A.M.~Litke$^\textrm{\scriptsize 143}$,    
J.D.~Little$^\textrm{\scriptsize 8}$,    
B.~Liu$^\textrm{\scriptsize 76}$,    
B.L~Liu$^\textrm{\scriptsize 6}$,    
H.B.~Liu$^\textrm{\scriptsize 29}$,    
H.~Liu$^\textrm{\scriptsize 103}$,    
J.B.~Liu$^\textrm{\scriptsize 58a}$,    
J.K.K.~Liu$^\textrm{\scriptsize 132}$,    
K.~Liu$^\textrm{\scriptsize 133}$,    
M.~Liu$^\textrm{\scriptsize 58a}$,    
P.~Liu$^\textrm{\scriptsize 18}$,    
Y.~Liu$^\textrm{\scriptsize 15a}$,    
Y.L.~Liu$^\textrm{\scriptsize 58a}$,    
Y.W.~Liu$^\textrm{\scriptsize 58a}$,    
M.~Livan$^\textrm{\scriptsize 68a,68b}$,    
A.~Lleres$^\textrm{\scriptsize 56}$,    
J.~Llorente~Merino$^\textrm{\scriptsize 15a}$,    
S.L.~Lloyd$^\textrm{\scriptsize 90}$,    
C.Y.~Lo$^\textrm{\scriptsize 61b}$,    
F.~Lo~Sterzo$^\textrm{\scriptsize 41}$,    
E.M.~Lobodzinska$^\textrm{\scriptsize 44}$,    
P.~Loch$^\textrm{\scriptsize 7}$,    
T.~Lohse$^\textrm{\scriptsize 19}$,    
K.~Lohwasser$^\textrm{\scriptsize 146}$,    
M.~Lokajicek$^\textrm{\scriptsize 138}$,    
J.D.~Long$^\textrm{\scriptsize 170}$,    
R.E.~Long$^\textrm{\scriptsize 87}$,    
L.~Longo$^\textrm{\scriptsize 65a,65b}$,    
K.A.~Looper$^\textrm{\scriptsize 123}$,    
J.A.~Lopez$^\textrm{\scriptsize 144b}$,    
I.~Lopez~Paz$^\textrm{\scriptsize 98}$,    
A.~Lopez~Solis$^\textrm{\scriptsize 146}$,    
J.~Lorenz$^\textrm{\scriptsize 112}$,    
N.~Lorenzo~Martinez$^\textrm{\scriptsize 5}$,    
M.~Losada$^\textrm{\scriptsize 22}$,    
P.J.~L{\"o}sel$^\textrm{\scriptsize 112}$,    
A.~L\"osle$^\textrm{\scriptsize 50}$,    
X.~Lou$^\textrm{\scriptsize 44}$,    
X.~Lou$^\textrm{\scriptsize 15a}$,    
A.~Lounis$^\textrm{\scriptsize 129}$,    
J.~Love$^\textrm{\scriptsize 6}$,    
P.A.~Love$^\textrm{\scriptsize 87}$,    
J.J.~Lozano~Bahilo$^\textrm{\scriptsize 171}$,    
H.~Lu$^\textrm{\scriptsize 61a}$,    
M.~Lu$^\textrm{\scriptsize 58a}$,    
Y.J.~Lu$^\textrm{\scriptsize 62}$,    
H.J.~Lubatti$^\textrm{\scriptsize 145}$,    
C.~Luci$^\textrm{\scriptsize 70a,70b}$,    
A.~Lucotte$^\textrm{\scriptsize 56}$,    
C.~Luedtke$^\textrm{\scriptsize 50}$,    
F.~Luehring$^\textrm{\scriptsize 63}$,    
I.~Luise$^\textrm{\scriptsize 133}$,    
L.~Luminari$^\textrm{\scriptsize 70a}$,    
B.~Lund-Jensen$^\textrm{\scriptsize 151}$,    
M.S.~Lutz$^\textrm{\scriptsize 100}$,    
P.M.~Luzi$^\textrm{\scriptsize 133}$,    
D.~Lynn$^\textrm{\scriptsize 29}$,    
R.~Lysak$^\textrm{\scriptsize 138}$,    
E.~Lytken$^\textrm{\scriptsize 94}$,    
F.~Lyu$^\textrm{\scriptsize 15a}$,    
V.~Lyubushkin$^\textrm{\scriptsize 77}$,    
T.~Lyubushkina$^\textrm{\scriptsize 77}$,    
H.~Ma$^\textrm{\scriptsize 29}$,    
L.L.~Ma$^\textrm{\scriptsize 58b}$,    
Y.~Ma$^\textrm{\scriptsize 58b}$,    
G.~Maccarrone$^\textrm{\scriptsize 49}$,    
A.~Macchiolo$^\textrm{\scriptsize 113}$,    
C.M.~Macdonald$^\textrm{\scriptsize 146}$,    
J.~Machado~Miguens$^\textrm{\scriptsize 134,137b}$,    
D.~Madaffari$^\textrm{\scriptsize 171}$,    
R.~Madar$^\textrm{\scriptsize 37}$,    
W.F.~Mader$^\textrm{\scriptsize 46}$,    
N.~Madysa$^\textrm{\scriptsize 46}$,    
J.~Maeda$^\textrm{\scriptsize 80}$,    
K.~Maekawa$^\textrm{\scriptsize 160}$,    
S.~Maeland$^\textrm{\scriptsize 17}$,    
T.~Maeno$^\textrm{\scriptsize 29}$,    
M.~Maerker$^\textrm{\scriptsize 46}$,    
A.S.~Maevskiy$^\textrm{\scriptsize 111}$,    
V.~Magerl$^\textrm{\scriptsize 50}$,    
D.J.~Mahon$^\textrm{\scriptsize 38}$,    
C.~Maidantchik$^\textrm{\scriptsize 78b}$,    
T.~Maier$^\textrm{\scriptsize 112}$,    
A.~Maio$^\textrm{\scriptsize 137a,137b,137d}$,    
O.~Majersky$^\textrm{\scriptsize 28a}$,    
S.~Majewski$^\textrm{\scriptsize 128}$,    
Y.~Makida$^\textrm{\scriptsize 79}$,    
N.~Makovec$^\textrm{\scriptsize 129}$,    
B.~Malaescu$^\textrm{\scriptsize 133}$,    
Pa.~Malecki$^\textrm{\scriptsize 82}$,    
V.P.~Maleev$^\textrm{\scriptsize 135}$,    
F.~Malek$^\textrm{\scriptsize 56}$,    
U.~Mallik$^\textrm{\scriptsize 75}$,    
D.~Malon$^\textrm{\scriptsize 6}$,    
C.~Malone$^\textrm{\scriptsize 31}$,    
S.~Maltezos$^\textrm{\scriptsize 10}$,    
S.~Malyukov$^\textrm{\scriptsize 35}$,    
J.~Mamuzic$^\textrm{\scriptsize 171}$,    
G.~Mancini$^\textrm{\scriptsize 49}$,    
I.~Mandi\'{c}$^\textrm{\scriptsize 89}$,    
J.~Maneira$^\textrm{\scriptsize 137a}$,    
L.~Manhaes~de~Andrade~Filho$^\textrm{\scriptsize 78a}$,    
J.~Manjarres~Ramos$^\textrm{\scriptsize 46}$,    
K.H.~Mankinen$^\textrm{\scriptsize 94}$,    
A.~Mann$^\textrm{\scriptsize 112}$,    
A.~Manousos$^\textrm{\scriptsize 74}$,    
B.~Mansoulie$^\textrm{\scriptsize 142}$,    
S.~Manzoni$^\textrm{\scriptsize 66a,66b}$,    
A.~Marantis$^\textrm{\scriptsize 159}$,    
G.~Marceca$^\textrm{\scriptsize 30}$,    
L.~March$^\textrm{\scriptsize 52}$,    
L.~Marchese$^\textrm{\scriptsize 132}$,    
G.~Marchiori$^\textrm{\scriptsize 133}$,    
M.~Marcisovsky$^\textrm{\scriptsize 138}$,    
C.~Marcon$^\textrm{\scriptsize 94}$,    
C.A.~Marin~Tobon$^\textrm{\scriptsize 35}$,    
M.~Marjanovic$^\textrm{\scriptsize 37}$,    
F.~Marroquim$^\textrm{\scriptsize 78b}$,    
Z.~Marshall$^\textrm{\scriptsize 18}$,    
M.U.F~Martensson$^\textrm{\scriptsize 169}$,    
S.~Marti-Garcia$^\textrm{\scriptsize 171}$,    
C.B.~Martin$^\textrm{\scriptsize 123}$,    
T.A.~Martin$^\textrm{\scriptsize 175}$,    
V.J.~Martin$^\textrm{\scriptsize 48}$,    
B.~Martin~dit~Latour$^\textrm{\scriptsize 17}$,    
M.~Martinez$^\textrm{\scriptsize 14,z}$,    
V.I.~Martinez~Outschoorn$^\textrm{\scriptsize 100}$,    
S.~Martin-Haugh$^\textrm{\scriptsize 141}$,    
V.S.~Martoiu$^\textrm{\scriptsize 27b}$,    
A.C.~Martyniuk$^\textrm{\scriptsize 92}$,    
A.~Marzin$^\textrm{\scriptsize 35}$,    
L.~Masetti$^\textrm{\scriptsize 97}$,    
T.~Mashimo$^\textrm{\scriptsize 160}$,    
R.~Mashinistov$^\textrm{\scriptsize 108}$,    
J.~Masik$^\textrm{\scriptsize 98}$,    
A.L.~Maslennikov$^\textrm{\scriptsize 120b,120a}$,    
L.H.~Mason$^\textrm{\scriptsize 102}$,    
L.~Massa$^\textrm{\scriptsize 71a,71b}$,    
P.~Massarotti$^\textrm{\scriptsize 67a,67b}$,    
P.~Mastrandrea$^\textrm{\scriptsize 152}$,    
A.~Mastroberardino$^\textrm{\scriptsize 40b,40a}$,    
T.~Masubuchi$^\textrm{\scriptsize 160}$,    
P.~M\"attig$^\textrm{\scriptsize 24}$,    
J.~Maurer$^\textrm{\scriptsize 27b}$,    
B.~Ma\v{c}ek$^\textrm{\scriptsize 89}$,    
S.J.~Maxfield$^\textrm{\scriptsize 88}$,    
D.A.~Maximov$^\textrm{\scriptsize 120b,120a}$,    
R.~Mazini$^\textrm{\scriptsize 155}$,    
I.~Maznas$^\textrm{\scriptsize 159}$,    
S.M.~Mazza$^\textrm{\scriptsize 143}$,    
S.P.~Mc~Kee$^\textrm{\scriptsize 103}$,    
A.~McCarn$^\textrm{\scriptsize 41}$,    
T.G.~McCarthy$^\textrm{\scriptsize 113}$,    
L.I.~McClymont$^\textrm{\scriptsize 92}$,    
W.P.~McCormack$^\textrm{\scriptsize 18}$,    
E.F.~McDonald$^\textrm{\scriptsize 102}$,    
J.A.~Mcfayden$^\textrm{\scriptsize 35}$,    
G.~Mchedlidze$^\textrm{\scriptsize 51}$,    
M.A.~McKay$^\textrm{\scriptsize 41}$,    
K.D.~McLean$^\textrm{\scriptsize 173}$,    
S.J.~McMahon$^\textrm{\scriptsize 141}$,    
P.C.~McNamara$^\textrm{\scriptsize 102}$,    
C.J.~McNicol$^\textrm{\scriptsize 175}$,    
R.A.~McPherson$^\textrm{\scriptsize 173,ad}$,    
J.E.~Mdhluli$^\textrm{\scriptsize 32c}$,    
Z.A.~Meadows$^\textrm{\scriptsize 100}$,    
S.~Meehan$^\textrm{\scriptsize 145}$,    
T.M.~Megy$^\textrm{\scriptsize 50}$,    
S.~Mehlhase$^\textrm{\scriptsize 112}$,    
A.~Mehta$^\textrm{\scriptsize 88}$,    
T.~Meideck$^\textrm{\scriptsize 56}$,    
B.~Meirose$^\textrm{\scriptsize 42}$,    
D.~Melini$^\textrm{\scriptsize 171,h}$,    
B.R.~Mellado~Garcia$^\textrm{\scriptsize 32c}$,    
J.D.~Mellenthin$^\textrm{\scriptsize 51}$,    
M.~Melo$^\textrm{\scriptsize 28a}$,    
F.~Meloni$^\textrm{\scriptsize 44}$,    
A.~Melzer$^\textrm{\scriptsize 24}$,    
S.B.~Menary$^\textrm{\scriptsize 98}$,    
E.D.~Mendes~Gouveia$^\textrm{\scriptsize 137a}$,    
L.~Meng$^\textrm{\scriptsize 88}$,    
X.T.~Meng$^\textrm{\scriptsize 103}$,    
S.~Menke$^\textrm{\scriptsize 113}$,    
E.~Meoni$^\textrm{\scriptsize 40b,40a}$,    
S.~Mergelmeyer$^\textrm{\scriptsize 19}$,    
S.A.M.~Merkt$^\textrm{\scriptsize 136}$,    
C.~Merlassino$^\textrm{\scriptsize 20}$,    
P.~Mermod$^\textrm{\scriptsize 52}$,    
L.~Merola$^\textrm{\scriptsize 67a,67b}$,    
C.~Meroni$^\textrm{\scriptsize 66a}$,    
F.S.~Merritt$^\textrm{\scriptsize 36}$,    
A.~Messina$^\textrm{\scriptsize 70a,70b}$,    
J.~Metcalfe$^\textrm{\scriptsize 6}$,    
A.S.~Mete$^\textrm{\scriptsize 168}$,    
C.~Meyer$^\textrm{\scriptsize 63}$,    
J.~Meyer$^\textrm{\scriptsize 157}$,    
J-P.~Meyer$^\textrm{\scriptsize 142}$,    
H.~Meyer~Zu~Theenhausen$^\textrm{\scriptsize 59a}$,    
F.~Miano$^\textrm{\scriptsize 153}$,    
R.P.~Middleton$^\textrm{\scriptsize 141}$,    
L.~Mijovi\'{c}$^\textrm{\scriptsize 48}$,    
G.~Mikenberg$^\textrm{\scriptsize 177}$,    
M.~Mikestikova$^\textrm{\scriptsize 138}$,    
M.~Miku\v{z}$^\textrm{\scriptsize 89}$,    
M.~Milesi$^\textrm{\scriptsize 102}$,    
A.~Milic$^\textrm{\scriptsize 164}$,    
D.A.~Millar$^\textrm{\scriptsize 90}$,    
D.W.~Miller$^\textrm{\scriptsize 36}$,    
A.~Milov$^\textrm{\scriptsize 177}$,    
D.A.~Milstead$^\textrm{\scriptsize 43a,43b}$,    
R.A.~Mina$^\textrm{\scriptsize 150,r}$,    
A.A.~Minaenko$^\textrm{\scriptsize 121}$,    
M.~Mi\~nano~Moya$^\textrm{\scriptsize 171}$,    
I.A.~Minashvili$^\textrm{\scriptsize 156b}$,    
A.I.~Mincer$^\textrm{\scriptsize 122}$,    
B.~Mindur$^\textrm{\scriptsize 81a}$,    
M.~Mineev$^\textrm{\scriptsize 77}$,    
Y.~Minegishi$^\textrm{\scriptsize 160}$,    
Y.~Ming$^\textrm{\scriptsize 178}$,    
L.M.~Mir$^\textrm{\scriptsize 14}$,    
A.~Mirto$^\textrm{\scriptsize 65a,65b}$,    
K.P.~Mistry$^\textrm{\scriptsize 134}$,    
T.~Mitani$^\textrm{\scriptsize 176}$,    
J.~Mitrevski$^\textrm{\scriptsize 112}$,    
V.A.~Mitsou$^\textrm{\scriptsize 171}$,    
M.~Mittal$^\textrm{\scriptsize 58c}$,    
A.~Miucci$^\textrm{\scriptsize 20}$,    
P.S.~Miyagawa$^\textrm{\scriptsize 146}$,    
A.~Mizukami$^\textrm{\scriptsize 79}$,    
J.U.~Mj\"ornmark$^\textrm{\scriptsize 94}$,    
T.~Mkrtchyan$^\textrm{\scriptsize 181}$,    
M.~Mlynarikova$^\textrm{\scriptsize 140}$,    
T.~Moa$^\textrm{\scriptsize 43a,43b}$,    
K.~Mochizuki$^\textrm{\scriptsize 107}$,    
P.~Mogg$^\textrm{\scriptsize 50}$,    
S.~Mohapatra$^\textrm{\scriptsize 38}$,    
S.~Molander$^\textrm{\scriptsize 43a,43b}$,    
R.~Moles-Valls$^\textrm{\scriptsize 24}$,    
M.C.~Mondragon$^\textrm{\scriptsize 104}$,    
K.~M\"onig$^\textrm{\scriptsize 44}$,    
J.~Monk$^\textrm{\scriptsize 39}$,    
E.~Monnier$^\textrm{\scriptsize 99}$,    
A.~Montalbano$^\textrm{\scriptsize 149}$,    
J.~Montejo~Berlingen$^\textrm{\scriptsize 35}$,    
F.~Monticelli$^\textrm{\scriptsize 86}$,    
S.~Monzani$^\textrm{\scriptsize 66a}$,    
N.~Morange$^\textrm{\scriptsize 129}$,    
D.~Moreno$^\textrm{\scriptsize 22}$,    
M.~Moreno~Ll\'acer$^\textrm{\scriptsize 35}$,    
P.~Morettini$^\textrm{\scriptsize 53b}$,    
M.~Morgenstern$^\textrm{\scriptsize 118}$,    
S.~Morgenstern$^\textrm{\scriptsize 46}$,    
D.~Mori$^\textrm{\scriptsize 149}$,    
M.~Morii$^\textrm{\scriptsize 57}$,    
M.~Morinaga$^\textrm{\scriptsize 176}$,    
V.~Morisbak$^\textrm{\scriptsize 131}$,    
A.K.~Morley$^\textrm{\scriptsize 35}$,    
G.~Mornacchi$^\textrm{\scriptsize 35}$,    
A.P.~Morris$^\textrm{\scriptsize 92}$,    
J.D.~Morris$^\textrm{\scriptsize 90}$,    
L.~Morvaj$^\textrm{\scriptsize 152}$,    
P.~Moschovakos$^\textrm{\scriptsize 10}$,    
M.~Mosidze$^\textrm{\scriptsize 156b}$,    
H.J.~Moss$^\textrm{\scriptsize 146}$,    
J.~Moss$^\textrm{\scriptsize 150,o}$,    
K.~Motohashi$^\textrm{\scriptsize 162}$,    
R.~Mount$^\textrm{\scriptsize 150}$,    
E.~Mountricha$^\textrm{\scriptsize 35}$,    
E.J.W.~Moyse$^\textrm{\scriptsize 100}$,    
S.~Muanza$^\textrm{\scriptsize 99}$,    
F.~Mueller$^\textrm{\scriptsize 113}$,    
J.~Mueller$^\textrm{\scriptsize 136}$,    
R.S.P.~Mueller$^\textrm{\scriptsize 112}$,    
D.~Muenstermann$^\textrm{\scriptsize 87}$,    
G.A.~Mullier$^\textrm{\scriptsize 94}$,    
F.J.~Munoz~Sanchez$^\textrm{\scriptsize 98}$,    
P.~Murin$^\textrm{\scriptsize 28b}$,    
W.J.~Murray$^\textrm{\scriptsize 175,141}$,    
A.~Murrone$^\textrm{\scriptsize 66a,66b}$,    
M.~Mu\v{s}kinja$^\textrm{\scriptsize 89}$,    
C.~Mwewa$^\textrm{\scriptsize 32a}$,    
A.G.~Myagkov$^\textrm{\scriptsize 121,ao}$,    
J.~Myers$^\textrm{\scriptsize 128}$,    
M.~Myska$^\textrm{\scriptsize 139}$,    
B.P.~Nachman$^\textrm{\scriptsize 18}$,    
O.~Nackenhorst$^\textrm{\scriptsize 45}$,    
K.~Nagai$^\textrm{\scriptsize 132}$,    
K.~Nagano$^\textrm{\scriptsize 79}$,    
Y.~Nagasaka$^\textrm{\scriptsize 60}$,    
M.~Nagel$^\textrm{\scriptsize 50}$,    
E.~Nagy$^\textrm{\scriptsize 99}$,    
A.M.~Nairz$^\textrm{\scriptsize 35}$,    
Y.~Nakahama$^\textrm{\scriptsize 115}$,    
K.~Nakamura$^\textrm{\scriptsize 79}$,    
T.~Nakamura$^\textrm{\scriptsize 160}$,    
I.~Nakano$^\textrm{\scriptsize 124}$,    
H.~Nanjo$^\textrm{\scriptsize 130}$,    
F.~Napolitano$^\textrm{\scriptsize 59a}$,    
R.F.~Naranjo~Garcia$^\textrm{\scriptsize 44}$,    
R.~Narayan$^\textrm{\scriptsize 11}$,    
D.I.~Narrias~Villar$^\textrm{\scriptsize 59a}$,    
I.~Naryshkin$^\textrm{\scriptsize 135}$,    
T.~Naumann$^\textrm{\scriptsize 44}$,    
G.~Navarro$^\textrm{\scriptsize 22}$,    
R.~Nayyar$^\textrm{\scriptsize 7}$,    
H.A.~Neal$^\textrm{\scriptsize 103,*}$,    
P.Y.~Nechaeva$^\textrm{\scriptsize 108}$,    
T.J.~Neep$^\textrm{\scriptsize 142}$,    
A.~Negri$^\textrm{\scriptsize 68a,68b}$,    
M.~Negrini$^\textrm{\scriptsize 23b}$,    
S.~Nektarijevic$^\textrm{\scriptsize 117}$,    
C.~Nellist$^\textrm{\scriptsize 51}$,    
M.E.~Nelson$^\textrm{\scriptsize 132}$,    
S.~Nemecek$^\textrm{\scriptsize 138}$,    
P.~Nemethy$^\textrm{\scriptsize 122}$,    
M.~Nessi$^\textrm{\scriptsize 35,f}$,    
M.S.~Neubauer$^\textrm{\scriptsize 170}$,    
M.~Neumann$^\textrm{\scriptsize 179}$,    
P.R.~Newman$^\textrm{\scriptsize 21}$,    
T.Y.~Ng$^\textrm{\scriptsize 61c}$,    
Y.S.~Ng$^\textrm{\scriptsize 19}$,    
Y.W.Y.~Ng$^\textrm{\scriptsize 168}$,    
H.D.N.~Nguyen$^\textrm{\scriptsize 99}$,    
T.~Nguyen~Manh$^\textrm{\scriptsize 107}$,    
E.~Nibigira$^\textrm{\scriptsize 37}$,    
R.B.~Nickerson$^\textrm{\scriptsize 132}$,    
R.~Nicolaidou$^\textrm{\scriptsize 142}$,    
D.S.~Nielsen$^\textrm{\scriptsize 39}$,    
J.~Nielsen$^\textrm{\scriptsize 143}$,    
N.~Nikiforou$^\textrm{\scriptsize 11}$,    
V.~Nikolaenko$^\textrm{\scriptsize 121,ao}$,    
I.~Nikolic-Audit$^\textrm{\scriptsize 133}$,    
K.~Nikolopoulos$^\textrm{\scriptsize 21}$,    
P.~Nilsson$^\textrm{\scriptsize 29}$,    
H.R.~Nindhito$^\textrm{\scriptsize 52}$,    
Y.~Ninomiya$^\textrm{\scriptsize 79}$,    
A.~Nisati$^\textrm{\scriptsize 70a}$,    
N.~Nishu$^\textrm{\scriptsize 58c}$,    
R.~Nisius$^\textrm{\scriptsize 113}$,    
I.~Nitsche$^\textrm{\scriptsize 45}$,    
T.~Nitta$^\textrm{\scriptsize 176}$,    
T.~Nobe$^\textrm{\scriptsize 160}$,    
Y.~Noguchi$^\textrm{\scriptsize 83}$,    
M.~Nomachi$^\textrm{\scriptsize 130}$,    
I.~Nomidis$^\textrm{\scriptsize 133}$,    
M.A.~Nomura$^\textrm{\scriptsize 29}$,    
T.~Nooney$^\textrm{\scriptsize 90}$,    
M.~Nordberg$^\textrm{\scriptsize 35}$,    
N.~Norjoharuddeen$^\textrm{\scriptsize 132}$,    
T.~Novak$^\textrm{\scriptsize 89}$,    
O.~Novgorodova$^\textrm{\scriptsize 46}$,    
R.~Novotny$^\textrm{\scriptsize 139}$,    
L.~Nozka$^\textrm{\scriptsize 127}$,    
K.~Ntekas$^\textrm{\scriptsize 168}$,    
E.~Nurse$^\textrm{\scriptsize 92}$,    
F.~Nuti$^\textrm{\scriptsize 102}$,    
F.G.~Oakham$^\textrm{\scriptsize 33,av}$,    
H.~Oberlack$^\textrm{\scriptsize 113}$,    
J.~Ocariz$^\textrm{\scriptsize 133}$,    
A.~Ochi$^\textrm{\scriptsize 80}$,    
I.~Ochoa$^\textrm{\scriptsize 38}$,    
J.P.~Ochoa-Ricoux$^\textrm{\scriptsize 144a}$,    
K.~O'Connor$^\textrm{\scriptsize 26}$,    
S.~Oda$^\textrm{\scriptsize 85}$,    
S.~Odaka$^\textrm{\scriptsize 79}$,    
S.~Oerdek$^\textrm{\scriptsize 51}$,    
A.~Oh$^\textrm{\scriptsize 98}$,    
S.H.~Oh$^\textrm{\scriptsize 47}$,    
C.C.~Ohm$^\textrm{\scriptsize 151}$,    
H.~Oide$^\textrm{\scriptsize 53b,53a}$,    
M.L.~Ojeda$^\textrm{\scriptsize 164}$,    
H.~Okawa$^\textrm{\scriptsize 166}$,    
Y.~Okazaki$^\textrm{\scriptsize 83}$,    
Y.~Okumura$^\textrm{\scriptsize 160}$,    
T.~Okuyama$^\textrm{\scriptsize 79}$,    
A.~Olariu$^\textrm{\scriptsize 27b}$,    
L.F.~Oleiro~Seabra$^\textrm{\scriptsize 137a}$,    
S.A.~Olivares~Pino$^\textrm{\scriptsize 144a}$,    
D.~Oliveira~Damazio$^\textrm{\scriptsize 29}$,    
J.L.~Oliver$^\textrm{\scriptsize 1}$,    
M.J.R.~Olsson$^\textrm{\scriptsize 36}$,    
A.~Olszewski$^\textrm{\scriptsize 82}$,    
J.~Olszowska$^\textrm{\scriptsize 82}$,    
D.C.~O'Neil$^\textrm{\scriptsize 149}$,    
A.~Onofre$^\textrm{\scriptsize 137a,137e}$,    
K.~Onogi$^\textrm{\scriptsize 115}$,    
P.U.E.~Onyisi$^\textrm{\scriptsize 11}$,    
H.~Oppen$^\textrm{\scriptsize 131}$,    
M.J.~Oreglia$^\textrm{\scriptsize 36}$,    
G.E.~Orellana$^\textrm{\scriptsize 86}$,    
Y.~Oren$^\textrm{\scriptsize 158}$,    
D.~Orestano$^\textrm{\scriptsize 72a,72b}$,    
E.C.~Orgill$^\textrm{\scriptsize 98}$,    
N.~Orlando$^\textrm{\scriptsize 61b}$,    
A.A.~O'Rourke$^\textrm{\scriptsize 44}$,    
R.S.~Orr$^\textrm{\scriptsize 164}$,    
B.~Osculati$^\textrm{\scriptsize 53b,53a,*}$,    
V.~O'Shea$^\textrm{\scriptsize 55}$,    
R.~Ospanov$^\textrm{\scriptsize 58a}$,    
G.~Otero~y~Garzon$^\textrm{\scriptsize 30}$,    
H.~Otono$^\textrm{\scriptsize 85}$,    
M.~Ouchrif$^\textrm{\scriptsize 34d}$,    
F.~Ould-Saada$^\textrm{\scriptsize 131}$,    
A.~Ouraou$^\textrm{\scriptsize 142}$,    
Q.~Ouyang$^\textrm{\scriptsize 15a}$,    
M.~Owen$^\textrm{\scriptsize 55}$,    
R.E.~Owen$^\textrm{\scriptsize 21}$,    
V.E.~Ozcan$^\textrm{\scriptsize 12c}$,    
N.~Ozturk$^\textrm{\scriptsize 8}$,    
J.~Pacalt$^\textrm{\scriptsize 127}$,    
H.A.~Pacey$^\textrm{\scriptsize 31}$,    
K.~Pachal$^\textrm{\scriptsize 149}$,    
A.~Pacheco~Pages$^\textrm{\scriptsize 14}$,    
L.~Pacheco~Rodriguez$^\textrm{\scriptsize 142}$,    
C.~Padilla~Aranda$^\textrm{\scriptsize 14}$,    
S.~Pagan~Griso$^\textrm{\scriptsize 18}$,    
M.~Paganini$^\textrm{\scriptsize 180}$,    
G.~Palacino$^\textrm{\scriptsize 63}$,    
S.~Palazzo$^\textrm{\scriptsize 48}$,    
S.~Palestini$^\textrm{\scriptsize 35}$,    
M.~Palka$^\textrm{\scriptsize 81b}$,    
D.~Pallin$^\textrm{\scriptsize 37}$,    
I.~Panagoulias$^\textrm{\scriptsize 10}$,    
C.E.~Pandini$^\textrm{\scriptsize 35}$,    
J.G.~Panduro~Vazquez$^\textrm{\scriptsize 91}$,    
P.~Pani$^\textrm{\scriptsize 35}$,    
G.~Panizzo$^\textrm{\scriptsize 64a,64c}$,    
L.~Paolozzi$^\textrm{\scriptsize 52}$,    
T.D.~Papadopoulou$^\textrm{\scriptsize 10}$,    
K.~Papageorgiou$^\textrm{\scriptsize 9,k}$,    
A.~Paramonov$^\textrm{\scriptsize 6}$,    
D.~Paredes~Hernandez$^\textrm{\scriptsize 61b}$,    
S.R.~Paredes~Saenz$^\textrm{\scriptsize 132}$,    
B.~Parida$^\textrm{\scriptsize 163}$,    
T.H.~Park$^\textrm{\scriptsize 33}$,    
A.J.~Parker$^\textrm{\scriptsize 87}$,    
K.A.~Parker$^\textrm{\scriptsize 44}$,    
M.A.~Parker$^\textrm{\scriptsize 31}$,    
F.~Parodi$^\textrm{\scriptsize 53b,53a}$,    
J.A.~Parsons$^\textrm{\scriptsize 38}$,    
U.~Parzefall$^\textrm{\scriptsize 50}$,    
V.R.~Pascuzzi$^\textrm{\scriptsize 164}$,    
J.M.P.~Pasner$^\textrm{\scriptsize 143}$,    
E.~Pasqualucci$^\textrm{\scriptsize 70a}$,    
S.~Passaggio$^\textrm{\scriptsize 53b}$,    
F.~Pastore$^\textrm{\scriptsize 91}$,    
P.~Pasuwan$^\textrm{\scriptsize 43a,43b}$,    
S.~Pataraia$^\textrm{\scriptsize 97}$,    
J.R.~Pater$^\textrm{\scriptsize 98}$,    
A.~Pathak$^\textrm{\scriptsize 178,l}$,    
T.~Pauly$^\textrm{\scriptsize 35}$,    
B.~Pearson$^\textrm{\scriptsize 113}$,    
M.~Pedersen$^\textrm{\scriptsize 131}$,    
L.~Pedraza~Diaz$^\textrm{\scriptsize 117}$,    
R.~Pedro$^\textrm{\scriptsize 137a,137b}$,    
S.V.~Peleganchuk$^\textrm{\scriptsize 120b,120a}$,    
O.~Penc$^\textrm{\scriptsize 138}$,    
C.~Peng$^\textrm{\scriptsize 15d}$,    
H.~Peng$^\textrm{\scriptsize 58a}$,    
B.S.~Peralva$^\textrm{\scriptsize 78a}$,    
M.M.~Perego$^\textrm{\scriptsize 129}$,    
A.P.~Pereira~Peixoto$^\textrm{\scriptsize 137a}$,    
D.V.~Perepelitsa$^\textrm{\scriptsize 29}$,    
F.~Peri$^\textrm{\scriptsize 19}$,    
L.~Perini$^\textrm{\scriptsize 66a,66b}$,    
H.~Pernegger$^\textrm{\scriptsize 35}$,    
S.~Perrella$^\textrm{\scriptsize 67a,67b}$,    
V.D.~Peshekhonov$^\textrm{\scriptsize 77,*}$,    
K.~Peters$^\textrm{\scriptsize 44}$,    
R.F.Y.~Peters$^\textrm{\scriptsize 98}$,    
B.A.~Petersen$^\textrm{\scriptsize 35}$,    
T.C.~Petersen$^\textrm{\scriptsize 39}$,    
E.~Petit$^\textrm{\scriptsize 56}$,    
A.~Petridis$^\textrm{\scriptsize 1}$,    
C.~Petridou$^\textrm{\scriptsize 159}$,    
P.~Petroff$^\textrm{\scriptsize 129}$,    
M.~Petrov$^\textrm{\scriptsize 132}$,    
F.~Petrucci$^\textrm{\scriptsize 72a,72b}$,    
M.~Pettee$^\textrm{\scriptsize 180}$,    
N.E.~Pettersson$^\textrm{\scriptsize 100}$,    
A.~Peyaud$^\textrm{\scriptsize 142}$,    
R.~Pezoa$^\textrm{\scriptsize 144b}$,    
T.~Pham$^\textrm{\scriptsize 102}$,    
F.H.~Phillips$^\textrm{\scriptsize 104}$,    
P.W.~Phillips$^\textrm{\scriptsize 141}$,    
M.W.~Phipps$^\textrm{\scriptsize 170}$,    
G.~Piacquadio$^\textrm{\scriptsize 152}$,    
E.~Pianori$^\textrm{\scriptsize 18}$,    
A.~Picazio$^\textrm{\scriptsize 100}$,    
R.H.~Pickles$^\textrm{\scriptsize 98}$,    
R.~Piegaia$^\textrm{\scriptsize 30}$,    
J.E.~Pilcher$^\textrm{\scriptsize 36}$,    
A.D.~Pilkington$^\textrm{\scriptsize 98}$,    
M.~Pinamonti$^\textrm{\scriptsize 71a,71b}$,    
J.L.~Pinfold$^\textrm{\scriptsize 3}$,    
M.~Pitt$^\textrm{\scriptsize 177}$,    
L.~Pizzimento$^\textrm{\scriptsize 71a,71b}$,    
M.-A.~Pleier$^\textrm{\scriptsize 29}$,    
V.~Pleskot$^\textrm{\scriptsize 140}$,    
E.~Plotnikova$^\textrm{\scriptsize 77}$,    
D.~Pluth$^\textrm{\scriptsize 76}$,    
P.~Podberezko$^\textrm{\scriptsize 120b,120a}$,    
R.~Poettgen$^\textrm{\scriptsize 94}$,    
R.~Poggi$^\textrm{\scriptsize 52}$,    
L.~Poggioli$^\textrm{\scriptsize 129}$,    
I.~Pogrebnyak$^\textrm{\scriptsize 104}$,    
D.~Pohl$^\textrm{\scriptsize 24}$,    
I.~Pokharel$^\textrm{\scriptsize 51}$,    
G.~Polesello$^\textrm{\scriptsize 68a}$,    
A.~Poley$^\textrm{\scriptsize 18}$,    
A.~Policicchio$^\textrm{\scriptsize 70a,70b}$,    
R.~Polifka$^\textrm{\scriptsize 35}$,    
A.~Polini$^\textrm{\scriptsize 23b}$,    
C.S.~Pollard$^\textrm{\scriptsize 44}$,    
V.~Polychronakos$^\textrm{\scriptsize 29}$,    
D.~Ponomarenko$^\textrm{\scriptsize 110}$,    
L.~Pontecorvo$^\textrm{\scriptsize 35}$,    
G.A.~Popeneciu$^\textrm{\scriptsize 27d}$,    
D.M.~Portillo~Quintero$^\textrm{\scriptsize 133}$,    
S.~Pospisil$^\textrm{\scriptsize 139}$,    
K.~Potamianos$^\textrm{\scriptsize 44}$,    
I.N.~Potrap$^\textrm{\scriptsize 77}$,    
C.J.~Potter$^\textrm{\scriptsize 31}$,    
H.~Potti$^\textrm{\scriptsize 11}$,    
T.~Poulsen$^\textrm{\scriptsize 94}$,    
J.~Poveda$^\textrm{\scriptsize 35}$,    
T.D.~Powell$^\textrm{\scriptsize 146}$,    
M.E.~Pozo~Astigarraga$^\textrm{\scriptsize 35}$,    
P.~Pralavorio$^\textrm{\scriptsize 99}$,    
S.~Prell$^\textrm{\scriptsize 76}$,    
D.~Price$^\textrm{\scriptsize 98}$,    
M.~Primavera$^\textrm{\scriptsize 65a}$,    
S.~Prince$^\textrm{\scriptsize 101}$,    
M.L.~Proffitt$^\textrm{\scriptsize 145}$,    
N.~Proklova$^\textrm{\scriptsize 110}$,    
K.~Prokofiev$^\textrm{\scriptsize 61c}$,    
F.~Prokoshin$^\textrm{\scriptsize 144b}$,    
S.~Protopopescu$^\textrm{\scriptsize 29}$,    
J.~Proudfoot$^\textrm{\scriptsize 6}$,    
M.~Przybycien$^\textrm{\scriptsize 81a}$,    
A.~Puri$^\textrm{\scriptsize 170}$,    
P.~Puzo$^\textrm{\scriptsize 129}$,    
J.~Qian$^\textrm{\scriptsize 103}$,    
Y.~Qin$^\textrm{\scriptsize 98}$,    
A.~Quadt$^\textrm{\scriptsize 51}$,    
M.~Queitsch-Maitland$^\textrm{\scriptsize 44}$,    
A.~Qureshi$^\textrm{\scriptsize 1}$,    
P.~Rados$^\textrm{\scriptsize 102}$,    
F.~Ragusa$^\textrm{\scriptsize 66a,66b}$,    
G.~Rahal$^\textrm{\scriptsize 95}$,    
J.A.~Raine$^\textrm{\scriptsize 52}$,    
S.~Rajagopalan$^\textrm{\scriptsize 29}$,    
A.~Ramirez~Morales$^\textrm{\scriptsize 90}$,    
K.~Ran$^\textrm{\scriptsize 15a}$,    
T.~Rashid$^\textrm{\scriptsize 129}$,    
S.~Raspopov$^\textrm{\scriptsize 5}$,    
M.G.~Ratti$^\textrm{\scriptsize 66a,66b}$,    
D.M.~Rauch$^\textrm{\scriptsize 44}$,    
F.~Rauscher$^\textrm{\scriptsize 112}$,    
S.~Rave$^\textrm{\scriptsize 97}$,    
B.~Ravina$^\textrm{\scriptsize 146}$,    
I.~Ravinovich$^\textrm{\scriptsize 177}$,    
J.H.~Rawling$^\textrm{\scriptsize 98}$,    
M.~Raymond$^\textrm{\scriptsize 35}$,    
A.L.~Read$^\textrm{\scriptsize 131}$,    
N.P.~Readioff$^\textrm{\scriptsize 56}$,    
M.~Reale$^\textrm{\scriptsize 65a,65b}$,    
D.M.~Rebuzzi$^\textrm{\scriptsize 68a,68b}$,    
A.~Redelbach$^\textrm{\scriptsize 174}$,    
G.~Redlinger$^\textrm{\scriptsize 29}$,    
R.~Reece$^\textrm{\scriptsize 143}$,    
R.G.~Reed$^\textrm{\scriptsize 32c}$,    
K.~Reeves$^\textrm{\scriptsize 42}$,    
L.~Rehnisch$^\textrm{\scriptsize 19}$,    
J.~Reichert$^\textrm{\scriptsize 134}$,    
D.~Reikher$^\textrm{\scriptsize 158}$,    
A.~Reiss$^\textrm{\scriptsize 97}$,    
A.~Rej$^\textrm{\scriptsize 148}$,    
C.~Rembser$^\textrm{\scriptsize 35}$,    
H.~Ren$^\textrm{\scriptsize 15d}$,    
M.~Rescigno$^\textrm{\scriptsize 70a}$,    
S.~Resconi$^\textrm{\scriptsize 66a}$,    
E.D.~Resseguie$^\textrm{\scriptsize 134}$,    
S.~Rettie$^\textrm{\scriptsize 172}$,    
E.~Reynolds$^\textrm{\scriptsize 21}$,    
O.L.~Rezanova$^\textrm{\scriptsize 120b,120a}$,    
P.~Reznicek$^\textrm{\scriptsize 140}$,    
E.~Ricci$^\textrm{\scriptsize 73a,73b}$,    
R.~Richter$^\textrm{\scriptsize 113}$,    
S.~Richter$^\textrm{\scriptsize 44}$,    
E.~Richter-Was$^\textrm{\scriptsize 81b}$,    
O.~Ricken$^\textrm{\scriptsize 24}$,    
M.~Ridel$^\textrm{\scriptsize 133}$,    
P.~Rieck$^\textrm{\scriptsize 113}$,    
C.J.~Riegel$^\textrm{\scriptsize 179}$,    
O.~Rifki$^\textrm{\scriptsize 44}$,    
M.~Rijssenbeek$^\textrm{\scriptsize 152}$,    
A.~Rimoldi$^\textrm{\scriptsize 68a,68b}$,    
M.~Rimoldi$^\textrm{\scriptsize 20}$,    
L.~Rinaldi$^\textrm{\scriptsize 23b}$,    
G.~Ripellino$^\textrm{\scriptsize 151}$,    
B.~Risti\'{c}$^\textrm{\scriptsize 87}$,    
E.~Ritsch$^\textrm{\scriptsize 35}$,    
I.~Riu$^\textrm{\scriptsize 14}$,    
J.C.~Rivera~Vergara$^\textrm{\scriptsize 144a}$,    
F.~Rizatdinova$^\textrm{\scriptsize 126}$,    
E.~Rizvi$^\textrm{\scriptsize 90}$,    
C.~Rizzi$^\textrm{\scriptsize 14}$,    
R.T.~Roberts$^\textrm{\scriptsize 98}$,    
S.H.~Robertson$^\textrm{\scriptsize 101,ad}$,    
D.~Robinson$^\textrm{\scriptsize 31}$,    
J.E.M.~Robinson$^\textrm{\scriptsize 44}$,    
A.~Robson$^\textrm{\scriptsize 55}$,    
E.~Rocco$^\textrm{\scriptsize 97}$,    
C.~Roda$^\textrm{\scriptsize 69a,69b}$,    
Y.~Rodina$^\textrm{\scriptsize 99}$,    
S.~Rodriguez~Bosca$^\textrm{\scriptsize 171}$,    
A.~Rodriguez~Perez$^\textrm{\scriptsize 14}$,    
D.~Rodriguez~Rodriguez$^\textrm{\scriptsize 171}$,    
A.M.~Rodr\'iguez~Vera$^\textrm{\scriptsize 165b}$,    
S.~Roe$^\textrm{\scriptsize 35}$,    
C.S.~Rogan$^\textrm{\scriptsize 57}$,    
O.~R{\o}hne$^\textrm{\scriptsize 131}$,    
R.~R\"ohrig$^\textrm{\scriptsize 113}$,    
C.P.A.~Roland$^\textrm{\scriptsize 63}$,    
J.~Roloff$^\textrm{\scriptsize 57}$,    
A.~Romaniouk$^\textrm{\scriptsize 110}$,    
M.~Romano$^\textrm{\scriptsize 23b,23a}$,    
N.~Rompotis$^\textrm{\scriptsize 88}$,    
M.~Ronzani$^\textrm{\scriptsize 122}$,    
L.~Roos$^\textrm{\scriptsize 133}$,    
S.~Rosati$^\textrm{\scriptsize 70a}$,    
K.~Rosbach$^\textrm{\scriptsize 50}$,    
N-A.~Rosien$^\textrm{\scriptsize 51}$,    
B.J.~Rosser$^\textrm{\scriptsize 134}$,    
E.~Rossi$^\textrm{\scriptsize 44}$,    
E.~Rossi$^\textrm{\scriptsize 72a,72b}$,    
E.~Rossi$^\textrm{\scriptsize 67a,67b}$,    
L.P.~Rossi$^\textrm{\scriptsize 53b}$,    
L.~Rossini$^\textrm{\scriptsize 66a,66b}$,    
J.H.N.~Rosten$^\textrm{\scriptsize 31}$,    
R.~Rosten$^\textrm{\scriptsize 14}$,    
M.~Rotaru$^\textrm{\scriptsize 27b}$,    
J.~Rothberg$^\textrm{\scriptsize 145}$,    
D.~Rousseau$^\textrm{\scriptsize 129}$,    
D.~Roy$^\textrm{\scriptsize 32c}$,    
A.~Rozanov$^\textrm{\scriptsize 99}$,    
Y.~Rozen$^\textrm{\scriptsize 157}$,    
X.~Ruan$^\textrm{\scriptsize 32c}$,    
F.~Rubbo$^\textrm{\scriptsize 150}$,    
F.~R\"uhr$^\textrm{\scriptsize 50}$,    
A.~Ruiz-Martinez$^\textrm{\scriptsize 171}$,    
Z.~Rurikova$^\textrm{\scriptsize 50}$,    
N.A.~Rusakovich$^\textrm{\scriptsize 77}$,    
H.L.~Russell$^\textrm{\scriptsize 101}$,    
J.P.~Rutherfoord$^\textrm{\scriptsize 7}$,    
E.M.~R{\"u}ttinger$^\textrm{\scriptsize 44,m}$,    
Y.F.~Ryabov$^\textrm{\scriptsize 135}$,    
M.~Rybar$^\textrm{\scriptsize 38}$,    
G.~Rybkin$^\textrm{\scriptsize 129}$,    
S.~Ryu$^\textrm{\scriptsize 6}$,    
A.~Ryzhov$^\textrm{\scriptsize 121}$,    
G.F.~Rzehorz$^\textrm{\scriptsize 51}$,    
P.~Sabatini$^\textrm{\scriptsize 51}$,    
G.~Sabato$^\textrm{\scriptsize 118}$,    
S.~Sacerdoti$^\textrm{\scriptsize 129}$,    
H.F-W.~Sadrozinski$^\textrm{\scriptsize 143}$,    
R.~Sadykov$^\textrm{\scriptsize 77}$,    
F.~Safai~Tehrani$^\textrm{\scriptsize 70a}$,    
P.~Saha$^\textrm{\scriptsize 119}$,    
M.~Sahinsoy$^\textrm{\scriptsize 59a}$,    
A.~Sahu$^\textrm{\scriptsize 179}$,    
M.~Saimpert$^\textrm{\scriptsize 44}$,    
M.~Saito$^\textrm{\scriptsize 160}$,    
T.~Saito$^\textrm{\scriptsize 160}$,    
H.~Sakamoto$^\textrm{\scriptsize 160}$,    
A.~Sakharov$^\textrm{\scriptsize 122,an}$,    
D.~Salamani$^\textrm{\scriptsize 52}$,    
G.~Salamanna$^\textrm{\scriptsize 72a,72b}$,    
J.E.~Salazar~Loyola$^\textrm{\scriptsize 144b}$,    
P.H.~Sales~De~Bruin$^\textrm{\scriptsize 169}$,    
D.~Salihagic$^\textrm{\scriptsize 113}$,    
A.~Salnikov$^\textrm{\scriptsize 150}$,    
J.~Salt$^\textrm{\scriptsize 171}$,    
D.~Salvatore$^\textrm{\scriptsize 40b,40a}$,    
F.~Salvatore$^\textrm{\scriptsize 153}$,    
A.~Salvucci$^\textrm{\scriptsize 61a,61b,61c}$,    
A.~Salzburger$^\textrm{\scriptsize 35}$,    
J.~Samarati$^\textrm{\scriptsize 35}$,    
D.~Sammel$^\textrm{\scriptsize 50}$,    
D.~Sampsonidis$^\textrm{\scriptsize 159}$,    
D.~Sampsonidou$^\textrm{\scriptsize 159}$,    
J.~S\'anchez$^\textrm{\scriptsize 171}$,    
A.~Sanchez~Pineda$^\textrm{\scriptsize 64a,64c}$,    
H.~Sandaker$^\textrm{\scriptsize 131}$,    
C.O.~Sander$^\textrm{\scriptsize 44}$,    
M.~Sandhoff$^\textrm{\scriptsize 179}$,    
C.~Sandoval$^\textrm{\scriptsize 22}$,    
D.P.C.~Sankey$^\textrm{\scriptsize 141}$,    
M.~Sannino$^\textrm{\scriptsize 53b,53a}$,    
Y.~Sano$^\textrm{\scriptsize 115}$,    
A.~Sansoni$^\textrm{\scriptsize 49}$,    
C.~Santoni$^\textrm{\scriptsize 37}$,    
H.~Santos$^\textrm{\scriptsize 137a}$,    
I.~Santoyo~Castillo$^\textrm{\scriptsize 153}$,    
A.~Santra$^\textrm{\scriptsize 171}$,    
A.~Sapronov$^\textrm{\scriptsize 77}$,    
J.G.~Saraiva$^\textrm{\scriptsize 137a,137d}$,    
O.~Sasaki$^\textrm{\scriptsize 79}$,    
K.~Sato$^\textrm{\scriptsize 166}$,    
E.~Sauvan$^\textrm{\scriptsize 5}$,    
P.~Savard$^\textrm{\scriptsize 164,av}$,    
N.~Savic$^\textrm{\scriptsize 113}$,    
R.~Sawada$^\textrm{\scriptsize 160}$,    
C.~Sawyer$^\textrm{\scriptsize 141}$,    
L.~Sawyer$^\textrm{\scriptsize 93,al}$,    
C.~Sbarra$^\textrm{\scriptsize 23b}$,    
A.~Sbrizzi$^\textrm{\scriptsize 23a}$,    
T.~Scanlon$^\textrm{\scriptsize 92}$,    
J.~Schaarschmidt$^\textrm{\scriptsize 145}$,    
P.~Schacht$^\textrm{\scriptsize 113}$,    
B.M.~Schachtner$^\textrm{\scriptsize 112}$,    
D.~Schaefer$^\textrm{\scriptsize 36}$,    
L.~Schaefer$^\textrm{\scriptsize 134}$,    
J.~Schaeffer$^\textrm{\scriptsize 97}$,    
S.~Schaepe$^\textrm{\scriptsize 35}$,    
U.~Sch\"afer$^\textrm{\scriptsize 97}$,    
A.C.~Schaffer$^\textrm{\scriptsize 129}$,    
D.~Schaile$^\textrm{\scriptsize 112}$,    
R.D.~Schamberger$^\textrm{\scriptsize 152}$,    
N.~Scharmberg$^\textrm{\scriptsize 98}$,    
V.A.~Schegelsky$^\textrm{\scriptsize 135}$,    
D.~Scheirich$^\textrm{\scriptsize 140}$,    
F.~Schenck$^\textrm{\scriptsize 19}$,    
M.~Schernau$^\textrm{\scriptsize 168}$,    
C.~Schiavi$^\textrm{\scriptsize 53b,53a}$,    
S.~Schier$^\textrm{\scriptsize 143}$,    
L.K.~Schildgen$^\textrm{\scriptsize 24}$,    
Z.M.~Schillaci$^\textrm{\scriptsize 26}$,    
E.J.~Schioppa$^\textrm{\scriptsize 35}$,    
M.~Schioppa$^\textrm{\scriptsize 40b,40a}$,    
K.E.~Schleicher$^\textrm{\scriptsize 50}$,    
S.~Schlenker$^\textrm{\scriptsize 35}$,    
K.R.~Schmidt-Sommerfeld$^\textrm{\scriptsize 113}$,    
K.~Schmieden$^\textrm{\scriptsize 35}$,    
C.~Schmitt$^\textrm{\scriptsize 97}$,    
S.~Schmitt$^\textrm{\scriptsize 44}$,    
S.~Schmitz$^\textrm{\scriptsize 97}$,    
J.C.~Schmoeckel$^\textrm{\scriptsize 44}$,    
U.~Schnoor$^\textrm{\scriptsize 50}$,    
L.~Schoeffel$^\textrm{\scriptsize 142}$,    
A.~Schoening$^\textrm{\scriptsize 59b}$,    
E.~Schopf$^\textrm{\scriptsize 132}$,    
M.~Schott$^\textrm{\scriptsize 97}$,    
J.F.P.~Schouwenberg$^\textrm{\scriptsize 117}$,    
J.~Schovancova$^\textrm{\scriptsize 35}$,    
S.~Schramm$^\textrm{\scriptsize 52}$,    
A.~Schulte$^\textrm{\scriptsize 97}$,    
H-C.~Schultz-Coulon$^\textrm{\scriptsize 59a}$,    
M.~Schumacher$^\textrm{\scriptsize 50}$,    
B.A.~Schumm$^\textrm{\scriptsize 143}$,    
Ph.~Schune$^\textrm{\scriptsize 142}$,    
A.~Schwartzman$^\textrm{\scriptsize 150}$,    
T.A.~Schwarz$^\textrm{\scriptsize 103}$,    
Ph.~Schwemling$^\textrm{\scriptsize 142}$,    
R.~Schwienhorst$^\textrm{\scriptsize 104}$,    
A.~Sciandra$^\textrm{\scriptsize 24}$,    
G.~Sciolla$^\textrm{\scriptsize 26}$,    
M.~Scornajenghi$^\textrm{\scriptsize 40b,40a}$,    
F.~Scuri$^\textrm{\scriptsize 69a}$,    
F.~Scutti$^\textrm{\scriptsize 102}$,    
L.M.~Scyboz$^\textrm{\scriptsize 113}$,    
C.D.~Sebastiani$^\textrm{\scriptsize 70a,70b}$,    
P.~Seema$^\textrm{\scriptsize 19}$,    
S.C.~Seidel$^\textrm{\scriptsize 116}$,    
A.~Seiden$^\textrm{\scriptsize 143}$,    
T.~Seiss$^\textrm{\scriptsize 36}$,    
J.M.~Seixas$^\textrm{\scriptsize 78b}$,    
G.~Sekhniaidze$^\textrm{\scriptsize 67a}$,    
K.~Sekhon$^\textrm{\scriptsize 103}$,    
S.J.~Sekula$^\textrm{\scriptsize 41}$,    
N.~Semprini-Cesari$^\textrm{\scriptsize 23b,23a}$,    
S.~Sen$^\textrm{\scriptsize 47}$,    
S.~Senkin$^\textrm{\scriptsize 37}$,    
C.~Serfon$^\textrm{\scriptsize 131}$,    
L.~Serin$^\textrm{\scriptsize 129}$,    
L.~Serkin$^\textrm{\scriptsize 64a,64b}$,    
M.~Sessa$^\textrm{\scriptsize 58a}$,    
H.~Severini$^\textrm{\scriptsize 125}$,    
F.~Sforza$^\textrm{\scriptsize 167}$,    
A.~Sfyrla$^\textrm{\scriptsize 52}$,    
E.~Shabalina$^\textrm{\scriptsize 51}$,    
J.D.~Shahinian$^\textrm{\scriptsize 143}$,    
N.W.~Shaikh$^\textrm{\scriptsize 43a,43b}$,    
D.~Shaked~Renous$^\textrm{\scriptsize 177}$,    
L.Y.~Shan$^\textrm{\scriptsize 15a}$,    
R.~Shang$^\textrm{\scriptsize 170}$,    
J.T.~Shank$^\textrm{\scriptsize 25}$,    
M.~Shapiro$^\textrm{\scriptsize 18}$,    
A.S.~Sharma$^\textrm{\scriptsize 1}$,    
A.~Sharma$^\textrm{\scriptsize 132}$,    
P.B.~Shatalov$^\textrm{\scriptsize 109}$,    
K.~Shaw$^\textrm{\scriptsize 153}$,    
S.M.~Shaw$^\textrm{\scriptsize 98}$,    
A.~Shcherbakova$^\textrm{\scriptsize 135}$,    
Y.~Shen$^\textrm{\scriptsize 125}$,    
N.~Sherafati$^\textrm{\scriptsize 33}$,    
A.D.~Sherman$^\textrm{\scriptsize 25}$,    
P.~Sherwood$^\textrm{\scriptsize 92}$,    
L.~Shi$^\textrm{\scriptsize 155,ar}$,    
S.~Shimizu$^\textrm{\scriptsize 79}$,    
C.O.~Shimmin$^\textrm{\scriptsize 180}$,    
Y.~Shimogama$^\textrm{\scriptsize 176}$,    
M.~Shimojima$^\textrm{\scriptsize 114}$,    
I.P.J.~Shipsey$^\textrm{\scriptsize 132}$,    
S.~Shirabe$^\textrm{\scriptsize 85}$,    
M.~Shiyakova$^\textrm{\scriptsize 77}$,    
J.~Shlomi$^\textrm{\scriptsize 177}$,    
A.~Shmeleva$^\textrm{\scriptsize 108}$,    
D.~Shoaleh~Saadi$^\textrm{\scriptsize 107}$,    
M.J.~Shochet$^\textrm{\scriptsize 36}$,    
S.~Shojaii$^\textrm{\scriptsize 102}$,    
D.R.~Shope$^\textrm{\scriptsize 125}$,    
S.~Shrestha$^\textrm{\scriptsize 123}$,    
E.~Shulga$^\textrm{\scriptsize 110}$,    
P.~Sicho$^\textrm{\scriptsize 138}$,    
A.M.~Sickles$^\textrm{\scriptsize 170}$,    
P.E.~Sidebo$^\textrm{\scriptsize 151}$,    
E.~Sideras~Haddad$^\textrm{\scriptsize 32c}$,    
O.~Sidiropoulou$^\textrm{\scriptsize 35}$,    
A.~Sidoti$^\textrm{\scriptsize 23b,23a}$,    
F.~Siegert$^\textrm{\scriptsize 46}$,    
Dj.~Sijacki$^\textrm{\scriptsize 16}$,    
J.~Silva$^\textrm{\scriptsize 137a}$,    
M.~Silva~Jr.$^\textrm{\scriptsize 178}$,    
M.V.~Silva~Oliveira$^\textrm{\scriptsize 78a}$,    
S.B.~Silverstein$^\textrm{\scriptsize 43a}$,    
S.~Simion$^\textrm{\scriptsize 129}$,    
E.~Simioni$^\textrm{\scriptsize 97}$,    
M.~Simon$^\textrm{\scriptsize 97}$,    
R.~Simoniello$^\textrm{\scriptsize 97}$,    
P.~Sinervo$^\textrm{\scriptsize 164}$,    
N.B.~Sinev$^\textrm{\scriptsize 128}$,    
M.~Sioli$^\textrm{\scriptsize 23b,23a}$,    
I.~Siral$^\textrm{\scriptsize 103}$,    
S.Yu.~Sivoklokov$^\textrm{\scriptsize 111}$,    
J.~Sj\"{o}lin$^\textrm{\scriptsize 43a,43b}$,    
P.~Skubic$^\textrm{\scriptsize 125}$,    
M.~Slater$^\textrm{\scriptsize 21}$,    
T.~Slavicek$^\textrm{\scriptsize 139}$,    
M.~Slawinska$^\textrm{\scriptsize 82}$,    
K.~Sliwa$^\textrm{\scriptsize 167}$,    
R.~Slovak$^\textrm{\scriptsize 140}$,    
V.~Smakhtin$^\textrm{\scriptsize 177}$,    
B.H.~Smart$^\textrm{\scriptsize 5}$,    
J.~Smiesko$^\textrm{\scriptsize 28a}$,    
N.~Smirnov$^\textrm{\scriptsize 110}$,    
S.Yu.~Smirnov$^\textrm{\scriptsize 110}$,    
Y.~Smirnov$^\textrm{\scriptsize 110}$,    
L.N.~Smirnova$^\textrm{\scriptsize 111}$,    
O.~Smirnova$^\textrm{\scriptsize 94}$,    
J.W.~Smith$^\textrm{\scriptsize 51}$,    
M.~Smizanska$^\textrm{\scriptsize 87}$,    
K.~Smolek$^\textrm{\scriptsize 139}$,    
A.~Smykiewicz$^\textrm{\scriptsize 82}$,    
A.A.~Snesarev$^\textrm{\scriptsize 108}$,    
I.M.~Snyder$^\textrm{\scriptsize 128}$,    
S.~Snyder$^\textrm{\scriptsize 29}$,    
R.~Sobie$^\textrm{\scriptsize 173,ad}$,    
A.M.~Soffa$^\textrm{\scriptsize 168}$,    
A.~Soffer$^\textrm{\scriptsize 158}$,    
A.~S{\o}gaard$^\textrm{\scriptsize 48}$,    
F.~Sohns$^\textrm{\scriptsize 51}$,    
G.~Sokhrannyi$^\textrm{\scriptsize 89}$,    
C.A.~Solans~Sanchez$^\textrm{\scriptsize 35}$,    
M.~Solar$^\textrm{\scriptsize 139}$,    
E.Yu.~Soldatov$^\textrm{\scriptsize 110}$,    
U.~Soldevila$^\textrm{\scriptsize 171}$,    
A.A.~Solodkov$^\textrm{\scriptsize 121}$,    
A.~Soloshenko$^\textrm{\scriptsize 77}$,    
O.V.~Solovyanov$^\textrm{\scriptsize 121}$,    
V.~Solovyev$^\textrm{\scriptsize 135}$,    
P.~Sommer$^\textrm{\scriptsize 146}$,    
H.~Son$^\textrm{\scriptsize 167}$,    
W.~Song$^\textrm{\scriptsize 141}$,    
W.Y.~Song$^\textrm{\scriptsize 165b}$,    
A.~Sopczak$^\textrm{\scriptsize 139}$,    
F.~Sopkova$^\textrm{\scriptsize 28b}$,    
C.L.~Sotiropoulou$^\textrm{\scriptsize 69a,69b}$,    
S.~Sottocornola$^\textrm{\scriptsize 68a,68b}$,    
R.~Soualah$^\textrm{\scriptsize 64a,64c,j}$,    
A.M.~Soukharev$^\textrm{\scriptsize 120b,120a}$,    
D.~South$^\textrm{\scriptsize 44}$,    
S.~Spagnolo$^\textrm{\scriptsize 65a,65b}$,    
M.~Spalla$^\textrm{\scriptsize 113}$,    
M.~Spangenberg$^\textrm{\scriptsize 175}$,    
F.~Span\`o$^\textrm{\scriptsize 91}$,    
D.~Sperlich$^\textrm{\scriptsize 19}$,    
T.M.~Spieker$^\textrm{\scriptsize 59a}$,    
R.~Spighi$^\textrm{\scriptsize 23b}$,    
G.~Spigo$^\textrm{\scriptsize 35}$,    
L.A.~Spiller$^\textrm{\scriptsize 102}$,    
D.P.~Spiteri$^\textrm{\scriptsize 55}$,    
M.~Spousta$^\textrm{\scriptsize 140}$,    
A.~Stabile$^\textrm{\scriptsize 66a,66b}$,    
R.~Stamen$^\textrm{\scriptsize 59a}$,    
S.~Stamm$^\textrm{\scriptsize 19}$,    
E.~Stanecka$^\textrm{\scriptsize 82}$,    
R.W.~Stanek$^\textrm{\scriptsize 6}$,    
C.~Stanescu$^\textrm{\scriptsize 72a}$,    
B.~Stanislaus$^\textrm{\scriptsize 132}$,    
M.M.~Stanitzki$^\textrm{\scriptsize 44}$,    
B.~Stapf$^\textrm{\scriptsize 118}$,    
S.~Stapnes$^\textrm{\scriptsize 131}$,    
E.A.~Starchenko$^\textrm{\scriptsize 121}$,    
G.H.~Stark$^\textrm{\scriptsize 143}$,    
J.~Stark$^\textrm{\scriptsize 56}$,    
S.H~Stark$^\textrm{\scriptsize 39}$,    
P.~Staroba$^\textrm{\scriptsize 138}$,    
P.~Starovoitov$^\textrm{\scriptsize 59a}$,    
S.~St\"arz$^\textrm{\scriptsize 101}$,    
R.~Staszewski$^\textrm{\scriptsize 82}$,    
M.~Stegler$^\textrm{\scriptsize 44}$,    
P.~Steinberg$^\textrm{\scriptsize 29}$,    
B.~Stelzer$^\textrm{\scriptsize 149}$,    
H.J.~Stelzer$^\textrm{\scriptsize 35}$,    
O.~Stelzer-Chilton$^\textrm{\scriptsize 165a}$,    
H.~Stenzel$^\textrm{\scriptsize 54}$,    
T.J.~Stevenson$^\textrm{\scriptsize 90}$,    
G.A.~Stewart$^\textrm{\scriptsize 35}$,    
M.C.~Stockton$^\textrm{\scriptsize 35}$,    
G.~Stoicea$^\textrm{\scriptsize 27b}$,    
P.~Stolte$^\textrm{\scriptsize 51}$,    
S.~Stonjek$^\textrm{\scriptsize 113}$,    
A.~Straessner$^\textrm{\scriptsize 46}$,    
J.~Strandberg$^\textrm{\scriptsize 151}$,    
S.~Strandberg$^\textrm{\scriptsize 43a,43b}$,    
M.~Strauss$^\textrm{\scriptsize 125}$,    
P.~Strizenec$^\textrm{\scriptsize 28b}$,    
R.~Str\"ohmer$^\textrm{\scriptsize 174}$,    
D.M.~Strom$^\textrm{\scriptsize 128}$,    
R.~Stroynowski$^\textrm{\scriptsize 41}$,    
A.~Strubig$^\textrm{\scriptsize 48}$,    
S.A.~Stucci$^\textrm{\scriptsize 29}$,    
B.~Stugu$^\textrm{\scriptsize 17}$,    
J.~Stupak$^\textrm{\scriptsize 125}$,    
N.A.~Styles$^\textrm{\scriptsize 44}$,    
D.~Su$^\textrm{\scriptsize 150}$,    
J.~Su$^\textrm{\scriptsize 136}$,    
S.~Suchek$^\textrm{\scriptsize 59a}$,    
Y.~Sugaya$^\textrm{\scriptsize 130}$,    
M.~Suk$^\textrm{\scriptsize 139}$,    
V.V.~Sulin$^\textrm{\scriptsize 108}$,    
M.J.~Sullivan$^\textrm{\scriptsize 88}$,    
D.M.S.~Sultan$^\textrm{\scriptsize 52}$,    
S.~Sultansoy$^\textrm{\scriptsize 4c}$,    
T.~Sumida$^\textrm{\scriptsize 83}$,    
S.~Sun$^\textrm{\scriptsize 103}$,    
X.~Sun$^\textrm{\scriptsize 3}$,    
K.~Suruliz$^\textrm{\scriptsize 153}$,    
C.J.E.~Suster$^\textrm{\scriptsize 154}$,    
M.R.~Sutton$^\textrm{\scriptsize 153}$,    
S.~Suzuki$^\textrm{\scriptsize 79}$,    
M.~Svatos$^\textrm{\scriptsize 138}$,    
M.~Swiatlowski$^\textrm{\scriptsize 36}$,    
S.P.~Swift$^\textrm{\scriptsize 2}$,    
A.~Sydorenko$^\textrm{\scriptsize 97}$,    
I.~Sykora$^\textrm{\scriptsize 28a}$,    
M.~Sykora$^\textrm{\scriptsize 140}$,    
T.~Sykora$^\textrm{\scriptsize 140}$,    
D.~Ta$^\textrm{\scriptsize 97}$,    
K.~Tackmann$^\textrm{\scriptsize 44,aa}$,    
J.~Taenzer$^\textrm{\scriptsize 158}$,    
A.~Taffard$^\textrm{\scriptsize 168}$,    
R.~Tafirout$^\textrm{\scriptsize 165a}$,    
E.~Tahirovic$^\textrm{\scriptsize 90}$,    
N.~Taiblum$^\textrm{\scriptsize 158}$,    
H.~Takai$^\textrm{\scriptsize 29}$,    
R.~Takashima$^\textrm{\scriptsize 84}$,    
E.H.~Takasugi$^\textrm{\scriptsize 113}$,    
K.~Takeda$^\textrm{\scriptsize 80}$,    
T.~Takeshita$^\textrm{\scriptsize 147}$,    
Y.~Takubo$^\textrm{\scriptsize 79}$,    
M.~Talby$^\textrm{\scriptsize 99}$,    
A.A.~Talyshev$^\textrm{\scriptsize 120b,120a}$,    
J.~Tanaka$^\textrm{\scriptsize 160}$,    
M.~Tanaka$^\textrm{\scriptsize 162}$,    
R.~Tanaka$^\textrm{\scriptsize 129}$,    
B.B.~Tannenwald$^\textrm{\scriptsize 123}$,    
S.~Tapia~Araya$^\textrm{\scriptsize 144b}$,    
S.~Tapprogge$^\textrm{\scriptsize 97}$,    
A.~Tarek~Abouelfadl~Mohamed$^\textrm{\scriptsize 133}$,    
S.~Tarem$^\textrm{\scriptsize 157}$,    
G.~Tarna$^\textrm{\scriptsize 27b,e}$,    
G.F.~Tartarelli$^\textrm{\scriptsize 66a}$,    
P.~Tas$^\textrm{\scriptsize 140}$,    
M.~Tasevsky$^\textrm{\scriptsize 138}$,    
T.~Tashiro$^\textrm{\scriptsize 83}$,    
E.~Tassi$^\textrm{\scriptsize 40b,40a}$,    
A.~Tavares~Delgado$^\textrm{\scriptsize 137a,137b}$,    
Y.~Tayalati$^\textrm{\scriptsize 34e}$,    
A.J.~Taylor$^\textrm{\scriptsize 48}$,    
G.N.~Taylor$^\textrm{\scriptsize 102}$,    
P.T.E.~Taylor$^\textrm{\scriptsize 102}$,    
W.~Taylor$^\textrm{\scriptsize 165b}$,    
A.S.~Tee$^\textrm{\scriptsize 87}$,    
R.~Teixeira~De~Lima$^\textrm{\scriptsize 150}$,    
P.~Teixeira-Dias$^\textrm{\scriptsize 91}$,    
H.~Ten~Kate$^\textrm{\scriptsize 35}$,    
J.J.~Teoh$^\textrm{\scriptsize 118}$,    
S.~Terada$^\textrm{\scriptsize 79}$,    
K.~Terashi$^\textrm{\scriptsize 160}$,    
J.~Terron$^\textrm{\scriptsize 96}$,    
S.~Terzo$^\textrm{\scriptsize 14}$,    
M.~Testa$^\textrm{\scriptsize 49}$,    
R.J.~Teuscher$^\textrm{\scriptsize 164,ad}$,    
S.J.~Thais$^\textrm{\scriptsize 180}$,    
T.~Theveneaux-Pelzer$^\textrm{\scriptsize 44}$,    
F.~Thiele$^\textrm{\scriptsize 39}$,    
D.W.~Thomas$^\textrm{\scriptsize 91}$,    
J.P.~Thomas$^\textrm{\scriptsize 21}$,    
A.S.~Thompson$^\textrm{\scriptsize 55}$,    
P.D.~Thompson$^\textrm{\scriptsize 21}$,    
L.A.~Thomsen$^\textrm{\scriptsize 180}$,    
E.~Thomson$^\textrm{\scriptsize 134}$,    
Y.~Tian$^\textrm{\scriptsize 38}$,    
R.E.~Ticse~Torres$^\textrm{\scriptsize 51}$,    
V.O.~Tikhomirov$^\textrm{\scriptsize 108,ap}$,    
Yu.A.~Tikhonov$^\textrm{\scriptsize 120b,120a}$,    
S.~Timoshenko$^\textrm{\scriptsize 110}$,    
P.~Tipton$^\textrm{\scriptsize 180}$,    
S.~Tisserant$^\textrm{\scriptsize 99}$,    
K.~Todome$^\textrm{\scriptsize 162}$,    
S.~Todorova-Nova$^\textrm{\scriptsize 5}$,    
S.~Todt$^\textrm{\scriptsize 46}$,    
J.~Tojo$^\textrm{\scriptsize 85}$,    
S.~Tok\'ar$^\textrm{\scriptsize 28a}$,    
K.~Tokushuku$^\textrm{\scriptsize 79}$,    
E.~Tolley$^\textrm{\scriptsize 123}$,    
K.G.~Tomiwa$^\textrm{\scriptsize 32c}$,    
M.~Tomoto$^\textrm{\scriptsize 115}$,    
L.~Tompkins$^\textrm{\scriptsize 150,r}$,    
K.~Toms$^\textrm{\scriptsize 116}$,    
B.~Tong$^\textrm{\scriptsize 57}$,    
P.~Tornambe$^\textrm{\scriptsize 50}$,    
E.~Torrence$^\textrm{\scriptsize 128}$,    
H.~Torres$^\textrm{\scriptsize 46}$,    
E.~Torr\'o~Pastor$^\textrm{\scriptsize 145}$,    
C.~Tosciri$^\textrm{\scriptsize 132}$,    
J.~Toth$^\textrm{\scriptsize 99,ac}$,    
F.~Touchard$^\textrm{\scriptsize 99}$,    
D.R.~Tovey$^\textrm{\scriptsize 146}$,    
C.J.~Treado$^\textrm{\scriptsize 122}$,    
T.~Trefzger$^\textrm{\scriptsize 174}$,    
F.~Tresoldi$^\textrm{\scriptsize 153}$,    
A.~Tricoli$^\textrm{\scriptsize 29}$,    
I.M.~Trigger$^\textrm{\scriptsize 165a}$,    
S.~Trincaz-Duvoid$^\textrm{\scriptsize 133}$,    
W.~Trischuk$^\textrm{\scriptsize 164}$,    
B.~Trocm\'e$^\textrm{\scriptsize 56}$,    
A.~Trofymov$^\textrm{\scriptsize 129}$,    
C.~Troncon$^\textrm{\scriptsize 66a}$,    
M.~Trovatelli$^\textrm{\scriptsize 173}$,    
F.~Trovato$^\textrm{\scriptsize 153}$,    
L.~Truong$^\textrm{\scriptsize 32b}$,    
M.~Trzebinski$^\textrm{\scriptsize 82}$,    
A.~Trzupek$^\textrm{\scriptsize 82}$,    
F.~Tsai$^\textrm{\scriptsize 44}$,    
J.C-L.~Tseng$^\textrm{\scriptsize 132}$,    
P.V.~Tsiareshka$^\textrm{\scriptsize 105,aj}$,    
A.~Tsirigotis$^\textrm{\scriptsize 159}$,    
N.~Tsirintanis$^\textrm{\scriptsize 9}$,    
V.~Tsiskaridze$^\textrm{\scriptsize 152}$,    
E.G.~Tskhadadze$^\textrm{\scriptsize 156a}$,    
I.I.~Tsukerman$^\textrm{\scriptsize 109}$,    
V.~Tsulaia$^\textrm{\scriptsize 18}$,    
S.~Tsuno$^\textrm{\scriptsize 79}$,    
D.~Tsybychev$^\textrm{\scriptsize 152,163}$,    
Y.~Tu$^\textrm{\scriptsize 61b}$,    
A.~Tudorache$^\textrm{\scriptsize 27b}$,    
V.~Tudorache$^\textrm{\scriptsize 27b}$,    
T.T.~Tulbure$^\textrm{\scriptsize 27a}$,    
A.N.~Tuna$^\textrm{\scriptsize 57}$,    
S.~Turchikhin$^\textrm{\scriptsize 77}$,    
D.~Turgeman$^\textrm{\scriptsize 177}$,    
I.~Turk~Cakir$^\textrm{\scriptsize 4b,u}$,    
R.~Turra$^\textrm{\scriptsize 66a}$,    
P.M.~Tuts$^\textrm{\scriptsize 38}$,    
S~Tzamarias$^\textrm{\scriptsize 159}$,    
E.~Tzovara$^\textrm{\scriptsize 97}$,    
G.~Ucchielli$^\textrm{\scriptsize 45}$,    
I.~Ueda$^\textrm{\scriptsize 79}$,    
M.~Ughetto$^\textrm{\scriptsize 43a,43b}$,    
F.~Ukegawa$^\textrm{\scriptsize 166}$,    
G.~Unal$^\textrm{\scriptsize 35}$,    
A.~Undrus$^\textrm{\scriptsize 29}$,    
G.~Unel$^\textrm{\scriptsize 168}$,    
F.C.~Ungaro$^\textrm{\scriptsize 102}$,    
Y.~Unno$^\textrm{\scriptsize 79}$,    
K.~Uno$^\textrm{\scriptsize 160}$,    
J.~Urban$^\textrm{\scriptsize 28b}$,    
P.~Urquijo$^\textrm{\scriptsize 102}$,    
G.~Usai$^\textrm{\scriptsize 8}$,    
J.~Usui$^\textrm{\scriptsize 79}$,    
L.~Vacavant$^\textrm{\scriptsize 99}$,    
V.~Vacek$^\textrm{\scriptsize 139}$,    
B.~Vachon$^\textrm{\scriptsize 101}$,    
K.O.H.~Vadla$^\textrm{\scriptsize 131}$,    
A.~Vaidya$^\textrm{\scriptsize 92}$,    
C.~Valderanis$^\textrm{\scriptsize 112}$,    
E.~Valdes~Santurio$^\textrm{\scriptsize 43a,43b}$,    
M.~Valente$^\textrm{\scriptsize 52}$,    
S.~Valentinetti$^\textrm{\scriptsize 23b,23a}$,    
A.~Valero$^\textrm{\scriptsize 171}$,    
L.~Val\'ery$^\textrm{\scriptsize 44}$,    
R.A.~Vallance$^\textrm{\scriptsize 21}$,    
A.~Vallier$^\textrm{\scriptsize 5}$,    
J.A.~Valls~Ferrer$^\textrm{\scriptsize 171}$,    
T.R.~Van~Daalen$^\textrm{\scriptsize 14}$,    
H.~Van~der~Graaf$^\textrm{\scriptsize 118}$,    
P.~Van~Gemmeren$^\textrm{\scriptsize 6}$,    
I.~Van~Vulpen$^\textrm{\scriptsize 118}$,    
M.~Vanadia$^\textrm{\scriptsize 71a,71b}$,    
W.~Vandelli$^\textrm{\scriptsize 35}$,    
A.~Vaniachine$^\textrm{\scriptsize 163}$,    
P.~Vankov$^\textrm{\scriptsize 118}$,    
R.~Vari$^\textrm{\scriptsize 70a}$,    
E.W.~Varnes$^\textrm{\scriptsize 7}$,    
C.~Varni$^\textrm{\scriptsize 53b,53a}$,    
T.~Varol$^\textrm{\scriptsize 41}$,    
D.~Varouchas$^\textrm{\scriptsize 129}$,    
K.E.~Varvell$^\textrm{\scriptsize 154}$,    
G.A.~Vasquez$^\textrm{\scriptsize 144b}$,    
J.G.~Vasquez$^\textrm{\scriptsize 180}$,    
F.~Vazeille$^\textrm{\scriptsize 37}$,    
D.~Vazquez~Furelos$^\textrm{\scriptsize 14}$,    
T.~Vazquez~Schroeder$^\textrm{\scriptsize 35}$,    
J.~Veatch$^\textrm{\scriptsize 51}$,    
V.~Vecchio$^\textrm{\scriptsize 72a,72b}$,    
L.M.~Veloce$^\textrm{\scriptsize 164}$,    
F.~Veloso$^\textrm{\scriptsize 137a,137c}$,    
S.~Veneziano$^\textrm{\scriptsize 70a}$,    
A.~Ventura$^\textrm{\scriptsize 65a,65b}$,    
N.~Venturi$^\textrm{\scriptsize 35}$,    
V.~Vercesi$^\textrm{\scriptsize 68a}$,    
M.~Verducci$^\textrm{\scriptsize 72a,72b}$,    
C.M.~Vergel~Infante$^\textrm{\scriptsize 76}$,    
C.~Vergis$^\textrm{\scriptsize 24}$,    
W.~Verkerke$^\textrm{\scriptsize 118}$,    
A.T.~Vermeulen$^\textrm{\scriptsize 118}$,    
J.C.~Vermeulen$^\textrm{\scriptsize 118}$,    
M.C.~Vetterli$^\textrm{\scriptsize 149,av}$,    
N.~Viaux~Maira$^\textrm{\scriptsize 144b}$,    
M.~Vicente~Barreto~Pinto$^\textrm{\scriptsize 52}$,    
I.~Vichou$^\textrm{\scriptsize 170,*}$,    
T.~Vickey$^\textrm{\scriptsize 146}$,    
O.E.~Vickey~Boeriu$^\textrm{\scriptsize 146}$,    
G.H.A.~Viehhauser$^\textrm{\scriptsize 132}$,    
S.~Viel$^\textrm{\scriptsize 18}$,    
L.~Vigani$^\textrm{\scriptsize 132}$,    
M.~Villa$^\textrm{\scriptsize 23b,23a}$,    
M.~Villaplana~Perez$^\textrm{\scriptsize 66a,66b}$,    
E.~Vilucchi$^\textrm{\scriptsize 49}$,    
M.G.~Vincter$^\textrm{\scriptsize 33}$,    
V.B.~Vinogradov$^\textrm{\scriptsize 77}$,    
A.~Vishwakarma$^\textrm{\scriptsize 44}$,    
C.~Vittori$^\textrm{\scriptsize 23b,23a}$,    
I.~Vivarelli$^\textrm{\scriptsize 153}$,    
S.~Vlachos$^\textrm{\scriptsize 10}$,    
M.~Vogel$^\textrm{\scriptsize 179}$,    
P.~Vokac$^\textrm{\scriptsize 139}$,    
G.~Volpi$^\textrm{\scriptsize 14}$,    
S.E.~von~Buddenbrock$^\textrm{\scriptsize 32c}$,    
E.~Von~Toerne$^\textrm{\scriptsize 24}$,    
V.~Vorobel$^\textrm{\scriptsize 140}$,    
K.~Vorobev$^\textrm{\scriptsize 110}$,    
M.~Vos$^\textrm{\scriptsize 171}$,    
J.H.~Vossebeld$^\textrm{\scriptsize 88}$,    
N.~Vranjes$^\textrm{\scriptsize 16}$,    
M.~Vranjes~Milosavljevic$^\textrm{\scriptsize 16}$,    
V.~Vrba$^\textrm{\scriptsize 139}$,    
M.~Vreeswijk$^\textrm{\scriptsize 118}$,    
T.~\v{S}filigoj$^\textrm{\scriptsize 89}$,    
R.~Vuillermet$^\textrm{\scriptsize 35}$,    
I.~Vukotic$^\textrm{\scriptsize 36}$,    
T.~\v{Z}eni\v{s}$^\textrm{\scriptsize 28a}$,    
L.~\v{Z}ivkovi\'{c}$^\textrm{\scriptsize 16}$,    
P.~Wagner$^\textrm{\scriptsize 24}$,    
W.~Wagner$^\textrm{\scriptsize 179}$,    
J.~Wagner-Kuhr$^\textrm{\scriptsize 112}$,    
H.~Wahlberg$^\textrm{\scriptsize 86}$,    
S.~Wahrmund$^\textrm{\scriptsize 46}$,    
K.~Wakamiya$^\textrm{\scriptsize 80}$,    
V.M.~Walbrecht$^\textrm{\scriptsize 113}$,    
J.~Walder$^\textrm{\scriptsize 87}$,    
R.~Walker$^\textrm{\scriptsize 112}$,    
S.D.~Walker$^\textrm{\scriptsize 91}$,    
W.~Walkowiak$^\textrm{\scriptsize 148}$,    
V.~Wallangen$^\textrm{\scriptsize 43a,43b}$,    
A.M.~Wang$^\textrm{\scriptsize 57}$,    
C.~Wang$^\textrm{\scriptsize 58b}$,    
F.~Wang$^\textrm{\scriptsize 178}$,    
H.~Wang$^\textrm{\scriptsize 18}$,    
H.~Wang$^\textrm{\scriptsize 3}$,    
J.~Wang$^\textrm{\scriptsize 154}$,    
J.~Wang$^\textrm{\scriptsize 59b}$,    
P.~Wang$^\textrm{\scriptsize 41}$,    
Q.~Wang$^\textrm{\scriptsize 125}$,    
R.-J.~Wang$^\textrm{\scriptsize 133}$,    
R.~Wang$^\textrm{\scriptsize 58a}$,    
R.~Wang$^\textrm{\scriptsize 6}$,    
S.M.~Wang$^\textrm{\scriptsize 155}$,    
W.T.~Wang$^\textrm{\scriptsize 58a}$,    
W.~Wang$^\textrm{\scriptsize 15c,ae}$,    
W.X.~Wang$^\textrm{\scriptsize 58a,ae}$,    
Y.~Wang$^\textrm{\scriptsize 58a,am}$,    
Z.~Wang$^\textrm{\scriptsize 58c}$,    
C.~Wanotayaroj$^\textrm{\scriptsize 44}$,    
A.~Warburton$^\textrm{\scriptsize 101}$,    
C.P.~Ward$^\textrm{\scriptsize 31}$,    
D.R.~Wardrope$^\textrm{\scriptsize 92}$,    
A.~Washbrook$^\textrm{\scriptsize 48}$,    
P.M.~Watkins$^\textrm{\scriptsize 21}$,    
A.T.~Watson$^\textrm{\scriptsize 21}$,    
M.F.~Watson$^\textrm{\scriptsize 21}$,    
G.~Watts$^\textrm{\scriptsize 145}$,    
S.~Watts$^\textrm{\scriptsize 98}$,    
B.M.~Waugh$^\textrm{\scriptsize 92}$,    
A.F.~Webb$^\textrm{\scriptsize 11}$,    
S.~Webb$^\textrm{\scriptsize 97}$,    
C.~Weber$^\textrm{\scriptsize 180}$,    
M.S.~Weber$^\textrm{\scriptsize 20}$,    
S.A.~Weber$^\textrm{\scriptsize 33}$,    
S.M.~Weber$^\textrm{\scriptsize 59a}$,    
A.R.~Weidberg$^\textrm{\scriptsize 132}$,    
J.~Weingarten$^\textrm{\scriptsize 45}$,    
M.~Weirich$^\textrm{\scriptsize 97}$,    
C.~Weiser$^\textrm{\scriptsize 50}$,    
P.S.~Wells$^\textrm{\scriptsize 35}$,    
T.~Wenaus$^\textrm{\scriptsize 29}$,    
T.~Wengler$^\textrm{\scriptsize 35}$,    
S.~Wenig$^\textrm{\scriptsize 35}$,    
N.~Wermes$^\textrm{\scriptsize 24}$,    
M.D.~Werner$^\textrm{\scriptsize 76}$,    
P.~Werner$^\textrm{\scriptsize 35}$,    
M.~Wessels$^\textrm{\scriptsize 59a}$,    
T.D.~Weston$^\textrm{\scriptsize 20}$,    
K.~Whalen$^\textrm{\scriptsize 128}$,    
N.L.~Whallon$^\textrm{\scriptsize 145}$,    
A.M.~Wharton$^\textrm{\scriptsize 87}$,    
A.S.~White$^\textrm{\scriptsize 103}$,    
A.~White$^\textrm{\scriptsize 8}$,    
M.J.~White$^\textrm{\scriptsize 1}$,    
R.~White$^\textrm{\scriptsize 144b}$,    
D.~Whiteson$^\textrm{\scriptsize 168}$,    
B.W.~Whitmore$^\textrm{\scriptsize 87}$,    
F.J.~Wickens$^\textrm{\scriptsize 141}$,    
W.~Wiedenmann$^\textrm{\scriptsize 178}$,    
M.~Wielers$^\textrm{\scriptsize 141}$,    
C.~Wiglesworth$^\textrm{\scriptsize 39}$,    
L.A.M.~Wiik-Fuchs$^\textrm{\scriptsize 50}$,    
F.~Wilk$^\textrm{\scriptsize 98}$,    
H.G.~Wilkens$^\textrm{\scriptsize 35}$,    
L.J.~Wilkins$^\textrm{\scriptsize 91}$,    
H.H.~Williams$^\textrm{\scriptsize 134}$,    
S.~Williams$^\textrm{\scriptsize 31}$,    
C.~Willis$^\textrm{\scriptsize 104}$,    
S.~Willocq$^\textrm{\scriptsize 100}$,    
J.A.~Wilson$^\textrm{\scriptsize 21}$,    
I.~Wingerter-Seez$^\textrm{\scriptsize 5}$,    
E.~Winkels$^\textrm{\scriptsize 153}$,    
F.~Winklmeier$^\textrm{\scriptsize 128}$,    
O.J.~Winston$^\textrm{\scriptsize 153}$,    
B.T.~Winter$^\textrm{\scriptsize 50}$,    
M.~Wittgen$^\textrm{\scriptsize 150}$,    
M.~Wobisch$^\textrm{\scriptsize 93}$,    
A.~Wolf$^\textrm{\scriptsize 97}$,    
T.M.H.~Wolf$^\textrm{\scriptsize 118}$,    
R.~Wolff$^\textrm{\scriptsize 99}$,    
J.~Wollrath$^\textrm{\scriptsize 50}$,    
M.W.~Wolter$^\textrm{\scriptsize 82}$,    
H.~Wolters$^\textrm{\scriptsize 137a,137c}$,    
V.W.S.~Wong$^\textrm{\scriptsize 172}$,    
N.L.~Woods$^\textrm{\scriptsize 143}$,    
S.D.~Worm$^\textrm{\scriptsize 21}$,    
B.K.~Wosiek$^\textrm{\scriptsize 82}$,    
K.W.~Wo\'{z}niak$^\textrm{\scriptsize 82}$,    
K.~Wraight$^\textrm{\scriptsize 55}$,    
M.~Wu$^\textrm{\scriptsize 36}$,    
S.L.~Wu$^\textrm{\scriptsize 178}$,    
X.~Wu$^\textrm{\scriptsize 52}$,    
Y.~Wu$^\textrm{\scriptsize 58a}$,    
T.R.~Wyatt$^\textrm{\scriptsize 98}$,    
B.M.~Wynne$^\textrm{\scriptsize 48}$,    
S.~Xella$^\textrm{\scriptsize 39}$,    
Z.~Xi$^\textrm{\scriptsize 103}$,    
L.~Xia$^\textrm{\scriptsize 175}$,    
D.~Xu$^\textrm{\scriptsize 15a}$,    
H.~Xu$^\textrm{\scriptsize 58a,e}$,    
L.~Xu$^\textrm{\scriptsize 29}$,    
T.~Xu$^\textrm{\scriptsize 142}$,    
W.~Xu$^\textrm{\scriptsize 103}$,    
Z.~Xu$^\textrm{\scriptsize 150}$,    
B.~Yabsley$^\textrm{\scriptsize 154}$,    
S.~Yacoob$^\textrm{\scriptsize 32a}$,    
K.~Yajima$^\textrm{\scriptsize 130}$,    
D.P.~Yallup$^\textrm{\scriptsize 92}$,    
D.~Yamaguchi$^\textrm{\scriptsize 162}$,    
Y.~Yamaguchi$^\textrm{\scriptsize 162}$,    
A.~Yamamoto$^\textrm{\scriptsize 79}$,    
T.~Yamanaka$^\textrm{\scriptsize 160}$,    
F.~Yamane$^\textrm{\scriptsize 80}$,    
M.~Yamatani$^\textrm{\scriptsize 160}$,    
T.~Yamazaki$^\textrm{\scriptsize 160}$,    
Y.~Yamazaki$^\textrm{\scriptsize 80}$,    
Z.~Yan$^\textrm{\scriptsize 25}$,    
H.J.~Yang$^\textrm{\scriptsize 58c,58d}$,    
H.T.~Yang$^\textrm{\scriptsize 18}$,    
S.~Yang$^\textrm{\scriptsize 75}$,    
Y.~Yang$^\textrm{\scriptsize 160}$,    
Z.~Yang$^\textrm{\scriptsize 17}$,    
W-M.~Yao$^\textrm{\scriptsize 18}$,    
Y.C.~Yap$^\textrm{\scriptsize 44}$,    
Y.~Yasu$^\textrm{\scriptsize 79}$,    
E.~Yatsenko$^\textrm{\scriptsize 58c,58d}$,    
J.~Ye$^\textrm{\scriptsize 41}$,    
S.~Ye$^\textrm{\scriptsize 29}$,    
I.~Yeletskikh$^\textrm{\scriptsize 77}$,    
E.~Yigitbasi$^\textrm{\scriptsize 25}$,    
E.~Yildirim$^\textrm{\scriptsize 97}$,    
K.~Yorita$^\textrm{\scriptsize 176}$,    
K.~Yoshihara$^\textrm{\scriptsize 134}$,    
C.J.S.~Young$^\textrm{\scriptsize 35}$,    
C.~Young$^\textrm{\scriptsize 150}$,    
J.~Yu$^\textrm{\scriptsize 8}$,    
J.~Yu$^\textrm{\scriptsize 76}$,    
X.~Yue$^\textrm{\scriptsize 59a}$,    
S.P.Y.~Yuen$^\textrm{\scriptsize 24}$,    
B.~Zabinski$^\textrm{\scriptsize 82}$,    
G.~Zacharis$^\textrm{\scriptsize 10}$,    
E.~Zaffaroni$^\textrm{\scriptsize 52}$,    
R.~Zaidan$^\textrm{\scriptsize 14}$,    
A.M.~Zaitsev$^\textrm{\scriptsize 121,ao}$,    
T.~Zakareishvili$^\textrm{\scriptsize 156b}$,    
N.~Zakharchuk$^\textrm{\scriptsize 33}$,    
S.~Zambito$^\textrm{\scriptsize 57}$,    
D.~Zanzi$^\textrm{\scriptsize 35}$,    
D.R.~Zaripovas$^\textrm{\scriptsize 55}$,    
S.V.~Zei{\ss}ner$^\textrm{\scriptsize 45}$,    
C.~Zeitnitz$^\textrm{\scriptsize 179}$,    
G.~Zemaityte$^\textrm{\scriptsize 132}$,    
J.C.~Zeng$^\textrm{\scriptsize 170}$,    
Q.~Zeng$^\textrm{\scriptsize 150}$,    
O.~Zenin$^\textrm{\scriptsize 121}$,    
D.~Zerwas$^\textrm{\scriptsize 129}$,    
M.~Zgubi\v{c}$^\textrm{\scriptsize 132}$,    
D.F.~Zhang$^\textrm{\scriptsize 58b}$,    
D.~Zhang$^\textrm{\scriptsize 103}$,    
F.~Zhang$^\textrm{\scriptsize 178}$,    
G.~Zhang$^\textrm{\scriptsize 58a}$,    
G.~Zhang$^\textrm{\scriptsize 15b}$,    
H.~Zhang$^\textrm{\scriptsize 15c}$,    
J.~Zhang$^\textrm{\scriptsize 6}$,    
L.~Zhang$^\textrm{\scriptsize 15c}$,    
L.~Zhang$^\textrm{\scriptsize 58a}$,    
M.~Zhang$^\textrm{\scriptsize 170}$,    
P.~Zhang$^\textrm{\scriptsize 15c}$,    
R.~Zhang$^\textrm{\scriptsize 58a}$,    
R.~Zhang$^\textrm{\scriptsize 24}$,    
X.~Zhang$^\textrm{\scriptsize 58b}$,    
Y.~Zhang$^\textrm{\scriptsize 15d}$,    
Z.~Zhang$^\textrm{\scriptsize 129}$,    
P.~Zhao$^\textrm{\scriptsize 47}$,    
Y.~Zhao$^\textrm{\scriptsize 58b,129,ak}$,    
Z.~Zhao$^\textrm{\scriptsize 58a}$,    
A.~Zhemchugov$^\textrm{\scriptsize 77}$,    
Z.~Zheng$^\textrm{\scriptsize 103}$,    
D.~Zhong$^\textrm{\scriptsize 170}$,    
B.~Zhou$^\textrm{\scriptsize 103}$,    
C.~Zhou$^\textrm{\scriptsize 178}$,    
M.S.~Zhou$^\textrm{\scriptsize 15d}$,    
M.~Zhou$^\textrm{\scriptsize 152}$,    
N.~Zhou$^\textrm{\scriptsize 58c}$,    
Y.~Zhou$^\textrm{\scriptsize 7}$,    
C.G.~Zhu$^\textrm{\scriptsize 58b}$,    
H.L.~Zhu$^\textrm{\scriptsize 58a}$,    
H.~Zhu$^\textrm{\scriptsize 15a}$,    
J.~Zhu$^\textrm{\scriptsize 103}$,    
Y.~Zhu$^\textrm{\scriptsize 58a}$,    
X.~Zhuang$^\textrm{\scriptsize 15a}$,    
K.~Zhukov$^\textrm{\scriptsize 108}$,    
V.~Zhulanov$^\textrm{\scriptsize 120b,120a}$,    
A.~Zibell$^\textrm{\scriptsize 174}$,    
D.~Zieminska$^\textrm{\scriptsize 63}$,    
N.I.~Zimine$^\textrm{\scriptsize 77}$,    
S.~Zimmermann$^\textrm{\scriptsize 50}$,    
Z.~Zinonos$^\textrm{\scriptsize 113}$,    
M.~Ziolkowski$^\textrm{\scriptsize 148}$,    
G.~Zobernig$^\textrm{\scriptsize 178}$,    
A.~Zoccoli$^\textrm{\scriptsize 23b,23a}$,    
K.~Zoch$^\textrm{\scriptsize 51}$,    
T.G.~Zorbas$^\textrm{\scriptsize 146}$,    
R.~Zou$^\textrm{\scriptsize 36}$,    
M.~Zur~Nedden$^\textrm{\scriptsize 19}$,    
L.~Zwalinski$^\textrm{\scriptsize 35}$.    
\bigskip
\\

$^{1}$Department of Physics, University of Adelaide, Adelaide; Australia.\\
$^{2}$Physics Department, SUNY Albany, Albany NY; United States of America.\\
$^{3}$Department of Physics, University of Alberta, Edmonton AB; Canada.\\
$^{4}$$^{(a)}$Department of Physics, Ankara University, Ankara;$^{(b)}$Istanbul Aydin University, Istanbul;$^{(c)}$Division of Physics, TOBB University of Economics and Technology, Ankara; Turkey.\\
$^{5}$LAPP, Universit\'e Grenoble Alpes, Universit\'e Savoie Mont Blanc, CNRS/IN2P3, Annecy; France.\\
$^{6}$High Energy Physics Division, Argonne National Laboratory, Argonne IL; United States of America.\\
$^{7}$Department of Physics, University of Arizona, Tucson AZ; United States of America.\\
$^{8}$Department of Physics, University of Texas at Arlington, Arlington TX; United States of America.\\
$^{9}$Physics Department, National and Kapodistrian University of Athens, Athens; Greece.\\
$^{10}$Physics Department, National Technical University of Athens, Zografou; Greece.\\
$^{11}$Department of Physics, University of Texas at Austin, Austin TX; United States of America.\\
$^{12}$$^{(a)}$Bahcesehir University, Faculty of Engineering and Natural Sciences, Istanbul;$^{(b)}$Istanbul Bilgi University, Faculty of Engineering and Natural Sciences, Istanbul;$^{(c)}$Department of Physics, Bogazici University, Istanbul;$^{(d)}$Department of Physics Engineering, Gaziantep University, Gaziantep; Turkey.\\
$^{13}$Institute of Physics, Azerbaijan Academy of Sciences, Baku; Azerbaijan.\\
$^{14}$Institut de F\'isica d'Altes Energies (IFAE), Barcelona Institute of Science and Technology, Barcelona; Spain.\\
$^{15}$$^{(a)}$Institute of High Energy Physics, Chinese Academy of Sciences, Beijing;$^{(b)}$Physics Department, Tsinghua University, Beijing;$^{(c)}$Department of Physics, Nanjing University, Nanjing;$^{(d)}$University of Chinese Academy of Science (UCAS), Beijing; China.\\
$^{16}$Institute of Physics, University of Belgrade, Belgrade; Serbia.\\
$^{17}$Department for Physics and Technology, University of Bergen, Bergen; Norway.\\
$^{18}$Physics Division, Lawrence Berkeley National Laboratory and University of California, Berkeley CA; United States of America.\\
$^{19}$Institut f\"{u}r Physik, Humboldt Universit\"{a}t zu Berlin, Berlin; Germany.\\
$^{20}$Albert Einstein Center for Fundamental Physics and Laboratory for High Energy Physics, University of Bern, Bern; Switzerland.\\
$^{21}$School of Physics and Astronomy, University of Birmingham, Birmingham; United Kingdom.\\
$^{22}$Centro de Investigaci\'ones, Universidad Antonio Nari\~no, Bogota; Colombia.\\
$^{23}$$^{(a)}$Dipartimento di Fisica e Astronomia, Universit\`a di Bologna, Bologna;$^{(b)}$INFN Sezione di Bologna; Italy.\\
$^{24}$Physikalisches Institut, Universit\"{a}t Bonn, Bonn; Germany.\\
$^{25}$Department of Physics, Boston University, Boston MA; United States of America.\\
$^{26}$Department of Physics, Brandeis University, Waltham MA; United States of America.\\
$^{27}$$^{(a)}$Transilvania University of Brasov, Brasov;$^{(b)}$Horia Hulubei National Institute of Physics and Nuclear Engineering, Bucharest;$^{(c)}$Department of Physics, Alexandru Ioan Cuza University of Iasi, Iasi;$^{(d)}$National Institute for Research and Development of Isotopic and Molecular Technologies, Physics Department, Cluj-Napoca;$^{(e)}$University Politehnica Bucharest, Bucharest;$^{(f)}$West University in Timisoara, Timisoara; Romania.\\
$^{28}$$^{(a)}$Faculty of Mathematics, Physics and Informatics, Comenius University, Bratislava;$^{(b)}$Department of Subnuclear Physics, Institute of Experimental Physics of the Slovak Academy of Sciences, Kosice; Slovak Republic.\\
$^{29}$Physics Department, Brookhaven National Laboratory, Upton NY; United States of America.\\
$^{30}$Departamento de F\'isica, Universidad de Buenos Aires, Buenos Aires; Argentina.\\
$^{31}$Cavendish Laboratory, University of Cambridge, Cambridge; United Kingdom.\\
$^{32}$$^{(a)}$Department of Physics, University of Cape Town, Cape Town;$^{(b)}$Department of Mechanical Engineering Science, University of Johannesburg, Johannesburg;$^{(c)}$School of Physics, University of the Witwatersrand, Johannesburg; South Africa.\\
$^{33}$Department of Physics, Carleton University, Ottawa ON; Canada.\\
$^{34}$$^{(a)}$Facult\'e des Sciences Ain Chock, R\'eseau Universitaire de Physique des Hautes Energies - Universit\'e Hassan II, Casablanca;$^{(b)}$Centre National de l'Energie des Sciences Techniques Nucleaires (CNESTEN), Rabat;$^{(c)}$Facult\'e des Sciences Semlalia, Universit\'e Cadi Ayyad, LPHEA-Marrakech;$^{(d)}$Facult\'e des Sciences, Universit\'e Mohamed Premier and LPTPM, Oujda;$^{(e)}$Facult\'e des sciences, Universit\'e Mohammed V, Rabat; Morocco.\\
$^{35}$CERN, Geneva; Switzerland.\\
$^{36}$Enrico Fermi Institute, University of Chicago, Chicago IL; United States of America.\\
$^{37}$LPC, Universit\'e Clermont Auvergne, CNRS/IN2P3, Clermont-Ferrand; France.\\
$^{38}$Nevis Laboratory, Columbia University, Irvington NY; United States of America.\\
$^{39}$Niels Bohr Institute, University of Copenhagen, Copenhagen; Denmark.\\
$^{40}$$^{(a)}$Dipartimento di Fisica, Universit\`a della Calabria, Rende;$^{(b)}$INFN Gruppo Collegato di Cosenza, Laboratori Nazionali di Frascati; Italy.\\
$^{41}$Physics Department, Southern Methodist University, Dallas TX; United States of America.\\
$^{42}$Physics Department, University of Texas at Dallas, Richardson TX; United States of America.\\
$^{43}$$^{(a)}$Department of Physics, Stockholm University;$^{(b)}$Oskar Klein Centre, Stockholm; Sweden.\\
$^{44}$Deutsches Elektronen-Synchrotron DESY, Hamburg and Zeuthen; Germany.\\
$^{45}$Lehrstuhl f{\"u}r Experimentelle Physik IV, Technische Universit{\"a}t Dortmund, Dortmund; Germany.\\
$^{46}$Institut f\"{u}r Kern-~und Teilchenphysik, Technische Universit\"{a}t Dresden, Dresden; Germany.\\
$^{47}$Department of Physics, Duke University, Durham NC; United States of America.\\
$^{48}$SUPA - School of Physics and Astronomy, University of Edinburgh, Edinburgh; United Kingdom.\\
$^{49}$INFN e Laboratori Nazionali di Frascati, Frascati; Italy.\\
$^{50}$Physikalisches Institut, Albert-Ludwigs-Universit\"{a}t Freiburg, Freiburg; Germany.\\
$^{51}$II. Physikalisches Institut, Georg-August-Universit\"{a}t G\"ottingen, G\"ottingen; Germany.\\
$^{52}$D\'epartement de Physique Nucl\'eaire et Corpusculaire, Universit\'e de Gen\`eve, Gen\`eve; Switzerland.\\
$^{53}$$^{(a)}$Dipartimento di Fisica, Universit\`a di Genova, Genova;$^{(b)}$INFN Sezione di Genova; Italy.\\
$^{54}$II. Physikalisches Institut, Justus-Liebig-Universit{\"a}t Giessen, Giessen; Germany.\\
$^{55}$SUPA - School of Physics and Astronomy, University of Glasgow, Glasgow; United Kingdom.\\
$^{56}$LPSC, Universit\'e Grenoble Alpes, CNRS/IN2P3, Grenoble INP, Grenoble; France.\\
$^{57}$Laboratory for Particle Physics and Cosmology, Harvard University, Cambridge MA; United States of America.\\
$^{58}$$^{(a)}$Department of Modern Physics and State Key Laboratory of Particle Detection and Electronics, University of Science and Technology of China, Hefei;$^{(b)}$Institute of Frontier and Interdisciplinary Science and Key Laboratory of Particle Physics and Particle Irradiation (MOE), Shandong University, Qingdao;$^{(c)}$School of Physics and Astronomy, Shanghai Jiao Tong University, KLPPAC-MoE, SKLPPC, Shanghai;$^{(d)}$Tsung-Dao Lee Institute, Shanghai; China.\\
$^{59}$$^{(a)}$Kirchhoff-Institut f\"{u}r Physik, Ruprecht-Karls-Universit\"{a}t Heidelberg, Heidelberg;$^{(b)}$Physikalisches Institut, Ruprecht-Karls-Universit\"{a}t Heidelberg, Heidelberg; Germany.\\
$^{60}$Faculty of Applied Information Science, Hiroshima Institute of Technology, Hiroshima; Japan.\\
$^{61}$$^{(a)}$Department of Physics, Chinese University of Hong Kong, Shatin, N.T., Hong Kong;$^{(b)}$Department of Physics, University of Hong Kong, Hong Kong;$^{(c)}$Department of Physics and Institute for Advanced Study, Hong Kong University of Science and Technology, Clear Water Bay, Kowloon, Hong Kong; China.\\
$^{62}$Department of Physics, National Tsing Hua University, Hsinchu; Taiwan.\\
$^{63}$Department of Physics, Indiana University, Bloomington IN; United States of America.\\
$^{64}$$^{(a)}$INFN Gruppo Collegato di Udine, Sezione di Trieste, Udine;$^{(b)}$ICTP, Trieste;$^{(c)}$Dipartimento di Chimica, Fisica e Ambiente, Universit\`a di Udine, Udine; Italy.\\
$^{65}$$^{(a)}$INFN Sezione di Lecce;$^{(b)}$Dipartimento di Matematica e Fisica, Universit\`a del Salento, Lecce; Italy.\\
$^{66}$$^{(a)}$INFN Sezione di Milano;$^{(b)}$Dipartimento di Fisica, Universit\`a di Milano, Milano; Italy.\\
$^{67}$$^{(a)}$INFN Sezione di Napoli;$^{(b)}$Dipartimento di Fisica, Universit\`a di Napoli, Napoli; Italy.\\
$^{68}$$^{(a)}$INFN Sezione di Pavia;$^{(b)}$Dipartimento di Fisica, Universit\`a di Pavia, Pavia; Italy.\\
$^{69}$$^{(a)}$INFN Sezione di Pisa;$^{(b)}$Dipartimento di Fisica E. Fermi, Universit\`a di Pisa, Pisa; Italy.\\
$^{70}$$^{(a)}$INFN Sezione di Roma;$^{(b)}$Dipartimento di Fisica, Sapienza Universit\`a di Roma, Roma; Italy.\\
$^{71}$$^{(a)}$INFN Sezione di Roma Tor Vergata;$^{(b)}$Dipartimento di Fisica, Universit\`a di Roma Tor Vergata, Roma; Italy.\\
$^{72}$$^{(a)}$INFN Sezione di Roma Tre;$^{(b)}$Dipartimento di Matematica e Fisica, Universit\`a Roma Tre, Roma; Italy.\\
$^{73}$$^{(a)}$INFN-TIFPA;$^{(b)}$Universit\`a degli Studi di Trento, Trento; Italy.\\
$^{74}$Institut f\"{u}r Astro-~und Teilchenphysik, Leopold-Franzens-Universit\"{a}t, Innsbruck; Austria.\\
$^{75}$University of Iowa, Iowa City IA; United States of America.\\
$^{76}$Department of Physics and Astronomy, Iowa State University, Ames IA; United States of America.\\
$^{77}$Joint Institute for Nuclear Research, Dubna; Russia.\\
$^{78}$$^{(a)}$Departamento de Engenharia El\'etrica, Universidade Federal de Juiz de Fora (UFJF), Juiz de Fora;$^{(b)}$Universidade Federal do Rio De Janeiro COPPE/EE/IF, Rio de Janeiro;$^{(c)}$Universidade Federal de S\~ao Jo\~ao del Rei (UFSJ), S\~ao Jo\~ao del Rei;$^{(d)}$Instituto de F\'isica, Universidade de S\~ao Paulo, S\~ao Paulo; Brazil.\\
$^{79}$KEK, High Energy Accelerator Research Organization, Tsukuba; Japan.\\
$^{80}$Graduate School of Science, Kobe University, Kobe; Japan.\\
$^{81}$$^{(a)}$AGH University of Science and Technology, Faculty of Physics and Applied Computer Science, Krakow;$^{(b)}$Marian Smoluchowski Institute of Physics, Jagiellonian University, Krakow; Poland.\\
$^{82}$Institute of Nuclear Physics Polish Academy of Sciences, Krakow; Poland.\\
$^{83}$Faculty of Science, Kyoto University, Kyoto; Japan.\\
$^{84}$Kyoto University of Education, Kyoto; Japan.\\
$^{85}$Research Center for Advanced Particle Physics and Department of Physics, Kyushu University, Fukuoka ; Japan.\\
$^{86}$Instituto de F\'{i}sica La Plata, Universidad Nacional de La Plata and CONICET, La Plata; Argentina.\\
$^{87}$Physics Department, Lancaster University, Lancaster; United Kingdom.\\
$^{88}$Oliver Lodge Laboratory, University of Liverpool, Liverpool; United Kingdom.\\
$^{89}$Department of Experimental Particle Physics, Jo\v{z}ef Stefan Institute and Department of Physics, University of Ljubljana, Ljubljana; Slovenia.\\
$^{90}$School of Physics and Astronomy, Queen Mary University of London, London; United Kingdom.\\
$^{91}$Department of Physics, Royal Holloway University of London, Egham; United Kingdom.\\
$^{92}$Department of Physics and Astronomy, University College London, London; United Kingdom.\\
$^{93}$Louisiana Tech University, Ruston LA; United States of America.\\
$^{94}$Fysiska institutionen, Lunds universitet, Lund; Sweden.\\
$^{95}$Centre de Calcul de l'Institut National de Physique Nucl\'eaire et de Physique des Particules (IN2P3), Villeurbanne; France.\\
$^{96}$Departamento de F\'isica Teorica C-15 and CIAFF, Universidad Aut\'onoma de Madrid, Madrid; Spain.\\
$^{97}$Institut f\"{u}r Physik, Universit\"{a}t Mainz, Mainz; Germany.\\
$^{98}$School of Physics and Astronomy, University of Manchester, Manchester; United Kingdom.\\
$^{99}$CPPM, Aix-Marseille Universit\'e, CNRS/IN2P3, Marseille; France.\\
$^{100}$Department of Physics, University of Massachusetts, Amherst MA; United States of America.\\
$^{101}$Department of Physics, McGill University, Montreal QC; Canada.\\
$^{102}$School of Physics, University of Melbourne, Victoria; Australia.\\
$^{103}$Department of Physics, University of Michigan, Ann Arbor MI; United States of America.\\
$^{104}$Department of Physics and Astronomy, Michigan State University, East Lansing MI; United States of America.\\
$^{105}$B.I. Stepanov Institute of Physics, National Academy of Sciences of Belarus, Minsk; Belarus.\\
$^{106}$Research Institute for Nuclear Problems of Byelorussian State University, Minsk; Belarus.\\
$^{107}$Group of Particle Physics, University of Montreal, Montreal QC; Canada.\\
$^{108}$P.N. Lebedev Physical Institute of the Russian Academy of Sciences, Moscow; Russia.\\
$^{109}$Institute for Theoretical and Experimental Physics (ITEP), Moscow; Russia.\\
$^{110}$National Research Nuclear University MEPhI, Moscow; Russia.\\
$^{111}$D.V. Skobeltsyn Institute of Nuclear Physics, M.V. Lomonosov Moscow State University, Moscow; Russia.\\
$^{112}$Fakult\"at f\"ur Physik, Ludwig-Maximilians-Universit\"at M\"unchen, M\"unchen; Germany.\\
$^{113}$Max-Planck-Institut f\"ur Physik (Werner-Heisenberg-Institut), M\"unchen; Germany.\\
$^{114}$Nagasaki Institute of Applied Science, Nagasaki; Japan.\\
$^{115}$Graduate School of Science and Kobayashi-Maskawa Institute, Nagoya University, Nagoya; Japan.\\
$^{116}$Department of Physics and Astronomy, University of New Mexico, Albuquerque NM; United States of America.\\
$^{117}$Institute for Mathematics, Astrophysics and Particle Physics, Radboud University Nijmegen/Nikhef, Nijmegen; Netherlands.\\
$^{118}$Nikhef National Institute for Subatomic Physics and University of Amsterdam, Amsterdam; Netherlands.\\
$^{119}$Department of Physics, Northern Illinois University, DeKalb IL; United States of America.\\
$^{120}$$^{(a)}$Budker Institute of Nuclear Physics and NSU, SB RAS, Novosibirsk;$^{(b)}$Novosibirsk State University Novosibirsk; Russia.\\
$^{121}$Institute for High Energy Physics of the National Research Centre Kurchatov Institute, Protvino; Russia.\\
$^{122}$Department of Physics, New York University, New York NY; United States of America.\\
$^{123}$Ohio State University, Columbus OH; United States of America.\\
$^{124}$Faculty of Science, Okayama University, Okayama; Japan.\\
$^{125}$Homer L. Dodge Department of Physics and Astronomy, University of Oklahoma, Norman OK; United States of America.\\
$^{126}$Department of Physics, Oklahoma State University, Stillwater OK; United States of America.\\
$^{127}$Palack\'y University, RCPTM, Joint Laboratory of Optics, Olomouc; Czech Republic.\\
$^{128}$Center for High Energy Physics, University of Oregon, Eugene OR; United States of America.\\
$^{129}$LAL, Universit\'e Paris-Sud, CNRS/IN2P3, Universit\'e Paris-Saclay, Orsay; France.\\
$^{130}$Graduate School of Science, Osaka University, Osaka; Japan.\\
$^{131}$Department of Physics, University of Oslo, Oslo; Norway.\\
$^{132}$Department of Physics, Oxford University, Oxford; United Kingdom.\\
$^{133}$LPNHE, Sorbonne Universit\'e, Paris Diderot Sorbonne Paris Cit\'e, CNRS/IN2P3, Paris; France.\\
$^{134}$Department of Physics, University of Pennsylvania, Philadelphia PA; United States of America.\\
$^{135}$Konstantinov Nuclear Physics Institute of National Research Centre "Kurchatov Institute", PNPI, St. Petersburg; Russia.\\
$^{136}$Department of Physics and Astronomy, University of Pittsburgh, Pittsburgh PA; United States of America.\\
$^{137}$$^{(a)}$Laborat\'orio de Instrumenta\c{c}\~ao e F\'isica Experimental de Part\'iculas - LIP;$^{(b)}$Departamento de F\'isica, Faculdade de Ci\^{e}ncias, Universidade de Lisboa, Lisboa;$^{(c)}$Departamento de F\'isica, Universidade de Coimbra, Coimbra;$^{(d)}$Centro de F\'isica Nuclear da Universidade de Lisboa, Lisboa;$^{(e)}$Departamento de F\'isica, Universidade do Minho, Braga;$^{(f)}$Departamento de F\'isica Teorica y del Cosmos, Universidad de Granada, Granada (Spain);$^{(g)}$Dep F\'isica and CEFITEC of Faculdade de Ci\^{e}ncias e Tecnologia, Universidade Nova de Lisboa, Caparica; Portugal.\\
$^{138}$Institute of Physics, Academy of Sciences of the Czech Republic, Prague; Czech Republic.\\
$^{139}$Czech Technical University in Prague, Prague; Czech Republic.\\
$^{140}$Charles University, Faculty of Mathematics and Physics, Prague; Czech Republic.\\
$^{141}$Particle Physics Department, Rutherford Appleton Laboratory, Didcot; United Kingdom.\\
$^{142}$IRFU, CEA, Universit\'e Paris-Saclay, Gif-sur-Yvette; France.\\
$^{143}$Santa Cruz Institute for Particle Physics, University of California Santa Cruz, Santa Cruz CA; United States of America.\\
$^{144}$$^{(a)}$Departamento de F\'isica, Pontificia Universidad Cat\'olica de Chile, Santiago;$^{(b)}$Departamento de F\'isica, Universidad T\'ecnica Federico Santa Mar\'ia, Valpara\'iso; Chile.\\
$^{145}$Department of Physics, University of Washington, Seattle WA; United States of America.\\
$^{146}$Department of Physics and Astronomy, University of Sheffield, Sheffield; United Kingdom.\\
$^{147}$Department of Physics, Shinshu University, Nagano; Japan.\\
$^{148}$Department Physik, Universit\"{a}t Siegen, Siegen; Germany.\\
$^{149}$Department of Physics, Simon Fraser University, Burnaby BC; Canada.\\
$^{150}$SLAC National Accelerator Laboratory, Stanford CA; United States of America.\\
$^{151}$Physics Department, Royal Institute of Technology, Stockholm; Sweden.\\
$^{152}$Departments of Physics and Astronomy, Stony Brook University, Stony Brook NY; United States of America.\\
$^{153}$Department of Physics and Astronomy, University of Sussex, Brighton; United Kingdom.\\
$^{154}$School of Physics, University of Sydney, Sydney; Australia.\\
$^{155}$Institute of Physics, Academia Sinica, Taipei; Taiwan.\\
$^{156}$$^{(a)}$E. Andronikashvili Institute of Physics, Iv. Javakhishvili Tbilisi State University, Tbilisi;$^{(b)}$High Energy Physics Institute, Tbilisi State University, Tbilisi; Georgia.\\
$^{157}$Department of Physics, Technion, Israel Institute of Technology, Haifa; Israel.\\
$^{158}$Raymond and Beverly Sackler School of Physics and Astronomy, Tel Aviv University, Tel Aviv; Israel.\\
$^{159}$Department of Physics, Aristotle University of Thessaloniki, Thessaloniki; Greece.\\
$^{160}$International Center for Elementary Particle Physics and Department of Physics, University of Tokyo, Tokyo; Japan.\\
$^{161}$Graduate School of Science and Technology, Tokyo Metropolitan University, Tokyo; Japan.\\
$^{162}$Department of Physics, Tokyo Institute of Technology, Tokyo; Japan.\\
$^{163}$Tomsk State University, Tomsk; Russia.\\
$^{164}$Department of Physics, University of Toronto, Toronto ON; Canada.\\
$^{165}$$^{(a)}$TRIUMF, Vancouver BC;$^{(b)}$Department of Physics and Astronomy, York University, Toronto ON; Canada.\\
$^{166}$Division of Physics and Tomonaga Center for the History of the Universe, Faculty of Pure and Applied Sciences, University of Tsukuba, Tsukuba; Japan.\\
$^{167}$Department of Physics and Astronomy, Tufts University, Medford MA; United States of America.\\
$^{168}$Department of Physics and Astronomy, University of California Irvine, Irvine CA; United States of America.\\
$^{169}$Department of Physics and Astronomy, University of Uppsala, Uppsala; Sweden.\\
$^{170}$Department of Physics, University of Illinois, Urbana IL; United States of America.\\
$^{171}$Instituto de F\'isica Corpuscular (IFIC), Centro Mixto Universidad de Valencia - CSIC, Valencia; Spain.\\
$^{172}$Department of Physics, University of British Columbia, Vancouver BC; Canada.\\
$^{173}$Department of Physics and Astronomy, University of Victoria, Victoria BC; Canada.\\
$^{174}$Fakult\"at f\"ur Physik und Astronomie, Julius-Maximilians-Universit\"at W\"urzburg, W\"urzburg; Germany.\\
$^{175}$Department of Physics, University of Warwick, Coventry; United Kingdom.\\
$^{176}$Waseda University, Tokyo; Japan.\\
$^{177}$Department of Particle Physics, Weizmann Institute of Science, Rehovot; Israel.\\
$^{178}$Department of Physics, University of Wisconsin, Madison WI; United States of America.\\
$^{179}$Fakult{\"a}t f{\"u}r Mathematik und Naturwissenschaften, Fachgruppe Physik, Bergische Universit\"{a}t Wuppertal, Wuppertal; Germany.\\
$^{180}$Department of Physics, Yale University, New Haven CT; United States of America.\\
$^{181}$Yerevan Physics Institute, Yerevan; Armenia.\\

$^{a}$ Also at Borough of Manhattan Community College, City University of New York, NY; United States of America.\\
$^{b}$ Also at California State University, East Bay; United States of America.\\
$^{c}$ Also at Centre for High Performance Computing, CSIR Campus, Rosebank, Cape Town; South Africa.\\
$^{d}$ Also at CERN, Geneva; Switzerland.\\
$^{e}$ Also at CPPM, Aix-Marseille Universit\'e, CNRS/IN2P3, Marseille; France.\\
$^{f}$ Also at D\'epartement de Physique Nucl\'eaire et Corpusculaire, Universit\'e de Gen\`eve, Gen\`eve; Switzerland.\\
$^{g}$ Also at Departament de Fisica de la Universitat Autonoma de Barcelona, Barcelona; Spain.\\
$^{h}$ Also at Departamento de F\'isica Teorica y del Cosmos, Universidad de Granada, Granada (Spain); Spain.\\
$^{i}$ Also at Departamento de Física, Instituto Superior Técnico, Universidade de Lisboa, Lisboa; Portugal.\\
$^{j}$ Also at Department of Applied Physics and Astronomy, University of Sharjah, Sharjah; United Arab Emirates.\\
$^{k}$ Also at Department of Financial and Management Engineering, University of the Aegean, Chios; Greece.\\
$^{l}$ Also at Department of Physics and Astronomy, University of Louisville, Louisville, KY; United States of America.\\
$^{m}$ Also at Department of Physics and Astronomy, University of Sheffield, Sheffield; United Kingdom.\\
$^{n}$ Also at Department of Physics, California State University, Fresno CA; United States of America.\\
$^{o}$ Also at Department of Physics, California State University, Sacramento CA; United States of America.\\
$^{p}$ Also at Department of Physics, King's College London, London; United Kingdom.\\
$^{q}$ Also at Department of Physics, St. Petersburg State Polytechnical University, St. Petersburg; Russia.\\
$^{r}$ Also at Department of Physics, Stanford University; United States of America.\\
$^{s}$ Also at Department of Physics, University of Fribourg, Fribourg; Switzerland.\\
$^{t}$ Also at Department of Physics, University of Michigan, Ann Arbor MI; United States of America.\\
$^{u}$ Also at Giresun University, Faculty of Engineering, Giresun; Turkey.\\
$^{v}$ Also at Graduate School of Science, Osaka University, Osaka; Japan.\\
$^{w}$ Also at Hellenic Open University, Patras; Greece.\\
$^{x}$ Also at Horia Hulubei National Institute of Physics and Nuclear Engineering, Bucharest; Romania.\\
$^{y}$ Also at II. Physikalisches Institut, Georg-August-Universit\"{a}t G\"ottingen, G\"ottingen; Germany.\\
$^{z}$ Also at Institucio Catalana de Recerca i Estudis Avancats, ICREA, Barcelona; Spain.\\
$^{aa}$ Also at Institut f\"{u}r Experimentalphysik, Universit\"{a}t Hamburg, Hamburg; Germany.\\
$^{ab}$ Also at Institute for Mathematics, Astrophysics and Particle Physics, Radboud University Nijmegen/Nikhef, Nijmegen; Netherlands.\\
$^{ac}$ Also at Institute for Particle and Nuclear Physics, Wigner Research Centre for Physics, Budapest; Hungary.\\
$^{ad}$ Also at Institute of Particle Physics (IPP); Canada.\\
$^{ae}$ Also at Institute of Physics, Academia Sinica, Taipei; Taiwan.\\
$^{af}$ Also at Institute of Physics, Azerbaijan Academy of Sciences, Baku; Azerbaijan.\\
$^{ag}$ Also at Institute of Theoretical Physics, Ilia State University, Tbilisi; Georgia.\\
$^{ah}$ Also at Instituto de Física Teórica de la Universidad Autónoma de Madrid; Spain.\\
$^{ai}$ Also at Istanbul University, Dept. of Physics, Istanbul; Turkey.\\
$^{aj}$ Also at Joint Institute for Nuclear Research, Dubna; Russia.\\
$^{ak}$ Also at LAL, Universit\'e Paris-Sud, CNRS/IN2P3, Universit\'e Paris-Saclay, Orsay; France.\\
$^{al}$ Also at Louisiana Tech University, Ruston LA; United States of America.\\
$^{am}$ Also at LPNHE, Sorbonne Universit\'e, Paris Diderot Sorbonne Paris Cit\'e, CNRS/IN2P3, Paris; France.\\
$^{an}$ Also at Manhattan College, New York NY; United States of America.\\
$^{ao}$ Also at Moscow Institute of Physics and Technology State University, Dolgoprudny; Russia.\\
$^{ap}$ Also at National Research Nuclear University MEPhI, Moscow; Russia.\\
$^{aq}$ Also at Physikalisches Institut, Albert-Ludwigs-Universit\"{a}t Freiburg, Freiburg; Germany.\\
$^{ar}$ Also at School of Physics, Sun Yat-sen University, Guangzhou; China.\\
$^{as}$ Also at The City College of New York, New York NY; United States of America.\\
$^{at}$ Also at The Collaborative Innovation Center of Quantum Matter (CICQM), Beijing; China.\\
$^{au}$ Also at Tomsk State University, Tomsk, and Moscow Institute of Physics and Technology State University, Dolgoprudny; Russia.\\
$^{av}$ Also at TRIUMF, Vancouver BC; Canada.\\
$^{aw}$ Also at Universita di Napoli Parthenope, Napoli; Italy.\\
$^{*}$ Deceased

\end{flushleft}

% Created with Glance <Atlas.Glance@cern.ch>

%-------------------------------------------------------------------------------

%-------------------------------------------------------------------------------
% Auxiliary material - comment out the following line if you do not have any.
\include{atlas-document-auxmat}
%-------------------------------------------------------------------------------

%-------------------------------------------------------------------------------
% Extra tables etc. for HepData - comment in the following line if you have any.
% \include{atlas-document-hepdata}
%-------------------------------------------------------------------------------

\end{document}
