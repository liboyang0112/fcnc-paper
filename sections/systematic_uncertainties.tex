%-------------------------------------------------------------------------------
\section{Systematic uncertainties}
\label{sec:systematics}
%-------------------------------------------------------------------------------
				   
Several sources of systematic uncertainty that can affect the normalisation of signal 
and background and/or the shape of their corresponding discriminant distributions are considered.
Each source is considered to be uncorrelated with the other sources.  
Correlations of a given systematic uncertainty are maintained across processes and channels 
as appropriate.
The following sections describe the systematic uncertainties considered.

\subsection{Luminosity}
\label{sec:syst_lumi}

The uncertainty in the integrated luminosity is 1.7\%, affecting the overall normalisation of all processes estimated from the simulation. 
It is derived, following a methodology similar to that detailed in Ref.~\cite{Aaboud:2016hhf}, and using the LUCID-2 detector 
for the baseline luminosity measurements \cite{Avoni:2018iuv}, from a calibration of the luminosity scale using $x$--$y$ beam-separation scans.

\subsection{Reconstructed objects}
\label{sec:syst_objects}

Uncertainties associated with electrons, muons, and $\had$ candidates arise from the trigger, reconstruction,  
identification and isolation (in the case of electrons and muons) efficiencies, as well as the momentum scale and resolution. 
These are measured using $Z\to \ell^+\ell^-$ and $J/\psi\to \ell^+\ell^-$ events ($\ell =e, \mu$)~\cite{ATLAS-CONF-2016-024,Aad:2016jkr} 
in the case of electrons and muons, and using $Z\to \tau^+\tau^-$ events in the case of $\had$ candidates~\cite{ATLAS-CONF-2017-029}.

Uncertainties associated with jets arise from the jet energy scale
and resolution, and the efficiency to pass the JVT requirements. 
The largest contribution results from the jet energy scale, whose uncertainty dependence on jet $\pt$ and $\eta$, jet flavour, and pile-up treatment, 
is split into 43 uncorrelated components that are treated independently~\cite{Aaboud:2017jcu}. The total JES uncertainty is
below 5\% for most jets and below 1\% for central jets with pT between 300 GeV and 2 TeV. The difference between the JER
in data and MC is represented by one NP. It is applied on the MC by smearing the jet pT within the prescribed uncertainty.

Uncertainties associated with energy scales and resolutions of leptons and jets 
are propagated to $\met$. Additional uncertainties originating from the modelling 
of the underlying event, in particular its impact on the $\pt$ scale and resolution 
of unclustered energy, are negligible.

Efficiencies to tag $b$-jets and $c$-jets in the simulation are corrected to match the efficiencies in data by $\pt$-dependent factors,
whereas the light-jet efficiency is scaled by $\pt$- and $\eta$-dependent factors.
The $b$-jet efficiency is measured in a data sample enriched in $\ttbar$ events~\cite{Aad:2019epj79}, %EPJ C79(2019)970 {Aaboud:2018xwy},
  while the $c$-jet efficiency is measured
using $\ttbar$ events~\cite{ATLAS-CONF-2018-001} or $W$+$c$-jet events~\cite{Aad:2015ydr}. 
The light-jet efficiency is measured in a multi-jet data sample enriched in light-flavour jets~\cite{ATLAS-CONF-2018-006}.
The uncertainties in these scale factors include a total of 44 independent sources affecting $b$-jets, 19 source affecting $c$-jets, and 19 sources affecting light-jets. 
These systematic uncertainties are taken as uncorrelated between $b$-jets, $c$-jets, and light-jets.

The uncertainty on the pileup reweighing is evaluated by varying the pileup scale factors
by 1$\sigma$ based on the reweighing of the average interactions per bunch crossing. However, this
uncertainty is highly correlated with the luminosity uncertainty and may be an overestimate.

\subsection{Background modelling}
\label{sec:syst_bkgmodeling}

A number of sources of systematic uncertainty affecting the modelling of $t\bar{t}$+jets are considered: the choice of the renormalisation and factorisation scale in the matrix-element calculation, the choice of the matching scale when matching the matrix elements to the parton show generator, the uncertainty in the value of $\alpha_s$ when modeling initial-state radiation (ISR), the choice of the renormalisation scale when modeling final-state radiation (FSR).
%%An uncertainty of  6\% is assigned to the inclusive $\ttbar$ production
%%cross section~\cite{Czakon:2011xx}, including contributions from varying the factorisation and renormalisation 
%%scales, as well as from the top-quark mass, the PDF and $\alpha_{\textrm{S}}$. The latter two represent the largest contribution 
%%to the overall theoretical uncertainty in the cross section and were calculated using the PDF4LHC prescription~\cite{Botje:2011sn} 
%%with the NNPDF3.0(NLO), CT10 NNLO~\cite{Lai:2010vv,Gao:2013xoa} and NNPDF2.3(LO) 5F FFN~\cite{Ball:2012cx} PDF sets.

The hdamp parameter (which controls the amount of radiation produced by the parton shower in
POWHEG-BOX v2) is set to 1.5$m_{top}$ in the nominal case. Alternative samples are generated with hdamp=3$m_{top}$. The
difference between two samples is treated as one of the systematics as ``$t\bar{t}$ hdamp''. 
%The uncertainty associated with the choice of NLO generator is derived by comparing the nominal prediction from
%{\powheg}+{\pythiaeight} with a prediction from \textsc{Sherpa}~2.2.1. For the latter, the matrix-element calculation is performed 
%for up to two partons at NLO and up to four partons at LO using \textsc{Comix} and \textsc{OpenLoops}, and
%merged with the {\sherpa} parton shower using the ME+PS@NLO prescription.

The uncertainty due to the choice of parton shower and hadronisation (PS \& Had) model is derived 
by comparing the predictions from {\powheg} interfaced either to {\pythiaeight} or {\herwig7}.
The latter uses the MMHT2014 LO~\cite{Harland-Lang:2014zoa} PDF set in combination with the H7UE tune~\cite{Bellm:2015jjp}.
The uncertainty in the modelling of additional radiation from the PS are assessed by
varying the corresponding parameter of the A14 set~\cite{ATL-PHYS-PUB-2016-004} and by varying the radiation renormalisation and factorisation scales
by a factor of 2.0 and 0.5, respectively. 
%The uncertainty in the modelling of additional radiation is assessed with two alternative {\powheg}+{\pythiaeight} samples:
%a sample with increased radiation (referred to as radHi) is obtained by decreasing the renormalisation and factorisation scales  
%by a factor of two, %doubling the $h_{\textrm{damp}}$ parameter,
%and using the Var3c upward variation of the A14 parameter set;
%a sample with decreased radiation (referred to as radLow) is obtained by increasing the scales by a factor of two 
%and using the Var3c downward variation of the A14 set~\cite{ATL-PHYS-PUB-2016-004}.


Another significant background in hadronic channel stems from the $Z\rightarrow \tau\tau$ samples, several sources of uncertainty are
considered for these samples: the PDF variation considering the standard deviation of the 100 NNPDF replicas event weights
of NNPDF3.0nnlo~\cite{Ball:2015NNPDF} PDF
set used in Sherpa, renormalisation ($\mu_{R}$) and factorisation scales ($\mu_{F}$), jet-to-parton matching uncertainty, resummation scale uncertainty,
variation in the choice of $\alpha_{S}$, alternative PDF variation evaluated comparing predictions from NNPDF3.0nnlo PDF set (nominal)
with MMHT2014nnlo68cl and CT14nnlo~\cite{Lai:2010vv,Gao:2013xoa} PDF sets.

Uncertainties affecting the normalisation of the $V$+jets background are estimated separately for $V$+light-jets, $V$+$\geq$1$c$+jets,
and $V$+$\geq$1$b$+jets subprocesses. The total normalisation uncertainty of $V$+jets processes is estimated approximately 30\% by comparing the
data and total background prediction in the different analysis regions considered, but requiring exactly zero $b$-tagged jets.
%Agreement between data and predicted background 
%in these modified regions, which are dominated by $V$+light-jets, is found to be within approximately 30\%. This bound is taken to 
%be the normalisation uncertainty, correlated across all $V$+jets subprocesses. 
%Since {\sherpa}~2.2 has been found to underestimate $V$+heavy-flavour production by about a factor
%of 1.3~\cite{Aaboud:2017xsd}, additional 30\% normalisation uncertainties are assumed for $V$+$\geq$1$c$+jets and $V$+$\geq$1$b$+jets
%subprocesses, considered uncorrelated between them.

%To estimate the uncertainty originating from ISR modelling in the single top samples, the same procedure as $t\bar{t}$ is used. One NP reflects the symmetrised effect of the two shower variations Var3cDown and Var3cUp, and the other NP with name ``scale'' describe the symmetrised effect from the independent variations of the renormalisation and factorisation scale. The impact of the uncertainty from FSR modelling is estimated by reweighing the nominal single top sample. The two variations $\mu^{FSR}_{R} \times 0.5$ and $\mu^{FSR}_{R} \times 2$ are considered.  The standard deviation of 100 NNPDF3.0nnlo variations is calculated Uncertainties on the PDF are evaluated following the recommended prescription in Ref.~\cite{ttRun2}.

%The values of the scale-parameters $\mu_{f}$ and $\mu_{f}$ are varied by factors of 2.0 and 0.5 with respect to their default values to obtain the uncertainties related to the renormalization and factorization scales in ttV samples. Uncertainties on the PDF are evaluated following the same procedure in Ref.~\cite{ttZRun2}.

%Following the recommendations of the PMG group, the uncertainties to be considered for diboson background include~\cite{dibosonRes} scale variations of the renormalization and factorization scale,
%PDF variations and $\alpha_{S}$ variations.

%A $+5.8\%-9.2\%$ normalization uncertainty is considered for the ttH background, corresponding to the scale and $\alpha_{S}$ uncertainties in the NLO cross-section computation. A constant PDF uncertainty of $\pm3.6$\% is assumed, as it was done for the previous measurement. Therefore, a overall systematics with variation 9\% is entered in the final fit model~\cite{ttZRun2}.

Uncertainties affecting the modelling of the single-top-quark background include a 
$+5\%$/$-4\%$ uncertainty of the total cross section estimated as a weighted average 
of the theoretical uncertainties in $t$-, $tW$- and $s$-channel production~\cite{Kidonakis:2011wy,Kidonakis:2010ux,Kidonakis:2010tc}.
Additional uncertainties associated with the parton shower, hadronisation and ISR/FSR radiations are also considered using the same procedure
as in $t\bar t$. Uncertainties of the diboson background normalisation include variations of the renormalization and factorization scale, NNPDF3.0nnlo variations and $\alpha_{S}$ variations. Uncertainties of the $\ttbar V$ and $\ttbar H$ cross sections are 12\% and $+5.8\%-9.2\%$, respectively,
from the uncertainties of their respective NLO theoretical cross sections~\cite{ttZRun2}. 

%by comparing the nominal
%%%samples with alternative samples where generator parameters are varied.
%%%For the $t$- and $tW$-channel processes, an uncertainty due to the choice of parton shower and hadronisation model is derived 
%%%by comparing events produced by {\powheg} interfaced to {\pythia}~8 or {\herwigpp}.
%%%These uncertainties are treated as fully correlated among single-top-quark production processes, but uncorrelated with the
%%%corresponding uncertainty of the $\ttbar$+jets background.
%%%An additional systematic uncertainty in $tW$-channel production concerning the separation 
%%%between $t\bar{t}$ and $tW$ at NLO is assessed by comparing
%%%the nominal sample, which uses the diagram removal scheme~\cite{Frixione:2008yi}, with an alternative sample
%%%using the diagram subtraction scheme~\cite{Frixione:2008yi}.

%%%%(it is assumed that two jets originate from the $W/Z$ decay, as in $WW/WZ \to \ell \nu jj$). 
%%%%Therefore, the total normalisation uncertainty is $5\% \oplus \sqrt{N-2}\times 24\%$, where $N$ is the selected jet multiplicity,  
%%%%resulting in 34\%, 42\%, and 48\%, for events with exactly 4 jets, exactly 5 jets, and $\geq$6 jets, respectively. 
%%%%Recent comparisons between data and {\sherpa}~2.1.1 for $WZ(\to \ell\nu\ell\ell) + \geq$4 jets show
%%%%agreement within the experimental uncertainty of approximately 40\%~\cite{Aaboud:2016yus}, which further justifies the above uncertainty.
%%%%Given the very small contribution of this background to the total prediction, the final result is not affected by the assumed modelling
%%%%uncertainties.
%%%%



The statistical uncertainties on the fake $\had$ background calibration in the leptonic channel are applied with uncorrelated uncertainties for different fake $\had$ sources and
different $\pt$ slices. The uncertainty of the ABCD method is applied with a normalisation factor for muon and electron respectively considering both,
its statistical fluctuation and the differences between signal regions.
The uncertainties in the fake factor method applied in hadronic channel includes statistical uncertainties of each fake factor and different sets of fake factors derived from different signal depleted CRs.

\subsection{Signal modelling}
\label{sec:syst_sigmodeling}

Several normalisation and shape uncertainties are taken into account for the $\Hq$ and $pp\to tH$ signals.
Uncertainties of the Higgs boson branching ratios are taken into account
following the recommendation in Ref.~\cite{deFlorian:2016spz}.
The uncertainty of ISR, FSR, scale and PDF are considered in all SRs. The parton shower uncertainties are estimated by comparing
the nominal to an alternative sample interfaced with {\herwig7}.
%the predictions from {\powheg} interfaced either to {\pythiaeight} or {\herwig7}. 
%treated correlatedly with $\ttbar$ samples. 



\begin{table}[htbp]
\caption{List of relative uncertainties of the signal strength from the combined fit in the hadronic channels. The uncertainties are symmetrised for presentation and grouped into the categories described in the text. %The quadrature sum of the individual uncertainties is not equal to the total uncertainty due to correlations introduced by the fit
}
\small
\centering
\begin{tabular}{cc} \toprule\toprule
$\text{Uncertainty}    $   & $\Delta\mu/\mu[\%]$ \\\midrule
$\text{Fake modelling} $   & $13.5$ \\
$\text{Instrument}     $   & $2.6$  \\
$\text{Jets}           $   & $1.7$   \\
$\text{Met}            $   & $0.4$   \\
$\tau                  $   & $7.2$    \\
$\text{b-tagging}       $  & $0.3$    \\
$\text{Signal Theory}  $   & $1.5$   \\
$\text{Other MC theory} $  & $2.7$    \\
$t\bar{t} \text{theory}$  & $7.6$     \\\midrule
$\text{Statistics}      $  & $11.2$   \\
$\text{Total systematic}$  & $24.2$  \\ \midrule
$\text{Total}  $           & $26.7$ \\
\bottomrule\bottomrule
\end{tabular}
\label{tab:had_sys_impact}
\end{table} 


\begin{table}[htbp]
\caption{List of relative uncertainties of the signal strength from the combined fit in the leptonic channels. The uncertainties are symmetrised for presentation and grouped into  the categories described in the text. %The quadrature sum of the individual uncertainties is not equal to the total uncertainty due to correlations introduced by the fit
}
\small
\centering
\begin{tabular}{cc} \toprule\toprule
$\text{Uncertainty}        $ & $  \Delta\mu/\mu[\%]  $\\\midrule
$\text{Fake modelling}     $ & $       2.1             $\\
$\text{Instrument}         $ & $       0.7             $\\
$\text{Jets}               $ & $       2.8             $\\
$\text{Met}                $ & $       1.2             $\\
$\text{Lepton}             $ & $       0.7             $\\
$\tau                      $ & $       2.3             $\\
$\text{b-tagging}          $ & $       4.8             $\\
$\text{Signal Theory}      $ & $       6.9             $\\
$\text{Other MC theory}    $ & $       2.1             $\\
$t\bar{t} \text{theory}    $ & $       3.7             $\\\midrule
$\text{Statistics}         $ & $       5.9             $\\
$\text{Total systematic}   $ & $       12.9            $\\\midrule
$\text{Total}              $ & $       14.2            $\\
\bottomrule\bottomrule
\end{tabular}
\label{tab:lep_sys_impact}
\end{table} 





