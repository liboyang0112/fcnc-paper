%-------------------------------------------------------------------------------
\section{Systematic uncertainties}
\label{sec:systematics}
%-------------------------------------------------------------------------------
				   
Several sources of systematic uncertainty are considered that can affect the normalisation of signal 
and background and/or the shape of their corresponding final discriminant distributions.  
Each source is considered to be uncorrelated with the other sources.  
Correlations of a given systematic uncertainty are maintained across processes and channels 
as appropriate.
The following sections describe the systematic uncertainties considered.

%Table~\ref{tab:SystSummary} presents a list of all systematic uncertainties considered in the analysis 
%and indicates whether they are taken to be normalisation-only, or to affect both shape and normalisation.
%
%%%%%%%%%%%%%%%
%\begin{table*}[htbp]
%\centering
%\begin{tabular}{lcc}
%\toprule\toprule
%Systematic uncertainty & Type  & Components \\
%\midrule
%\textbf{Luminosity}                  &  N & 1\\\midrule
%\textbf{Reconstructed Objects}                 &   & \\
%Electron                  & SN & 5 \\
%Muon                      &  SN & 6 \\
%Jet reconstruction      & SN & 1\\ 
%Jet vertex fraction         & SN    & 1\\
%Jet energy scale            & SN & 22\\
%Jet energy resolution       & SN & 1\\
%Missing transverse momentum  & SN & 2\\ 
%$b$-tagging efficiency      & SN & 6\\
%$c$-tagging efficiency      & SN & 4\\
%Light-jet tagging efficiency    & SN & 12\\ 
%High-\pt\ tagging   & SN & 1 \\ \midrule
%\textbf{Background Model}                 &   & \\
%$t\bar{t}$ cross section    &  N & 1\\
%$t\bar{t}$ modelling: $\pt$ reweighting   & SN & 9\\
%$t\bar{t}$ modelling: parton shower & SN & 3\\
%$t\bar{t}$+HF: normalisation & N & 2 \\
%$t\bar{t}$+$c\bar{c}$: $\pt$ reweighting  & SN & 2 \\
%$t\bar{t}$+$c\bar{c}$: generator & SN & 4 \\
%$t\bar{t}$+$b\bar{b}$: NLO shape & SN & 8 \\
%$W$+jets normalisation      &  N & 3\\
%$W$ $\pt$ reweighting     &  SN & 1\\
%$Z$+jets normalisation      &  N & 3\\
%$Z$ $\pt$ reweighting     &  SN & 1\\
%Single top normalisation    &  N & 3\\
%Single top model            &  SN & 1\\
%Diboson normalisation  &  N & 3\\
%$t\bar{t}V$ cross section   &  N & 1\\
%$t\bar{t}V$ model           &  SN & 1\\ 
%$t\bar{t}H$ cross section & N & 1 \\
%$t\bar{t}H$ model       & SN & 2 \\ 
%Multijet normalisation  &  N & 4\\ \midrule
%\textbf{Signal Model}                 &   & \\
%$t\bar{t}$ cross section    &  N & 1\\
%Higgs boson branching ratios & N & 3 \\
%$t\bar{t}$ modelling: $\pt$ reweighting   & SN & 9\\
%$t\bar{t}$ modelling: $\pt$ reweighting non-closure  & N & 1\\
%$t\bar{t}$ modelling: parton shower & N & 1\\
%\bottomrule\bottomrule
%\end{tabular}
%\caption{\label{tab:SystSummary} List of systematic uncertainties considered. 
%An ``N'' means that the uncertainty is taken as affecting only the normalisation for all relevant 
%processes and channels, whereas ``SN'' means that the uncertainty is 
%taken on both shape and normalisation.
%Some of the systematic uncertainties are split into several components for a more
%accurate treatment.}
%\end{table*}
%%%%%%%%%%%%%%%
%
%The leading sources of systematic uncertainty vary depending on the analysis channel considered, but they 
%typically originate from $\ttbar$+jets modelling (including $\ttbar$+HF) and $b$-tagging. 
%For example, the total systematic uncertainty  in the background normalisation in the (4 j, 4 b) channel, 
%which dominates the sensitivity in the case of the $\Hc$ search, is approximately 20\%, 
%with the largest contributions originating from $\ttbar$+HF normalisation, $b$-tagging efficiency, 
%$c$-tagging efficiency, light-jet tagging efficiency and $\ttbar$ cross section.
%However, as shown in section~\ref{sec:result}, the fit to data in the nine analysis channels allows the overall background 
%uncertainty to be reduced significantly, to approximately 4.4\%. 
%The reduced uncertainty results from the significant constraints provided by the data on some
%systematic uncertainties, as well as the anti-correlations among sources of systematic uncertainty resulting
%from the fit to the data.
%The total systematic uncertainty on the $\Hc$ signal normalisation in the (4 j, 4 b) channel is approximately 17\%, 
%with similar contributions from uncertainties related to $b$-tagging and overall signal modelling. After the fit,
%this uncertainty is reduced to 7.8\%.
%Table~\ref{tab:SystSummary_WbHc} presents a summary of the sys\-te\-ma\-tic uncertainties for the $\Hc$ 
%search and their impact on the normalisation of the signal and the main backgrounds in the (4 j, 4 b) channel.
%
%The following sections describe each of the systematic uncertainties considered in the analyses. 
%
%\begin{table*}[htbp]
%\centering
%\begin{tabular}{l | c c c c  | c c c c}
%\toprule\toprule
% & \multicolumn{4}{c|}{Pre-fit} & \multicolumn{4}{c}{Post-fit} \\ 
% &  $WbHc$ & $t\bar{t}$+LJ & $t\bar{t}+c\bar{c}$ & $t\bar{t}+b\bar{b}$ &  $WbHc$ & $t\bar{t}$+LJ & $t\bar{t}+c\bar{c}$ & $t\bar{t}+b\bar{b}$ \\
%\midrule
%Luminosity  & $\pm 2.8 $  & $\pm 2.8 $  & $\pm 2.8 $  & $\pm 2.8 $  & $\pm 2.6$ & $\pm 2.6 $  & $\pm 2.6 $  & $\pm 2.6 $ \\ 
%Lepton efficiencies  & $\pm 1.5 $  & $\pm 1.5 $  & $\pm 1.5 $  & $\pm 1.5 $  & $\pm 1.5$ &  $\pm 1.5 $  & $\pm 1.5 $  & $\pm 1.5 $ \\ 
%Jet energy scale  & $\pm 3.3 $  & $\pm 2.9 $  & $\pm 2.3 $  & $\pm 5.8 $  & $\pm 1.4$ &  $\pm 1.2 $  & $\pm 1.8 $  & $\pm 4.1 $ \\ 
%Jet efficiencies  & $\pm 1.2 $  & --  & $\pm 1.9 $  & $\pm 1.7 $  & $\pm 0.9$ &  --  & $\pm 1.4 $  & $\pm 1.2 $ \\
%Jet energy resolution  & --  & $\pm 1.2 $  & $\pm 2.8 $  & $\pm 2.9 $  & -- &  -- & $\pm 1.0 $  & $\pm 1.1 $ \\ 
%$b$-tagging eff.  & $\pm 7.9 $  & $\pm 5.5 $  & $\pm 5.2 $  & $\pm 10 $  & $\pm 5.7$ & $\pm 3.9 $  & $\pm 3.7 $  & $\pm 6.6 $ \\ 
%$c$-tagging eff.  & $\pm 7.0 $  & $\pm 6.6 $  & $\pm 13 $  & $\pm 3.5 $  & $\pm 6.3$ &  $\pm 6.0 $  & $\pm 11 $  & $\pm 3.2 $ \\ 
%Light-jet tagging eff.  & $\pm 0.8 $  & $\pm 18 $  & $\pm 3.2 $  & $\pm 1.5 $  & $\pm 0.6$ &  $\pm 13 $  & $\pm 2.3 $  & $\pm 1.1 $ \\
%$t\bar{t}$: reweighting  & $\pm 5.9 $  & $\pm 2.7 $  & $\pm 4.2 $  & --  & $\pm 3.8$ &  $\pm 1.9 $  & $\pm 2.3 $  & -- \\ 
%$t\bar{t}$: parton shower  & $\pm 5.4 $  & $\pm 4.8 $  & $\pm 10 $  & $\pm 4.9 $  & $\pm 1.7$ &  $\pm 1.5 $  & $\pm 6.5 $  & $\pm 3.1 $ \\ 
%$t\bar{t}$+HF: normalisation  & --  & --  & $\pm 50 $  & $\pm 50 $  & -- &  --  & $\pm 32 $  & $\pm 16 $ \\ 
%$t\bar{t}$+HF: modelling  & --  & --  & --  & $\pm 7.7 $  & -- &  --  & --  & $\pm 7.4 $ \\ 
%Signal modelling  & $\pm 6.9 $  & --  & --  & --  & $\pm 6.9$ &  --  & --  & -- \\ 
%Theor. cross sections  & $\pm 6.2 $  & $\pm 6.2 $  & $\pm 6.2 $  & $\pm 6.2 $  & $\pm 3.9$ &  $\pm 3.9 $  & $\pm 3.9 $  & $\pm 3.9 $ \\ 
%\midrule                                                                                                                                                                                               
%Total   & $\pm 17 $  & $\pm 22 $  & $\pm 54 $  & $\pm 53 $  & $\pm 7.8$ &  $\pm 14 $  & $\pm 28 $  & $\pm 15 $ \\ 
%\bottomrule\bottomrule
%\end{tabular}
%\caption{$\Hcbb$ search: summary of the systematic uncertainties considered in the (4 j, 4 b) 
%channel and their impact (in \%) on the normalisation of the signal and the main backgrounds, before and after the fit to data.
%The $\Hc$ signal and the $t\bar{t}$+light-jets background are denoted by ``$WbHc$''  and ``$t\bar{t}$+LJ'' respectively.
%Only sources of systematic uncertainty resulting in a normalisation change of at least 0.5\% are displayed.
%The total post-fit uncertainty can differ from the sum in quadrature of individual sources due to the 
%anti-correlations between them resulting from the fit to the data.}
%\label{tab:SystSummary_WbHc}
%\end{table*}
%%%%%%%%%%%

\subsection{Luminosity}
\label{sec:syst_lumi}

The uncertainty in the integrated luminosity is 2.1\%, affecting the overall normalisation of
all processes estimated from the simulation. It is derived, following a methodology similar to that detailed in Ref.~\cite{Aaboud:2016hhf}, 
from a calibration of the luminosity scale using $x$--$y$ beam-separation scans performed in August 2015 and May 2016.

\subsection{Reconstructed objects}
\label{sec:syst_objects}

%Uncertainties associated with electrons and muons arise from the trigger, reconstruction, identification, and isolation
%efficiencies, as well as the lepton momentum scale and resolution. These are measured in data using 
%$Z\to \ell^+\ell^-$ and $J/\psi\to \ell^+\ell^-$ events~\cite{ATLAS-CONF-2016-024,Aad:2016jkr}. 
%The combined effect of all these uncertainties results in an overall normalisation 
%uncertainty in signal and background of approximately 1\%.

Uncertainties associated with electrons, muons, and $\tauhad$ candidates arise from the trigger, reconstruction,  
identification and isolation (in the case of electrons and muons) efficiencies, as well as the momentum scale and resolution. 
These are measured in data using $Z\to \ell^+\ell^-$ and $J/\psi\to \ell^+\ell^-$ events ($\ell =e, \mu$)~\cite{ATLAS-CONF-2016-024,Aad:2016jkr} 
in the case of electrons and muons, and using $Z\to \tau^+\tau^-$ events in the case of $\tauhad$ candidates~\cite{ATLAS-CONF-2017-029}.

Uncertainties associated with jets arise from the jet energy scale
and resolution, and the efficiency to pass the JVT requirements. 
The largest contribution results from the jet energy scale, whose uncertainty dependence on jet $\pt$ and $\eta$, jet flavour, and pile-up treatment, 
is split into 21 uncorrelated components that are treated independently~\cite{Aaboud:2017jcu}.  

Uncertainties associated with energy scales and resolutions of leptons and jets 
are propagated to $\met$. Additional uncertainties originating from the modelling 
of the underlying event, in particular its impact on the $\pt$ scale and resolution 
of unclustered energy, are negligible.

Efficiencies to tag jets from $b$- and $c$-quarks in the simulation are corrected to match the efficiencies in data by $\pt$-dependent factors,
whereas the light-jet efficiency is scaled by $\pt$- and $\eta$-dependent factors.
The $b$-jet efficiency is measured in a data sample enriched in $\ttbar$ events~\cite{Aaboud:2018xwy}, while the $c$-jet efficiency is measured
using $\ttbar$ events~\cite{ATLAS-CONF-2018-001} or $W$+$c$-jet events~\cite{Aad:2015ydr}. 
The light-jet efficiency is measured in a multijet data sample enriched in light-flavour jets~\cite{ATLAS-CONF-2018-006}.
Since the $\ttbar$ sample used to measure the $c$-jet tagging efficiency overlaps with the analysis sample, the $\Hbb$ search uses
instead the $W$+$c$-jet based scale factors.  
In the case of the $\Hbb$ ($\Htautau$) search, the uncertainties on these scale factors include a total of six independent sources affecting $b$-jets, one (three) source(s) affecting $c$-jets, and 17 sources affecting light jets. 
These systematic uncertainties are taken as uncorrelated between $b$-jets, $c$-jets, and light-jets. 
An additional uncertainty is included due to the extrapolation of these corrections to jets 
with $\pt$ beyond the kinematic reach of the data calibration samples used ($\pt>300~\gev$ for $b$- and $c$-jets, 
and $\pt>750~\gev$ for light-jets); it is taken to be correlated among the three jet flavours. 
%This uncertainty is evaluated in the simulation by comparing the tagging efficiencies while varying e.g. the fraction 
%of tracks with shared hits in the silicon detectors or the fraction of fake tracks resulting from random combinations of hits, 
%both of which typically increase at high $\pt$ due to growing track multiplicity and density of hits within the jet. 
Finally, an uncertainty related to the application of $c$-jet scale factors to $\tau$-jets is considered,
but has a negligible impact in these analyses.

\subsection{Background modelling}
\label{sec:syst_bkgmodeling}

A number of sources of systematic uncertainty affecting the modelling of $t\bar{t}$+jets are considered. 
An uncertainty of  6\% is assigned to the inclusive $\ttbar$ production
cross section~\cite{Czakon:2011xx}, including contributions from varying the factorisation and renormalisation 
scales, as well as from the top quark mass, the PDF and $\alpha_{\textrm{S}}$. The latter two represent the largest contribution 
to the overall theoretical uncertainty in the cross section and were calculated using the PDF4LHC prescription~\cite{Botje:2011sn} 
with the MSTW 2008 68\% CL NNLO, CT10 NNLO~\cite{Lai:2010vv,Gao:2013xoa} and NNPDF2.3 5F FFN~\cite{Ball:2012cx} PDF sets.
The uncertainty associated with the choice of NLO generator is derived by comparing the nominal prediction from
{\powheg}+{\pythiaeight} with a prediction from \textsc{Sherpa}~2.2.1. For the latter, the matrix element calculation is performed 
for up to two partons at NLO and up to four partons at LO using \textsc{Comix} and \textsc{OpenLoops}, and
merged with the {\sherpa} parton shower using the ME+PS@NLO prescription.
%either {\amcatnlo}+{\pythiaeight} (in the case of the $\Htautau$ search) or
%\textsc{Sherpa} (in the case of the $\Hbb$ search). 
The uncertainty due to the choice of parton shower and hadronisation (PS \& Had) model is derived 
by comparing the predictions from {\powheg} interfaced either to {\pythiaeight} or {\herwig7}. 
The latter uses MMHT2014 LO~\cite{Harland-Lang:2014zoa} PDF set in combination with the H7UE tune~\cite{Bellm:2015jjp}.
Additionally, the uncertainty on the modelling of additional radiation is assessed with two alternative {\powheg}+{\pythiaeight} samples:
a sample with increased radiation (referred to as ``radHi'') is obtained by decreasing the renormalisation and factorisation scales  
by a factor of two, doubling the $h_{\textrm{damp}}$ parameter, and using the Var3c upward variation of the A14 parameter set;
a sample with decreased radiation (referred to as ``radLow'') is obtained by increasing the scales by a factor of two 
and using the Var3c downward variation of the A14 set~\cite{ATL-PHYS-PUB-2016-004}.

In the case of the $\Hbb$ search, where the $\ttbar$+HF background plays a more prominent role, a more 
detailed treatment of its associated systematic uncertainties is made. In particular, since several analysis 
regions have a sufficiently large number of events of \ttbin\ background, its normalisation is 
determined in the fit to data.
%, with no prior uncertainty assumed. 
In the case of the \ttcin\ normalisation, an uncertainty of 50\% is assumed, as the fit to the data is unable 
to precisely determine it, and the analysis has very limited sensitivity to this uncertainty.
Since the diagrams that contribute to $\ttbar$+light-jets, \ttcin, \ttbin\
production are different, all above uncertainties on $\ttbar$+jets
background modelling (NLO generator, PS \& Had, and radHi/radLow), except the uncertainty on the inclusive cross-section, are
considered to be uncorrelated among these processes.
Additional uncertainties on the \ttbin\ background are considered associated with the NLO prediction from {\ShOL}, 
which is used for reweighting the nominal {\powheg}+{\pythiaeight} prediction. 
These include three different scale variations,  a different shower-recoil model scheme, and 
two alternative PDF sets (MSTW 2008 NLO and NNPDF2.3 NLO). Additional uncertainties are assessed for
the contributions to the \ttbin\ background originating from multiple parton interactions.
Finally, an extra uncertainty on the \ttbin\ background is assigned by comparing 
the predictions from {\powheg}+{\pythiaeight} and {\ShOL} 4F (5F vs 4F).
In the derivation of the above uncertainties, the overall normalisations of the \ttcin\ and \ttbin\ backgrounds 
at the particle level are fixed to the nominal prediction. In order to maintain the inclusive $\ttbar$ cross section, 
the normalisation of the $\ttbar$+light-jets background at the particle level is adjusted accordingly.
Further details are available in Ref.~\cite{Aaboud:2017rss}.

Uncertainties affecting the normalisation of the $V$+jets background are estimated for the sum
of $W$+jets and $Z$+jets, and separately for $V$+light-jets, $V$+$\geq$1$c$+jets, and $V$+$\geq$1$b$+jets subprocesses.
The total normalisation uncertainty of $V$+jets processes is estimated by comparing the data and total background prediction in 
the different analysis regions considered, but requiring exactly zero $b$-tagged jets. Agreement between data and predicted background 
in these modified regions, which are dominated by $V$+light-jets, is found to be within approximately 30\%. This bound is taken to 
be the normalisation uncertainty, correlated across all $V$+jets subprocesses. 
Since {\sherpa}~2.2 has been found to underestimate $V$+heavy-flavour by about a factor
of 1.3~\cite{Aaboud:2017xsd}, additional 30\% normalisation uncertainties are assumed for $V$+$\geq$1$c$+jets and $V$+$\geq$1$b$+jets
subprocesses, considered uncorrelated between them.

Uncertainties affecting the modelling of the single-top-quark background include a 
$+5\%$/$-4\%$ uncertainty in the total cross section estimated as a weighted average 
of the theoretical uncertainties in $t$-, $tW$- and $s$-channel production~\cite{Kidonakis:2011wy,Kidonakis:2010ux,Kidonakis:2010tc}.
Additional uncertainties associated with the modelling of additional radiation are assessed by comparing the nominal
samples with alternative samples where generator parameters are varied.
For the $t$- and $tW$-channel processes, an uncertainty due to the choice of parton shower and hadronisation model is derived 
by comparing events produced by {\powheg} interfaced to {\pythia}~6 or {\herwigpp}.
These uncertainties are treated as fully correlated among single-top-quark production processes, but uncorrelated with the
corresponding uncertainty in the $\ttbar$+jets background.
An additional systematic uncertainty on $tW$-channel production concerning the separation 
between $t\bar{t}$ and $tW$ at NLO is assessed by comparing
the nominal sample, which uses the so-called ``diagram removal'' scheme~\cite{Frixione:2008yi}, with an alternative sample
using the ``diagram subtraction'' scheme~\cite{Frixione:2008yi}.

Uncertainties on the diboson background normalisation include 5\% from the NLO theory cross sections~\cite{Campbell:1999ah},
as well as an additional 24\% normalisation uncertainty added in quadrature for each additional inclusive jet-multiplicity bin, based on a 
comparison among different algorithms for merging LO matrix elements and parton showers~\cite{Alwall:2007fs}
(it is assumed that two jets originate from the $W/Z$ decay, as in $WW/WZ \to \ell \nu jj$). 
Therefore, the total normalisation uncertainty is $5\% \oplus \sqrt{N-2}\times 24\%$, where $N$ is the selected jet multiplicity,  
i.e. 34\%, 42\%, and 48\%, for events with exactly 4 jets, exactly 5 jets, and $\geq$6 jets, respectively.
Recent comparisons between data and {\sherpa}~2.1.1 for $WZ(\to \ell\nu\ell\ell) + \geq$4 jets show
agreement within the experimental uncertainty of approximately 40\%~\cite{Aaboud:2016yus}, which further justifies the above uncertainty.

Uncertainties in the $\ttbar V$ and $\ttbar H$ cross sections are 15\% and $+10\%$/$-13\%$, respectively,
from the uncertainties in their respective NLO theoretical cross sections~\cite{Campbell:2012dh,Garzelli:2012bn,deFlorian:2016spz}. 

Uncertainties on the data-driven multijet background in the $\Hbb$ search estimate include
contributions from the limited sample size in data, particularly at high jet and $b$-tag multiplicities, as 
well as from the uncertainty on the rate of fake leptons, estimated in 
different control regions (e.g. selected with a requirement on either the maximum $\met$ or $\mtw$). 
A combined normalisation uncertainty of 50\% due 
to all these effects is assigned, which is taken as correlated across jet
and $b$-tag multiplicity bins, but uncorrelated between electron and muon channels. 
No explicit shape uncertainty is assigned since the large statistical uncertainties associated with
the multijet background prediction, which are uncorrelated 
between bins in the final discriminant distribution, effectively cover all possible shape uncertainties. 

Uncertainties on the data-driven fake $\tauhad$ background in the $\Htautau$ search are obtained by using additional signal-depleted regions. The construction is similar to that of the SRs and corresponding CRs discussed in Section~\ref{sec:fakeleptons}, but employing further loosened $\tauhad$ identification criteria, and thus referred to as ``loose SR''  and ``loose CR''. 
%The shape difference between the fake $\tauhad$ background prediction in each loose SR and its associated loose CR
%is applied as an uncertainty on the prediction in the nominal SR.
%In each loose SR, the shape difference between the data and the fake $\tauhad$ background estimate based on its associated loose CR,
%in both cases having subtracted the small contribution from real $\tauhad$ candidates based on the simulation, 
%is applied as an uncertainty on the prediction in the nominal SR. 
%In each loose SR, the shape difference between the data and the data-driven prediction of its associated loose CR is extracted and applied as an uncertainty on the prediction in the nominal SR. 
In each loose SR, after subtracting the small contribution from real $\tauhad$ candidates based on the simulation, 
the relative difference in the shape of the distribution between the remaining data and the fake $\tauhad$ background estimate based on its associated loose CR 
is applied as an uncertainty on the prediction in the nominal SR.
In addition, a 30\% uncertainty is applied to the fraction of $\ttbar$ events with a fake $\tauhad$ candidate from the simulation that are added to the fake $\tauhad$ template in the $\hadhad$ channel as part of the fake $\tauhad$ background estimation procedure (see Section~\ref{sec:faketaus}). This uncertainty, associated with the modelling of the fake $\tauhad$ rate by the simulation, is estimated by comparing data and simulation in a sample enriched in $\ttbar$ dilepton events plus a fake $\tauhad$ candidate. 
The same uncertainty is assigned to the selected signal events with fake $\tauhad$ candidates.
In addition, a systematic uncertainty is assigned to account for the different fractional composition of particles (various types of leptons and partons) originating the fake $\tauhad$ candidates 
between each SR and its corresponding CR in the $\ttbar$ simulation. 
Finally, the normalisation of the fake $\tauhad$ background in each SR is determined in the fit to data. 
%In order to be conservative, the yield of different origins of the fake taus is studied and a dedicated uncertainty is added to the fake template. The MC studies shows that non-H-to-tautau signal with fake taus can contribute to the $\lephad$ channel. So a 30\% uncertainty is  applied to these signal-faking-tau events.

\subsection{Signal modelling}
\label{sec:syst_sigmodeling}

Several normalisation and shape uncertainties are taken into account for the $\Hq$ signal.
The uncertainty on the $\ttbar$ cross section also applies to the $\Hq$ signal and is taken to be the same as, 
and fully correlated with, the uncertainty assigned to the $\ttbar \to WbWb$ background (see Sect.~\ref{sec:syst_bkgmodeling}).
Uncertainties on the Higgs boson branching ratios are taken into account
following the recommendation in Ref.~\cite{deFlorian:2016spz}.
%: $\pm 1.1\%$ ($\Delta\alpha_\textrm{S}$),  $\pm 1.4\%$ ($\Delta m_b$) and $\pm 0.8\%$ (theory).
Additional uncertainties associated with the modelling of additional radiation, with the choice of NLO generator, and
with the choice of parton shower and hadronisation model, are estimated from the comparison of the nominal
and alternative $\ttbar \to WbWb$ background samples (see discussion in Sect.~\ref{sec:syst_bkgmodeling}) and applied to $\Hq$ signal. 
These modelling uncertainties are taken to be uncorrelated with those affecting the $\ttbar \to WbWb$ background.
