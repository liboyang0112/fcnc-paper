%-------------------------------------------------------------------------------
\section{Systematic uncertainties}
\label{sec:systematics}
%-------------------------------------------------------------------------------
				   
Several sources of systematic uncertainty that can affect the normalisation of signal 
and background and/or the shape of their corresponding discriminant distributions are considered.
Each source is considered to be uncorrelated with the other sources.  
Correlations of a given systematic uncertainty are maintained across processes and channels 
as appropriate.
The following sections describe the systematic uncertainties considered.

\subsection{Luminosity}
\label{sec:syst_lumi}

The uncertainty in the integrated luminosity is 1.7\%, affecting the overall normalisation of all processes estimated from the simulation. 
It is derived, following a methodology similar to that detailed in Ref.~\cite{Aaboud:2016hhf}, and using the LUCID-2 detector 
for the baseline luminosity measurements \cite{Avoni:2018iuv}, from a calibration of the luminosity scale using $x$--$y$ beam-separation scans.

\subsection{Reconstructed objects}
\label{sec:syst_objects}

Uncertainties associated with electrons, muons, and $\had$ candidates arise from the trigger, reconstruction,  
identification and isolation (in the case of electrons and muons) efficiencies, as well as the momentum scale and resolution. 
These are measured using $Z\to \ell^+\ell^-$ and $J/\psi\to \ell^+\ell^-$ events ($\ell =e, \mu$)~\cite{ATLAS-CONF-2016-024,Aad:2016jkr} 
in the case of electrons and muons, and using $Z\to \tau^+\tau^-$ events in the case of $\had$ candidates~\cite{ATLAS-CONF-2017-029}.

Uncertainties associated with jets arise from the jet energy scale
and resolution, and the efficiency to pass the JVT requirements. 
The largest contribution results from the jet energy scale, whose uncertainty dependence on jet $\pt$ and $\eta$, jet flavour, and pile-up treatment, 
is split into 21 uncorrelated components that are treated independently~\cite{Aaboud:2017jcu}.  

Uncertainties associated with energy scales and resolutions of leptons and jets 
are propagated to $\met$. Additional uncertainties originating from the modelling 
of the underlying event, in particular its impact on the $\pt$ scale and resolution 
of unclustered energy, are negligible.

Efficiencies to tag $b$-jets and $c$-jets in the simulation are corrected to match the efficiencies in data by $\pt$-dependent factors,
whereas the light-jet efficiency is scaled by $\pt$- and $\eta$-dependent factors.
The $b$-jet efficiency is measured in a data sample enriched in $\ttbar$ events~\cite{Aaboud:2018xwy}, while the $c$-jet efficiency is measured
using $\ttbar$ events~\cite{ATLAS-CONF-2018-001} or $W$+$c$-jet events~\cite{Aad:2015ydr}. 
The light-jet efficiency is measured in a multijet data sample enriched in light-flavour jets~\cite{ATLAS-CONF-2018-006}.
The uncertainties in these scale factors include a total of 44 independent sources affecting $b$-jets, 19 source affecting $c$-jets, and 19 sources affecting light-jets. 
These systematic uncertainties are taken as uncorrelated between $b$-jets, $c$-jets, and light-jets. 

\subsection{Background modelling}
\label{sec:syst_bkgmodeling}

A number of sources of systematic uncertainty affecting the modelling of $t\bar{t}$+jets are considered. 
An uncertainty of  6\% is assigned to the inclusive $\ttbar$ production
cross section~\cite{Czakon:2011xx}, including contributions from varying the factorisation and renormalisation 
scales, as well as from the top-quark mass, the PDF and $\alpha_{\textrm{S}}$. The latter two represent the largest contribution 
to the overall theoretical uncertainty in the cross section and were calculated using the PDF4LHC prescription~\cite{Botje:2011sn} 
with the MSTW 2008 68\% CL NNLO, CT10 NNLO~\cite{Lai:2010vv,Gao:2013xoa} and NNPDF2.3 5F FFN~\cite{Ball:2012cx} PDF sets.
The uncertainty associated with the choice of NLO generator is derived by comparing the nominal prediction from
{\powheg}+{\pythiaeight} with a prediction from \textsc{Sherpa}~2.2.1. For the latter, the matrix-element calculation is performed 
for up to two partons at NLO and up to four partons at LO using \textsc{Comix} and \textsc{OpenLoops}, and
merged with the {\sherpa} parton shower using the ME+PS@NLO prescription.

The uncertainty due to the choice of parton shower and hadronisation (PS \& Had) model is derived 
by comparing the predictions from {\powheg} interfaced either to {\pythiaeight} or {\herwig7}. 
The latter uses the MMHT2014 LO~\cite{Harland-Lang:2014zoa} PDF set in combination with the H7UE tune~\cite{Bellm:2015jjp}.
The uncertainty in the modelling of additional radiation is assessed with two alternative {\powheg}+{\pythiaeight} samples:
a sample with increased radiation (referred to as radHi) is obtained by decreasing the renormalisation and factorisation scales  
by a factor of two, doubling the $h_{\textrm{damp}}$ parameter, and using the Var3c upward variation of the A14 parameter set;
a sample with decreased radiation (referred to as radLow) is obtained by increasing the scales by a factor of two 
and using the Var3c downward variation of the A14 set~\cite{ATL-PHYS-PUB-2016-004}.

Uncertainties affecting the normalisation of the $V$+jets background are estimated for the sum
of $W$+jets and $Z$+jets, and separately for $V$+light-jets, $V$+$\geq$1$c$+jets, and $V$+$\geq$1$b$+jets subprocesses.
The total normalisation uncertainty of $V$+jets processes is estimated by comparing the data and total background prediction in 
the different analysis regions considered, but requiring exactly zero $b$-tagged jets. Agreement between data and predicted background 
in these modified regions, which are dominated by $V$+light-jets, is found to be within approximately 30\%. This bound is taken to 
be the normalisation uncertainty, correlated across all $V$+jets subprocesses. 
Since {\sherpa}~2.2 has been found to underestimate $V$+heavy-flavour production by about a factor
of 1.3~\cite{Aaboud:2017xsd}, additional 30\% normalisation uncertainties are assumed for $V$+$\geq$1$c$+jets and $V$+$\geq$1$b$+jets
subprocesses, considered uncorrelated between them.

Uncertainties affecting the modelling of the single-top-quark background include a 
$+5\%$/$-4\%$ uncertainty of the total cross section estimated as a weighted average 
of the theoretical uncertainties in $t$-, $tW$- and $s$-channel production~\cite{Kidonakis:2011wy,Kidonakis:2010ux,Kidonakis:2010tc}.
Additional uncertainties associated with the modelling of additional radiation are assessed by comparing the nominal
samples with alternative samples where generator parameters are varied.
For the $t$- and $tW$-channel processes, an uncertainty due to the choice of parton shower and hadronisation model is derived 
by comparing events produced by {\powheg} interfaced to {\pythia}~6 or {\herwigpp}.
These uncertainties are treated as fully correlated among single-top-quark production processes, but uncorrelated with the
corresponding uncertainty of the $\ttbar$+jets background.
An additional systematic uncertainty in $tW$-channel production concerning the separation 
between $t\bar{t}$ and $tW$ at NLO is assessed by comparing
the nominal sample, which uses the diagram removal scheme~\cite{Frixione:2008yi}, with an alternative sample
using the diagram subtraction scheme~\cite{Frixione:2008yi}.

Uncertainties of the diboson background normalisation include 5\% from the NLO theory cross sections~\cite{Campbell:1999ah,Campbell:2011bn},
as well as an additional 24\% normalisation uncertainty added in quadrature for each additional inclusive jet-multiplicity bin, based on a 
comparison among different algorithms for merging LO matrix elements and parton showers~\cite{Alwall:2007fs}
(it is assumed that two jets originate from the $W/Z$ decay, as in $WW/WZ \to \ell \nu jj$). 
Therefore, the total normalisation uncertainty is $5\% \oplus \sqrt{N-2}\times 24\%$, where $N$ is the selected jet multiplicity,  
resulting in 34\%, 42\%, and 48\%, for events with exactly 4 jets, exactly 5 jets, and $\geq$6 jets, respectively. 
Recent comparisons between data and {\sherpa}~2.1.1 for $WZ(\to \ell\nu\ell\ell) + \geq$4 jets show
agreement within the experimental uncertainty of approximately 40\%~\cite{Aaboud:2016yus}, which further justifies the above uncertainty.
Given the very small contribution of this background to the total prediction, the final result is not affected by the assumed modelling
uncertainties.

Uncertainties of the $\ttbar V$ and $\ttbar H$ cross sections are 15\% and $+10\%$/$-13\%$, respectively,
from the uncertainties of their respective NLO theoretical cross sections~\cite{Campbell:2012dh,Garzelli:2012bn,deFlorian:2016spz}. 

The statistical uncertainties of the fake $\had$ background calibration in the leptonic channel is applied with uncorrelated uncertainties for different fake $\had$ source and different $\pt$ slices. The uncertainties of the ABCD method is applied with a normalisation factor for muon and electron respectively considering both statistical fluctuation and the differences between signal regions.

The uncertainties of the fake factor method applied in hadronic channel includes statistical uncertainties of each fake factor and different sets of fake factors derived from different signal depleted CRs.

\subsection{Signal modelling}
\label{sec:syst_sigmodeling}

Several normalisation and shape uncertainties are taken into account for the $\Hq$ signal.
Uncertainties of the Higgs boson branching ratios are taken into account
following the recommendation in Ref.~\cite{deFlorian:2016spz}.
The uncertainty of ISR, FSR and PDF are treated correlatedly with $\ttbar$ samples. The parton shower uncertainties is estimated by comparing the predictions from {\powheg} interfaced either to {\pythiaeight} or {\herwig7}. 