%-------------------------------------------------------------------------------
\section{Results}
\label{sec:result}
%-------------------------------------------------------------------------------

This section presents the results obtained from the individual channels, as well as their combination,
following the statistical analysis discussed in Section~\ref{sec:stat_analysis}.

A binned likelihood fit under the signal-plus-background hypothesis is performed on the BDT discriminant distributions in the seven 
analysis regions considered. The unconstrained parameter of the fit is the signal strength.
%and four independent parameters associated with the normalisation of the fake $\had$ background in each of the analysis regions. 
No significant pulls or constraints are obtained for the fitted nuisance parameters, resulting in a post-fit background prediction in each analysis region that is
very close to the pre-fit prediction, albeit with reduced uncertainties due to the anti-correlations among sources of systematic uncertainty resulting from the fit.
%Figure~\ref{fig:Bonlyfit_data} shows a background only fit to the BDT discriminant distribution in the data.
Figure~\ref{fig:asimov_postfitbdtHu} and~\ref{fig:asimov_postfitbdtHc} show a post-fit with signal plus background (S+B) to the data for
the $tcH$ and $tuH$ search separately.
%both pre- and post-fit to data, in the case of the $\Hc$ search.  
%A similar comparison for the leptonic channel is shown in Figure~\ref{fig:tthML_trexPrefit} and \ref{fig:tthML_trexPrefit_1}.
The observed and predicted yields after fit to the data with background only are summarized in Table~\ref{tab:HtautauPostfitYieldsUnblind}.
%pre-fit and post-fit yields can be found in Appendix~\ref{sec:prepostfit_yields_Htautau_appendix}.
A slight excess of data is observed above background with a significance of 2.3 $\sigma$, which is mainly in the most sensitive $t_{\ell}\thadhad$ channel in the high BDT score region as shown in
Table~\ref{tab:limits_summary}.
The kinematic distributions for the observed excess in
the high BDT region are checked. Within the large statistical uncertainty, the observed distributions are compatible with the background shapes, but also with a small signal contribution. There is no indication that the excess is from a specific data period.
The background modelling in this signal region has also been checked using the VR in which both $\thad$ candidates have the same charge.
%The BDT distribution in this VR is shown in Fig.##,
In this VR the signal contribution is negligble in the highest BDT bins and the background shape is well reproduced by the data.
%The fitted signal strength and its significance from individual channels and their combination are shown in Table~\ref{tab:limits_summary}.
%and found there are nothing unusual between data and expectation.
% \begin{figure}[H]
% \centering
% \begin{tabular}{@{}ccc@{}}
% %\includegraphics[width=0.33\textwidth]{\FCNCFigures/unblinded/tthML/tcH_reg1l2tau1bnj_os_postFit_BOnly.pdf}&
% %\includegraphics[width=0.33\textwidth]{\FCNCFigures/unblinded/tthML/tcH_reg1l1tau1b1j_ss_postFit_BOnly.pdf}&
% %\includegraphics[width=0.33\textwidth]{\FCNCFigures/unblinded/tthML/tcH_reg1l1tau1b2j_ss_postFit_BOnly.pdf}\\
% %(a1) BDT discriminant in $t_{\ell}\thadhad$ & (a2) BDT discriminant in  $t_{\ell}\tauhad$-1j& (a3) BDT discriminant in $t_{\ell}\tauhad$-2j\\
% %\includegraphics[width=0.33\textwidth]{\FCNCFigures/unblinded/tthML/tcH_reg1l1tau1b2j_os_postFit_BOnly.pdf}&
% %\includegraphics[width=0.33\textwidth]{\FCNCFigures/unblinded/tthML/tcH_reg1l1tau1b3j_os_postFit_BOnly.pdf}&
% %\includegraphics[width=0.33\textwidth]{\FCNCFigures/unblinded/xTFW/tcH_reg2mtau1b2jos_vetobtagwp70_highmet_postFit_BOnly.pdf}\\
% %(b1) BDT discriminant in $t_h\tlhad$-2j & (b2) BDT discriminant in  $t_h\tlhad$-3j & (b3) BDT discriminant in $t_h\thadhad$-2j \\
% %\includegraphics[width=0.33\textwidth]{\FCNCFigures/unblinded/xTFW/tcH_reg2mtau1b3jos_vetobtagwp70_highmet_postFit_BOnly.pdf}& \\
% %(c1) BDT discriminant in$t_h\thadhad$-3j\\
%   \includegraphics[page=9,width=0.33\textwidth]{\FCNCFigures/tthML/showFake/faketau/postfit/NOMINAL/reg1l2tau1bnj_os/BDTG_test.pdf}&
%   \includegraphics[page=9,width=0.33\textwidth]{\FCNCFigures/tthML/showFake/faketau/postfit/NOMINAL/reg1l1tau1b1j_ss_vetobtagwp70_highmet/BDTG_test.pdf}&
%   \includegraphics[page=9,width=0.33\textwidth]{\FCNCFigures/tthML/showFake/faketau/postfit/NOMINAL/reg1l1tau1b2j_ss_vetobtagwp70_highmet/BDTG_test.pdf}\\
% (a1) BDT in $t_{\ell}\thadhad$ & (a2) BDT in  $t_{\ell}\tauhad$-1j& (a3) BDT in $t_{\ell}\tauhad$-2j\\
%   \includegraphics[page=9,width=0.33\textwidth]{\FCNCFigures/tthML/showFake/faketau/postfit/NOMINAL/reg1l1tau1b2j_os_vetobtagwp70_highmet/BDTG_test.pdf}&
%   \includegraphics[page=9,width=0.33\textwidth]{\FCNCFigures/tthML/showFake/faketau/postfit/NOMINAL/reg1l1tau1b3j_os_vetobtagwp70_highmet/BDTG_test.pdf}&
%   \includegraphics[width=0.33\textwidth]{figures/reg2mtau1b2jos_vetobtagwp70_highmet_pre_bonly.pdf}\\
% (b1) BDT in $t_h\tlhad$-2j & (b2) BDT in $t_h\tlhad$-3j & (b3) BDT in $t_h\thadhad$-2j \\
%   \includegraphics[width=0.33\textwidth]{figures/reg2mtau1b3jos_vetobtagwp70_highmet_pre_bonly.pdf}& \\
% (c1) BDT in$t_h\thadhad$-3j\\
% \end{tabular}
% \caption{ The BDT output distributions are fitted with background only to the data: $t_{\ell}\thadhad$ (a1),  $t_{\ell}\tauhad$-1j (a2),  $t_{\ell}\tauhad$-2j (a3),
%   $t_h\tlhad$-2j (b1), $t_h\tlhad$-3j (b2), $t_h\thadhad$-2j (b3), and $t_h\thadhad$-3j (c1).
%   Statistical and systematic uncertainties are being shown. For different type of signals are also shown for comparing their shapes. ({\textbf to be updated?})}
% \label{fig:Bonlyfit_data}
% \end{figure}



%\input{\FCNCFigures/tex/tthML_trexPrefit}
%\input{\FCNCFigures/tex/xTFW_trexPrefit}
\begin{figure}[H]
\centering
%\begin{tabular}{@{}ccc@{}}
%\includegraphics[width=0.33\textwidth]{\FCNCFigures/tthML/Limit/tcH_reg1l2tau1bnj_os_postFit.pdf}&
%\includegraphics[width=0.33\textwidth]{\FCNCFigures/tthML/Limit/tcH_reg1l1tau1b1j_ss_postFit.pdf}&
%\includegraphics[width=0.33\textwidth]{\FCNCFigures/tthML/Limit/tcH_reg1l1tau1b2j_ss_postFit.pdf}\\
%(a1) BDT discriminant in $t_{\ell}\thadhad$ & (a2) BDT discriminant in  $t_{\ell}\tauhad$-1j& (a3) BDT discriminant in $t_{\ell}\tauhad$-2j\\
%\includegraphics[width=0.33\textwidth]{\FCNCFigures/tthML/Limit/tcH_reg1l1tau1b2j_os_postFit.pdf}&
%\includegraphics[width=0.33\textwidth]{\FCNCFigures/tthML/Limit/tcH_reg1l1tau1b3j_os_postFit.pdf}&
%\includegraphics[width=0.33\textwidth]{\FCNCFigures/xTFW/Limit/tcH_reg2mtau1b2jos_vetobtagwp70_highmet_postFit.pdf}\\
%(b1) BDT discriminant in $t_h\tlhad$-2j & (b2) BDT discriminant in  $t_h\tlhad$-3j & (b3) BDT discriminant in $t_h\thadhad$-2j \\
%\includegraphics[width=0.33\textwidth]{\FCNCFigures/xTFW/Limit/tcH_reg2mtau1b3jos_vetobtagwp70_highmet_postFit.pdf}& \\
%(c1) BDT discriminant in$t_h\thadhad$-3j\\
%\end{tabular}
%\caption{ The BDT output distributions are fitted to the asimov S+B data in $tHc$ search: $t_{\ell}\thadhad$ (a1),  $t_{\ell}\tauhad$-1j (a2),  $t_{\ell}\tauhad$-2j (a3),
%  $t_h\tlhad$-2j (b1), $t_h\tlhad$-3j (b2), $t_h\thadhad$-2j (b3), and $t_h\thadhad$-3j (c1). Statistical and systematic uncertainties are being shown.}
%\label{fig:asimov_postfitbdtHc}


\begin{tabular}{@{}ccc@{}}
\includegraphics[width=0.3\textwidth]{figures/tuH_reg1l2tau1bnj_os.pdf}&
\includegraphics[width=0.3\textwidth]{figures/tuH_reg1l1tau1b1j_ss.pdf}&
\includegraphics[width=0.3\textwidth]{figures/tuH_reg1l1tau1b2j_ss.pdf}\\
%(a1) BDT in $t_{\ell}\thadhad$ & (a2) BDT in  $t_{\ell}\tauhad$-1j& (a3) BDT in $t_{\ell}\tauhad$-2j\\
(a)  & (b) & (c) \\
\includegraphics[width=0.3\textwidth]{figures/tuH_reg1l1tau1b2j_os.pdf}&
\includegraphics[width=0.3\textwidth]{figures/tuH_reg1l1tau1b3j_os.pdf}&
\includegraphics[width=0.3\textwidth]{figures/tuH_reg1l2tau1bnj_ss.pdf}\\
%\includegraphics[width=0.3\textwidth]{figures/tcH_reg1l2tau1bnj_ss.pdf}\\
%(b1) BDT in $t_h\tlhad$-2j & (b2) BDT in  $t_h\tlhad$-3j & (b3) BDT in $t_h\thadhad$-2j \\
(d) & (e)  & (f) \\
%(d) & (e)\\
\includegraphics[width=0.3\textwidth]{figures/tuH_reg2mtau1b2jos.pdf}&
\includegraphics[width=0.3\textwidth]{figures/tuH_reg2mtau1b3jos.pdf}&
\includegraphics[width=0.3\textwidth]{figures/tuH_reg2mtau1b3jss.pdf}\\
%(c1) BDT in$t_h\thadhad$-3j\\
(g) & (h)  & (i) \\
%(f) & (g)  &  \\
\end{tabular}
\caption{ BDT output distributions obtained from a signal+background fit to the data for the $tuH$ search: 
%$t_{\ell}\thadhad$ (a),  $t_{\ell}\tauhad$-1j (b),  $t_{\ell}\tauhad$-2j (c), $t_h\tlhad$-2j (d), $t_h\tlhad$-3j (e), $t_{\ell}\thadhad$ SS (f), $t_h\thadhad$-2j (g), and $t_h\thadhad$-3j (h), where $t_{\ell}\thadhad$ SS is used for the background validation. 
$t_{\ell}\thadhad$ (a),$t_{\ell}\tauhad$-1j (b),  $t_{\ell}\tauhad$-2j (c), $t_h\tlhad$-2j (d), $t_h\tlhad$-3j (e), $t_{\ell}\thadhad$ same sign CR (f), $t_h\thadhad$-2j (g), $t_h\thadhad$-3j (h) and $t_h\thadhad$-3j same sign CR (i).  
The total size of the statistical and systematic uncertainties is indicated by the hatched band. The signal shapes of $tt(uH)$, $tH$, and their sum are also shown using a normalisation of $2 \times\BR(t\to uH)$ of 0.1\%. 
}
\label{fig:asimov_postfitbdtHu}
\end{figure}



\begin{figure}[H]
\centering
\begin{tabular}{@{}ccc@{}}
\includegraphics[width=0.3\textwidth]{figures/tcH_reg1l2tau1bnj_os.pdf}&
\includegraphics[width=0.3\textwidth]{figures/tcH_reg1l1tau1b1j_ss.pdf}&
\includegraphics[width=0.3\textwidth]{figures/tcH_reg1l1tau1b2j_ss.pdf}\\
%(a1) BDT in $t_{\ell}\thadhad$ & (a2) BDT in  $t_{\ell}\tauhad$-1j& (a3) BDT in $t_{\ell}\tauhad$-2j\\
(a)  & (b) & (c) \\
\includegraphics[width=0.3\textwidth]{figures/tcH_reg1l1tau1b2j_os.pdf}&
\includegraphics[width=0.3\textwidth]{figures/tcH_reg1l1tau1b3j_os.pdf}&
\includegraphics[width=0.3\textwidth]{figures/tcH_reg1l2tau1bnj_ss.pdf}\\
%\includegraphics[width=0.3\textwidth]{figures/tcH_reg1l2tau1bnj_ss.pdf}\\
%(b1) BDT in $t_h\tlhad$-2j & (b2) BDT in  $t_h\tlhad$-3j & (b3) BDT in $t_h\thadhad$-2j \\
(d) & (e)  & (f) \\
%(d) & (e)\\
\includegraphics[width=0.3\textwidth]{figures/tcH_reg2mtau1b2jos.pdf}&
\includegraphics[width=0.3\textwidth]{figures/tcH_reg2mtau1b3jos.pdf}&
\includegraphics[width=0.3\textwidth]{figures/tcH_reg2mtau1b3jss.pdf}\\
%(c1) BDT in$t_h\thadhad$-3j\\
(g) & (h)  & (i) \\
%(f) & (g)  &  \\
\end{tabular}
\caption{ BDT output distributions obtained from a signal+background fit to the data for the $tcH$ search: 
%$t_{\ell}\thadhad$ (a),  $t_{\ell}\tauhad$-1j (b),  $t_{\ell}\tauhad$-2j (c), $t_h\tlhad$-2j (d), $t_h\tlhad$-3j (e), $t_{\ell}\thadhad$ SS (f), $t_h\thadhad$-2j (g), and $t_h\thadhad$-3j (h), where $t_{\ell}\thadhad$ SS is used for the background validation.
$t_{\ell}\thadhad$ (a),$t_{\ell}\tauhad$-1j (b),  $t_{\ell}\tauhad$-2j (c), $t_h\tlhad$-2j (d), $t_h\tlhad$-3j (e), $t_{\ell}\thadhad$ same sign CR (f), $t_h\thadhad$-2j (g), $t_h\thadhad$-3j (h) and $t_h\thadhad$-3j same sign CR (i). 
The total size of the statistical and systematic
uncertainties is indicated by the hatched band. The signal shapes of $tt(cH)$, $tH$, and their sum are also shown using a normalisation of $2 \times\BR(t\to cH)$ of 0.1\%.
}
\label{fig:asimov_postfitbdtHc}
\end{figure}

\begin{table}[htbp]
\caption{
  Predicted and observed yields in each of the analysis regions considered. The background prediction is shown after a background-only fit to data.
  Also shown are the signal expectations for $\Hc$ and
  $\Hu$ assuming $\BR(t\to cH)=0.1\%$ and $\BR(t\to uH)=0.1\%$ respectively. The contributions with real $\had$ candidates from $\ttbar$ and  $Z\to \ell^+\ell^-$ ($\ell = e, \mu$),
  diboson, $\ttbar V$, $\ttbar H$, single-top-quark, and other small backgrounds are combined into a single background source referred to as ``Other MC'' in the leptonic channels ,
  whereas single-top-quark and the small contributions are combined into ``Rare'' in the hadronic channels.
  The quoted uncertainties are the sum in quadrature of statistical and systematic uncertainties of the yields.}
\small
\centering

\begin{tabular}{cccccc} \toprule\toprule
 & $t_l\tauhad$-1j & $t_l\tauhad$-2j & $t_h\tlhad$-3j &$t_h\tlhad$-2j  & $t_l\thadhad$ \\\midrule
  Double Fake            & $--       $  & $--       $  & $--       $  &  $--       $  & $73 \pm 24    $ \\ 
  $\bar{t}tV$            & $9.3 \pm 1.2 $  & $22.6 \pm 2.8$  & $23.5 \pm 3.0$  &  $13.7 \pm 1.7$  & $2.57 \pm 0.35$ \\ 
  SM Higgs               & $5.8 \pm 0.8 $  & $13.7 \pm 1.7$  & $32.8 \pm 3.5$  &  $13.5 \pm 2.5$  & $16.7 \pm 1.9 $ \\ 
  Diboson                & $32.6 \pm 3.4$  & $19.9 \pm 2.1$  & $36 \pm 4    $  &  $46 \pm 5    $  & $13.2 \pm 1.4 $ \\ 
  Other MC               & $35.6 \pm 3.1$  & $15.9 \pm 1.7$  & $226 \pm 21  $  &  $620 \pm 40  $  & $6.7 \pm 0.6  $  \\ 
  $Z\rightarrow\tau\tau$ & $0 \pm 6     $  & $9.1 \pm 2.2 $  & $500 \pm 60  $  &  $880 \pm 90  $  & $2.1 \pm 0.7  $ \\ 
  Lep Fake               & $212 \pm 30  $  & $80 \pm 10   $  & $292 \pm 26  $  &  $490 \pm 70  $  & $0.9 \pm 0.4  $ \\ 
  QCD Fake               & $670 \pm 200 $  & $310 \pm 90  $  & $180 \pm 70  $  &  $330 \pm 110 $  & $--        $  \\ 
  b Fake                 & $960 \pm 140 $  & $1250 \pm 230$  & $710 \pm 140 $  &  $710 \pm 130 $  & $82 \pm 13    $ \\ 
  W-jet Fake             & $970 \pm 200 $  & $1090 \pm 240$  & $3300 \pm 500$  &  $3800 \pm 600$  & $5.5 \pm 1.8  $ \\ 
  Other Fake             & $3020 \pm 260$  & $2470 \pm 160$  & $1420 \pm 220$  &  $1320 \pm 320$  & $129 \pm 14   $ \\ 
  $\bar{t}t$             & $281 \pm 14  $  & $195 \pm 24  $  & $7100 \pm 400$  &  $11800 \pm 500$ & $7.7 \pm 2.7  $ \\ \midrule
  Total background       & $6200 \pm 170$  & $5480 \pm 100$  & $13820 \pm 140$ &  $20000 \pm 170$ & $339 \pm 27   $ \\  \midrule
  tcH                    & $30 \pm 5    $  & $27 \pm 4    $  & $51 \pm 8    $  &  $34 \pm 6    $  & $36 \pm 5     $ \\
  tuH                    & $36 \pm 8    $  & $32 \pm 5    $  & $63 \pm 10   $  &  $45 \pm 7    $  & $48 \pm 7     $ \\ \midrule
  Data                   & $6353        $  & $5410        $  & $13804       $  &  $20000       $  & $351          $ \\ 
\bottomrule\bottomrule
\end{tabular}\\




\begin{tabular}{ccc} \toprule\toprule
& $t_{h}\thadhad$-2j & $t_{h}\thadhad$-3j\\\midrule
  $t\bar{t}V$              & $0.7 \pm 0.4 $ & $5.5 \pm 1.0 $  \\
  Diboson                  & $8.4 \pm 1.6 $ & $10.8 \pm 1.5$  \\
  Rare                     & $17.9 \pm 3.1$ & $10.2 \pm 2.6$  \\ 
  SM Higgs                 & $17.4 \pm 2.5$ & $25.9 \pm 3.1$  \\ 
  only $\tau_{sub}$ real   & $56 \pm 30   $ & $80 \pm 50   $  \\  
  $t\bar{t}$               & $221 \pm 28  $ & $220 \pm 40  $  \\
  Fake $\tau$              & $220 \pm 70  $ & $270 \pm 70  $  \\  
  $Z\rightarrow\tau\tau$   & $490 \pm 50  $ & $420 \pm 50  $  \\ \midrule
  Total background         & $1040 \pm 35 $ & $1040 \pm 40 $  \\ \midrule
  tcH                      & $15.6 \pm 2.5$ & $42 \pm 8    $  \\ 
  tuH                      & $23 \pm 4    $ & $52 \pm 10   $  \\ \midrule
  Data                     & $1033       $& $1052 $       \\
\bottomrule\bottomrule
\end{tabular}
\label{tab:HtautauPostfitYieldsUnblind}
\end{table}



%Comparison between the data and prediction for the BDT discriminant distribution in the
%$\lephad$ channel, before and after the fit to data  (``Pre-Fit'' and ``Post-Fit'', respectively) under the signal-plus-background hypothesis.
%Shown are the ($\lephad$, 3j) region (a) pre-fit and (c) post-fit, and the ($\lephad$, $\geq$4j) region (b) pre-fit and (d) post-fit.
%The contributions with real $\had$ candidates from $\ttbar$,  $\ttbar V$, $\ttbar H$, and single-top-quark backgrounds are combined into
%a single background source referred to as ``Top (real $\had$)'', whereas the small contributions from 
%$Z\to \ell^+\ell^-$ ($\ell = e, \mu$) and diboson backgrounds are combined into ``Other''. 
%In the pre-fit figures the expected $\Hc$ signal (solid red) corresponding to $\BR(t\to Hc)=1\%$ is also shown,
%added to the background prediction. In the post-fit figures, the $\Hc$ signal is normalised using the best-fit branching ratio, 
%$\BR(t\to Hc)=(-4.4^{+9.9}_{-8.5})\times 10^{-4}$.
%The bottom panels display the ratios of data to either the SM background prediction before the fit (``Bkg'')  or the total signal-plus-background
%prediction after the fit (``Pred''). 
%The hashed area represents the total uncertainty of the background. 
%In the case of the pre-fit background uncertainty, the normalisation uncertainty of the fake $\had$ background is not included.
%The results are given in terms of exclusion limits despite a small excess of data events above the background expectation is found.
Upper limits are derived using the CL$_{\textrm{s}}$ method~\cite{Junk:1999kv,Read:2002hq}, and  
observed (expected) 95\% CL limits are set on $\BR(t\to cH)$ and $\BR(t\to uH)$:
$\BR(t\to cH)<9.9 \times 10^{-4}\,(5.0^{+2.2}_{-1.4}\times10^{-4})$, assuming $\BR(t\to uH)=0$,and $\BR(t\to uH)<7.2 \times 10^{-4}\,(3.6^{+1.7}_{-1.0}\times10^{-4})$, assuming $\BR(t\to cH)=0$.
These results are dominated by the leptonic channels, which have a sensitivity a factor of two better than that of the hadronic channels.
The expected sensitivity has been improved upon the previous ATLAS result based on 36 fb$^{-1}$ of data using $H\to \tau\tau$ decay~\cite{fcnc36} by a factor of~5. A factor of~2 improvement in sensitivity comes from the larger dataset, and a further factor of~2.5 comes from including
additional leptonic channels, $tH$ production, and improved techniques.

%The best-fit branching ratio obtained is $\BR(t\to Hc)=[xxx^{+yy}_{-yy}\,(\mathrm{stat})^{+zz}_{-zz}\,(\mathrm{syst})] \times 10^{-4}$, assuming $\BR(t\to Hu)=0$. 
%The best-fit normalisation factors for the fake $\had$ background are: $0.82 \pm 0.23$ in the ($\lephad$, 3j) region, $0.84^{+0.25}_{-0.28}$ in the ($\lephad$, $\geq$4j) region,
%$0.94^{+0.18}_{-0.17}$ in the ($\hadhad$, 3j) region, and $0.90 \pm 0.26$ in the ($\hadhad$, $\geq$4j) region.
%A similar fit is performed for the $tuH$ search, yielding $\BR(t\to Hu)=[xxx^{+yy}_{-yy}\,(\mathrm{stat})^{+zz}_{-zz}\,(\mathrm{syst})] \times 10^{-4}$,
%assuming $\BR(t\to Hc)=0$.
%The obtained normalisation factors for the fake $\had$ background agree within 1\% with those obtained by the $\Hc$ search.
In both cases, the results are dominated by the statistical uncertainty.
The main contributions to the total systematic uncertainty arise from  the size of the MC samples,
the uncertainties on the measured $\BR(H\to \tautau)$ branching ratio,
the normalization and factorization scales, $b$-tagging, the choice of parton shower and hadronization for $t\bar t$ modelling, and
the fake $\had$ background estimation in the hadronic channels. Their absolute impacts on the signal strength are summarized in Table~\ref{tab:had_sys_impact}.
%the fake $\had$ background estimation in the hadronic channels and the uncertainty associated
%with the different responses to quark-initiated and gluon-initiated jets. 
%o significant excess of data events above the background expectation is found, 
%nd observed (expected) 95\% CL limits are set on $\BR(t\to Hc)$ and $\BR(t\to Hu)$:
%\BR(t\to Hc)<xxx \times 10^{-3}\,(yyy \times 10^{-3})$ and $\BR(t\to Hu)<xxx \times 10^{-3}\,(yyy \times 10^{-3})$.
%hese results are dominated by the leptonic channels, which has a sensitivity a factor of two better than that of the hadronic channels.
%\begin{figure*}[t!]
%\begin{center}
%\includegraphics[width=0.7\textwidth]{\FCNCFigures/tcH_combined_Limit.pdf}
%\caption{\small {Summary of the best-fit $\BR(t\to Hc)$ for the individual channels as well as their combination,
%assuming $\BR(t\to Hu)=0$. (TBD: updated with best fit plots.)}}
%\label{fig:summary_printnum_hc} 
%\end{center}
%\end{figure*}
%%%%%%%%%%%%%%
%%%%%%%%%%%%%%
%\begin{figure*}[h!]
%\begin{center}
%\includegraphics[width=0.7\textwidth]{\FCNCFigures/tuH_combined_Limit.pdf}
%\caption{\small {Summary of the best-fit $\BR(t\to Hu)$ for the individual channels as well as their combination,
%assuming $\BR(t\to Hc)=0$. (TBD: updated with best fit plots.)}}
%\label{fig:summary_printnum_hu} 
%\end{center}
%\end{figure*}
%%%%%%%%%%%%%%
%The first set of combined results is obtained for each branching ratio separately, setting the other branching ratio to zero.
%The best-fit combined branching ratios are $\BR(t\to Hc)=[3.0^{+3.0}_{-2.7}\,(\mathrm{stat})^{+2.6}_{-2.1}\,(\mathrm{syst})] \times 10^{-4}$ and 
%$\BR(t\to Hu)=[4.2^{+3.2}_{-2.9}\,(\mathrm{stat})^{+2.6}_{-2.1}\,(\mathrm{syst})] \times 10^{-4}$.  
%%The difference between the central values of $\BR(t\to Hc)$ and $\BR(t\to Hu)$ originates from the ability of the $H \to b\bar{b}$ search to 
%%probe both decay modes separately.
%A comparison of the best-fit branching ratios for the individual searches and their combination is shown in Figure~\ref{fig:summary_printnum_hc} 
%for $\BR(t\to Hc)$ and Figure~\ref{fig:summary_printnum_hu} for $\BR(t\to Hu)$.
%The observed (expected) 95\% CL combined upper limits on the branching ratios are 
%$\BR(t\to Hc)<1.1 \times 10^{-3}\,(8.3 \times 10^{-4})$ and $\BR(t\to Hu)<1.2 \times 10^{-3}\,(8.3 \times 10^{-4})$.
A summary of the upper limits, significance and best-fit values of the branching ratios obtained by the individual searches, as well as their combination, is given  
%%in Table~\ref{tab:limits_summary}, as is displayed in Figures~\ref{fig:limits_combo_1D_hc} and~\ref{fig:limits_combo_1D_hu}.
in Table~\ref{tab:limits_summary} and in Figures~\ref{fig:limits_combo_1D_hc}(a) and~\ref{fig:limits_combo_1D_hc}(b).


\begin{table}[h!]
  \caption{Absolute uncertainties on $\BR(t\to qH)$ ($q=u,c$) obtained from the combined fit to data. The uncertainties are symmetrised for
     a presentation purpose and grouped into the categories described in the Section~\ref{sec:systematics}.}
\label{tab:had_sys_impact}
% \begin{center}
%   \begin{tabular}{%
%       @{}l%
%       S
%       S
%       @{}
%     }
%     \toprule\toprule
%     \multirow{2}{*}{Source of uncertainty}      & \multicolumn{2}{c}{$\Delta\BR$ [$10^{-5}$]} \\
%     & \multicolumn{1}{c}{$t\rightarrow uH$} & \multicolumn{1}{c}{$t\rightarrow cH$} \\\midrule
%     Lepton ID                               & 0.6           &1.0         \\
%     $\met$                                  & 0.7           &0.8         \\
%     Fake lepton  modeling                   & 0.9           &1.1         \\
%     JES and JER                             & 2.4           &3.2         \\
%     Flavour tagging                         & 2.7           &3.7         \\
%     $t\bar{t}$ modeling                     & 2.9           &4.3         \\
%     Other MC modeling                       & 2.1           &2.9         \\
%     Fake $\tau$ modeling                    & 3.2           &4.6         \\
%     Signal modeling including Br($H\to\tau\tau$)            &5.3           &7.0         \\
%     $\tau$ ID                               & 3.3           &4.4         \\
%     Luminosity and Pileup                   & 0.9           &1.3         \\    
%     MC statistics              & 5.1           &7.0         \\\midrule
%     %Other MC modeling                       & 2.1           &2.9         \\
%     %JES and JER                             & 2.4           &3.2         \\
%     %Flavour tagging                         & 2.7           &3.7         \\
%     %$t\bar{t}$ modeling                     & 2.9           &4.3         \\
%     %Fake $\tau$ modeling                    & 3.2           &4.6         \\
%     %$\tau$ ID                               & 3.3           &4.4         \\
%     %Signal modeling including Br($H\to\tau\tau$)       & 5.3           &7.0         \\\midrule
%     Total systematic uncertainty                            & 11.2          &15.5        \\
%     Data statistical uncertainty                           & 14.1          &19.6         \\\midrule
%     Total uncertainties                       & 18            &25        \\
%     %Total systematic  uncertainty            & 11.2          &15.5        \\
%     %Total statistical uncertainty            & 14.1          &19.6         \\\midrule
%     %Total                                    & 18            &25        \\
%     \bottomrule\bottomrule
%   \end{tabular}
% \end{center}
% \end{table}
 \begin{center}
   \begin{tabular}{%
       @{}l%
       S
       S
       @{}
     }
     \toprule\toprule
     \multirow{2}{*}{Source of uncertainty}      & \multicolumn{2}{c}{$\Delta\BR$ [$10^{-5}$]} \\
     & \multicolumn{1}{c}{$t\rightarrow uH$} & \multicolumn{1}{c}{$t\rightarrow cH$} \\\midrule
     Lepton ID                               & 0.6           &0.8         \\
     $\met$                                  & 0.7           &0.7         \\
     Fake lepton  modeling                   & 1.2           &1.7         \\
     JES and JER                             & 2.5           &3.3         \\
     Flavour tagging                         & 2.7           &3.7         \\
     $t\bar{t}$ modeling                     & 2.6           &3.9         \\
     Other MC modeling                       & 2.1           &3.0         \\
     Fake $\tau$ modeling                    & 3.3           &4.7         \\
     Signal modeling including Br($H\to\tau\tau$)&1.8        &1.5         \\
     $\tau$ ID                               & 3.3           &4.4         \\
     Luminosity and Pileup                   & 1.7           &2.4         \\    
     MC statistics                           & 5.1           &7.1         \\\midrule
     Total systematic uncertainty            & 10.1          &14.1        \\
     Data statistical uncertainty            & 14.9          &19.4        \\\midrule
     Total uncertainties                     & 18            &24          \\
     \bottomrule\bottomrule
   \end{tabular}
 \end{center}
 \end{table}


Upper limits on the branching ratios $\BR(t\to qH)$ ($q=u,c$) can be translated into upper limits on the dimension-6 (D6) operator Wilson coefficients appearing in the effective field theory Lagrangian for the $tqH$ interaction~\cite{fcnc_production_theory}:
%
\begin{equation}
  \mathcal{L}_{EFT} = \frac{C^{i3}_{u\phi}}{\Lambda^{2}}(\phi^{\dagger}\phi)(\bar{q_{i}}t)\tilde{\phi} + \frac{C^{3i}_{u\phi}}{\Lambda^{2}}(\phi^{\dagger}\phi)(\bar{t}q_{i})\tilde{\phi}
  \label{eq:eq01}
\end{equation}
%
where the subscript i= 1, 2 represents the generation of the light quark fields ($q=u, c$).
The branching ratio $\BR(t\to qH)$ is estimated as the ratio of its partial width to the SM $t \to Wb$ partial width including next-to-leading-order QCD corrections and the coefficients can be extracted as $C_{q\phi} = \sqrt{1946.6~\BR(t\to qH)}$~\cite{fcnc_production_theory}. The $C_{q\phi}$ coefficient corresponds to the sum in quadrature of the coefficients relative to the two possible chirality combinations of the quark fields,
$C_{q\phi} =\sqrt{(C^{i3}_{q\phi})^2 + (C^{3i}_{q\phi})^2}$~\cite{fcnc_production_theory}. The observed (expected) upper limits on the D6 Wilson coefficients from the combination of the searches are $C_{c\phi}<1.38\,(0.97)$ and $C_{u\phi}<1.18\,(0.83)$ for a new physics scale $\Lambda=1$~TeV. 

%Upper limits on the branching ratios $\BR(t\to Hq)$ ($q=u,c$) can be translated into upper limits on the non-flavour-diagonal Yukawa couplings $\lamHq$ 
%appearing in the Lagrangian~\cite{Harnik:2012pb}:
%\begin{equation*}
%{\cal L}_\mathrm{FCNC} = -\lambda_{t_\mathrm{L} q_\mathrm{R}} \bar{t}_\mathrm{L} q_\mathrm{R} H - \lambda_{q_\mathrm{L} t_\mathrm{R}} \bar{q}_\mathrm{L} t_\mathrm{R} H  + \mathrm{h.c.}
%\end{equation*}
%The branching ratio $\BR(t\to Hq)$ is estimated as the ratio of its partial width~\cite{Zhang:2013xya} to the SM $t \to Wb$ partial width~\cite{Denner:1990ns}, 
%which is assumed to be dominant. Both predicted partial widths include next-to-leading-order QCD corrections.
%Using the expression derived in Ref.~\cite{Aad:2014dya}, the coupling $|\lamHq|$ can be extracted as $| \lamHq | = (1.92 \pm 0.02) \sqrt{\BR(t\to Hq)}$.
%The $\lamHq$ coupling corresponds to the sum in quadrature of the couplings relative to the two possible chirality combinations of the quark fields, 
%$\lamHq \equiv \sqrt{ |\lambda_{t_\mathrm{L} q_\mathrm{R}}|^2 +   |\lambda_{q_\mathrm{L} t_\mathrm{R}}|^2 }$~\cite{Harnik:2012pb}.
%The observed (expected) upper limits on the couplings from the combination of the searches are $|\lamHc|<0.064\,(0.055)$ and $|\lamHu|<0.066\,(0.055)$.

%%%%%%%%%%%%%%%

%%%%% mu benchmark is 0.1%

% \begin{table}[t!]
% \caption{\small{Summary of 95\% CL upper limits on $\BR(t \to cH)$ and $\BR(t \to uH)$, in each case neglecting the other decay mode. }}
% \begin{center}
% \begin{tabular}{lcc}
% \toprule\toprule
%  & \multicolumn{1}{c}{95\% CL upper limits} & \multicolumn{1}{c}{95\% CL upper limits}  \\
%  & \multicolumn{1}{c}{on $\BR(t \to cH)$} & \multicolumn{1}{c}{on $\BR(t \to uH)$} \\
%  &  Observed (Expected) & Observed (Expected)  \\
% \midrule\midrule
% hadronic  & $1.0 \times 10^{-3}$ ($9.8 \times 10^{-4}$) & $7.8 \times 10^{-4}$ ($7.8 \times 10^{-4}$) \\ 
% leptonic  & $1.3 \times 10^{-3}$ ($5.9 \times 10^{-4}$) & $9.2 \times 10^{-4}$ ($4.2 \times 10^{-4}$) \\
% \midrule
% Combination  & $9.9 \times 10^{-4}$ ($5.0 \times 10^{-4}$) & $7.2 \times 10^{-4}$ ($3.6 \times 10^{-4}$) \\
% \bottomrule\bottomrule
% \end{tabular}
% \label{tab:limits_summary}
% \end{center}
% \end{table}
% %%%%%%%%%%%%%%%

%      \begin{table}[t!]
%        \caption{\small{Summary of 95\% CL upper limits on $\BR(t \to cH)$ and $\BR(t \to uH)$, significance and best-fit branching ratio in signal regions with a
%        benchmark branching ratio of $\BR(t \to qH)=0.1\%$}. Expected significance is obtained from Asimov fit with a signal injection corresponding to a branching ratio of 0.1\%. }
%      \begin{center}
%      \resizebox{\textwidth}{!}{
%      \begin{tabular}{lcccccc}
%      \toprule\toprule
%      
%      \multirow{3}{*}{Signal Regions} & \multicolumn{3}{c}{$t\to cH$}                                & \multicolumn{3}{c}{$t\to uH$}  \\
%                                      &  95\% CL upper limits[$10^{-3}$]  & Significance             &  $\BR[10^{-3}]$ &     95\% CL upper limits[$10^{-3}$]  & Significance   &  $\BR%[10^{-      3}]$  \\
%                                      &  \multicolumn{2}{c}{Observed (Expected)}                     &        &     \multicolumn{2}{c}{Observed (Expected)}& \\
%      \midrule
%      $t_h\thadhad$-2j                & $1.85$($2.80^{+1.30}_{-0.78}$)&$-0.96$($0.78$) & $-1.03^{+1.04}_{-1.04}$&$1.10$($1.65^{+0.79}_{-0.46}$)&$-0.90$($1.25$) &$-0.55^{+0.59}_{-0.59%}$ \\
%      $t_h\thadhad$-3j                & $1.18$($1.06^{+0.50}_{-0.30}$)&$0.34$($1.87$)  & $0.16^{+0.47}_{-0.47}$ &$1.00$($0.89^{+0.42}_{-0.25}$)&$0.36$($2.13$)  &$0.14^{+0.40}_{-0.40}%$  \\       \midrule
%      Hadronic Combination            & $1.04$($0.98^{+0.46}_{-0.28}$)&$0.26$ ($1.99$) & $0.11^{+0.43}_{-0.43}$ &$0.78$($0.78^{+0.37}_{-0.22}$)&$0.11$($2.33$)  &$0.04^{+0.34}_{-0.34}%$  \\
%      \midrule
%      $t_l\tauhad$-2j                 &$4.86$($4.32^{+1.89}_{-1.21}$) &$0.40$($0.48$)   &$0.81^{+2.04}_{-2.04}$  & $3.93$($3.55^{+1.56}_{-0.99}$) & $0.34$($0.58$)  &$0.57^{+1.66}_{-%1.66}$      \\
%      $t_l\tauhad$-1j                 &$3.94$($3.67^{+1.66}_{-1.03}$) &$0.24$($0.57$)   &$0.40^{+1.70}_{-1.70}$  & $3.10$($2.87^{+1.29}_{-0.80}$) & $0.24$($0.73$)  &$0.31^{+1.33}_{-%1.33}$      \\
%      $t_h\tlhad$-2j                  &$4.81$($5.85^{+2.90}_{-1.63}$) &$-0.52$($0.39$)  &$-1.36^{+2.56}_{-2.56}$ & $2.56$($3.05^{+1.38}_{-0.85}$) & $-0.48$($0.69$) &$-0.66^{+1.38}_{-%1.38}$      \\
%      $t_h\tlhad$-3j                  &$2.78$($2.79^{+1.36}_{-0.78}$) &$-0.04$($0.76$)  &$-0.04^{+1.26}_{-1.26}$ & $2.07$($2.09^{+0.94}_{-0.58}$) & $-0.05$($0.98$) &$-0.04^{+0.98}_{-%0.98}$      \\
%      $t_l\thadhad$                   &$1.41$($0.63^{+0.29}_{-0.18}$) &$2.64$($3.24$)   &$0.74^{+0.34}_{-0.34}$  & $1.01$($0.45^{+0.21}_{-0.13}$) & $2.64$($4.08$)  &$0.53^{+0.25}_{-%0.25}$      \\ \midrule
%      Leptonic Combination            &$1.29$($0.59^{+0.27}_{-0.17}$) &$2.59$($3.34$)   &$0.68^{+0.32}_{-0.32}$  & $0.92$($0.42^{+0.19}_{-0.12}$) & $2.59$($4.23$)  &$0.48^{+0.23}_{-%0.23}$      \\
%      \midrule
%      Combination                     &$0.99$ ($0.50^{+0.22}_{-0.14}$)&$2.34$($3.69$)   &$0.51^{+0.25}_{-0.25}$ & $0.72$ ($0.36^{+0.17}_{-0.10}$)& $2.31$($4.49$)&$0.37^{+0.18}_{-0.18%}$  \\
%      \bottomrule\bottomrule
%      \end{tabular}
%      }
%      \label{tab:limits_summary}
%      \end{center}
%      \end{table}
%%%%%%%%%%%%%%%


\begin{table}[t!]
  \caption{\small{Summary of 95\% CL upper limits on $\BR(t \to cH)$ and $\BR(t \to uH)$, significance and best-fit branching ratio in signal regions with a
  benchmark branching ratio of $\BR(t \to qH)=0.1\%$}. Expected significance is obtained from Asimov fit with a signal injection corresponding to a branching ratio of 0.1\%. }
\begin{center}
\resizebox{\textwidth}{!}{
\begin{tabular}{lcccccc}
\toprule\toprule

\multirow{3}{*}{Signal Regions} & \multicolumn{3}{c}{$t\to cH$}                                & \multicolumn{3}{c}{$t\to uH$}  \\
                                &  95\% CL upper limits[$10^{-3}$]  & Significance             &  $\BR[10^{-3}]$ &     95\% CL upper limits[$10^{-3}$]  & Significance   &  $\BR[10^{-3}]$  \\
                                &  \multicolumn{2}{c}{Observed (Expected)}                     &        &     \multicolumn{2}{c}{Observed (Expected)}& \\
\midrule
$t_h\thadhad$-2j                & $1.80$($2.72^{+1.18}_{-0.76}$)&$-0.96$($0.78$) & $-1.03^{+1.03}_{-1.03}$&$1.07$($1.60^{+0.71}_{-0.45}$)&$-0.90$($1.31$) &$-0.55^{+0.58}_{-0.58}$ \\
$t_h\thadhad$-3j                & $1.14$($1.02^{+0.45}_{-0.29}$)&$0.34$($1.87$)  & $0.16^{+0.47}_{-0.47}$ &$0.97$($0.86^{+0.38}_{-0.24}$)&$0.36$($2.25$)  &$0.14^{+0.40}_{-0.40}$  \\ \midrule
Hadronic Combination            & $1.00$($0.95^{+0.42}_{-0.27}$)&$0.26$ ($1.99$) & $0.11^{+0.43}_{-0.43}$ &$0.76$($0.76^{+0.33}_{-0.21}$)&$0.12$($2.52$)  &$0.04^{+0.34}_{-0.34}$  \\

\midrule
$t_l\tauhad$-2j                 &$4.77$($4.23^{+1.72}_{-1.18}$) &$0.41$($0.47$)   &$0.85^{+2.06}_{-2.06}$  & $3.84$($3.48^{+1.42}_{-0.97}$) & $0.36$($0.58$)  &$0.61^{+1.68}_{-1.68}$\\
$t_l\tauhad$-1j                 &$3.80$($3.56^{+1.51}_{-0.99}$) &$0.22$($0.58$)   &$0.36^{+1.70}_{-1.70}$  & $2.98$($2.78^{+1.17}_{-0.78}$) & $0.22$($0.73$)  &$0.29^{+1.33}_{-1.33}$\\
$t_h\tlhad$-2j                  &$4.71$($5.71^{+2.68}_{-1.60}$) &$-0.52$($0.38$)  &$-1.36^{+2.56}_{-2.56}$ & $2.50$($2.97^{+1.25}_{-0.83}$) & $-0.47$($0.70$) &$-0.66^{+1.38}_{-1.38}$\\
$t_h\tlhad$-3j                  &$2.71$($2.71^{+1.25}_{-0.76}$) &$-0.03$($0.77$)  &$-0.03^{+1.26}_{-1.26}$ & $2.02$($2.03^{+0.86}_{-0.57}$) & $-0.05$($0.99$) &$-0.03^{+0.98}_{-0.98}$\\
$t_l\thadhad$                   &$1.35$($0.61^{+0.27}_{-0.17}$) &$2.64$($3.31$)   &$0.74^{+0.33}_{-0.33}$  & $0.97$($0.44^{+0.19}_{-0.12}$) & $2.64$($4.38$)  &$0.53^{+0.24}_{-0.24}$\\ \midrule
Leptonic Combination            &$1.25$($0.58^{+0.25}_{-0.16}$) &$2.61$($3.46$)   &$0.69^{+0.31}_{-0.31}$  & $0.88$($0.41^{+0.18}_{-0.11}$) & $2.60$($4.62$)  &$0.49^{+0.22}_{-0.22}$\\
\midrule
Combination                     &$0.94$($0.48^{+0.20}_{-0.14}$)&$2.34$($4.02$)   &$0.51^{+0.24}_{-0.24}$ & $0.69$($0.35^{+0.15}_{-0.10}$)& $2.31$($5.18$)&$0.37^{+0.18}_{-0.18}$  \\
\bottomrule\bottomrule
\end{tabular}
}
\label{tab:limits_summary}
\end{center}
\end{table}



%%%%%%%%%%%%%%
\begin{figure*}[h!]
\begin{center}
\begin{tabular}{@{}cc@{}}
\includegraphics[width=0.49\textwidth]{figures/tcH_Limits.pdf}&
\includegraphics[width=0.49\textwidth]{figures/tuH_Limits.pdf}\\
(a) tcH & (b) tuH \\
\end{tabular}
\caption{\small {95\% CL upper limits on $\BR(t\to cH)$(a) for the individual searches as well as their
combination, assuming $\BR(t\to uH)=0$. 95\% CL upper limits on $\BR(t\to uH)$(b) for the individual searches as well as their
combination, assuming $\BR(t\to cH)=0$. The observed limits (solid lines) are compared with the 
expected (median) limits under the background-only hypothesis (dotted lines). The surrounding shaded bands correspond to the 68\% and 95\% CL intervals around the expected limits, 
denoted by $\pm 1\sigma$ and $\pm 2\sigma$, respectively.
}}
\label{fig:limits_combo_1D_hc} 
\end{center}
\end{figure*}
%%%%%%%%%%%%%%

%%%%%%%%%%%%%%%%%%
%%%%\begin{figure*}[h!]
%%%%\begin{center}
%%%%\includegraphics[width=0.7\textwidth]{figures/tuH_combined_Limit.pdf}
%%%%\caption{\small {95\% CL upper limits on $\BR(t\to Hu)$ for the individual searches as well as their
%%%%combination, assuming $\BR(t\to Hc)=0$. The observed limits (solid lines) are compared with the 
%%%%expected (median) limits under the background-only
%%%%hypothesis (dotted lines). The surrounding shaded bands correspond to the 68\% and 95\% CL intervals around the expected limits, 
%%%%denoted by $\pm 1\sigma$ and $\pm 2\sigma$, respectively.
%%%%}}
%%%%\label{fig:limits_combo_1D_hu} 
%%%%\end{center}
%%%%\end{figure*}

A similar set of results can be obtained by simultaneously varying both branching ratios in the likelihood function.
Figure~\ref{fig:limits_combo_2D}(a) shows the 95\% CL upper limits on the branching ratios in the $\BR(t\to uH)$ versus $\BR(t\to cH)$ plane. 
%The small differences between the limiting values (on the $x$- and $y$-axes) of the branching ratio limits obtained in the two-dimensional scan and 
%those reported in Table~\ref{tab:limits_summary}, result from slightly different choices in the $\HML$ search  
%regarding the final discriminant, which in the two-dimensional case should be common to both signals, and its binning.
%\textbf{Add comment of what discriminant is used in this case and the caveat regarding the corresponding 1D limit.}
The corresponding upper limits on the D6 Wilson coefficients couplings in the $C_{u\phi}$ versus $C_{c\phi}$ plane are shown in Figure~\ref{fig:limits_combo_2D}(b).

\begin{figure*}[t!]
\begin{center}
\subfloat[]{\includegraphics[width=0.49\textwidth]{figures/2DLimits.pdf}}
\subfloat[]{\includegraphics[width=0.49\textwidth]{figures/Wilson_coefficient_smooth.pdf}}
\caption{\small {95\% CL upper limits (a) on the plane of $\BR(t\to uH)$ versus $\BR(t\to cH)$ and (b) on the plane 
of $C_{c\phi}$ versus $C_{u\phi}$ for the combination of the searches. The observed limits (solid lines) are compared with the expected (median) limits under the background-only hypothesis (dotted lines). The surrounding shaded bands correspond to the 68\% and 95\% CL intervals around the expected limits, 
denoted by $\pm 1\sigma$ and $\pm 2\sigma$, respectively.}}
%%=======
%%of $C_{u\phi}$ versus $C_{c\phi}$ for the combination of the searches. The observed limits (solid lines) are compared with the expected (median) limits under the background-only hypothesis (dotted %%lines). The surrounding shaded bands correspond to the 68\% and 95\% CL intervals around the expected limits, 
%%denoted by $\pm 1\sigma$ and $\pm 2\sigma$, respectively. ({\color{red} plot (b) needs to be updated}) }}
%%>>>>>>> e84999e0023e50e93b4b7507152380195248366c
\label{fig:limits_combo_2D} 
\end{center}
\end{figure*}
%%%%%%%%%%%%%%





%\begin{table}
%\caption{ Summary of fake tau scale factors derived in ttCRs. The numbers are shown as: nominal values,statistical errors and systematics erros. }
%\begin{center}
%\begin{tabular}{lcccccc}
%\toprule\toprule
%
%\multirow{2}{*}{Fake Factor Types} & \multicolumn{3}{c}{1 prong}                                                        & \multicolumn{3}{c}{3 prong}  \\
%                                &  $25-35$ GeV  & $35-45$ GeV  &  $45-$ GeV                                             &  $25-35$ GeV  & $35-45$ GeV       &  $45-$ GeV  \\
%\midrule
%Type-1                          &$0.71 \pm 0.01 \pm 0.03 $ &$0.61 \pm 0.02 \pm 0.04 $ &$0.38 \pm 0.02 \pm 0.05 $        & $1.01 \pm 0.03 \pm 0.04 $ & $1.09 \pm 0.04 \pm 0.05 $ & $0.30 \pm 0.05 \pm 0.07 $ \\
%% W fake OS       
%Type-2                          &$0.76 \pm 0.06 \pm 0.04 $ & $0.37 \pm 0.08 \pm 0.02$ & $0.74 \pm 0.08 \pm 0.02 $       & $0.93 \pm 0.10 \pm 0.04 $ & $1.05 \pm 0.09 \pm 0.03 $ & $0.79 \pm 0.09 \pm 0.04 $ \\
%% W fake SS          
%Type-3                          &$0.62 \pm 0.10 \pm 0.03 $ &$0.83 \pm 0.09 \pm 0.03 $ &$0.94 \pm 0.07 \pm 0.02 $        & $1.07 \pm 0.13 \pm 0.03 $ &$1.39 \pm 0.12 \pm 0.03 $ &$1.26 \pm 0.10 \pm 0.04 $  \\
%%  b fake       
%Type-4                          &$1.20 \pm 0.02 \pm 0.01 $ & $1.01 \pm 0.04 \pm 0.02 $ &$0.76 \pm 0.03 \pm 0.03 $       &$1.28 \pm 0.07 \pm 0.02 $ &$0.66 \pm 0.08 \pm 0.01 $ & $0.71 \pm 0.07 \pm 0.02 $ \\
%% other fake
%\bottomrule\bottomrule
%\end{tabular}
%\label{tab:ff_summary}
%\end{center}
%\end{table}


\begin{table}
\caption{ Summary of fake tau (1-prong) scale factors derived in ttCRs. The numbers are shown as: nominal values,statistical errors and systematics erros. }
\begin{center}
\begin{tabular}{lcccccc}
\toprule\toprule

\multirow{2}{*}{Fake Factor Types} & \multicolumn{3}{c}{1 prong}  \\                                                
                                   &  $25-35$ GeV  & $35-45$ GeV  &  $45-$ GeV            \\                           
\midrule
Type-1                          &$0.71 \pm 0.01 \pm 0.03 $ &$0.61 \pm 0.02 \pm 0.04 $ &$0.38 \pm 0.02 \pm 0.05 $  \\
% W fake OS       
Type-2                          &$0.76 \pm 0.06 \pm 0.04 $ & $0.37 \pm 0.08 \pm 0.02$ & $0.74 \pm 0.08 \pm 0.02 $ \\
% W fake SS          
Type-3                          &$0.62 \pm 0.10 \pm 0.03 $ &$0.83 \pm 0.09 \pm 0.03 $ &$0.94 \pm 0.07 \pm 0.02 $  \\
%  b fake       
Type-4                          &$1.20 \pm 0.02 \pm 0.01 $ & $1.01 \pm 0.04 \pm 0.02 $ &$0.76 \pm 0.03 \pm 0.03 $  \\
% other fake
\bottomrule\bottomrule
\end{tabular}
\label{tab:ff1_summary}
\end{center}
\end{table}


\begin{table}
\caption{ Summary of fake tau (3-prong) scale factors derived in ttCRs. The numbers are shown as: nominal values,statistical errors and systematics erros. }
\begin{center}
\begin{tabular}{lcccccc}
\toprule\toprule

\multirow{2}{*}{Fake Factor Types}        & \multicolumn{3}{c}{3 prong}  \\
                                          &  $25-35$ GeV  & $35-45$ GeV       &  $45-$ GeV  \\
\midrule
Type-1                                    & $1.01 \pm 0.03 \pm 0.04 $ & $1.09 \pm 0.04 \pm 0.05 $ & $0.30 \pm 0.05 \pm 0.07 $ \\
% W fake OS       
Type-2                                    & $0.93 \pm 0.10 \pm 0.04 $ & $1.05 \pm 0.09 \pm 0.03 $ & $0.79 \pm 0.09 \pm 0.04 $ \\
% W fake SS          
Type-3                                    & $1.07 \pm 0.13 \pm 0.03 $ &$1.39 \pm 0.12 \pm 0.03 $ &$1.26 \pm 0.10 \pm 0.04 $  \\
%  b fake       
Type-4                                    &$1.28 \pm 0.07 \pm 0.02 $ &$0.66 \pm 0.08 \pm 0.01 $ & $0.71 \pm 0.07 \pm 0.02 $ \\
% other fake
\bottomrule\bottomrule
\end{tabular}
\label{tab:ff2_summary}
\end{center}
\end{table}


