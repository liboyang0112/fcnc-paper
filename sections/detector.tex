%-------------------------------------------------------------------------------
\section{ATLAS detector}
\label{sec:detector}
%-------------------------------------------------------------------------------

The ATLAS detector~\cite{PERF-2007-01} at the LHC covers almost the entire solid angle around the collision point,\footnote{ATLAS 
uses a right-handed coordinate system with its origin at the nominal interaction point (IP) in the 
centre of the detector.  
%and the $z$-axis coinciding with the axis of the beam pipe.
The $x$-axis points from
the IP to the centre of the LHC ring, %and
the $y$-axis points upward,
and the $z$-axis coincides with the axis of the beam pipe.
Cylindrical coordinates ($r$,$\phi$) are used 
in the transverse plane, $\phi$ being the azimuthal angle around the beam pipe. The pseudorapidity is defined in 
terms of the polar angle $\theta$ as $\eta = - \ln \tan(\theta/2)$.
Angular separation is measured in units of $\Delta R\equiv \sqrt{(\Delta\eta)^2+(\Delta\phi)^2}$.} and it consists of an inner tracking detector surrounded by a thin superconducting solenoid producing a
2~T axial magnetic field, electromagnetic and hadronic calorimeters, and a muon spectrometer incorporating three large toroid magnet assemblies with eight coils each. The inner detector contains a high-granularity silicon pixel detector, including the %newly-installed
insertable B-layer~\cite{IBL1,IBL2,Abbott:2018ikt}, installed in 2014, and a silicon microstrip tracker, together providing a precise reconstruction of tracks of charged particles in the pseudorapidity range $|\eta|<2.5$.
The inner detector also includes a transition radiation tracker that provides tracking and electron identification for $|\eta|<2.0$.
%The electromagnetic (EM) sampling calorimeter uses lead as the absorber material 
%and liquid-argon (LAr) as the active medium, and is divided into barrel ($|\eta|<1.475$) and end-cap ($1.375<|\eta|<3.2$) regions.  
%Hadron calorimetry is also based on the sampling technique, with either scintillator tiles or LAr as the active medium, and with 
%steel, copper, or tungsten as the absorber material. The calorimeters cover $|\eta|<4.9$. 
%Hadronic calorimetry is provided by the steel/scintillating-tile calorimeter,
%segmented into three barrel structures within $|\eta| < 1.7$, and two copper/LAr hadronic endcap calorimeters.
%The solid angle coverage is completed with forward copper/LAr and tungsten/LAr calorimeter modules
%optimised for electromagnetic and hadronic measurements respectively.
The calorimeter system covers the pseudorapidity range $|\eta| < 4.9$. Within the region $|\eta|< 3.2$, electromagnetic (EM) calorimetry is provided by barrel and endcap high-granularity lead/liquid-argon (LAr) sampling calorimeters, with an additional thin LAr presampler covering $|\eta| < 1.8$ to correct for energy loss in material upstream of the calorimeters. Hadronic calorimetry is provided by %the
a steel/scintillator-tile calorimeter, segmented into three barrel structures within $|\eta| < 1.7$, and two copper/LAr hadronic endcap calorimeters.
The solid angle coverage is completed with forward copper/LAr and tungsten/LAr calorimeter modules optimised for electromagnetic and hadronic measurements, respectively.
The calorimeters are surrounded by a muon spectrometer within a magnetic field provided by air-core toroid magnets with a bending integral of about \SI{2.5}{\tesla\metre} in the barrel and up to \SI{6.0}{\tesla\metre} in the endcaps. 
The muon spectrometer measures the trajectories of muons with $|\eta|<2.7$ using multiple layers of high-precision tracking chambers, and it is instrumented with separate trigger chambers covering $|\eta|<2.4$. A two-level trigger system~\cite{Aaboud:2016leb}, consisting of a hardware-based level-1 trigger followed by a software-based high-level trigger, is used to reduce the event rate to a maximum of around \SI{1}{\kHz} for offline storage.
An extensive software suite~\cite{ATL-SOFT-PUB-2021-001} is used in the reconstruction and analysis of real and simulated data, in detector operations, and in the trigger and data acquisition systems of the experiment.
