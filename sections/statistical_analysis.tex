\section{Statistical analysis}
\label{sec:stat_analysis}

The final discriminant distributions across all analysis regions considered are jointly analysed to test for the 
presence of a signal. The statistical analysis uses a binned likelihood function ${\cal L}(\mu,\theta)$ constructed as
a product of Poisson probability terms over all bins considered in the search. This function depends
on the signal-strength parameter $\mu$, defined as a factor multiplying the expected yield of $\Hq$ signal events
normalised to a reference branching ratio $\BR_{\mathrm{ref}}(t\to Hq)=0.2\%$,
and $\theta$, a set of nuisance parameters that encode the effect of systematic uncertainties on the signal and background expectations. 
Therefore, the expected total number of events in a given bin depends on $\mu$ and $\theta$. 
All nuisance parameters are subject to Gaussian constraints in the likelihood.
For a given value of $\mu$, the nuisance parameters $\theta$ allow variations of the expectations for signal and background
according to the corresponding systematic uncertainties, and their fitted values result in the deviations from
the nominal expectations that globally provide the best fit to the data.
This procedure allows a reduction of the impact of systematic uncertainties on 
the search sensitivity by taking advantage of the highly populated background-dominated bins included in the likelihood fit.
%To verify the improved background prediction, fits under the background-only hypothesis are performed, 
%and differences between the data and the post-fit background prediction are checked 
%using kinematic variables other than the ones used in the fit. 
Statistical uncertainties in each bin of the predicted final discriminant distributions are taken into account by dedicated parameters in the fit.     
The best-fit $\BR(t\to Hq)$ is obtained by performing a binned likelihood fit to the data under the signal-plus-background
hypothesis, maximising the likelihood function ${\cal L}(\mu,\theta)$ over $\mu$ and $\theta$.

The fitting procedure was initially validated through extensive studies using mock data, defined as the sum of all predicted backgrounds 
plus an injected signal of variable strength, as well as by performing fits to real data where bins of the final discriminant variable with 
a signal contamination above 5\% are excluded (referred to as blinding requirements).
In both cases, the robustness of the model for systematic uncertainties is established by verifying the stability of the fitted background 
when varying assumptions about some of the leading sources of uncertainty. 
After this, the blinding requirements
are removed in the data and a fit under the signal-plus-background hypothesis is performed. Further checks involve the comparison of the fitted 
nuisance parameters before and after removal of the blinding requirements, and their values are found to be consistent. In addition, it is verified that the 
fit is able to correctly determine the strength of a simulated signal injected into the real data.

The test statistic $q_\mu$ is defined as the profile likelihood ratio, 
$q_\mu = -2\ln({\cal L}(\mu,{\hat{\theta}}_\mu)/{\cal L}(\hat{\mu},\hat{\theta}))$,
where $\hat{\mu}$ and $\hat{\theta}$ are the values of the parameters that
maximise the likelihood function (subject to the constraint $0\leq \hat{\mu} \leq \mu$), and ${\hat{\theta}}_\mu$ are the values of the
nuisance parameters that maximise the likelihood function for a given value of $\mu$. 
The test statistic $q_\mu$ is evaluated with the {\textsc RooFit} package~\cite{Verkerke:2003ir,RooFitManual}.
A related statistic is used to determine whether the observed data is compatible with the background-only hypothesis (the so-called discovery test)  
by setting $\mu=0$ in the profile likelihood ratio and leaving $\hat{\mu}$ unconstrained: $q_0 = -2\ln({\cal L}(0,{\hat{\theta}}_0)/{\cal L}(\hat{\mu},\hat{\theta}))$.
The $p$-value (referred to as $p_0$), representing the level of agreement between the data and the background-only hypothesis, is estimated by integrating
%representing the probability of the data being compatible with the background-only hypothesis is estimated by integrating
%the distribution of $q_0$ obtained from background-only pseudo-experiments, approximated using the asymptotic formulae given in Refs.~\cite{Cowan:2010js}, 
the distribution of $q_0$ based on the asymptotic formulae in Ref.~\cite{Cowan:2010js}, 
above the observed value of $q_0$ in the data. 
%The observed $p_0$-value is checked for each explored signal scenario.
%In the case of the data being compatible with the background-only hypothesis, 
Upper limits on $\mu$, and thus on 
$\BR(t\to Hq)$, are derived by using $q_\mu$ in the CL$_{\textrm{s}}$ method~\cite{Junk:1999kv,Read:2002hq}.
For a given signal scenario, values of the $\BR(t\to Hq)$ yielding CL$_{\textrm{s}} < 0.05$, 
where CL$_{\textrm{s}}$ is computed using the asymptotic approximation~\cite{Cowan:2010js}, are excluded at $\geq 95\%$ CL.


