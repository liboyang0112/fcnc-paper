\section{Statistical analysis}
\label{sec:stat_analysis}

For each search, the final discriminant distributions across all analysis regions considered are jointly analysed to test for the 
presence of a signal. The statistical analysis uses a binned likelihood function ${\cal L}(\mu,\theta)$ constructed as
a product of Poisson probability terms over all bins considered in the search. This function depends
on the signal-strength parameter $\mu$, defined as a multiplicative factor to the yield for $\Hq$ signal events
normalised to a reference branching ratio $\BR_{\mathrm{ref}}(t\to Hq)=1\%$,
and $\theta$, a set of nuisance parameters that encode the effect of systematic uncertainties in the signal and background expectations. 
Therefore, the expected total number of events in a given bin depends on $\mu$ and $\theta$. 
All nuisance parameters are implemented in the likelihood function as Gaussian or log-normal constraints, with the exception of a few parameters 
that control the normalisation of the some of the background components (e.g. the \ttbin\ background in the case of the $\Hbb$ search) 
and are treated as free parameters in the fit.

For a given value of $\mu$, the nuisance parameters $\theta$ allow variations of the expectations for signal and background
according to the corresponding systematic uncertainties, and their fitted values result in the deviations from
the nominal expectations that globally provide the best fit to the data.
This procedure allows a reduction of the impact of systematic uncertainties on 
the search sensitivity by taking advantage of the highly populated background-dominated bins included in the likelihood fit.
To verify the improved background prediction, fits under the background-only hypothesis are performed, 
and differences between the data and the post-fit background prediction are checked 
using kinematic variables other than the ones used in the fit. 
Statistical uncertainties in each bin of the predicted final discriminant distributions due to the limited size of the simulated samples 
are taken into account by dedicated parameters in the fit.     
The best-fit $\BR(t\to Hq)$ is obtained by performing a binned likelihood fit to the data under the signal-plus-background
hypothesis, i.e. maximising the likelihood function $L(\mu,\theta)$ over $\mu$ and $\theta$.

The test statistic $q_\mu$ is defined as the profile likelihood ratio: 
$q_\mu = -2\ln({\cal L}(\mu,{\hat{\theta}}_\mu)/{\cal L}(\hat{\mu},\hat{\theta}))$,
where $\hat{\mu}$ and $\hat{\theta}$ are the values of the parameters that
maximise the likelihood function (subject to the constraint $0\leq \hat{\mu} \leq \mu$), and ${\hat{\theta}}_\mu$ are the values of the
nuisance parameters that maximise the likelihood function for a given value of $\mu$. 
The test statistic $q_\mu$ is evaluated with the {\textsc RooFit} package~\cite{Verkerke:2003ir,RooFitManual}.
A related statistic is used to determine the probability that the observed data are compatible with the background-only hypothesis (i.e.~the discovery test)  
by setting $\mu=0$ in the profile likelihood ratio and leaving $\hat{\mu}$ unconstrained: $q_0 = -2\ln({\cal L}(0,{\hat{\theta}}_0)/{\cal L}(\hat{\mu},\hat{\theta}))$.
The $p$-value (referred to as $p_0$) representing the probability of the data being compatible with the 
background-only hypothesis is estimated by integrating
the distribution of $q_0$ obtained from background-only pseudo-experiments, approximated using the asymptotic formulae given in Refs.~\cite{Cowan:2010js,ErratumCowan:2010js}, above the observed value of $q_0$ in the data. The observed $p_0$-value is checked for each explored signal scenario.
In the absence of any significant excess above the background expectation, upper limits on $\mu$, and thus on 
$\BR(t\to Hq)$, are derived by using $q_\mu$ in the CL$_{\textrm{s}}$ method~\cite{Junk:1999kv,Read:2002hq}.
For a given signal scenario, values of the $\BR(t\to Hq)$ yielding CL$_{\textrm{s}} < 0.05$, 
where CL$_{\textrm{s}}$ is computed using the asymptotic approximation~\cite{Cowan:2010js,ErratumCowan:2010js}, are excluded at $\geq 95\%$ CL.


