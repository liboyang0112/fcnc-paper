%-------------------------------------------------------------------------------
\section{Data sample and event preselection}
\label{sec:data_presel}
%-------------------------------------------------------------------------------

The search is based on a dataset of $pp$ collisions at $\sqrt{s}=13~\tev$ with 25 ns bunch spacing collected from 2015 to 2018, corresponding to an integrated luminosity of $139~\ifb$.
Only events recorded with a single-electron trigger, a single-muon trigger, or a di-$\tau$-lepton trigger~\cite{TRIG-2018-05,TRIG-2018-01,id_trigger,l1topo_trigger} under stable beam conditions 
and for which all detector subsystems were operational are considered for analysis. The events recorded by dilepton triggers that fall into the control regions are used for fake-$\tau$-lepton background
estimation as discussed in Section~\ref{sec:background_model}. The number of $pp$ interactions per bunch crossing in this dataset ranges from about 8 to 45, with an average of 24.

Single-electron and single-muon triggers with low $\pt$ thresholds and lepton isolation requirements are combined in a logical OR 
with higher-threshold triggers that have a looser identification criterion and no isolation requirement.
The lowest $\pt$ threshold used for muons is 20 (26)~\gev\ in 2015 (2016--2018), while for electrons the threshold is 24 (26)~\gev.
%For ditau triggers, the $\pt$ threshold of the leading (trailing) $\tauhad$ candidate is 35 (25)~\gev\ with the medium identification required.
For di-$\tau$ triggers, the $\pt$ threshold for the leading (subleading) $\had$ candidate is 35 (25)~\gev.
%Events satisfying the trigger selection are required to have at least one primary vertex candidate.
To reduce the impact of the trigger efficiency uncertainty around the threshold, the leptons are required to have a \pt that is at least 1~\GeV\ above the threshold. 
The reconstructed $\tau$-leptons are required to have a \pt at least 5~\GeV\ higher than the trigger threshold.
%and $|\eta|<2.5$, excluding the EM calorimeter's transition region.
%The signal final state including a light lepton (electron or muon) or $\lep$ are referred to as leptonic channel, otherwise hadronic channel.
%The events in the leptonic channel are recorded by single-electron or single-muon trigger, required to have exactly one electron or muon that matches, with $\Delta R < 0.15$, the lepton reconstructed by the trigger. Further requirements are defined targeting different signal decay modes:  
The events in the leptonic channels are recorded by a single-electron or single-muon trigger, and are required to have exactly one electron or muon that matches, within $\Delta R < 0.15$, the lepton reconstructed by at least one of the possible triggers. The following additional requirements are applied.
%Requirements are placed on the following additional quantities.
%Further requirements are defined targeting different signal decay modes:  
\begin{itemize}
\item $t_h\lephad$: To enhance selection of the $t_hH$ and $t_ht(qH)$ final states with a $H\to\lephad$ decay, exactly one $\had$ with opposite-sign charge to $\lep$ is required, plus at least three jets with exactly one $b$-jet.
\item $t_{\ell}\hadhad$: To enhance selection of $t_{\ell}H$ and $t_{\ell}t(qH)$ final states with a $H\to\hadhad$ decay, exactly one light lepton and two opposite-sign $\had$ are required,
  plus jets with exactly one $b$-jet.
\item $t_{\ell}\had$: This channel enhances the selection of $t_{\ell}H$ and $t_{\ell}t(qH)$ final states with a $H\to\hadhad$ decay where one $\had$ fails the applied reconstruction or identification
  criterion so that there is only one reconstructed $\had$ candidate.
  To reduce the SM backgrounds and avoid overlaps with the final states used in other ATLAS searches, exactly one $\had$ with the same charge as that assigned to the light lepton is required.
  In addition, at least two jets including exactly one $b$-jet are required.
\end{itemize}

The events in the hadronic channel are selected by a di-$\tau$ trigger. Further requirements are:
%recorded by di-tau trigger, required to have both leading and subleading $\had$ passed the trigger. Further requirements are defined targeting signal decay modes:
\begin{itemize}
\item $t_h\hadhad$: The $t_hH$ and $t_ht(qH)$ final states with $H\to\hadhad$ decay are targeted. Exactly two $\had$ with opposite-sign charge and at least three jets, including exactly one $b$-jet, are required.
  %are selected, including at least 3 jet with exactly one $b$-jet.
\end{itemize}

The above requirements apply to the reconstructed objects defined in Section~\ref{sec:objects}.
%, which are in general different between both searches. 
These requirements are referred to as the preselection and are summarised in Table~\ref{tab:preselection}. 

%%%%%%%%%%%%%%%
\begin{table*}[t!]
\caption{\small{Summary of the preselection requirements. 
The leading and subleading $\had$ candidates are denoted by $\tau_{\mathrm{had1}}$ and $\tau_{\mathrm{had2}}$ respectively.}}
\begin{center}
\begin{tabular}{c|ccc|c}
\toprule\toprule
\multirow{2}{*}{Requirement} &  \multicolumn{3}{c|}{Leptonic channels}  & \multicolumn{1}{c}{Hadronic channel} \\ 
%\multirow{2}{*}{Requirement} &  \multicolumn{3}{c|}{leptonic channel}  & hadronic channel \\ 
& $t_h\lephad$ & $t_{\ell}\hadhad$ &  $t_{\ell}\had$ & $t_h\hadhad$\\
\midrule
Trigger & \multicolumn{3}{c|}{single-lepton trigger} & di-$\tau$ trigger  \\
Leptons  & \multicolumn{3}{c|}{=1 isolated $e$ or $\mu$}  & =0 isolated $e$ or $\mu$ \\
$\had$  & $=$1 $\had$ & $=$2 $\had$ & $=$1 $\had$ & $=$2 $\had$ \\
Electric charge ($Q$) & $Q_\ell \times Q_{\tau_{\mathrm{had1}}} = -1$ & $Q_{\tau_{\mathrm{had1}}} \times Q_{\tau_{\mathrm{had2}}} = -1$ & $Q_\ell \times Q_{\tau_{\mathrm{had1}}} = 1$ & $Q_{\tau_{\mathrm{had1}}} \times Q_{\tau_{\mathrm{had2}}} = -1$ \\
Jets  &   $\geq$3 jets & $\geq$1 jets & $\geq$2 jets & $\geq$3 jets \\
$b$-tagging & \multicolumn{3}{c|}{=1 $b$-jets} & =1 $b$-jets\\
\bottomrule\bottomrule
\end{tabular}
\label{tab:preselection}
\end{center}
\end{table*}
%%%%%%%%%%%%%%%
