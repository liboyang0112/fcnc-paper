%-------------------------------------------------------------------------------
\section{Data sample and event preselection}
\label{sec:data_presel}
%-------------------------------------------------------------------------------

Both searches are based on a dataset of $pp$ collisions at $\sqrt{s}=13~\tev$ with 25 ns bunch spacing collected in 2015 and 2016, corresponding to an integrated luminosity of $36.1~\ifb$.
Only events recorded with a single-electron trigger, a single-muon trigger, or a di-$\tau$ trigger under stable beam conditions 
and for which all detector subsystems were operational are considered.
The number of $pp$ interactions per bunch crossing in this dataset ranges from about 8 to 45, with an average of 24.
%The mean number of $pp$ interactions per bunch crossing in the dataset is 24.

%Single-electron and single-muon triggers with low $\pt$ threshold and lepton isolation requirements are combined in a logical OR 
%with higher-threshold triggers without isolation requirements to ensure maximum efficiency. 
%For muon triggers, the lowest $\pt$ threshold is 20 (26)~\gev\ in 2015 (2016), while the higher $\pt$ threshold is 50~\gev\ in both years. 
%For electrons, triggers with a $\pt$ threshold of 24 (26)~\gev\ in 2015 (2016) and isolation requirements are used
%along with triggers with a 60~\gev\ threshold and no isolation requirements, and with a 120 (140)~\gev\ threshold 
%with looser identification criteria and no isolation requirements.
Single-electron and single-muon triggers with low $\pt$ thresholds and lepton isolation requirements are combined in a logical OR 
with higher-threshold triggers but with a looser identification criterion and without any isolation requirement.
The lowest $\pt$ threshold used for muons is 20 (26)~\gev\ in 2015 (2016), while for electrons the threshold is 24 (26)~\gev.
%For ditau triggers, the $\pt$ threshold of the leading (trailing) $\tauhad$ candidate is 35 (25)~\gev\ with the medium identification required.
For di-$\tau$ triggers, the $\pt$ threshold of the leading (trailing) $\had$ candidate is 35 (25)~\gev.
In both searches, events satisfying the trigger selection are required to have at least one primary vertex candidate.

Events selected by the $\Hbb$ search are recorded with a single-electron or single-muon trigger and 
are required to have exactly one electron or muon that matches, with $\Delta R < 0.15$, the lepton reconstructed by the trigger.  
Furthermore, at least four jets are required, of which at least two must be $b$-tagged.

In the $\Htautau$ search, events are classified into $\lephad$ and $\hadhad$ channels depending on the 
multiplicity of selected leptons. Events in the $\lephad$ channel are recorded with a single-electron or single-muon trigger 
and are required to have exactly one selected electron or muon and at least one $\had$ candidate. 
The selected electron or muon is required to match, with $\Delta R < 0.15$, the lepton reconstructed by the trigger 
and to have a $\pt$ exceeding the trigger $\pt$ threshold by 1~\gev\ or 2~\gev\ (depending on the lepton trigger and 
data-taking conditions). In addition, its electric charge is required to be of opposite sign to that of the leading $\had$ candidate.
Events in the $\hadhad$ channel are recorded with a di-$\tau$ trigger, and are required to have at least two $\had$ candidates and 
no selected electrons or muons. The two leading $\had$ candidates are required to have charges of opposite sign. 
In addition, in both $\Htautau$ search channels, trigger matching for $\had$ candidates, at least three jets and exactly one $b$-tagged jet are required.

The above requirements apply to the reconstructed objects defined in Section~\ref{sec:objects}.
%, which are in general different between both searches. 
These requirements, which ensure a negligible overlap between the $\Hbb$ and $\Htautau$ searches,
are referred to as the preselection and are summarised in Table~\ref{tab:preselection}. 

%%%%%%%%%%%%%%%
\begin{table*}[t!]
\caption{\small{Summary of preselection requirements for the $\Hbb$ and $\Htautau$ searches. 
The leading and trailing $\had$ candidates are denoted by $\tau_{\mathrm{had,1}}$ and $\tau_{\mathrm{had,2}}$ respectively.}}
\begin{center}
\begin{tabular}{l|c|cc}
\toprule\toprule
\multicolumn{4}{c}{Preselection requirements} \\      
\midrule
Requirement &  $\Hbb$ search & \multicolumn{2}{c}{$\Htautau$ search} \\      
& & $\lephad$ channel & $\hadhad$ channel \\
\midrule
Trigger & single-lepton trigger & single-lepton trigger & di-$\tau$ trigger  \\
Leptons  & =1 isolated $e$ or $\mu$ & =1 isolated $e$ or $\mu$ & no isolated $e$ or $\mu$ \\
               & -- & $\geq$1 $\had$ & $\geq$2 $\had$ \\
Electric charge ($q$) & -- & $q_\ell \times q_{\tau_{\mathrm{had,1}}} < 0$ & $q_{\tau_{\mathrm{had,1}}} \times q_{\tau_{\mathrm{had,2}}} < 0$ \\
Jets  &  $\geq$4 jets & $\geq$3 jets & $\geq$3 jets \\
$b$-tagging & $\geq$2 $b$-tagged jets & =1 $b$-tagged jets & =1 $b$-tagged jets  \\
\bottomrule\bottomrule
\end{tabular}
\label{tab:preselection}
\end{center}
\end{table*}
%%%%%%%%%%%%%%%
