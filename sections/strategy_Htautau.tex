%-------------------------------------------------------------------------------
\section{Analysis strategy}
\label{sec:strategy_Htautau}
%-------------------------------------------------------------------------------

The analysis strategy adopted in the $\Htautau$ search is similar to the one used in Ref.~\cite{fcnc36,Chen:2015nta} but extended to more search channels.
%\subsection{Event categorisation and kinematic reconstruction}
%\label{sec:htautau_reco_cat}
The $tt(qH)$ and $tH$ signal being probed are characterised by the presence of $\tau$-leptons from the decay of 
the Higgs boson, where the remaining top quark decays into $Wb$. There is an additional $q$-jet from the FCNC $t\to qH$ decay in the top pair production. 
%at least  four (three) jets in the decay (production) mode, only one of which originates from a $b$-quark.
If the $W$ boson or one of the $\tau$-leptons decays leptonically, an isolated electron or muon, together with significant $\met$ is also expected.
In a significant fraction of the events, the lowest $\pt$ jet from the hadronic $W$ boson decay fails the minimum $\pt$ requirement of $25~\gev$,
resulting in only three reconstructed jets where the production mode is dominant. 
%some signal events mixed with production mode having only three reconstructed jets.
In order to optimise the sensitivity of the search, the selected events are categorised into seven SRs based on the number of light leptons,
$\had$ candidates, and on the number of light flavored jets:
$t_{\ell}\hadhad$, $t_{\ell}\had$-1j, $t_{\ell}\had$-2j, $t_h\lephad$-2j, $t_h\lephad$-3j, $t_h\hadhad$-2j, and $t_h\hadhad$-3j, as shown in Table~\ref{tab:srcr}. 

%   \begin{table}
%   \centering
%   \caption{Overview of the signal regions and the control regions used for fake tau scale factor derivation in leptonic channels.}
%   \label{tab:srcr}
%   \begin{tabular}[h]{c|c|c|c|c|c|c}
%   \hline \hline
%   \multicolumn{2}{c|}{Regions} & $b$-jet & light flavor jets        & lepton & hadronic taus & charge\\ \hline
%   \multirow{7}{*}{SR}&$t_{\ell}\thadhad$     & 1     & any                                & 1      & 2             & $\thadhad$ OS\\ \cline{2-7}
%   &$t_{\ell}\tauhad$-1j  & 1     & 1                                   & 1      & 1                     & $t_{\ell}\tauhad$ SS\\ \cline{2-7}
%   &$t_{\ell}\tauhad$-2j  & 1     & 2                                        & 1      & 1                     & $t_{\ell}\tauhad$ SS\\ \cline{2-7}
%   &$t_h\tlhad$-2j   & 1     & 2                           & 1      & 1             & $\tlhad$ OS\\ \cline{2-7}
%   &$t_h\tlhad$-3j   & 1     & $\ge3$                      & 1      & 1             & $\tlhad$ OS\\ \cline{2-7}
%   &$t_h\thadhad$-2j & 1     & 2                            & 0      & 2             & $\thadhad$ OS\\ \cline{2-7}
%   &$t_h\thadhad$-3j & 1     & $\ge3$                       & 0      & 2             & $\thadhad$ OS\\ \hline
%   \multirow{6}{*}{CRtt}&$t_{\ell}t_{\ell}1b\tauhad$ & 1     & any                           & 2      & 1                     & $t_{\ell}t_{\ell}$ OS\\ \cline{2-7}
%   &$t_{\ell}t_{\ell}2b\tauhad$      & 2     & any                           & 2      & 1                     & $t_{\ell}t_{\ell}$ OS\\ \cline{2-7}
%   &$t_{\ell}t_h2b\tauhad$-2jSS & 2     & 2                             & 1      & 1             & $t_{\ell}\tauhad$ SS\\ \cline{2-7}
%   &$t_{\ell}t_h2b\tauhad$-2jOS & 2     & 2                             & 1      & 1             & $t_{\ell}\tauhad$ OS\\ \cline{2-7}
%   &$t_{\ell}t_h2b\tauhad$-3jSS & 2     & $\ge3$                        & 1      & 1             & $t_{\ell}\tauhad$ SS\\ \cline{2-7}
%   &$t_{\ell}t_h2b\tauhad$-3jOS & 2     & $\ge3$                & 1      & 1             & $t_{\ell}\tauhad$ OS\\ \hline
%   \end{tabular}
%   \end{table}


\begin{table}
\centering
\caption{Overview of the signal regions (SR) and the $t\bar{t}$ control regions (CRtt) used for fake tau scale factor derivation in leptonic channels. Leptons are required to have either same-sign (SS) or opposite-sign (OS) charges in each region.}
\label{tab:srcr}
\begin{tabular}[h]{c|c|c|c|c|c|c}
\hline \hline
\multicolumn{2}{c|}{Regions} & $b$-jet & light flavor jets        & lepton & hadronic taus & charge\\ \hline
\multirow{7}{*}{SR}&$t_{\ell}\thadhad$     & 1     & $\ge0$                                & 1      & 2             & $\thadhad$ OS\\ \cline{2-7}
&$t_{\ell}\tauhad$-1j  & 1     & 1                                   & 1      & 1                     & $t_{\ell}\tauhad$ SS\\ \cline{2-7}
&$t_{\ell}\tauhad$-2j  & 1     & 2                                        & 1      & 1                     & $t_{\ell}\tauhad$ SS\\ \cline{2-7}
&$t_h\tlhad$-2j   & 1     & 2                           & 1      & 1             & $\tlhad$ OS\\ \cline{2-7}
&$t_h\tlhad$-3j   & 1     & $\ge3$                      & 1      & 1             & $\tlhad$ OS\\ \cline{2-7}
&$t_h\thadhad$-2j & 1     & 2                            & 0      & 2             & $\thadhad$ OS\\ \cline{2-7}
&$t_h\thadhad$-3j & 1     & $\ge3$                       & 0      & 2             & $\thadhad$ OS\\ \hline
\multirow{6}{*}{CRtt}&$t_{\ell}t_{\ell}1b\tauhad$ & 1     & $\ge0$                            & 2      & 1                     & $t_{\ell}t_{\ell}$ OS\\ \cline{2-7}
&$t_{\ell}t_{\ell}2b\tauhad$      & 2     & $\ge0$                            & 2      & 1                     & $t_{\ell}t_{\ell}$ OS\\ \cline{2-7}
&$t_{\ell}t_h2b\tauhad$-2jSS & 2     & 2                             & 1      & 1             & $t_{\ell}\tauhad$ SS\\ \cline{2-7}
&$t_{\ell}t_h2b\tauhad$-2jOS & 2     & 2                             & 1      & 1             & $t_{\ell}\tauhad$ OS\\ \cline{2-7}
&$t_{\ell}t_h2b\tauhad$-3jSS & 2     & $\ge3$                        & 1      & 1             & $t_{\ell}\tauhad$ SS\\ \cline{2-7}
&$t_{\ell}t_h2b\tauhad$-3jOS & 2     & $\ge3$                & 1      & 1             & $t_{\ell}\tauhad$ OS\\ \hline
\end{tabular}
\end{table}







%This event categorisation is primarily motivated by the different quality of the event kinematic reconstruction, depending on the amount 
%of $\met$ in the event (larger in $\lephad$ events compared with $\hadhad$ events), and whether a jet from the hadronic top-quark decay %is missing or not (events with exactly three jets or at least four jets).
This event categorisation is primarily motivated by optimizing the sensitivity in each signal region that targets either leptonic or hadronic top-quark
decays as well as the Higgs decays into either $\lephad$ or $\hadhad$ final states.   
The event kinematic reconstruction is used in the $t_h\hadhad$ and $t_h\lephad$ channels to suppress the background by constraining the
di-tau mass and the missing transverse energy in the event~\cite{Chen:2015nta}.

%based on the strategy used in Ref.~\cite{Chen:2015nta}, and is summarised below.

For the $t_ht(qH)$ events, the jet from $t\to qH$, denoted as the FCNC jet ($q$-jet), should be a hard narrow jet from the
decay chain $t\to qH\to q\tau\tau$, with taus reconstructed as $\lephad$ or $\hadhad$.
There should be at least four jets among which the one with the smaller angular distance to the visible decay of the di-tau,
is considered as the $q$-jet since the FCNC top decay products are likely boosted closer together than other jets. 
If there are more than two jets besides the $q$- and $b$-jet, the jets from $W$ boson decay are chosen from the combination
with the invariant mass closest to the $W$ boson mass. There is the chance that one of the jets fails the $\pt$ requirement and is not reconstructed.
This kind of events will fall into $t_hH$ final state.
%while the FCNC top resonance is
%still reconstructable that the missing jet is likely from the decay of $W$ boson.
%given the big chance that the jet which is missing is from $W$ decay.
There are three jets coming from top hadronic decay including the $b$-jet and a pair of opposite-sign taus from the Higgs boson decay, where
both $t_h$ and $H$ can be reconstructed.  
%. So a Higgs resonance formed by the taus and a top resonance formed by
%the jets are expected.

The four-momenta of the invisible decay products from the decay of the $\tau$-leptons 
are estimated using a kinematics fit. The fit is done by minimising a $\chi^2$ function based on the Gaussian constraints on the Higgs boson mass and the
measured $\met$ within their expected resolutions. The resolution on the Higgs boson mass is assumed to be $20~\gev$, while the resolution on the measured $\met$ is parameterised as a linear function of 
$\sqrt{\sum E_{\text{T}}}$, where $\sum E_{\text{T}}$ is the scalar sum of the $\pt$ of all physics objects contributing to the $\met$ reconstruction~\cite{Aaboud:2018tkc}.
After the $\chi^2$ minimisation, the Higgs boson $\pt$ as well as the 
$\pt$ of the parent top quarks are determined with better resolution in the signal events. 

For the  $t_{\ell}t(qH)$ and $t_{\ell}H$ events where $W$ boson from $t\to W b$ decay decays leptonically,
%has a large momentum and the flying direction is unknown.
the kinematic fit is no longer feasible due to extra neutrino from $W\rightarrow \ell\nu$ decay. The kinematic variables are calculated using the visible
objects only. After the event reconstruction, a number of variables are used to discriminate the signal from the background using a multivariate analysis described
in Section~\ref{sec:tmva}.

%With the event topology reconstructed, a number of variables are defined for signal and background separation used in the multivariate analysis discussed below.

%In the  $t_{\ell}t(qH)$ and $t_{\ell}H$ signal, the additional $q$ jet in the decay mode can still be found, but since the $W$ boson decays leptonically, there is a neutrino with large momentum and the flying direction is unknown. The kinematics fit is no longer feasible in the $t_{\ell}\hadhad$ channel. The variables calculated from the visible objects are directly used in the multivariate analysis. However, the kinematics fit is still performed for the $t_{\ell}\had$ channels where the lepton and $\had$ are treated as di-tau candidate.
