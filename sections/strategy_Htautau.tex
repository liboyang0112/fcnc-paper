%-------------------------------------------------------------------------------
\section{Analysis strategy}
\label{sec:strategy_Htautau}
%-------------------------------------------------------------------------------

The analysis strategy adopted in this FCNC $\Htautau$ search is similar to the one used in Refs.~\cite{fcnc36,Chen:2015nta} but extended to more search channels.
%\subsection{Event categorisation and kinematic reconstruction}
%\label{sec:htautau_reco_cat}
The $tt(qH)$ and $tH$ signal being probed is characterised by the presence of $\tau$-leptons from the decay of 
the Higgs boson, where the remaining top quark decays into $Wb$. There is an additional $q$-jet from the FCNC $t\to qH$ decay in the top pair production. 
%at least  four (three) jets in the decay (production) mode, only one of which originates from a $b$-quark.
If the $W$ boson or one of the $\tau$-leptons decays leptonically, an isolated electron or muon, together with significant $\met$, is also expected.
In a significant fraction of the events, the lowest-$\pt$ jet from the hadronic $W$ boson decay fails the minimum $\pt$ requirement of $25~\gev$,
resulting in only three reconstructed jets where the production mode is dominant. 
%some signal events mixed with production mode having only three reconstructed jets.
In order to optimise the sensitivity of the search, the selected events are categorised into seven signal regions (SRs) based on the  numbers of light leptons,
$\had$ candidates, and light-flavour jets:
%$t_{\ell}\hadhad$, $t_{\ell}\had$(1j and 2j), $t_{\ell}\had$-2j, $t_h\lephad$-2j, $t_h\lephad$-3j, $t_h\hadhad$-2j, and $t_h\hadhad$-3j, as shown in Table~\ref{tab:srcr}. 
$t_{\ell}\hadhad$, $t_{\ell}\had$ (1j and 2j), $t_h\lephad$ (2j and 3j), $t_h\hadhad$ (2j and 3j), as shown in Table~\ref{tab:srcr}. 

%   \begin{table}
%   \centering
%   \caption{Overview of the signal regions and the control regions used for fake tau scale factor derivation in leptonic channels.}
%   \label{tab:srcr}
%   \begin{tabular}[h]{c|c|c|c|c|c|c}
%   \hline \hline
%   \multicolumn{2}{c|}{Regions} & $b$-jet & light flavor jets        & lepton & hadronic taus & charge\\ \hline
%   \multirow{7}{*}{SR}&$t_{\ell}\thadhad$     & 1     & any                                & 1      & 2             & $\thadhad$ OS\\ \cline{2-7}
%   &$t_{\ell}\tauhad$-1j  & 1     & 1                                   & 1      & 1                     & $t_{\ell}\tauhad$ SS\\ \cline{2-7}
%   &$t_{\ell}\tauhad$-2j  & 1     & 2                                        & 1      & 1                     & $t_{\ell}\tauhad$ SS\\ \cline{2-7}
%   &$t_h\tlhad$-2j   & 1     & 2                           & 1      & 1             & $\tlhad$ OS\\ \cline{2-7}
%   &$t_h\tlhad$-3j   & 1     & $\ge3$                      & 1      & 1             & $\tlhad$ OS\\ \cline{2-7}
%   &$t_h\thadhad$-2j & 1     & 2                            & 0      & 2             & $\thadhad$ OS\\ \cline{2-7}
%   &$t_h\thadhad$-3j & 1     & $\ge3$                       & 0      & 2             & $\thadhad$ OS\\ \hline
%   \multirow{6}{*}{CRtt}&$t_{\ell}t_{\ell}1b\tauhad$ & 1     & any                           & 2      & 1                     & $t_{\ell}t_{\ell}$ OS\\ \cline{2-7}
%   &$t_{\ell}t_{\ell}2b\tauhad$      & 2     & any                           & 2      & 1                     & $t_{\ell}t_{\ell}$ OS\\ \cline{2-7}
%   &$t_{\ell}t_h2b\tauhad$-2jSS & 2     & 2                             & 1      & 1             & $t_{\ell}\tauhad$ SS\\ \cline{2-7}
%   &$t_{\ell}t_h2b\tauhad$-2jOS & 2     & 2                             & 1      & 1             & $t_{\ell}\tauhad$ OS\\ \cline{2-7}
%   &$t_{\ell}t_h2b\tauhad$-3jSS & 2     & $\ge3$                        & 1      & 1             & $t_{\ell}\tauhad$ SS\\ \cline{2-7}
%   &$t_{\ell}t_h2b\tauhad$-3jOS & 2     & $\ge3$                & 1      & 1             & $t_{\ell}\tauhad$ OS\\ \hline
%   \end{tabular}
%   \end{table}


\begin{table}
%\centering
\caption{Overview of the signal regions (SR), validation region (VR), and $t\bar{t}$ control regions (CRtt) used for the fake-$\tau$-lepton scale factor derivation in the leptonic channels. Leptons are required to have either same-sign (SS) or opposite-sign (OS) charges in each region.}
\label{tab:srcr}
\begin{center}
\begin{tabular}[h]{c|c|c|c|c|c|c}
\hline \hline
\multicolumn{2}{c|}{Regions} & $b$-jets & Light-flavour jets        & Leptons & Hadronic $\tau$ decays & Charge\\ \hline
\multirow{7}{*}{SR}&$t_{\ell}\thadhad$     & 1     & $\ge0$~~~~                                & 1      & 2             & $\thadhad$ OS\\ \cline{2-7}
&$t_{\ell}\tauhad$-1j  & 1     & 1                                   & 1      & 1                     & $t_{\ell}\tauhad$ SS\\ \cline{2-7}
&$t_{\ell}\tauhad$-2j  & 1     & 2                                        & 1      & 1                     & $t_{\ell}\tauhad$ SS\\ \cline{2-7}
&$t_h\tlhad$-2j   & 1     & 2                           & 1      & 1             & $\tlhad$ OS\\ \cline{2-7}
&$t_h\tlhad$-3j   & 1     & $\ge3$~~~~                      & 1      & 1             & $\tlhad$ OS\\ \cline{2-7}
&$t_h\thadhad$-2j & 1     & 2                            & 0      & 2             & $\thadhad$ OS\\ \cline{2-7}
&$t_h\thadhad$-3j & 1     & $\ge3$~~~~                       & 0      & 2             & $\thadhad$ OS\\ \hline
\multirow{2}{*}{VR}&$t_{\ell}\thadhad$-SS     & 1     & $\ge0$~~~~                                & 1      & 2             & $\thadhad$ SS\\ \hline
&$t_h\thadhad$-3j SS & 1     & $\ge3$~~~~                       & 0      & 2             & $\thadhad$ SS\\ \hline
\multirow{6}{*}{CRtt}&$t_{\ell}t_{\ell}1b\tauhad$ & 1     & $\ge0$~~~~                            & 2      & 1                     & $t_{\ell}t_{\ell}$ OS\\ \cline{2-7}
&$t_{\ell}t_{\ell}2b\tauhad$      & 2     & $\ge0$~~~~                            & 2      & 1                     & $t_{\ell}t_{\ell}$ OS\\ \cline{2-7}
&$t_{\ell}t_h2b\tauhad$-2jSS & 2     & 2                             & 1      & 1             & $t_{\ell}\tauhad$ SS\\ \cline{2-7}
&$t_{\ell}t_h2b\tauhad$-2jOS & 2     & 2                             & 1      & 1             & $t_{\ell}\tauhad$ OS\\ \cline{2-7}
&$t_{\ell}t_h2b\tauhad$-3jSS & 2     & $\ge3$~~~~                        & 1      & 1             & $t_{\ell}\tauhad$ SS\\ \cline{2-7}
&$t_{\ell}t_h2b\tauhad$-3jOS & 2     & $\ge3$~~~~                & 1      & 1             & $t_{\ell}\tauhad$ OS\\ \hline
\end{tabular}
\end{center}
\end{table}







%This event categorisation is primarily motivated by the different quality of the event kinematic reconstruction, depending on the amount 
%of $\met$ in the event (larger in $\lephad$ events compared with $\hadhad$ events), and whether a jet from the hadronic top-quark decay %is missing or not (events with exactly three jets or at least four jets).
This event categorisation is primarily used to optimise the sensitivity in each signal region that targets either leptonic or hadronic top-quark
decays as well as the Higgs boson decays into either the $\lephad$ or $\hadhad$ final state.   
%The event kinematic reconstruction is used in the $t_h\hadhad$ and $t_h\lephad$ channels to suppress the background by constraining the di-tau mass and the missing transverse energy in the event~\cite{Chen:2015nta}.
The contribution of background to the signal regions of the $t_h\hadhad$ and $t_h\lephad$ channels is reduced by placing kinematic constraints on the di-$\tau$ mass and the $\met$ in the event~\cite{Chen:2015nta}.

%based on the strategy used in Ref.~\cite{Chen:2015nta}, and is summarised below.

For the $t_ht(qH)$ events, the jet from $t\to qH$, referred to as the FCNC jet ($q$-jet), should be a high-$\pt$ jet from the
decay chain $t\to qH\to q\tau\tau$, with $\tau$-leptons reconstructed as $\lephad$ or $\hadhad$.
%There should be at least four jets among which the one with the smaller angular distance to the visible decay of the di-tau,
%is considered as the $q$-jet 
Events should contain four jets, with the one having the smallest angular separation from the visible di-$\tau$ system being labelled the $q$-jet since the FCNC top-quark decay products are likely to be boosted closer together. 
If there are more than two jets besides the $q$-jet and $b$-jet, the jets from the $W$ boson decay are chosen to be those from the combination
with an invariant mass closest to the $W$ boson mass. It is possible that one of the jets fails the $\pt$ requirement and is not reconstructed.
Events of this kind produce the $t_hH$ final state.
%while the FCNC top resonance is
%still reconstructable that the missing jet is likely from the decay of $W$ boson.
%given the big chance that the jet which is missing is from $W$ decay.
When both the $t_h$ and $H$ can be reconstructed, three jets come from the top quark's hadronic decay, including a $b$-jet, and a pair of opposite-sign $\had$ come from the Higgs boson decay.  
%. So a Higgs resonance formed by the taus and a top resonance formed by
%the jets are expected.

The four-momenta of the invisible decay products from the decay of the $\tau$-leptons 
are estimated using a kinematic fit and assuming a collinear approximation for the $\tau$ decay products.
The fit is done by minimising a $\chi^2$ function based on the Gaussian constraints placed on the Higgs boson mass ($m_H=125~\gev$) and the
measured $E_{x,y}^{\text{miss}}$  within their expected resolutions ($\sigma_{E_{x,y}^{\text{miss}}}$), defined as
\begin{eqnarray}
\begin{array}{ll}
\chi^2 =
\left( \frac{m_{\tau\tau,\text{fit}} - m_{H}}{\sigma_{\tau\tau}} \right)^2 +
\left( \frac{E_{x,\text{fit}}^{\text{miss}} - E_{x}^{\text{miss}}}{\sigma_{E_{x}^{\text{miss}}}} \right)^2
  +\left( \frac{E_{y,\text{fit}}^{\text{miss}} - E_{y}^{\text{miss}}}{\sigma_{E_{y}^{\text{miss}}}} \right)^2.
\end{array}
\label{eq:eq1}
\end{eqnarray}

The Higgs boson mass resolution ($\sigma_{\tau\tau}$) is estimated to be $20~\gev$ from a fit of the mass
distribution of the simulated  $tqH$ signal events, while the $\met$ measurement's resolution is parameterised as a linear function of 
$\sqrt{\sum E_{\text{T}}}$, where $\sum E_{\text{T}}$ is the scalar sum of the $E_{\text{T}}$ values of all physics objects contributing to the $\met$ reconstruction~\cite{Aaboud:2018tkc}.
After the $\chi^2$ minimisation, both the Higgs boson $\pt$ and the 
$\pt$ values of the parent top quarks are determined with better resolution in the signal events. 

For the  $t_{\ell}t(qH)$ and $t_{\ell}H$ events where the $W$ boson from $t\to W b$ decay decays leptonically,
%has a large momentum and the flying direction is unknown.
the kinematic fit is no longer feasible due to the neutrino from the $W\rightarrow \ell\nu$ decay. The kinematic variables are calculated using the visible
objects only. After the event reconstruction, a number of these variables are used in a multivariate analysis to discriminate the signal from the background, as described in Section~\ref{sec:tmva}.

%With the event topology reconstructed, a number of variables are defined for signal and background separation used in the multivariate analysis discussed below.

%In the  $t_{\ell}t(qH)$ and $t_{\ell}H$ signal, the additional $q$ jet in the decay mode can still be found, but since the $W$ boson decays leptonically, there is a neutrino with large momentum and the flying direction is unknown. The kinematics fit is no longer feasible in the $t_{\ell}\hadhad$ channel. The variables calculated from the visible objects are directly used in the multivariate analysis. However, the kinematics fit is still performed for the $t_{\ell}\had$ channels where the lepton and $\had$ are treated as di-tau candidate.
