
%-------------------------------------------------------------------------------
\section{Introduction}
\label{sec:intro}
%-------------------------------------------------------------------------------

Following the observation of the Higgs boson by the ATLAS and CMS experiments~\cite{Aad:2012tfa,Chatrchyan:2012ufa} at
the Large Hadron Collider (LHC), a comprehensive programme of measurements of its properties is underway.
%looking for deviations from the Standard Model (SM) predictions.  
An interesting possibility is the presence of flavour-changing neutral-current (FCNC) interactions 
between the Higgs boson, the top quark, and a $u$- or $c$-quark, $tqH$ ($q=u,c$). Since the Higgs boson is lighter than the top quark~\cite{Aad:2015zhl}, 
%with a measured mass $m_H=125.09 \pm 0.24~\gev$~\cite{Aad:2015zhl}, 
such interactions would manifest themselves as FCNC top-quark decays~\cite{Agashe:2013hma}, $t\to H q$.  
In the Standard Model (SM), such decays are suppressed relative to the dominant $t\to Wb$ decay mode, since $tqH$ 
interactions are forbidden at the tree level and suppressed even at higher orders in the perturbative expansion due to the 
Glashow--Iliopoulos--Maiani (GIM) mechanism~\cite{Glashow:1970gm}.  As a result, the SM predictions for the $t \to Hq$ branching 
ratios ($\BR$) are exceedingly small, $\BR(t\to Hu) \sim 10^{-17} $ and $\BR(t\to Hc) \sim 10^{-15}$~\cite{Eilam:1990zc,Mele:1998ag,AguilarSaavedra:2004wm,Zhang:2013xya}, making them undetectable in the foreseeable future.
In contrast, large enhancements of these branching ratios are possible in some scenarios beyond the SM.
Examples include quark-singlet models~\cite{AguilarSaavedra:2002kr}, two-Higgs-doublet models (2HDM) of type I, with explicit flavour conservation,
and of type II, such as the minimal supersymmetric SM (MSSM)~\cite{Bejar:2000ub, Guasch:1999jp,Cao:2007dk,Cao:2014udj}, supersymmetric models
with R-parity violation~\cite{Eilam:2001dh}, composite Higgs models with partial compositeness~\cite{Azatov:2014lha}, 
or warped extra dimensions models with SM fermions in the bulk~\cite{Azatov:2009na}. 
In these scenarios, branching ratios can be as high as $\BR(t\to Hq) \sim 10^{-5}$. 
An even larger branching ratio of  $\BR(t\to Hc) \sim 10^{-3}$ can be reached in 2HDM without explicit flavour conservation (type III),
since a tree-level FCNC coupling is not forbidden by any symmetry~\cite{Cheng:1987rs,Baum:2008qm,Chen:2013qta,Chiang:2015cba,Crivellin:2015hha,Botella:2015hoa, Gori:2017tvg,Chiang:2017fjr}. 
While other FCNC top couplings ($tq\gamma$, $tqZ$, $tqg$) are also enhanced in these scenarios beyond the SM, 
the largest enhancements are typically found for the $tqH$ couplings, and in particular the $tcH$ coupling~\cite{Agashe:2013hma}.

Searches for $t \to Hq$ decays have been performed by the ATLAS and CMS collaborations, taking advantage of the large samples
of top-quark pair ($\ttbar$) events collected in proton-proton ($pp$) collisions at centre-of-mass energies of $\sqrt{s}=7~\tev$ and $8~\tev$~\cite{Aad:2014dya,Aad:2015pja,Khachatryan:2016atv} during Run~1 of the LHC, as well as at $\sqrt{s}=13~\tev$~\cite{Aaboud:2017mfd,Aaboud:2018pob,Sirunyan:2017uae} using early Run~2 data.
In these searches, one of the top quarks is required to decay into $Wb$, while the other top quark decays into $Hq$, yielding $\ttbar \to WbHq$.\footnote{ 
In the following, $WbHq$ is used to denote both $W^+b H\bar{q}$ and its charge conjugate, $HqW^- \bar{b}$. Similarly, 
$WbWb$ is used to denote $W^+b W^- \bar{b}$.}  The Higgs boson is assumed to have a mass of $m_H=125~\gev$ and to decay as predicted by
the SM. The simplifying assumption of SM-like Higgs boson branching ratios is motivated by the fact that measurements of the flavour-diagonal Higgs boson couplings by the ATLAS and CMS collaborations are in agreement with the SM prediction within about 10\%~\cite{Khachatryan:2016vau,Sirunyan:2018koj}. 
Furthermore, typical beyond-the-SM scenarios that predict significant enhancements to $\BR(t\to Hq)$, also predict modifications to the Higgs boson branching ratios at the few percent level or below, well beyond the current experimental precision.
Some of the most sensitive single-channel searches have been performed in the $H\to\gamma\gamma$ decay mode, which
has a small branching ratio of $\BR(H\to \gamma\gamma)\simeq 0.2\%$, but benefits from having a very small background contamination 
and excellent diphoton mass re\-so\-lu\-tion. 
Searches targeting signatures with two same-charge leptons or three leptons (electrons or muons), generically referred to as multileptons,
are able to exploit a branching ratio that is significantly larger for the $H \rightarrow WW^*, \tau\tau$ decay modes than for the $H \rightarrow \gamma\gamma$ decay mode,
%are able to exploit a significantly larger branching ratio for the Higgs boson decay into $H \to WW^*, \tau\tau$ compared to the $H\to\gamma\gamma$ decay mode, 
and are also characterised by relatively small backgrounds.
%However, in general they do not have good mass resolution,\footnote{An exception is the $H\to ZZ^*\to \ell^+\ell^- \ell^{\prime +}\ell^{\prime -}$ 
%($\ell, \ell^\prime = e, \mu$) decay mode, which has a very small branching ratio and thus is not promising for this search.} 
%so any excess would be hard to interpret as originating from $t \to Hq$ decays.
Finally, searches have also been performed exploiting the dominant Higgs boson decay mode, $H\to b\bar{b}$, which has a branching ratio 
of $\BR(H\to b\bar{b})\simeq 58\%$. Compared with Run~1, the Run~2 searches benefit from the increased $\ttbar$ cross section at $\sqrt{s}=13~\tev$, 
as well as the larger integrated luminosity.
Using 36.1 fb$^{-1}$ of data at $\sqrt{s}=13~\tev$, the ATLAS Collaboration has derived upper limits at 95\% confidence level (CL) of  
$\BR(t\to Hc)<0.22\%$ using $H\to \gamma\gamma$ decays~\cite{Aaboud:2017mfd}, and of $\BR(t\to Hc)<0.16\%$ based on
%signatures with two same-charge light leptons (electrons or muons) or three light leptons 
multilepton signatures resulting from 
$H \to WW^*$, $H\to \tau^+\tau^-$ in which both $\tau$-leptons decay leptonically, or $H \to ZZ^*$~\cite{Aaboud:2018pob}.
These upper limits are derived assuming that $\BR(t\to Hu)=0$. Similar upper limits are obtained for $\BR(t\to Hu)$ if $\BR(t\to Hc)=0$. 
The CMS Collaboration has performed a search using  
$H\to b\bar{b}$ decays~\cite{Sirunyan:2017uae} with 35.9 fb$^{-1}$ of data at $\sqrt{s}=13~\tev$, resulting 
in upper limits of $\BR(t\to Hc)<0.47\%$ and $\BR(t\to Hu)<0.47\%$, in each case neglecting the other decay mode.
Compared with previous searches, the search in Ref.~\cite{Sirunyan:2017uae} considers in addition the contribution to the signal from 
$pp \to tH$ production~\cite{Greljo:2014dka}.

%Upper limits on the branching ratios $\BR(t\to Hq)$ ($q=u,c$) can be translated to upper limits on the non-flavour-diagonal Yukawa couplings $\lamHq$ 
%appearing in the following Lagrangian~\cite{Harnik:2012pb}:
%\begin{equation}
%{\cal L}_{\rm FCNC} = -\lambda_{t_L q_R} \bar{t}_L q_R H - \lambda_{q_L t_R} \bar{q}_L t_R H  + h.c.
%\end{equation}
%The branching ratio $\BR(t\to Hq)$ is estimated as the ratio of its partial width~\cite{Zhang:2013xya} to the SM $t \to Wb$ partial width~\cite{Denner:1990ns}, 
%which is assumed to be dominant. Both predicted partial widths include next-to-leading-order (NLO) QCD corrections.
%Using the expression derived in Sect. 1.2 of Ref.~\cite{ATL-COM-PHYS-2016-1664}, the coupling $|\lamHq|$ can be extracted as $| \lamHq | = (1.92 \pm 0.02) \sqrt{\BR(t\to Hq)}$.
%The $\lamHq$ coupling corresponds to the sum in quadrature of the couplings relative to the two possible chirality combinations of the quark fields, 
%$\lamHq \equiv \sqrt{ |\lambda_{t_Lq_R}|^2 +   |\lambda_{q_L t_R}|^2 }$~\cite{Harnik:2012pb}.

%The searches presented in this paper are focused on the dominant fermionic decay modes of the Higgs boson.
The searches presented in this paper are focussed on fermionic decay modes of the Higgs boson. Therefore, they help to complete the 
ATLAS experiment's programme of searches for $t \to Hq$ decays based on $pp$ collision
%programme of searches for $t \to Hq$ decays of the ATLAS experiment based on $pp$ collision 
data at $\sqrt{s}=13~\tev$ recorded in 2015 and 2016. The corresponding integrated luminosity is 36.1 fb$^{-1}$.
Two analyses are performed, searching for $\ttbar \to WbHq$ production (ignoring $pp \to tH$ production) and targeting the 
$H \to b\bar{b}$ and $H \to \tau^+\tau^-$ decay modes, which this paper refers to as ``$\Hbb$ search'' and ``$\Htautau$ search'', respectively.
The $\Hbb$ search selects events with one isolated electron or muon from the $W \to \ell\nu$ decay, and multiple jets, several 
of which are identified with high purity as originating from the hadronisation of $b$-quarks. 
%The $\Htautau$ search selects events with either one or two hadronically decaying $\tau$-lepton candidates, as well as multiple jets. 
The $\Htautau$ search selects events with two $\tau$-lepton candidates, at least one of which decays hadronically, as well as multiple jets.
The latter requirement aims to select events with a hadronically decaying $W$ boson, since this allows an improved reconstruction of the
event kinematics.

%two of which originating from hadronic $W$ decays. 
Both searches employ multivariate techniques to discriminate between the signal and the background on the basis of their different kinematics. 
These two searches are combined with previous ATLAS searches in the diphoton and multilepton final states using the same dataset~\cite{Aaboud:2017mfd,Aaboud:2018pob}, and bounds are set on $\BR(t\to Hc)$ and $\BR(t\to Hu)$, as well as on the corresponding non-flavour-diagonal Yukawa couplings. 
The combination is performed after verifying the overall consistency of the results obtained by the different searches, which exploit very different
experimental signatures and thus are affected by different backgrounds and related systematic uncertainties. 
By combining all searches, the expected sensitivity is improved by about a factor of two relative to the most sensitive individual results. 





