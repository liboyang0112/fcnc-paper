%-------------------------------------------------------------------------------
\section{Introduction}
\label{sec:intro}
%-------------------------------------------------------------------------------

Since the observation of Higgs boson by the ATLAS and CMS experiments~\cite{Aad:2012tfa,Chatrchyan:2012ufa} at
the Large Hadron Collider (LHC), a comprehensive programme of measurements of its properties have been conducted for nearly a decade, by which no significant sign of deviation from Standard Model (SM) was observed until today. However there is still chance finding new physics with searches already attempted but with improved sensitivities. The presence of flavour-changing neutral-current (FCNC) interactions 
between the Higgs boson, the top quark, and a $u$- or $c$-quark, $tqH$ ($q=u,c$) is still an active topic. Since the Higgs boson is lighter than the top quark~\cite{Aad:2015zhl},
%with a measured mass $m_H=125.09 \pm 0.24~\gev$~\cite{Aad:2015zhl}, 
such interactions would manifest themselves as FCNC top-quark decays~\cite{Agashe:2013hma}, $t\to qH$.  
In the Standard Model (SM), the FCNC interaction is forbidden at tree level and suppressed at higher orders due to the Glashow-Iliopoulos-Maiani (GIM) mechanism~\cite{Glashow:1970gm}. The $t\to u/c+H$ branching fraction in the SM is calculated to be exceedingly small around $10^{-15}$~\cite{Eilam:1990zc,Mele:1998ag,AguilarSaavedra:2004wm,Zhang:2013xya}. 
%such decays are suppressed relative to the dominant $t\to Wb$ decay mode, since $tqH$ 
%interactions are forbidden at the tree level and suppressed even at higher orders in the perturbative expansion due to the 
%Glashow--Iliopoulos--Maiani (GIM) mechanism~\cite{Glashow:1970gm}.
%As a result, the SM predictions for the $t \to Hq$ branching 
%ratios ($\BR$) are exceedingly small, $\BR(t\to Hu) \sim 10^{-17} $ and $\BR(t\to Hc) \sim 10^{-15}$~\cite{Eilam:1990zc,Mele:1998ag,AguilarSaavedra:2004wm,Zhang:2013xya}, making them undetectable in the foreseeable future.
In contrast, large enhancements of these branching ratios are possible in some scenarios beyond the SM.
Examples include quark-singlet models~\cite{AguilarSaavedra:2002kr}, two-Higgs-doublet models (2HDM) with or without the flavour violation,
%of type I, with explicit flavour conservation,
%and of type II, such as
the minimal supersymmetric SM (MSSM)~\cite{Bejar:2000ub, Guasch:1999jp,Cao:2007dk,Cao:2014udj},
supersymmetric models with R-parity violation~\cite{Eilam:2001dh}, composite Higgs models with partial compositeness~\cite{Azatov:2014lha}, 
or warped extra dimensions models with SM fermions in the bulk~\cite{Azatov:2009na}. 
In these scenarios, branching ratios can be as high as $\BR(t\to qH) \sim 10^{-5}$. 
An even larger branching ratio of  $\BR(t\to cH) \sim 10^{-3}$ can be reached in 2HDM without explicit flavour conservation (type III),
since a tree-level FCNC coupling is no longer forbidden by any symmetry~\cite{Cheng:1987rs,Baum:2008qm,Chen:2013qta,Chiang:2015cba,Crivellin:2015hha,Botella:2015hoa, Gori:2017tvg,Chiang:2017fjr}. 
%While other FCNC top couplings ($tq\gamma$, $tqZ$, $tqg$) are also enhanced in these scenarios beyond the SM, 
%the largest enhancements are typically found for the $tqH$ couplings, and in particular the $tcH$ coupling~\cite{Agashe:2013hma}.
Therefore, an observation of this decay would be a clear evidence for new physics. The $tqH$ interaction could also potentially open up more
Higgs decay channels, such as $H\rightarrow t^*q\rightarrow Wbq$, but they are likely suppressed due to the $t-H$ mass difference,
which could be interesting for future studies.

On the other hand, if the $tqH$ interaction exists, the single-top, Higgs associated production through this interaction should also be enhanced.
The $tH$ associated production in the SM prediction is expected to be small at LHC~\cite{Greljo:2014dka}.
So the study on this process will also contribute to the FCNC interaction searches.

Searches for $t \to Hq$ decays have been performed by the ATLAS and CMS collaborations, taking advantage of the large samples
of top-quark pair ($\ttbar$) events collected in proton-proton ($pp$) collisions at centre-of-mass energies of $\sqrt{s}=7~\tev$ and $8~\tev$~\cite{Aad:2014dya,Aad:2015pja,Khachatryan:2016atv} during Run~1 of the LHC, as well as at $\sqrt{s}=13~\tev$~\cite{fcnc36} using early Run~2 data.

In these searches, one of the top quarks is required to decay into $Wb$, while the other top quark decays into $Hq$, yielding $\ttbar \to WbHq$.\footnote{In the following, $WbHq$ is used to denote both $W^+b H\bar{q}$ and its charge conjugate, $HqW^- \bar{b}$. Similarly, 
$WbWb$ is used to denote $W^+b W^- \bar{b}$.}  The Higgs boson is assumed to have a mass of $m_H=125~\gev$ and to decay as predicted by
the SM. The assumption of using SM-like Higgs boson branching ratios is motivated by the fact that measurements of the flavour-diagonal Higgs boson couplings by the ATLAS and CMS collaborations are in agreement with the SM prediction within about 10\%~\cite{Khachatryan:2016vau,Sirunyan:2018koj}. 
%Furthermore, typical beyond-the-SM scenarios that predict significant enhancements to $\BR(t\to Hq)$, also predict modifications to the Higgs boson branching ratios at the few percent level or below, well beyond the current experimental precision.
%Some of the most sensitive single-channel searches have been performed in the $H\to\gamma\gamma$ decay mode, which
%has a small branching ratio of $\BR(H\to \gamma\gamma)\simeq 0.2\%$, but benefits from having a very small background contamination 
%and excellent diphoton mass re\-so\-lu\-tion. 
%Searches targeting signatures with two same-charge leptons or three leptons (electrons or muons), generically referred to as multileptons,
%are able to exploit a branching ratio that is significantly larger for the $H \rightarrow WW^*, \tau\tau$ decay modes than for the $H \rightarrow \gamma\gamma$ decay mode,
%are able to exploit a significantly larger branching ratio for the Higgs boson decay into $H \to WW^*, \tau\tau$ compared to the $H\to\gamma\gamma$ decay mode, 
%and are also characterised by relatively small backgrounds.
%However, in general they do not have good mass resolution,\footnote{An exception is the $H\to ZZ^*\to \ell^+\ell^- \ell^{\prime +}\ell^{\prime -}$ 
%($\ell, \ell^\prime = e, \mu$) decay mode, which has a very small branching ratio and thus is not promising for this search.} 
%so any excess would be hard to interpret as originating from $t \to Hq$ decays.
%Finally, searches have also been performed exploiting the dominant Higgs boson decay mode, $H\to b\bar{b}$, which has a branching ratio 
%of $\BR(H\to b\bar{b})\simeq 58\%$.
Compared to Run~1, the Run~2 searches benefit from the increased $\ttbar$ cross section at $\sqrt{s}=13~\tev$, as well as the larger integrated luminosity.
Using 36.1 fb$^{-1}$ of data at $\sqrt{s}=13~\tev$, the ATLAS Collaboration has derived upper limits at 95\% confidence level (CL) of
$\BR(t\to cH)<0.22\%$ using $H\to \gamma\gamma$ decays~\cite{Aaboud:2017mfd}, $\BR(t\to cH)<0.16\%$ based on
%signatures with two same-charge light leptons (electrons or muons) or three light leptons 
multilepton (electron or muon) signatures resulting from 
$H \to  WW^*, ZZ^*$, $H\to \tau^+\tau^-$ in which both $\tau$-leptons decay leptonically~\cite{Aaboud:2018pob},
$\BR(t\to cH)<0.42\%$ using $H\to b\bar{b}$ decay~\cite{fcnc36}, and of $\BR(t\to cH)<0.19\%$ using $H\to tau\bar{tau}$ decay in which at least
one of $\tau$-leptons decays hadronically~\cite{fcnc36}.  
These upper limits are derived assuming that $\BR(t\to uH)=0$. Similar upper limits are obtained for $\BR(t\to uH)$ if $\BR(t\to cH)=0$.
%The searches with  $H\to \tau^+\tau^-$ and $H\to b\bar{b}$ decays are later published~\cite{fcnc36}, unfortunately with no FCNC signal found.
Combining all of the ATLAS searches using 36fb$^{-1}$ Run~2 data, the limits at 95\% CL are set for  $\BR(t\to cH)<0.11\%$ and $\BR(t\to uH)<0.12\%$ respectively assuming $\BR(t\to uH)=0$ and $\BR(t\to cH)=0$~\cite{fcnc36}.

The CMS Collaboration has performed a search using  
$H\to b\bar{b}$ decays~\cite{Sirunyan:2017uae} with 35.9 fb$^{-1}$ of data at $\sqrt{s}=13~\tev$, resulting 
in upper limits of $\BR(t\to Hc)<0.47\%$ and $\BR(t\to Hu)<0.47\%$, in each case neglecting the other decay mode.
Compared with previous searches, the search in Ref.~\cite{Sirunyan:2017uae} considers in addition the contribution to the signal from 
$pp \to tH$ production~\cite{Greljo:2014dka}.

%Upper limits on the branching ratios $\BR(t\to Hq)$ ($q=u,c$) can be translated to upper limits on the non-flavour-diagonal Yukawa couplings $\lamHq$ 
%appearing in the following Lagrangian~\cite{Harnik:2012pb}:
%\begin{equation}
%{\cal L}_{\rm FCNC} = -\lambda_{t_L q_R} \bar{t}_L q_R H - \lambda_{q_L t_R} \bar{q}_L t_R H  + h.c.
%\end{equation}
%The branching ratio $\BR(t\to Hq)$ is estimated as the ratio of its partial width~\cite{Zhang:2013xya} to the SM $t \to Wb$ partial width~\cite{Denner:1990ns}, 
%which is assumed to be dominant. Both predicted partial widths include next-to-leading-order (NLO) QCD corrections.
%Using the expression derived in Sect. 1.2 of Ref.~\cite{ATL-COM-PHYS-2016-1664}, the coupling $|\lamHq|$ can be extracted as $| \lamHq | = (1.92 \pm 0.02) \sqrt{\BR(t\to Hq)}$.
%The $\lamHq$ coupling corresponds to the sum in quadrature of the couplings relative to the two possible chirality combinations of the quark fields, 
%$\lamHq \equiv \sqrt{ |\lambda_{t_Lq_R}|^2 +   |\lambda_{q_L t_R}|^2 }$~\cite{Harnik:2012pb}.

%The searches presented in this paper are focused on the dominant fermionic decay modes of the Higgs boson.
This paper is focused on the tau decay mode of the Higgs boson with the complete dataset collected during the Run~2 from 2016 to 2018. The corresponding integrated luminosity is 139.0 fb$^{-1}$. Both $\ttbar \to WbHq$ decay and $pp \to tH$ production are targeted and the signals are divided into several final states depending on the production modes and
the  $W$ boson decay either hadronically or leptonically from the $t\rightarrow Wb$ decay referred as $t_h$ and $t_l$. The contribution of $W\rightarrow\tau\nu$ is included either in
$t_l$ when the $\tau$ decays into a light lepton(electron or muon) as $\tau_{lep}$ or in $t_h$ when the $\tau$ decays hadronically as $\thad$.
The $H\rightarrow \tau\tau$ decay is detected to be in
either $\tlhad$ or $\thadhad$ final states when the top decays hadronically ($t_h$) and only in $\thadhad$ when the top decay leptonically ($t_l$). 
Events with two hadronically decaying $\tau$-lepton candidate (\tauhad) without electron or muon are selected as the hadronic channels. Events with at least one \tauhad with an additional electron or muon are selected as the leptonic channels. More signal regions are exploited here compared to the previous FCNC $\Htautau$ search conducted using partial Run~2 data as well as
various $t\bar{t}$ control regions (CRtt) used for calibration of tau fakes in Monte Carlo and data-driven fake estimations from QCD multi-jet background.

Multivariate technique is used to discriminate between the signal and the background on the basis of their different kinematics. 

%The expected sensitivity is about a factor of two relative to the most sensitive results~\cite{fcnc36} up until now. 
