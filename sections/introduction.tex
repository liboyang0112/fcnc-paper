%-------------------------------------------------------------------------------
\section{Introduction}
\label{sec:intro}
%-------------------------------------------------------------------------------
Since the discovery of the Higgs boson in 2012 at the Large Hadron Collider (LHC) by the ATLAS~\cite{Aad:2012tfa} and 
CMS~\cite{Chatrchyan:2012ufa} collaborations, 
%Since the observation of Higgs boson by the ATLAS and CMS experiments~\cite{Aad:2012tfa,Chatrchyan:2012ufa} at
a comprehensive programme of measurements % of its properties
has been conducted to explore this particle. Measurements so far have proved to be consistent with the Standard Model (SM) predictions. 
The programme is ongoing and precision measurements as well as searches for rare new-physics processes beyond the Standard Model (BSM)
are underway. One such possibility is flavour-changing neutral current (FCNC) interactions between the Higgs boson, 
the top quark, and an up-type quark, $tqH$ ($q=u, c$), which have been searched for by the ATLAS and CMS collaborations.  
%with increased precision and proved to be consistent with the Standard Model (SM) predictions. 
%for nearly a decade, by which no significant sign of deviation from Standard Model (SM) was observed until today.
%However the LHC still provides an excellent opportunity to search for rare new physics processes beyond the Standard Model (BSM).  
%there are still chances to find new physics by improving the sensitivity of previous searches.
%The flavour-changing neutral-current (FCNC) interactions 
%between the Higgs boson, the top quark, and a $u$- or $c$-quark, $tqH$ ($q=u,c$) has been actively searched for by the ATLAS and CMS collaborations. %is still an active topic.
Since the Higgs boson is lighter than the top quark~\cite{Aad:2015zhl},
%with a measured mass $m_H=125.09 \pm 0.24~\gev$~\cite{Aad:2015zhl}, 
such interactions could manifest themselves as FCNC top-quark decays~($t\to qH$)~\cite{Agashe:2013hma}.
%Furthermore, if the $tqH$ interaction exists, the associated single-top and  Higgs  production process through this interaction would enhance 
%becomes non-zero, enhancing 
%the total
%production cross section of $pp\rightarrow tH$.

In the Standard Model (SM), the FCNC interaction is forbidden at tree level and suppressed at higher orders through the Glashow--Iliopoulos--Maiani (GIM) mechanism~\cite{Glashow:1970gm}. An observation of an enhanced rate of this decay would be clear evidence of new physics.
Furthermore, if the $tqH$ interaction exists, the associated single-top and  Higgs  production process through this interaction would enhance
%becomes non-zero, enhancing
the total production cross section of $pp\rightarrow tH$.
The $t\to qH$ branching fraction in the SM is calculated to be exceedingly small, $\BR(t\to qH)\approx10^{-15}$~\cite{Eilam:1990zc,Mele:1998ag,AguilarSaavedra:2004wm,Zhang:2013xya}. 
%such decays are suppressed relative to the dominant $t\to Wb$ decay mode, since $tqH$ 
%interactions are forbidden at the tree level and suppressed even at higher orders in the perturbative expansion due to the 
%Glashow--Iliopoulos--Maiani (GIM) mechanism~\cite{Glashow:1970gm}.
%As a result, the SM predictions for the $t \to Hq$ branching 
%ratios ($\BR$) are exceedingly small, $\BR(t\to Hu) \sim 10^{-17} $ and $\BR(t\to Hc) \sim 10^{-15}$~\cite{Eilam:1990zc,Mele:1998ag,AguilarSaavedra:2004wm,Zhang:2013xya}, making them undetectable in the foreseeable future.
However, these branching ratios can be large enough to be observed at LHC 
%as high as $\BR(t\to qH) \sim 10^{-5}$
when processes beyond the SM are included. Examples of these processes include: quark-singlet models~\cite{AguilarSaavedra:2002kr}, two-Higgs-doublet models (2HDMs)~\cite{ Branco:2hdm2012} with or without flavour violation,
 %of type I, with explicit flavour conservation,
 %and of type II, such as
the minimal supersymmetric SM (MSSM)~\cite{Bejar:2000ub, Guasch:1999jp,Cao:2007dk,Cao:2014udj},
supersymmetric models with R-parity violation~\cite{Eilam:2001dh}, composite Higgs models with partial  compositeness~\cite{Azatov:2014lha}, 
and warped extra dimensions models with SM fermions in the bulk~\cite{Azatov:2009na}. 
% In contrast, large enhancements of these branching ratios are possible in some scenarios beyond the SM.
% Examples include quark-singlet models~\cite{AguilarSaavedra:2002kr}, two-Higgs-doublet models (2HDM)~\cite{ %Branco:2hdm2012} with or without flavour violation,
% %of type I, with explicit flavour conservation,
% %and of type II, such as
% the minimal supersymmetric SM (MSSM)~\cite{Bejar:2000ub, Guasch:1999jp,Cao:2007dk,Cao:2014udj},
% supersymmetric models with R-parity violation~\cite{Eilam:2001dh}, composite Higgs models with partial  %compositeness~\cite{Azatov:2014lha}, 
% or warped extra dimensions models with SM fermions in the bulk~\cite{Azatov:2009na}. 
% In these scenarios, branching ratios can be as high as $\BR(t\to qH) \sim 10^{-5}$. 
An even larger branching ratio of  $\BR(t\to cH) \sim 10^{-3}$ can be reached in 2HDMs without explicit flavour conservation, since in these models the tree-level FCNC coupling is no longer forbidden by any symmetry~\cite{Cheng:1987rs,Baum:2008qm,Chen:2013qta,Chiang:2015cba,Crivellin:2015hha,Botella:2015hoa, Gori:2017tvg,Chiang:2017fjr}. 
%While other FCNC top couplings ($tq\gamma$, $tqZ$, $tqg$) are also enhanced in these scenarios beyond the SM, 
%the largest enhancements are typically found for the $tqH$ couplings, and in particular the $tcH$ coupling~\cite{Agashe:2013hma}.
%Therefore, an observation of an enhanced rate of this decay would be a clear evidence for new physics.
%The $tqH$ interaction could also potentially open up more
%Higgs decay channels, such as $H\rightarrow t^*q\rightarrow Wbq$, but they are likely suppressed due to the $t-H$ mass difference,
%which could be interesting for future studies.
%Furthermore, if the $tqH$ interaction exists, the associated single-top and  Higgs  production process through %this interaction would enhance 
%%becomes non-zero, enhancing 
%the total
%production cross section of $pp\rightarrow tH$.
The study of $pp\rightarrow tH$ processes will also contribute to the FCNC interaction searches~\cite{Greljo:2014dka}.
In the SM, associated production of $tH$ in $pp$ collisions is expected to have a cross section of $\sigma_{tH}=92^{+7}_{-12}$ fb at a centre-of-mass energy of $\sqrt{s}= 13~\tev$~\cite{deFlorian:2016spz}.

Searches for $t \to qH$ decays have been performed by the ATLAS and CMS collaborations, taking advantage of the large samples
of top-quark pair ($\ttbar$) events collected in proton--proton ($pp$) collisions at centre-of-mass energies of $\sqrt{s}=7~\tev$ and $8~\tev$~\cite{Aad:2014dya,Aad:2015pja,Khachatryan:2016atv} during Run~1 of the LHC, as well as at $\sqrt{s}=13~\tev$~\cite{fcnc36} using early Run~2 data.
In these searches, one of the top quarks is required to decay into $Wb$, while the other top quark decays into $qH$ with a small branching ratio 
 $\BR(t\to qH)$, a process denoted by $\ttbar \to WbHq$.\footnote{In the following, $WbHq$ is used to denote both $W^+b H\bar{q}$ and its charge conjugate, $HqW^- \bar{b}$. Similarly, 
$WbWb$ is used to denote $W^+b W^- \bar{b}$.}  The Higgs boson is assumed to have a mass of $m_H=125~\gev$ and to decay as predicted by
the SM.
%The assumption of using SM-like Higgs boson branching ratios is motivated by the fact that measurements of the flavour-diagonal Higgs boson couplings by the ATLAS and CMS collaborations are in agreement with the SM prediction within about 10\%~\cite{Khachatryan:2016vau,Sirunyan:2018koj}. 
%Furthermore, typical beyond-the-SM scenarios that predict significant enhancements to $\BR(t\to Hq)$, also predict modifications to the Higgs boson branching ratios at the few percent level or below, well beyond the current experimental precision.
%Some of the most sensitive single-channel searches have been performed in the $H\to\gamma\gamma$ decay mode, which
%has a small branching ratio of $\BR(H\to \gamma\gamma)\simeq 0.2\%$, but benefits from having a very small background contamination 
%and excellent diphoton mass re\-so\-lu\-tion. 
%Searches targeting signatures with two same-charge leptons or three leptons (electrons or muons), generically referred to as multileptons,
%are able to exploit a branching ratio that is significantly larger for the $H \rightarrow WW^*, \tau\tau$ decay modes than for the $H \rightarrow \gamma\gamma$ decay mode,
%are able to exploit a significantly larger branching ratio for the Higgs boson decay into $H \to WW^*, \tau\tau$ compared to the $H\to\gamma\gamma$ decay mode, 
%and are also characterised by relatively small backgrounds.
%However, in general they do not have good mass resolution,\footnote{An exception is the $H\to ZZ^*\to \ell^+\ell^- \ell^{\prime +}\ell^{\prime -}$ 
%($\ell, \ell^\prime = e, \mu$) decay mode, which has a very small branching ratio and thus is not promising for this search.} 
%so any excess would be hard to interpret as originating from $t \to Hq$ decays.
%Finally, searches have also been performed exploiting the dominant Higgs boson decay mode, $H\to b\bar{b}$, which has a branching ratio 
%of $\BR(H\to b\bar{b})\simeq 58\%$.
Compared to Run~1, the Run~2 searches, summarised in Table~\ref{tab:limits_summary_ref}, benefit from the increased $\ttbar$ cross section at $\sqrt{s}=13~\tev$, as well as the larger integrated luminosity.
Using 36.1~fb$^{-1}$ of data at $\sqrt{s}=13~\tev$, the ATLAS Collaboration has derived upper limits at 95\% confidence level (CL) on the 
$t\to cH$ branching ratio: $\BR(t\to cH)<0.22\%$ using $H\to \gamma\gamma$ decays~\cite{Aaboud:2017mfd} and $\BR(t\to cH)<0.16\%$ based on
%signatures with two same-charge light leptons (electrons or muons) or three light leptons 
multi-lepton (electron or muon) signatures resulting from 
$H \to  WW^*, ZZ^*$, $\tau^+\tau^-$ in which both $\tau$-leptons decay leptonically~\cite{Aaboud:2018pob}.
ATLAS also set upper limits of $\BR(t\to cH)<0.42\%$ using $H\to b\bar{b}$ decay~\cite{fcnc36} and $\BR(t\to cH)<0.19\%$ using $H\to \tau^+\tau^-$ decays in which at least
one of the $\tau$-leptons decays hadronically~\cite{fcnc36}.  
These upper limits are derived assuming that the branching ratio $\BR(t\to uH)=0$. Similar upper limits are obtained for $\BR(t\to uH)$ when assuming $\BR(t\to cH)=0$.
%The searches with  $H\to \tau^+\tau^-$ and $H\to b\bar{b}$ decays are later published~\cite{fcnc36}, unfortunately with no FCNC signal found.
Combining all of the ATLAS searches using 36.1~fb$^{-1}$ of Run~2 data, upper limits at 95\% CL on the branching fractions are 
set at $\BR(t\to cH)<0.11\%$ assuming $\BR(t\to uH)=0$, and at $\BR(t\to uH)<0.12\%$ assuming $\BR(t\to cH)=0$~\cite{fcnc36}.

The CMS Collaboration performed a similar search using  
$H\to b\bar{b}$ decays~\cite{Sirunyan:2017uae} with 35.9 fb$^{-1}$ of data at $\sqrt{s}=13~\tev$, resulting 
in upper limits of $\BR(t\to cH)<0.47\%$ and $\BR(t\to uH)<0.47\%$, in each case neglecting the other decay mode.
%Compared with previous searches, 
The search in Ref.~\cite{Sirunyan:2017uae} also considers the contribution to the signal from 
$pp \to tH$ production~\cite{Greljo:2014dka}. 
CMS subsequently updated their search for $tqH$ in the $H\to b\bar{b}$ channel using
the full Run~2 dataset, obtaining observed (expected) upper limits of  $\BR(t\to cH) < 9.4\times10^{-4}\, (8.6\times10^{-4})$
and $\BR(t\to uH) < 7.9\times10^{-4}\, (1.1\times10^{-3})$~\cite{CMS:2021gfa}.
%CMS also recently reported a search for $tqH$ in the $H\rightarrow \gamma\gamma$ mode using the full Run-2 data and set observed(expected) upper limits on $\mathcal{B}(t\to Hu)$ of $1.9\times10^{-4}(3.1\times10^{-4})$ and $\mathcal{B}(t\to Hc)$ of $7.3\times10^{-4}(5.1\times10^{-4})$~\cite{CMS-PAS-TOP-20-007}.


\begin{table}[t!]
\caption{\small{Summary of 95\% CL upper limits on $\BR(t \to cH)$ and $\BR(t \to uH)$ obtained by the ATLAS and CMS collaborations with Run~2 data. Each limit is obtained assuming the other branching ratio is zero.}}
\begin{center}
\small 
\begin{tabular}{ccccc}
\toprule\toprule
& &\multirow{2}{*}{$\mathcal{L}$ [fb$^{-1}$]} & \multicolumn{2}{c}{95\% CL observed upper limits}  \\
& & 										    & \multicolumn{1}{c}{on $\BR(t \to cH)$}            & \multicolumn{1}{c}{on $\BR(t \to uH)$} \\
\midrule
\multirow{5}{*}{ATLAS}
& $H \to b\bar{b}$~\cite{fcnc36}                                          & 36.1         & $4.2 \times 10^{-3}$ & $5.2 \times 10^{-3}$ \\
& $H \to \gamma\gamma$~\cite{Aaboud:2017mfd}                              & 36.1         & $2.2 \times 10^{-3}$  & $2.4 \times 10^{-3}$  \\
& $H \to \tau\tau$ ($\lephad$, $\hadhad$)~\cite{fcnc36}                   & 36.1         & $1.9 \times 10^{-3}$  & $1.7 \times 10^{-3}$  \\ 
& $H \to WW^*, \tau\tau, ZZ^*$ ($2\ell$SS, $3\ell$)~\cite{Aaboud:2018pob} & 36.1         & $1.6 \times 10^{-3}$  & $1.9 \times 10^{-3}$\\ 
& Combination~\cite{fcnc36}                                               & 36.1         & $1.1 \times 10^{-3}$  & $1.2 \times 10^{-3}$  \\\midrule
\multirow{2}{*}{CMS} 
& $H \to b\bar{b}$~\cite{Sirunyan:2017uae}                                & 35.9         & $4.7 \times 10^{-3}$  & $4.7 \times 10^{-3}$  \\
& $H \to b\bar{b}$~\cite{CMS:2021gfa}                                     & 137~~~~~          & $9.4 \times 10^{-4}$  & $7.9 \times 10^{-4}$  \\
%& $H \to \gamma\gamma$~\cite{CMS-PAS-TOP-20-007}                          & 137          & $7.3 \times 10^{-4}$  & $1.9 \times 10^{-4}$  \\
% 
\bottomrule\bottomrule
\end{tabular}
\label{tab:limits_summary_ref}
\end{center}
\end{table}




% \begin{center}
% \begin{tabular}{lccc}
% \hline \hline
% Analysis                           & Dataset & $\mathcal{L}$ [fb$^{-1}$] & Ref. \\
% \hline
% \Hyy\ (including \ttH, \Hyy)       & \multirow{4}{*}{2015--2017} & $79.8$ & (\cite{HIGG-2016-21}), \cite{HIGG-2018-13} \\
% \hfourl\ (including \ttH, \hfourl) &                             & $79.8$ & (\cite{HIGG-2016-22}), \cite{HIGG-2018-13} \\
% \VH, \hbb                          &                             & $79.8$ & \cite{HIGG-2018-04,HIGG-2018-50} \\
% \hmm                               &                             & $79.8$ & (\cite{HIGG-2016-10}) \\
% \hline
% \hwwenmun                          & \multirow{6}{*}{2015--2016} & $36.1$ & \cite{HIGG-2016-07} \\
% \htt                               &                             & $36.1$ & \cite{HIGG-2017-07} \\
% \VBF, \hbb                         &                             & $24.5$ -- $30.6$ & \cite{HIGG-2016-30} \\
% \ttH, \hbb\ and \ttH\ multilepton  &                             & $36.1$ & \cite{HIGG-2017-03,HIGG-2017-02,HIGG-2018-13}\\
% \Hinv                              &                             & $36.1$ & \cite{EXOT-2016-37,HIGG-2016-28,EXOT-2016-23,HIGG-2018-54}\\
% Off-shell \Hllll\ and \Hllvv\      &                             & $36.1$ & \cite{HIGG-2017-06} \\
% \hline \hline
% \end{tabular}
% \end{center}
% \label{tab:lumi}
% \end{table}
 



%Upper limits on the branching ratios $\BR(t\to Hq)$ ($q=u,c$) can be translated to upper limits on the non-flavour-diagonal Yukawa couplings $\lamHq$ 
%appearing in the following Lagrangian~\cite{Harnik:2012pb}:
%\begin{equation}
%{\cal L}_{\rm FCNC} = -\lambda_{t_L q_R} \bar{t}_L q_R H - \lambda_{q_L t_R} \bar{q}_L t_R H  + h.c.
%\end{equation}
%The branching ratio $\BR(t\to Hq)$ is estimated as the ratio of its partial width~\cite{Zhang:2013xya} to the SM $t \to Wb$ partial width~\cite{Denner:1990ns}, 
%which is assumed to be dominant. Both predicted partial widths include next-to-leading-order (NLO) QCD corrections.
%Using the expression derived in Sect. 1.2 of Ref.~\cite{ATL-COM-PHYS-2016-1664}, the coupling $|\lamHq|$ can be extracted as $| \lamHq | = (1.92 \pm 0.02) \sqrt{\BR(t\to Hq)}$.
%The $\lamHq$ coupling corresponds to the sum in quadrature of the couplings relative to the two possible chirality combinations of the quark fields, 
%$\lamHq \equiv \sqrt{ |\lambda_{t_Lq_R}|^2 +   |\lambda_{q_L t_R}|^2 }$~\cite{Harnik:2012pb}.

%The searches presented in this paper are focused on the dominant fermionic decay modes of the Higgs boson.

%This paper is focused on the tau decay mode of the Higgs boson with the complete dataset collected during the Run~2 from 2016 to 2018.
The analysis reported here targets Higgs boson decays into 
%This paper is focused on the decay mode of the Higgs boson into
$\tau$-leptons in the complete Run~2 dataset collected by ATLAS in 2015--2018. The corresponding integrated luminosity is 139 fb$^{-1}$. Both $\ttbar \to WbHq$ decays and $pp \to tH$ production are sought. The dataset is divided into several final states depending on the production mode and
the  $W$ boson and $\tau$-lepton decays. Top quark pair production events where the $W$ boson decays hadronically (leptonically) are denoted by $t_h$ ($t_{\ell}$).
The decays $\tau \to \ell \nu_{\ell} \nu_{\tau}$ are denoted by $\lep$, while the decays $\tau \to \text{hadrons} + \nu_{\tau}$ are denoted by $\thad$.
The contribution of 
$W\rightarrow\tau\nu$ is included as $\lep$ in $t_{\ell}$ when the $\tau$-lepton decays into a light lepton (electron or muon) or as $\thad$ 
in $t_h$ when the $\tau$-lepton decays hadronically.
%either hadronically or leptonically from the $t\rightarrow Wb$ decay, which are referred as $t_h$ and $t_l$. The contribution of $W\rightarrow\tau\nu$ is included either in
%$t_l$ when the $\tau$ decays into a light lepton (electron or muon) as $\tau_{\text{lep}}$ or in $t_h$ when the $\tau$ decays hadronically as $\thad$.
The $H\rightarrow \tau\tau$ decay is detected in
either the $\tlhad$ or $\thadhad$ final states when the top quark decays hadronically ($t_h$), but only the $\thadhad$ final state is considered
when the top quark decays leptonically ($t_{\ell}$), in order to avoid overlaps with
other ATLAS searches~\cite{Aaboud:2018pob}. 
In this search, events with two hadronically decaying $\tau$-leptons and no electron or muon define the hadronic channel. Events with at least one \tauhad and an additional electron or muon (corresponding to $t_{\ell}$ or $\lep$ events) are assigned to leptonic channels. More signal regions (with the top quark decaying leptonically) are exploited here than in the previous FCNC $\Htautau$ search, which was conducted using the partial Run~2 dataset~\cite{fcnc36}.
In addition, an improved treatment of misidentified $\tau$-leptons (`fakes') in simulation and in data-driven
estimations of fakes from multi-jet background is implemented.
% and more
%$$t\bar{t}$ control regions (CRtt) are used for calibration of misidentified (fake) taus in Monte Carlo and data-driven fake estimations from QCD multi-jet background.
Finally, a multivariate technique based on boosted decision trees is used to discriminate between the signal and the background on the basis of their different kinematical distributions. 

%The expected sensitivity is about a factor of two relative to the most sensitive results~\cite{fcnc36} up until now. 
