%-------------------------------------------------------------------------------
\section{Background estimation}
\label{sec:background_model}
%-------------------------------------------------------------------------------

Most background processes are modelled using MC simulation.
After the event preselection, the main background is $\ttbar$ production, often in association with jets, denoted by $\ttbar$+jets in the following.
Small contributions arise from single-top-quark, $W/Z$+jets, multi-jet and diboson ($WW,WZ,ZZ$) production, as well as from the associated 
production of a vector boson $V$ ($V=W,Z$) or a Higgs boson and a $\ttbar$ pair ($\ttbar V$ and $\ttbar H$). All backgrounds 
with prompt leptons, i.e.\ those originating from the decay of a $W$ boson, a $Z$ boson, or a $\tau$-lepton,
are estimated using samples of simulated events and initially normalised to their theoretical cross sections.
In the simulation, the top-quark and SM Higgs boson masses are set to $172.5~\gev$ and $125~\gev$, respectively,
and the Higgs boson is forced to decay into $H\to \tau\tau$ with branching ratio calculated using \textsc{Hdecay}~\cite{Djouadi:1997yw}.  
Backgrounds with non-prompt light leptons (electron or muon), with photons or jets misidentified as electrons, or with jets misidentified as tau candidates, 
generically referred to as \texttt{fake} leptons, are estimated using data-driven methods. 
The background prediction is further improved during the statistical analysis by performing a likelihood 
fit to data using several signal-depleted control regions as shown in Table~\ref{tab:srcr}.
The events with one light lepton, two same-sign $\had$, and one $b$-jet, are also selected, enter in a validation region (VR) for the background estimation in the leptonic channels. 
%-------------------------------------------------------------------------------
%\subsection{Backgrounds with fake leptons}
%\label{sec:fakeleptons}
%-------------------------------------------------------------------------------
\subsection{Backgrounds with fake tau leptons}
\label{sec:faketaus}
The background with one or more fake tau candidates mainly arises from $\ttbar$ or
multi-jet production, depending on the search channels.
Studies based on simulation show that, for all the above processes, fake tau candidates primarily result from the
misidentification of light jets and $b$-quark jets.
It is also found that the fake rate decreases for all jet flavours as the tau candidate $\pt$ increases.

In the leptonic channels, the events with prompt electron or muon and fake taus are modelled by calibrating MC with scale factors (SF)
derived from the dedicated $t\bar t$
control regions (CRtt) using dileptonic decays of $t\bar t$ events and semileptonic decays of $t\bar t$ events with two
$b$-jets, summarized in Table~\ref{tab:srcr}. 	
%the SM $t\bar t$ decay of dilepton events and semileptonically double-btagged lepton-jet events, summarized in Table~\ref{tab:srcr},
%aimed at fake taus from different origins.
There are four kinds of fake taus that need to be calibrated: Type-1 fake taus from hadronic $W$ boson decay ($\tau_{W}$) %$W$-jets ($\tau_{W}$)
with opposite-sign (OS) of charge to the light lepton;
Type-2 $\tau_{W}$'s with same-sign (SS) charge to the light lepton; Type-3 fake taus originating from $b$-hadron decays; Type-4 fake taus from light-flavour hadron decays.
The control regions are defined similar to the signal regions but with an additional $b$-jet or light lepton.
%$t_{\ell}t_{\ell}1b\thad$, $t_{\ell}t_{\ell}2b\thad$, $t_{\ell}t_h2b\thad$-2jSS, $t_{\ell}t_h2b\thad$-2jOS, $t_{\ell}t_h2b\thad$-3jSS, and $t_{\ell}t_h2b\thad$-3jOS.
The dilepton regions ($t_{\ell}t_{\ell}1b\thad$ and $t_{\ell}t_{\ell}2b\thad$) are used to calibrate Type-3 and Type-4 fake taus. The semileptonic
regions ($t_{\ell}t_h2b\thad$-2jOS and $t_{\ell}t_h2b\thad$-3jOS) where $\thad$ and light lepton have opposite charge are used to calibrate Type-1 fake taus.
Similarly for Type-2, the semileptonic regions ($t_{\ell}t_h2b\thad$-2jSS and $t_{\ell}t_h2b\thad$-3jSS) where $\thad$ and light lepton have same charge are used.
A simultaneous fit to data is made to derive the scale factors for fake taus in MC, which consist of total of 24 parameters
depending on four types, three $\pt$ bins, two bins for 1- and 3-prong taus separately.
The values of these scale factors range from 0.3-1.28 are then used to correct the MC estimated fakes in the corresponding  signal regions with a single $b$-jet.
In the $t_{\ell}\thadhad$ channel, both taus can be misidentified, so the calibration is applied to each tau separately, following the same procedure used in the $\tlhad$ channel.
%using the lepton and fake tau charges, then the scale factors are multiplied together.
The nominal value of scale factors will vary according to their uncertainties in the final fit. A closure test is made for the fake tau estimations using the same procedure
in the $t_{\ell}\thadhad$-SS validation region, which shows a good agreement between data and the background prediction. 
%In the control regions with single lepton, $\met > 20$GeV and at least 2 light jets and 2 b-tagged jets are required to ensure that QCD contribution is negligible.
%The calibration is done depending on different source of the fake taus and \pt. The calibration factor are derived in the dedicated
%control regions discussed in Section~\ref{sec:strategy_Htautau}.

In the hadronic channels, the contribution of fakes is estimated partially from data by defining control regions (CR)
enriched in misidentified
tau candidates via loosened tau requirements with fake factors and partially from MC~\cite{ATLAS-CONF-2021-044}.
These CRs do not overlap with the main signal regions, discussed in Section~\ref{sec:strategy_Htautau}.
The CR selection requirements are analogous to those used to define signal regions, except
that the subleading tau candidate is required to fail the medium tau identification, but it still passes a loose requirement.
The contribution of fakes with subleading tau candidate can be calculated by rescaling the templates of loose taus in the CR
with fake factors (FF).
The templates are produced by subtracting all MC background contributions with real subleading taus from data.
The FFs are computed as the ratio of misidentified $\thad$ candidates that either pass or fail the medium tau ID selection in regions enriched in events originating from 
$W$+jets processes~\cite{ATLAS-CONF-2021-044}. 
%the Data-MC ($\mathrm{real~tau}$) ratio of the yields of events in which the tau candidate is passing or failing the medium tau ID selection
%in the $W$+jets samples~\cite{ATLAS-CONF-2021-044}.
%passing to failing the medium tau ID in the $W$+jets control region.
%We have compared
FFs obtained from the $W$+jets and from the same-sign $\thadhad$ control regions are compared and the differences are treated as a systematic uncertainty.

\subsection{Background with fake light leptons}
The background originating from non-prompt or misidentified light leptons is primarily from the multi-jet production.
%Electrons and muons only appear in the leptonic channels where exactly one electron or muon and at least one $\had$ are required.
%Since the rate of $\had$ candidate is much larger than for electron or muon, in the majority of the fake lepton events, the $\had$ candidate is misidentified and
%the electron or muon is prompt based on MC studies.
%However the multi-jet background has large event rate and can contribute with both electron or muon and $\had$ faked, especially to $t_{\ell}\had$ where the fake-lepton dominates. and the jet multiplicity is low.
The contribution from these events is estimated with a data-driven method called ABCD.
The estimate is based on the number of tight and loose light-leptons that passed and failed the PLIV cut in the low ($<25~\gev$) and high
($>25~\gev$) $\met$ regions. The number of
fake light leptons in the signal region is evaluated by scaling the number of loose leptons in the high $\met$ region by the ratio of tight and loose leptons in the
low $\met$ region. The contributions of prompt leptons and calibrated fake taus in the number of tight and loose leptons are subtracted using MC simulation.
A closure test is made for the background estimations in the signal depleted low BDT ($<-0.6$) regions, defined in Section~\ref{sec:tmva}.
The data are in good agreement with the background prediction in the
$t_{\ell}\had$ channels, while the fake light leptons are negligible in other leptonic channels. 
