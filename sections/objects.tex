%-------------------------------------------------------------------------------
\section{Event reconstruction}
\label{sec:objects}
%-------------------------------------------------------------------------------

Events are selected from $pp$ collisions at $\sqrt(s)$=13 TeV recorded by the ATLAS detector during 2015-2018.
Only events for which all relevant subsystems were operational are considered.
Events are required to have at least one vertex with two or more tracks with transverse momentum
%The event reconstruction is affected by multiple $pp$ collisions in a single bunch crossing and by collisions
%in neighbouring bunch crossings, referred to as pile-up. 
%Interaction vertices from the $pp$ collisions are reconstructed from at least two tracks with transverse momentum
($\pt$) larger than $400~\mev$ that are consistent with originating from the 
beam collision region in the $x$--$y$ plane. If more than one primary vertex candidate is found, the
candidate whose associated tracks form the largest sum of squared $\pt$~\cite{ATL-PHYS-PUB-2015-026}
is selected as the hard-scatter primary vertex.

Electron candidates~\cite{ATLAS-CONF-2016-024,Aaboud:2018ugz} are reconstructed from energy 
clusters in the EM calorimeter that are matched to reconstructed tracks in the inner detector;
electron candidates in the transition region between the EM barrel and endcap calorimeters 
($1.37 < |\eta_{\textrm{cluster}}| < 1.52$) are excluded.
Electron candidates are required to have $|\eta_{\textrm{cluster}}| < 2.47$, and to satisfy tight likelihood-based identification 
criteria~\cite{ATLAS-CONF-2016-024} based on calorimeter, tracking and combined variables that provide 
separation between electrons and jets. 

Muon candidates~\cite{Aad:2016jkr} are reconstructed by matching track segments in %the various
different layers of the muon spectrometer to tracks found in the inner detector;
the resulting muon candidates are re-fitted using the complete track information from both detector systems.
Muon candidates are required to satisfy medium identification criteria~\cite{Aad:2016jkr}. 

Electron (muon) candidates are matched to the primary vertex by requiring that the significance of their transverse impact parameter, $d_0$, 
satisfies $|d_0/\sigma(d_0)|<5\,(3)$, where $\sigma(d_0)$ is the measured uncertainty in $d_0$,
and by requiring that their longitudinal impact parameter, $z_0$, satisfies $|z_0 \sin\theta|<0.5$~mm.
%To reduce the impact of the trigger efficiency uncertainty around the threshold, the leptons are required to have \pt 1~GeV above the threshold.
To further reduce the background from non-prompt leptons, photon conversions and hadrons, lepton candidates are also required to be isolated 
in the tracker and in the calorimeter.
A track-based lepton isolation criterion is defined by calculating the quantity $I_R = \sum \pt^{\textrm{trk}}$ and $E_R = \sum E^{\textrm{clst}}$, where
the scalar sum includes all tracks and energy deposits (excluding the lepton candidate itself) within the cone defined by $\Delta R<0.2 (0.3)$ around the %axis
direction of the electron (muon). The electron (muon) candidates are required to satisfy both $I_R/\pt^\ell < 0.2~(0.3)$ and $E_R/\pt^\ell < 0.2~(0.3)$.

Tight isolation working points are also applied in some channels to reduce fake and non-prompt lepton contributions based on a trained isolation boosted decision tree \texttt{PromptLeptonVeto}(PLIV), which is to identify non-prompt light leptons using lifetime information associated with a track jet that matches the selected light lepton~\cite{ATLAS-CONF-2019-045}. These additional reconstructed charged particle tracks inside the jet can be
used to increase the precision for identifying the displaced decay vertex of heavy flavor (b, c) hadrons that produced non-prompt leptons.
The tight working points are used for high $\pt$ lepton.
The scale factors for the efficiencies of the tight \texttt{PLIV} working points are measured using the tag and probe method
with $Z\rightarrow l^+l^-$ ($l=e$ or $\mu$) events. These scale factors have also been checked for the electron or muon from the tau decay
using $Z\rightarrow\tau\tau\rightarrow e\mu4\nu$ samples, which is consistent at 2\% level. To be conservative, an
additional uncertainty of $\pm 2\%$ is considered for the PLIV efficiency for the tau-lepton in the lepton+$\thad$ channels.
%Additionally, the $\Htautau$ search requires leptons to
%satisfy a calorimeter-based isolation criterion; a requirement is made on the sum of the transverse energy of
%topological clusters within the cone defined by $\Delta R<0.2$ around the lepton, after subtracting the contribution
%from the energy deposit of the lepton itself and correcting for pile-up effects, to be less than a $\pt$-dependent 
%percentage of the lepton energy. 
%The combined efficiency of track-based and calorimeter-based isolation requirements made by the $\Htautau$ 
%search is 90\% (99\%) at $\pt=25\;(60)~\gev$.

%a Gradient isolation working points are used for both electrons and muons. The working points are based on cut maps derived from the $Z\to l^+l^-$ Monte Carlo samples. The efficiency maps are simply derived from the sum of transverse energy of the clusters deposited in the calorimeter in the given cone. The efficiency for the leptons with a certain $\pt$ is $(0.1143*\pt[\GeV]+92.14)\%*(0.1143*\pt[\GeV]+92.14)\%$.
%draft 1 version
%Candidate jets are reconstructed with the anti-$k_t$ algorithm~\cite{Cacciari:2008gp,Cacciari:2005hq,Cacciari:2011ma} with a
%radius parameter $R=0.4$, using topological clusters~\cite{Aad:2016upy} 
%built from energy deposits in the calorimeters calibrated to the electromagnetic scale. 
Candidate jets are reconstructed using the anti-$k_t$ algorithm~\cite{Cacciari:2008gp,Cacciari:2005hq} with a
radius parameter $R=0.4$ applied to topological energy clusters~\cite{Aad:2016upy} and charged-particle tracks, processed using
a particle-flow algorithm~\cite{Aad:2017epj77}. %newref Eur. Phys. J. C 77 (2017) 466
%as implemented in the \fastjet\ package~\cite{Cacciari:2011ma}.  
%Jet reconstruction in the calorimeter starts from topological clustering~\cite{Aad:2016upy} of individual calorimeter cells calibrated to the electromagnetic energy scale. 
The reconstructed jets are then calibrated to the particle level by the application of a jet energy scale 
derived from simulation and in situ corrections based on $\sqrt{s}=13~\tev$ data~\cite{Aaboud:2017jcu}. %(TBD: check PFlow jets description)
The calibrated jets used are required to have $\pt > 25~\gev$ and $|\eta| < 2.5$.
Jet four-momenta are corrected for pile-up effects using the jet-area method~\cite{Cacciari:2008gn}.

Quality criteria are imposed to reject events that contain any jets arising from non-collision sources
or detector noise~\cite{ATLAS-CONF-2015-029}.  To reduce the contamination due to jets originating from pile-up interactions,
additional requirements are imposed on the jet vertex tagger (JVT)~\cite{Aad:2015ina} output for jets with $\pt<60~\gev$ and $|\eta| < 2.4$.

Jets containing $b$-hadrons are identified ($b$-tagged) via the DL1r tagger~\cite{Aad:2019epj79,ATL-PHYS-PUB-2017-013} %newref
%\cite{Aad:2015ydr,ATL-PHYS-PUB-2016-012} 
that uses multivariate techniques to combine information about the impact parameters of displaced tracks and the  topological properties 
of secondary and tertiary decay vertices reconstructed within the jet. For each jet, a value for the multivariate $b$-tagging discriminant is 
calculated. A jet is considered $b$-tagged if this value is above the threshold corresponding to
an average 70\% efficiency to tag a $b$-quark jet, with a light-jet\footnote{Light-jet refers to a jet originating from the hadronisation of a light quark ($u$, $d$, $s$) or a gluon.} rejection factor of about 385 and a charm-jet rejection factor of about 12, as determined for jets with
$\pt >20~\gev$ and $|\eta|<2.5$ in simulated $\ttbar$ events~\cite{Aad:2019epj79}.

Hadronically decaying $\tau$-lepton ($\had$) candidates are reconstructed from energy clusters in the calorimeters and
associated inner-detector tracks~\cite{ATL-PHYS-PUB-2019-033}. %newref
%~\cite{ATL-PHYS-PUB-2015-045}.
Candidates are required to have either one or three associated tracks,
with a total charge of $\pm 1$. Candidates are required to have \pt 5 GeV higher than the trigger threshold, and $|\eta|<2.5$,
excluding the EM calorimeter's transition region.
A Recurrent Neural Network (RNN)~\cite{Graves:2012SCI}
%~\cite{Breiman:1984jka,Friedman:2002we,Freund:1997xna}
%A. Graves, Supervised Sequence Labelling with Recurrent Neural Networks, Studies in Computational Intelligence 385, Springer, 2012.
using calorimeter- and tracking-based variables is used to identify $\had$ candidates and reject jet backgrounds.
Three working points labelled loose, medium and tight are defined, and correspond to different $\had$ identification efficiency values, with the efficiency designed to be independent of $\pt$. The $\Htautau$ search uses the medium
working point for the nominal selection.
%, while the loose working point is used for background estimation.
The medium working point has a combined reconstruction and identification efficiency of 75\% (60\%) for one-prong (three-prong) $\had$ 
decays, and an expected rejection factor against light-jets of 35 (240)~\cite{ATL-PHYS-PUB-2019-033}. 
Electrons that are reconstructed as one-prong $\had$ candidates are removed via a BDT trained to reject electrons.
Events with a $b$-tagged $\had$ candidate are rejected.
%(TBD: check RNN description and update then numbers)

%Overlaps between candidate objects are removed sequentially, following this order: muons, electrons,
%$\tauhad$ (only for the $\Htautau$ search), and jets. In the $\Hbb$ search, firstly, electron candidates that lie 
%within $\Delta R = 0.01$ of a muon candidate are removed to suppress contributions from muon bremsstrahlung. 
%Overlaps between electron and jet candidates are resolved next, and finally, overlaps between remaining jet candidates 
%and muon candidates are removed. Clusters from identified electrons are not excluded during jet reconstruction. 
%In order to avoid double-counting of electrons as jets, the closest jet whose axis is within ${\Delta}R = 0.2$ of an electron 
%is discarded. If the electron is within ${\Delta}R = 0.4$ of the axis of any jet after this initial removal, the jet is retained and  the electron is removed.
%The overlap removal procedure between the remaining jet candidates and muon candidates is designed to remove those muons 
%that are likely to have arisen in the decay chain of hadrons and to retain the overlapping jet instead. 
%Jets and muons may also appear in close proximity when the jet results from high-$\pt$ muon bremsstrahlung, 
%and in such cases the jet should be removed and the muon retained. Such jets are characterised by having very 
%few matching inner-detector tracks. Selected muons that satisfy $\Delta R(\mu,{\textrm{jet}}) < 0.04+10~\gev/\pt^\mu$ are rejected
%if the jet has at least three tracks originating from the primary vertex; otherwise the jet is removed and the muon is kept.
%In the $\Htautau$ search, a fixed cone size of $\Delta R=0.2$ is used to determine the overlap between
%candidate objects, and only the highest-$\pt$ (leading) or the two leading $\tauhad$ candidates (depending on the
%analysis channel, see Section~\ref{sec:data_presel}) are considered to resolve their overlap with jets.

To avoid double counting reconstructed objects, an overlap removal procedure is applied.
%Overlaps between reconstructed objects are removed sequentially.
%Firstly,
Electron candidates that lie 
within $\Delta R = 0.01$ of a muon candidate are removed to suppress contributions from muon bremsstrahlung. 
%Overlaps between electron and jet candidates are resolved next, and finally, overlaps between remaining jet candidates 
%and muon candidates are removed.
Energy clusters from identified electrons are not excluded during jet reconstruction. 
In order to avoid double-counting of electrons as jets, the closest jet whose axis is within ${\Delta}R = 0.2$ of an electron 
is discarded if the jet is not $b$-tagged, otherwise the electron is removed.
If the electron is within ${\Delta}R = 0.4$ of the axis of any jet after this initial removal, the jet is retained and  the electron is removed.
The overlap removal procedure between the remaining jet candidates and muon candidates is designed to remove those muons 
that are likely to have arisen in the decay of hadrons and to retain the overlapping jet instead. 
Jets and muons may also appear in close proximity when the jet results from high-$\pt$ muon bremsstrahlung, 
and in such cases the jet should be removed and the muon retained. Such jets are characterised by having very 
few matching inner-detector tracks. Selected muons that satisfy $\Delta R(\mu,{\textrm{jet}}) < 0.2$ are rejected
if the jet is either $b$-tagged or has at least three tracks originating from the primary vertex; otherwise the jet is removed and the muon is kept.
The $\tauhad$ within a $\Delta R=0.2$ cone of an electron or muon are removed.
In order to avoid double-counting of $\tauhad$ as jets, the closest jet whose axis is
within ${\Delta}R = 0.2$ of a $\tauhad$ is discarded if the jet is not $b$-tagged, otherwise the $\tauhad$ is removed. 
% TEXT FROM BOYANG. NEED TO RECHECK WHAT WE HAVE ABOVE IS ACCURATE
%In the $\Htautau$ search, the objects are removed with the following sequence: if two electrons have overlapping second-layer cluster, or shared tracks, the electron with lower $\pt$ is removed; $\tauhad$ within a $\Delta R=0.2$ cone of an electron or muon are removed; if a muon sharing an ID track with an electron and the muon is calo-tagged, the muon is removed, otherwise the electron is removed; jets within a $\Delta R=0.2$ cone of an electron are removed; electrons within a $\Delta R=0.4$ cone of a jet are removed; when a muon ID track is ghost associated to a jet or within a $\Delta R=0.2$ cone of a jet, the jet is removed if it has less than 3 tracks with $\pt>500$ MeV or has a relative small $\pt$ ($\pt^{\mu}>0.5\pt^{\text{jet}} \text{ and } \pt^{\mu}>0.7[\text{the scalar sum of the } \pt \text{'s of the jet tracks with } \pt>500$ MeV]); muons within a $\Delta R=0.4$ cone of a jet are removed; jets within a $\Delta R=0.2$ cone of the leading $\tauhad$ ($\lephad$), or with the two leading $\tauhad$'s ($\hadhad$), are removed.

%firstly, the electron within a ${\Delta}R = 0.2$ cone of muons are excluded, then the $\tau_{had}$'s or jets within a ${\Delta}R = 0.2$ cone of an electron or muon are excluded. Finally, the jets within a ${\Delta}R = 0.2$ cone of the leading $\had$ ($\lephad$), or with the two leading $\had$’s ($\hadhad$) are excluded.

The missing transverse momentum $\mpt$ (with magnitude $\met$) is defined as the negative vector sum of the 
$\pt$ of all selected and calibrated objects in the event, including a term to account for momentum from soft particles 
in the event which are not associated with any of the selected objects. 
This soft term is calculated from inner-detector tracks matched to the selected primary vertex to make it more resilient to
contamination from pile-up interactions~\cite{Aaboud:2018tkc}.


