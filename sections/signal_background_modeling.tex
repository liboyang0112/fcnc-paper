%-------------------------------------------------------------------------------
\section{Signal and background modelling}
\label{sec:signal_background_model}
%-------------------------------------------------------------------------------

Signal and most background processes are modelled using Monte Carlo (MC) simulation.
After the event preselection, the main background is $\ttbar$ production, often in association with jets, denoted by $\ttbar$+jets in the following.
Small contributions arise from single-top-quark, $W/Z$+jets, multijet and diboson ($WW,WZ,ZZ$) production, as well as from the associated 
production of a vector boson $V$ ($V=W,Z$) or a Higgs boson and a $\ttbar$ pair ($\ttbar V$ and $\ttbar H$). All backgrounds 
with prompt leptons, i.e.\ those originating from the decay of a $W$ boson, a $Z$ boson, or a $\tau$-lepton,
are estimated using samples of simulated events and initially normalised to their theoretical cross sections.
In the simulation, the top-quark and SM Higgs boson masses are set to $172.5~\gev$ and $125~\gev$, respectively,
and the Higgs boson is forced to decay into $H\to \tau\tau$ with branching ratio calculated using \textsc{Hdecay}~\cite{Djouadi:1997yw}.  
Backgrounds with non-prompt electrons or muons, with photons or jets misidentified as electrons, or with jets misidentified as $\had$ candidates, 
generically referred to as fake leptons, are estimated using data-driven methods. 
The background prediction is further improved during the statistical analysis by performing a likelihood 
fit to data using several signal-depleted analysis regions, as discussed in Sections~\ref{sec:strategy_Htautau}.

%-------------------------------------------------------------------------------
\subsection{Simulated signal and background processes}
\label{sec:simulations}
%-------------------------------------------------------------------------------

Samples of simulated $\ttbar \to WbHq$ events were generated with the next-to-leading-order (NLO) generator\footnote{In the following, 
the order of a generator should be understood as referring to the order in the strong coupling constant at which the matrix-element calculation 
is performed.} {\powheg}~v2 \cite{Frixione:2007nw,Nason:2004rx,Frixione:2007vw,Alioli:2010xd}
%{\amcatnlolong}~2.4.3~\cite{Alwall:2014hca}  (referred to in the following as {\amcatnlo})
with the NNPDF3.0 NLO~\cite{Ball:2014uwa} parton distribution function (PDF) set and interfaced to {\pythia} 8.212~\cite{Sjostrand:2007gs} with the NNPDF2.3 LO~\cite{Ball:2012cx} PDF set for the modelling of parton showering, hadronisation, and the underlying event. 
The A14~\cite{ATLASUETune4} set of tuned parameters in {\pythia} controlling the description of multiparton interactions and  
initial- and final-state radiation, referred to as the tune, was used.
The signal sample is normalised to the same total cross section as used for the inclusive $t\bar{t}\to WbWb$ sample (see discussion below) and
assuming an arbitrary branching ratio of $\BR_{\mathrm{ref}}(t\to Hq)=0.1\%$.
The case of both top quarks decaying into $Hq$ is neglected in the analysis given the existing upper limits on $\BR(t \to Hq)$ (Section~\ref{sec:intro}).

The $tH$ signal events were generated by {\amcatnlolong}~2.6.2~\cite{Alwall:2014hca}  (referred to in the following as {\amcatnlo})
with the NNPDF3.0 NLO parton distribution function (PDF) set and interfaced to {\pythia} 8.212 with the NNPDF2.3 LO PDF set for the modelling of parton showering,
hadronisation, and the underlying event with the A14 tune.
Depending on either up quark or charm quark involved in the FCNC production, the effective Lagrangian of $tqH$ interaction are parametrised using
dim-6 operators~\cite{fcnc_production_theory}. We obtain $\sigma(ug\to tH)$ = 0.711 pb and $\sigma(cg\to tH)$ = 0.103 pb using $\BR_{\mathrm{ref}}(t\to Hq)=0.1\%$ as benchmark.   

The nominal sample used to model the $\ttbar$ background was generated with the NLO generator {\powheg}~v2
%\cite{Frixione:2007nw,Nason:2004rx,Frixione:2007vw,Alioli:2010xd}
using the NNPDF3.0 NLO PDF set. The {\powheg} model parameter $h_{\textrm{damp}}$, which controls 
matrix element to parton shower matching and effectively regulates the high-$\pt$ radiation, was set to 1.5 times the top-quark mass. 
The parton showers, hadronisation, and underlying event were modelled by {\pythia}~8.210 with the NNPDF2.3 LO PDF set in combination with the A14 tune.
Alternative $\ttbar$ simulation samples used to derive systematic uncertainties are described in Section~\ref{sec:syst_bkgmodeling}. 
The generated $\ttbar$ samples are normalised to a theoretical cross section of $832^{+46}_{-51}$~pb, 
computed using \textsc{Top++}~v2.0~\cite{Czakon:2011xx} at next-to-next-to-leading order (NNLO), 
including resummation of next-to-next-to-leading logarithmic (NNLL) soft gluon 
terms~\cite{Cacciari:2011hy,Baernreuther:2012ws,Czakon:2012zr,Czakon:2012pz,Czakon:2013goa}.

Samples of single-top-quark events corresponding to the $t$-channel production mechanism were generated with the 
{\powheg}~v1~\cite{Frederix:2012dh} generator, using the 4F scheme  for the NLO matrix-element calculations
and the fixed 4F \textsc{CT10}f\textsc{4}~\cite{Lai:2010vv} PDF set.
Samples corresponding to the $tW$- and $s$-channel production mechanisms were generated 
with {\powheg}~v1 using the CT10 PDF set. Overlaps between the $\ttbar$ and $tW$ final states were avoided by using 
the diagram removal scheme~\cite{Frixione:2005vw}.
The parton showers, hadronisation and the underlying event were modelled using {\pythia}~6.428~\cite{Sjostrand:2006za} 
with the CTEQ6L1~\cite{Pumplin:2002vw,Nadolsky:2008zw} PDF set 
in combination with the Perugia 2012 tune~\cite{Skands:2010ak}.
The single-top-quark samples are normalised to the approximate NNLO theoretical cross 
sections~\cite{Kidonakis:2011wy,Kidonakis:2010ux,Kidonakis:2010tc}. 

Samples of $W/Z$+jets events were generated with the {\sherpa}~2.2.1~\cite{Gleisberg:2008ta} generator. 
The matrix element was calculated for up to two partons at NLO and up to four partons at LO using 
\textsc{Comix}~\cite{Gleisberg:2008fv} and \textsc{OpenLoops}~\cite{Cascioli:2011va}. The matrix-element calculation 
is merged with the {\sherpa} parton shower~\cite{Schumann:2007mg} using the ME+PS@NLO prescription~\cite{Hoeche:2012yf}. 
The PDF set used for the matrix-element calculation is NNPDF3.0 NNLO~\cite{Ball:2014uwa} with a dedicated parton shower tuning developed for {\sherpa}. 
Separate samples were generated for different $W/Z$+jets categories using filters for a $b$-jet 
($W/Z$+$\geq$1$b$+jets), a $c$-jet and no $b$-jet ($W/Z$+$\geq$1$c$+jets), and with a veto on $b$- and $c$-jets 
($W/Z$+light-jets), which are combined into the inclusive $W/Z$+jets samples.
Both the $W$+jets and $Z$+jets samples are normalised to their respective inclusive NNLO theoretical 
cross sections calculated with \textsc{FEWZ}~\cite{Anastasiou:2003ds}.

Samples of $WW/WZ/ZZ$+jets events were generated with {\sherpa}~2.2.1 using the CT10 PDF set
and include processes containing up to four electroweak vertices. 
In the case of $WW/WZ$+jets ($ZZ$+jets) the matrix element was calculated for zero (up to one) additional partons 
at NLO and up to three partons at LO using the same procedure as for the $W/Z$+jets samples. 
The final states simulated require one of the bosons to decay leptonically and the other hadronically.
All diboson samples are normalised to their NLO theoretical cross sections provided by {\sherpa}. 

Samples of $\ttbar V$ and $\ttbar H$ events were generated with {\amcatnlo}~2.2.1, using NLO matrix elements and the NNPDF3.0 NLO PDF set,
and interfaced to {\pythia}~8.210 with the NNPDF2.3 LO PDF set and the A14 tune. 
%Instead, the $\ttbar V$ samples used in the $\Hbb$ search are based on LO matrix elements computed for up to two additional partons 
%using the NNPDF3.0 NLO PDF set, and merged using the CKKW-L approach~\cite{Lonnblad:2001iq}.
The $\ttbar V$ samples are normalised to the NLO cross section computed with {\amcatnlo}, while the $\ttbar H$ sample is normalised using 
the NLO cross section recommended in Ref.~\cite{deFlorian:2016spz}.

All generated samples, except those produced with the {\sherpa}~\cite{Gleisberg:2008ta} event generator, 
utilise \textsc{EvtGen}~1.2.0~\cite{Lange:2001uf} to model the decays of heavy-flavour hadrons. 
To model the effects of pile-up, events from minimum-bias interactions were generated using {\pythia}~8.186~\cite{Sjostrand:2007gs}  
in combination with the A2 tune~\cite{ATL-PHYS-PUB-2011-014}, 
and overlaid onto the simulated hard-scatter events according to the luminosity profile of the recorded data. 
The generated events were processed through a simulation~\cite{Aad:2010ah} of the ATLAS detector geometry and response 
using \textsc{Geant4}~\cite{Agostinelli:2002hh}. A faster simulation, where the full \textsc{Geant4} simulation of
the calorimeter response is replaced by a detailed parameterisation of the shower shapes~\cite{FastCaloSim},
was adopted for some of the samples used to estimate systematic uncertainties in background modelling.
Simulated events were processed through the same reconstruction software as the data, and corrections were applied so that the object identification 
efficiencies, energy scales and energy resolutions match those determined from data control samples.

%-------------------------------------------------------------------------------
\subsection{Backgrounds with fake leptons}
\label{sec:fakeleptons}
%-------------------------------------------------------------------------------

\subsubsection{Fake electrons and muons}
Electrons and muons only appear in the leptonic channels where there is exactly one electron or muon and at least one $\had$. Since the fake rate of $\had$ candidate is much larger than electron or muon, in the majority of the fake lepton events, the $\had$ candidate is misidentified and the electron or muon is real, which is also proved by monte carlo studies. However the multi-jet background has large event rate and can contribute with both electron or muon and $\had$ faked, espcially to $t_l\had$ where the fake-lepton dominates.
%and the jet multiplicity is low.
This kind of event is modelled by data-driven ABCD method by cutting on $\met$ and the tight lepton isolation using PLIV, which is widely used in ATLAS physics analyses.

%\subsubsection{Fake $\tau$-lepton candidates}
%\label{sec:faketaus}
%The background with one or more fake $\had$ candidates mainly arises from $\ttbar$ or
%multijet production, depending on the search channel. 
%Studies based on the simulation show that, for all the above processes, fake $\had$ candidates primarily result from the 
%misidentification of light-quark jets, with the contribution from $b$-quarks and gluon jets playing a subdominant role.
%It is also found that the fake rate decreases for all jet flavours as the $\had$ candidate $\pt$ increases.

%In the hadronic channels, this background is estimated partially from data by defining control regions (CR) enriched in fake $\had$ candidates via loosened $\had$ requirements with% proper fake factors and partially from monte carlo. These CRs do not overlap with the main search regions (SRs), discussed in Section~\ref{sec:strategy_Htautau}. The CR selection %requirements are analogous to those used to define the different SRs, except that the trailing $\had$ candidate is required to fail the medium $\had$ identification.

%In the leptonic channels, the events with real electron or muon and fake taus are modelled by calibrated monte carlo with scale factors derived from the dedicated $t\bar t$
%control regions ($CR_{tt}$) using the SM $t\bar t$ decay of dilepton events and semileptonically double-btagged lepton-jet events.
%The calibration is done depending on different source of the fake taus and \pt. The calibration factor are derived in the dedicated control regions discussed in Section~\ref{sec:st%rategy_Htautau}.

\subsubsection{Fake $\tau$-lepton candidates}
\label{sec:faketaus}
The background with one or more fake $\had$ candidates mainly arises from $\ttbar$ or
multijet production, depending on the search channels.
Studies based on the simulation show that, for all the above processes, fake $\had$ candidates primarily result from the
misidentification of light-quark jets, with the contribution from $b$-quarks and gluon jets playing a subdominant role.
It is also found that the fake rate decreases for all jet flavours as the $\had$ candidate $\pt$ increases.

In the leptonic channels, the events with real electron or muon and fake taus are modelled by calibrating monte carlo with scale factors
derived from the dedicated $t\bar t$
control regions ($CR_{tt}$) using the SM $t\bar t$ decay of dilepton events and semileptonically double-btagged lepton-jet events, summarized in Table~\ref{tab:srcr},
aimed at fake taus from different origins. There are four kinds of fake taus that need to be calibrated: Type1) the fake taus from $W$-jets ($\tau_{W}$)
with the opposite charge to the lepton;
Type2) $\tau_{W}$'s with the same charge as the lepton; Type 3) the fake taus from $b$-jets; Type4) the fake taus from other origins (mainly radiations).
These control regions are similar to the signal regions but with an addition $b$-jet or lepton defined as following:
$t_lt_l1b\thad$, $t_lt_l2b\thad$, $t_lt_h2b\thad$-2jSS, $t_lt_h2b\thad$-2jOS, $t_lt_h2b\thad$-3jSS, and $t_lt_h2b\thad$-3jOS.
The di-lepton regions ($t_lt_l1b\thad$ and $t_lt_l2b\thad$) are used to calibrate the Type3 and Type4 fake taus. The regions of
$t_lt_h2b\thad$-2jOS and $t_lt_h2b\thad$-3jOS are used to calibrate Type1 fake taus.
Similarly for the Type2, the regions of $t_lt_h2b\thad$-2jSS and $t_lt_h2b\thad$-3jSS are used.
A simultaneous fit to data is made to derive the scale factors for the fake taus in MC, which consist of total 24 parameters
including four types, 3 $\pt$ bins (25-35, 35-45, 45- GeV), 2 bins for 1 and 3 prong taus separately.
The values of these scale factors range from 0.3 to 1.28 and are applied to the corresponding signal regions with single b-tagged jet.

In the $t_l\thadhad$ channel, both taus can be fake, so the calibration is applied to each tau separately, following the same procedure as $\tlhad$ channels using the
lepton and fake tau charges, then the scale factors are multiplied together. The nominal value of the scale factors will vary along with other uncertainties from
combined preformace (CP) recommendations and theory uncertainties in the final fit.
%In the control regions with single lepton, $\met > 20$GeV and at least 2 light jets and 2 b-tagged jets are required to ensure that QCD contribution is negligible.
%The calibration is done depending on different source of the fake taus and \pt. The calibration factor are derived in the dedicated
%control regions discussed in Section~\ref{sec:strategy_Htautau}.

In the hadronic channels, this background is estimated partially from data by defining control regions (CR) enriched in fake $\had$
candidates via loosened $\had$ requirements with proper fake factors and partially from monte carlo~\cite{ATLAS-CONF-2021-044}.
These CRs do not overlap with the main search regions (SRs), discussed in Section~\ref{sec:strategy_Htautau}.
The CR selection requirements are analogous to those used to define the different SRs, except
that the trailing $\had$ candidate is required to fail the medium $\had$ identification.
The events with fake trailing $\had$ candidate can be calculated by rescaling the templates of fake taus in the Fake-CR with proper fake factors (FF).
The templates are aquired by subtracting all MC background contributions with real sub-leading taus from data. The FFs are computed as
the ratio of the Data$-$MC$_\mathrm{real~tau}$ yields passing the medium tau ID to which failing the medium tau ID from different control regions and
the differences are treated as part of systematics.
